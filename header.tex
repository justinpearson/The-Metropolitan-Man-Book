\documentclass[ebook,12pt]{memoir}
\usepackage[
    bookmarks=true,
    unicode=true,
    pdfborder={0 0 0},
    pdftitle={The Metropolitan Man},
    pdfauthor={Alexander Wales}, 
    breaklinks={true},
    pdfkeywords={superman, rationality},
    pdfencoding=auto
]{hyperref}                            % for \url

\usepackage[utf8]{inputenc}            % Use unicode or pdflatex deletes non-ascii chars like é

\linespread{1.1}                       % space between lines
\setlrmarginsandblock{2cm}{1.5cm}{*}     % left-right margins
\setulmarginsandblock{2.5cm}{*}{1}     % top-bottom margins
\checkandfixthelayout                  % weird-ass memoir hack
\usepackage{charter}                   % change font
\usepackage{verbatim}                  % for \verbatiminput

% Use \Emdash*, \Endash* (no need to use \Hyphdash) (* means non-breaking)
% and shortcuts: \=== \==
% https://tex.stackexchange.com/questions/103608/how-to-force-latex-not-to-break-the-line-after-a-hyphen
\usepackage[shortcuts]{extdash}

\begin{document}

%%%%%%%%%%%%%%%%%%%%%% Page 1 & 2: blank %%%%%%%%%%%%%%%%%
\thispagestyle{empty}  % no page number on this page
\phantom{lol}
\cleardoublepage

%%%%%%%%%%%%%%%%%%%%%% Page 3: Title %%%%%%%%%%%%%%%%%%%%%
\begin{center}
\thispagestyle{empty}
\vspace*{0.2in}  % vspace* : * means "do not discard this whitespace, 
                 %  despite it being at the top or bottom of the page"
\Huge\MakeUppercase{The Metropolitan Man}       \\
\vspace{0.5in}                                  
\large BY                                       \\
\vspace{0.1in}                                  
\LARGE \MakeUppercase{Alexander Wales}          \\
\normalsize                                     
\vspace{4in}
Find the original text at:                      \\
\footnotesize{\url{http://fanfiction.net/s/10360716/1/The-Metropolitan-Man}}
\end{center}
\clearpage

%%%%%%%%%%%%%%%%%%%%%% Page 4: Copyright %%%%%%%%%%%%%%%%%%%%%
\thispagestyle{empty}
\footnotesize

\noindent The Metropolitan Man \copyright\ 2014 Alexander Wales.

\vspace{.2in}

\noindent Alexander Wales website:

\url{https://alexanderwales.com/}

\vspace{.2in}

\noindent Text downloaded from fanfiction.net:

\url{https://www.fanfiction.net/s/10360716/1/The-Metropolitan-Man}

\vspace{.2in}

\noindent Other fanfics from Alexander Wales:

\url{https://www.fanfiction.net/u/4976703/alexanderwales}

\vspace{.2in}

\noindent Support Alexander Wales on Patreon:

\url{https://www.patreon.com/alexanderwales}

\vspace{.2in}

\noindent Cover illustrations \copyright\ Justin Maller:

\url{http://justinmaller.com/wallpaper/356/}

\vspace{.2in}

\noindent Cover art downloaded from Mike Schw\"orer's GitHub repo:

\url{https://github.com/Mikescher/Metropolitan-Man-Lyx}

\vspace{.2in}

\noindent Typeset by Justin P. Pearson:

\url{http://justinppearson.com}


\vfill

\begin{description}
    \item[Category] Superman
    \item[Genre] Adventure, Mystery 
    \item[Language] English 
    \item[Published] May 18, 2014
    \item[Updated] July 25, 2014
    \item[Status] Complete
    \item[Rating] M 
    \item[Chapters] 13 
    \item[Words] 80,698
    \item[Publisher] www.fanfiction.net 
\end{description}

\normalsize
\cleartorecto

%%%%%%%%%%%%%%%%%%%%% Page 5: preamble %%%%%%%%%%%%%%%%%%%%%%%

\thispagestyle{empty}

\vspace*{2in}

\textbf{How to generate and print this PDF yourself}

\vspace*{.5cm}

\footnotesize

I can't resist showing you how to create the PDF for \emph{The Metropolitan Man} yourself. 
The following two pages show the code that produced this 
PDF\footnote{ \tiny{Actually the code that produced this PDF --- \texttt{build.py} --- is slightly more complicated
because it fixes some minor typesetting bugs in hyphenation and quotation marks.}}. 
By running it in your terminal with \texttt{python3 build.py}, the code will
download the \emph{The Metropolitan Man} from \url{fanfiction.net} 
and typeset it into a PDF with LaTeX (a typesetting program common in academia).

The code lives at this GitHub repository:

\noindent \url{https://github.com/justinpearson/The-Metropolitan-Man-Book}

Armed with the PDF and some cover art, you can then order a physical copy to be printed by 
an online book-printer like \url{http://lulu.com}.

I hope you can tweak this code and use it to download, typeset, 
and print your own books!

\ \ \ \ --- Justin Pearson, Apr 2019

\normalsize
\cleartoverso


%%%%%%%%%%%%%%%%%%%% Page 6: python script %%%%%%%%%%%%%%%%%%%%%%%%%%%
\thispagestyle{empty}

% \footnotesize
% \noindent The following shell script will download and typeset \emph{The Metropolitan Man} 
%using curl, sed, Python, BeautifulSoup, pandoc, and pdflatex. 

% \vspace{.2cm}
% \vfill

\tiny

\verbatiminput{build-simple.py}

\normalsize
\clearpage


%%%%%%%%%%%%%%%%%%%% Page 7: python script %%%%%%%%%%%%%%%%%%%%%%%%%%%
\thispagestyle{empty}

\footnotesize
\noindent The script has four stages:

\begin{description}

\item[Download.] We use the Selenium Webdriver browser automation tool\footnote{
    \tiny{ \url{https://selenium-python.readthedocs.io/} }
} to programatically drive the Firefox web browser to fanfiction.net to
download the HTML of each of the story's 13 chapters. Originally (Apr 2019),
I used the simpler command-line tool \texttt{curl}, but currently (Sep 2021) fanfiction.net
has a Cloudflare-powered captcha system that detects and rejects requests from non-browser web scraping tools.
Surprisingly, the captcha even detects a re-used automated browser session,
so you have to tear down and re-initialize the browser each iteration.

\item[Parse HTML.] We use BeautifulSoup\footnote{
    \tiny{ \url{https://www.crummy.com/software/BeautifulSoup/} }
}, a Python HTML-parsing library,
to extract the text of each chapter from its scraped HTML webpage.
We extract both the chapter title and the HTML \texttt{<div>} tag containing
the chapter's text contents. We embed the chapter title in an \texttt{<h1>} header tag
and prepend it to the story's \texttt{<div>}, because pandoc --- used later in the pipeline ---
converts header tags to \LaTeX\ chapters. 

\item[HTML to TeX.] We use Pandoc\footnote{\tiny{\url{https://pandoc.org/}}} to convert
each chapter's HTML to TeX format, converting all double-quote characters to ``smart quotes''.

\item[Assemble chapters.] We concatenate the TeX code of the 13 chapters, bookending them with a
LaTeX preamble and a footer, saving the result as \texttt{mm.tex}. Interestingly, the header file (\texttt{header.tex})
contains a LaTeX command to inject the source code of \texttt{build-simple.py} into the final tex file! How meta.

\item[TeX to PDF.] We use the \LaTeX\ document preparation system (installed via MacTeX\footnote{
    \tiny{ \url{http://tug.org/mactex/} }
}) to build the final PDF, \texttt{mm.pdf}. We run it twice, since the first run does not
generate the table of contents.

\end{description}


\normalsize
\cleartorecto


%%%%%%%%%%%%%%%%%%%%% Page 9: TOC %%%%%%%%%%%%%%%%%%%%%%%%%%%%
% \frontmatter        % Use lowercase Roman numerals as page numbers.
\tableofcontents*  % * means "no self-reference to TOC in the TOC"
\addtocontents{toc}{\protect\thispagestyle{empty}}   % no page numbers on the TOC pages (ch 1 should be page 1)
\pagenumbering{gobble}

\mainmatter         % Restart page number and use Arabic numbers.


