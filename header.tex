\documentclass[ebook,12pt]{memoir}
\usepackage[
    bookmarks=true,
    unicode=true,
    pdfborder={0 0 0},
    pdftitle={The Metropolitan Man},
    pdfauthor={Alexander Wales}, 
    breaklinks={true},
    pdfkeywords={superman, rationality},
    pdfencoding=auto
]{hyperref}                            % for \url

\usepackage[utf8]{inputenc}            % Use unicode or pdflatex deletes non-ascii chars like é

\linespread{1.1}                       % space between lines
\setlrmarginsandblock{2cm}{1.5cm}{*}     % left-right margins
\setulmarginsandblock{2.5cm}{*}{1}     % top-bottom margins
\checkandfixthelayout                  % weird-ass memoir hack
\usepackage{charter}                   % change font
\usepackage{verbatim}                  % for \verbatiminput


% Fixes the annoying linebreak at ---'' , for example,
% “Ma’am, if you don’t want me bringing pistols into your home—
% ” Floyd began.
% Use \Emdash*, \Endash* (no need to use \Hyphdash) (* means non-breaking)
% and shortcuts: \=== \==
% https://tex.stackexchange.com/questions/103608/how-to-force-latex-not-to-break-the-line-after-a-hyphen
\usepackage[shortcuts]{extdash}

\begin{document}

%%%%%%%%%%%%%%%%%%%%%% Page 1 & 2: blank %%%%%%%%%%%%%%%%%
\thispagestyle{empty}  % no page number on this page
\phantom{lol}
\cleardoublepage

%%%%%%%%%%%%%%%%%%%%%% Page 3: Title %%%%%%%%%%%%%%%%%%%%%
\begin{center}
\thispagestyle{empty}
\vspace*{0.2in}  % vspace* : * means "do not discard this whitespace, 
                 %  despite it being at the top or bottom of the page"
\Huge\MakeUppercase{The Metropolitan Man}       \\
\vspace{0.5in}                                  
\large BY                                       \\
\vspace{0.1in}                                  
\LARGE \MakeUppercase{Alexander Wales}          \\
\normalsize                                     
\vspace{4in}
Find the original text at:                      \\
\footnotesize{\url{http://fanfiction.net/s/10360716/1/The-Metropolitan-Man}}
\end{center}
\clearpage

%%%%%%%%%%%%%%%%%%%%%% Page 4: Copyright %%%%%%%%%%%%%%%%%%%%%
\thispagestyle{empty}
\footnotesize

\noindent The Metropolitan Man \copyright\ 2014 Alexander Wales.

\vspace{.2in}

\noindent Alexander Wales website:

\url{https://alexanderwales.com/}

\vspace{.2in}

\noindent Text downloaded from fanfiction.net:

\url{https://www.fanfiction.net/s/10360716/1/The-Metropolitan-Man}

\vspace{.2in}

\noindent Other fanfics from Alexander Wales:

\url{https://www.fanfiction.net/u/4976703/alexanderwales}

\vspace{.2in}

\noindent Support Alexander Wales on Patreon:

\url{https://www.patreon.com/alexanderwales}

\vspace{.2in}

\noindent Cover illustrations \copyright\ Justin Maller:

\url{http://justinmaller.com/wallpaper/356/}

\vspace{.2in}

\noindent Cover art downloaded from Mike Schw\"orer's GitHub repo:

\url{https://github.com/Mikescher/Metropolitan-Man-Lyx}

\vspace{.2in}

\noindent Typeset by Justin P. Pearson:

\url{http://justinppearson.com}


\vfill

\begin{description}
    \item[Category] Superman
    \item[Genre] Adventure, Mystery 
    \item[Language] English 
    \item[Published] May 18, 2014
    \item[Updated] July 25, 2014
    \item[Status] Complete
    \item[Rating] M 
    \item[Chapters] 13 
    \item[Words] 80,698
    \item[Publisher] www.fanfiction.net 
\end{description}

\normalsize
\cleartorecto

%%%%%%%%%%%%%%%%%%%%% Page 5: preamble %%%%%%%%%%%%%%%%%%%%%%%

\thispagestyle{empty}

\vspace*{2in}

\textbf{How to generate and print this PDF yourself}

\vspace*{.5cm}

\footnotesize

I can't resist showing you how to create the PDF for \emph{The Metropolitan Man} yourself. 
The following two pages show the code that produced this 
PDF\footnote{ \tiny{Actually the code to produce this PDF is slightly more complicated 
because it fixes some minor typesetting bugs in hyphenation and quotation marks.}}. 
By running it in your terminal, the code will
download the \emph{The Metropolitan Man} from \url{fanfiction.net} 
and typeset it into a PDF with LaTeX (a typesetting program common in academia).
I tried to use only common programs like \texttt{curl}, \texttt{sed}, and \texttt{python3},
and if you're not familiar with \texttt{pandoc} (an excellent document conversion tool), 
I hope this project will show you how easy and powerful it is.

The code lives at this GitHub repository:

\noindent \url{https://github.com/justinpearson/The-Metropolitan-Man-Book}

Armed with the PDF and some cover art, you can then order a physical copy to be printed by 
an online book-printer like \url{http://lulu.com}.

I hope you can tweak this code and use it to download, typeset, 
and print your own books!

\ \ \ \ --- Justin Pearson, Apr 2019

\normalsize
\cleartoverso


%%%%%%%%%%%%%%%%%%%% Page 6: bash script %%%%%%%%%%%%%%%%%%%%%%%%%%%
\thispagestyle{empty}

% \footnotesize
% \noindent The following shell script will download and typeset \emph{The Metropolitan Man} 
%using curl, sed, Python, BeautifulSoup, pandoc, and pdflatex. 

% \vspace{.2cm}
% \vfill

\tiny

\verbatiminput{build-simple.sh}

\normalsize
\clearpage


%%%%%%%%%%%%%%%%%%%% Page 7: python script %%%%%%%%%%%%%%%%%%%%%%%%%%%
\thispagestyle{empty}

\footnotesize
\noindent The shell script on the previous page uses the following Python script 
to parse the HTML from fanfiction.net. This script uses the BeautifulSoup package to
select both the chapter title and the HTML \texttt{<div>} tag containing
the story. It prepends the chapter title to the story's \texttt{<div>}
as an \texttt{<h1>} header tag, because pandoc --- used later in the pipeline --- 
converts header tags to \LaTeX\ chapters. 
The script reads from stdin and prints the pruned HTML to stdout, making this script suitable 
for use in Unix pipelines. 

\vspace{.5cm}

\tiny

\verbatiminput{prune_html.py}

\normalsize
\cleartorecto


%%%%%%%%%%%%%%%%%%%%% Page 9: TOC %%%%%%%%%%%%%%%%%%%%%%%%%%%%
% \frontmatter        % Use lowercase Roman numerals as page numbers.
\tableofcontents*  % * means "no self-reference to TOC in the TOC"
\addtocontents{toc}{\protect\thispagestyle{empty}}   % no page numbers on the TOC pages (ch 1 should be page 1)
\pagenumbering{gobble}

\mainmatter         % Restart page number and use Arabic numbers.


