\documentclass[ebook,12pt]{memoir}
\usepackage[
    bookmarks=true,
    unicode=true,
    pdfborder={0 0 0},
    pdftitle={The Metropolitan Man},
    pdfauthor={Alexander Wales}, 
    breaklinks={true},
    pdfkeywords={superman, rationality},
    pdfencoding=auto
]{hyperref}                            % for \url

\usepackage[utf8]{inputenc}            % Use unicode or pdflatex deletes non-ascii chars like é

\linespread{1.1}                       % space between lines
\setlrmarginsandblock{2cm}{1.5cm}{*}     % left-right margins
\setulmarginsandblock{2.5cm}{*}{1}     % top-bottom margins
\checkandfixthelayout                  % weird-ass memoir hack
\usepackage{charter}                   % change font
\usepackage{verbatim}                  % for \verbatiminput

% Use \Emdash*, \Endash* (no need to use \Hyphdash) (* means non-breaking)
% and shortcuts: \=== \==
% https://tex.stackexchange.com/questions/103608/how-to-force-latex-not-to-break-the-line-after-a-hyphen
\usepackage[shortcuts]{extdash}

\begin{document}

%%%%%%%%%%%%%%%%%%%%%% Page 1 & 2: blank %%%%%%%%%%%%%%%%%
\thispagestyle{empty}  % no page number on this page
\phantom{lol}
\cleardoublepage

%%%%%%%%%%%%%%%%%%%%%% Page 3: Title %%%%%%%%%%%%%%%%%%%%%
\begin{center}
\thispagestyle{empty}
\vspace*{0.2in}  % vspace* : * means "do not discard this whitespace, 
                 %  despite it being at the top or bottom of the page"
\Huge\MakeUppercase{The Metropolitan Man}       \\
\vspace{0.5in}                                  
\large BY                                       \\
\vspace{0.1in}                                  
\LARGE \MakeUppercase{Alexander Wales}          \\
\normalsize                                     
\vspace{4in}
Find the original text at:                      \\
\footnotesize{\url{http://fanfiction.net/s/10360716/1/The-Metropolitan-Man}}
\end{center}
\clearpage

%%%%%%%%%%%%%%%%%%%%%% Page 4: Copyright %%%%%%%%%%%%%%%%%%%%%
\thispagestyle{empty}
\footnotesize

\noindent The Metropolitan Man \copyright\ 2014 Alexander Wales.

\vspace{.2in}

\noindent Alexander Wales website:

\url{https://alexanderwales.com/}

\vspace{.2in}

\noindent Text downloaded from fanfiction.net:

\url{https://www.fanfiction.net/s/10360716/1/The-Metropolitan-Man}

\vspace{.2in}

\noindent Other fanfics from Alexander Wales:

\url{https://www.fanfiction.net/u/4976703/alexanderwales}

\vspace{.2in}

\noindent Support Alexander Wales on Patreon:

\url{https://www.patreon.com/alexanderwales}

\vspace{.2in}

\noindent Cover illustrations \copyright\ Justin Maller:

\url{http://justinmaller.com/wallpaper/356/}

\vspace{.2in}

\noindent Cover art downloaded from Mike Schw\"orer's GitHub repo:

\url{https://github.com/Mikescher/Metropolitan-Man-Lyx}

\vspace{.2in}

\noindent Typeset by Justin P. Pearson:

\url{http://justinppearson.com}


\vfill

\begin{description}
    \item[Category] Superman
    \item[Genre] Adventure, Mystery 
    \item[Language] English 
    \item[Published] May 18, 2014
    \item[Updated] July 25, 2014
    \item[Status] Complete
    \item[Rating] M 
    \item[Chapters] 13 
    \item[Words] 80,698
    \item[Publisher] www.fanfiction.net 
\end{description}

\normalsize
\cleartorecto

%%%%%%%%%%%%%%%%%%%%% Page 5: preamble %%%%%%%%%%%%%%%%%%%%%%%

\thispagestyle{empty}

\vspace*{2in}

\textbf{How to generate and print this PDF yourself}

\vspace*{.5cm}

\footnotesize

I can't resist showing you how to create the PDF for \emph{The Metropolitan Man} yourself. 
The following two pages show the code that produced this 
PDF\footnote{ \tiny{Actually the code that produced this PDF --- \texttt{build.py} --- is slightly more complicated
because it fixes some minor typesetting bugs in hyphenation and quotation marks.}}. 
By running it in your terminal with \texttt{python3 build.py}, the code will
download the \emph{The Metropolitan Man} from \url{fanfiction.net} 
and typeset it into a PDF with LaTeX (a typesetting program common in academia).

The code lives at this GitHub repository:

\noindent \url{https://github.com/justinpearson/The-Metropolitan-Man-Book}

Armed with the PDF and some cover art, you can then order a physical copy to be printed by 
an online book-printer like \url{http://lulu.com}.

I hope you can tweak this code and use it to download, typeset, 
and print your own books!

\ \ \ \ --- Justin Pearson, Apr 2019

\normalsize
\cleartoverso


%%%%%%%%%%%%%%%%%%%% Page 6: python script %%%%%%%%%%%%%%%%%%%%%%%%%%%
\thispagestyle{empty}

% \footnotesize
% \noindent The following shell script will download and typeset \emph{The Metropolitan Man} 
%using curl, sed, Python, BeautifulSoup, pandoc, and pdflatex. 

% \vspace{.2cm}
% \vfill

\tiny

\verbatiminput{build-simple.py}

\normalsize
\clearpage


%%%%%%%%%%%%%%%%%%%% Page 7: python script %%%%%%%%%%%%%%%%%%%%%%%%%%%
\thispagestyle{empty}

\footnotesize
\noindent The script has four stages:

\begin{description}

\item[Download.] We use the Selenium Webdriver browser automation tool\footnote{
    \tiny{ \url{https://selenium-python.readthedocs.io/} }
} to programatically drive the Firefox web browser to fanfiction.net to
download the HTML of each of the story's 13 chapters. Originally (Apr 2019),
I used the simpler command-line tool \texttt{curl}, but currently (Sep 2021) fanfiction.net
has a Cloudflare-powered captcha system that detects and rejects requests from non-browser web scraping tools.
Surprisingly, the captcha even detects a re-used automated browser session,
so you have to tear down and re-initialize the browser each iteration.

\item[Parse HTML.] We use BeautifulSoup\footnote{
    \tiny{ \url{https://www.crummy.com/software/BeautifulSoup/} }
}, a Python HTML-parsing library,
to extract the text of each chapter from its scraped HTML webpage.
We extract both the chapter title and the HTML \texttt{<div>} tag containing
the chapter's text contents. We embed the chapter title in an \texttt{<h1>} header tag
and prepend it to the story's \texttt{<div>}, because pandoc --- used later in the pipeline ---
converts header tags to \LaTeX\ chapters. 

\item[HTML to TeX.] We use Pandoc\footnote{\tiny{\url{https://pandoc.org/}}} to convert
each chapter's HTML to TeX format, converting all double-quote characters to ``smart quotes''.

\item[Assemble chapters.] We concatenate the TeX code of the 13 chapters, bookending them with a
LaTeX preamble and a footer, saving the result as \texttt{mm.tex}. Interestingly, the header file (\texttt{header.tex})
contains a LaTeX command to inject the source code of \texttt{build-simple.py} into the final tex file! How meta.

\item[TeX to PDF.] We use the \LaTeX\ document preparation system (installed via MacTeX\footnote{
    \tiny{ \url{http://tug.org/mactex/} }
}) to build the final PDF, \texttt{mm.pdf}. We run it twice, since the first run does not
generate the table of contents.

\end{description}


\normalsize
\cleartorecto


%%%%%%%%%%%%%%%%%%%%% Page 9: TOC %%%%%%%%%%%%%%%%%%%%%%%%%%%%
% \frontmatter        % Use lowercase Roman numerals as page numbers.
\tableofcontents*  % * means "no self-reference to TOC in the TOC"
\addtocontents{toc}{\protect\thispagestyle{empty}}   % no page numbers on the TOC pages (ch 1 should be page 1)
\pagenumbering{gobble}

\mainmatter         % Restart page number and use Arabic numbers.




%%%%%%%%%%%%%%%%%%% NEW CHAPTER %%%%%%%%%%%%%%%%%%%%%%%%

\hypertarget{literally-incredible}{%
\chapter{Literally Incredible}\label{literally-incredible}}

\emph{Author's Note: This story is rated M for adult language and
themes, including discussion of sexual violence.}

\begin{center}\rule{0.5\linewidth}{0.5pt}\end{center}

Lex Luthor had been lounging in the Skylight Club when he'd first heard
of Superman.

``He's a fellow that flies through the sky!'' declared one of the
patrons, whose name Lex couldn't recall, a good sign that the man was
someone unimportant.

``Impossible,'' said Lex with a mild tone that nevertheless carried
across the room. The conversation at the table stopped, and Lex unfolded
himself from his customary chair and walked over to join the three men,
holding a martini in one hand. Lex wore a suit, one of his more casual
ones that had only cost what a dockworker made in a month. His head was
completely free of hair, save for his thick, expressive eyebrows. He was
a dashing figure, he knew, muscular and well‐proportioned, the result of
the delicate care he gave to his body. The man who had been speaking,
the one who had said that a man flew through the sky, was wearing a
charcoal grey suit that was three months out of fashion.

``Alexander Luthor,'' he said, holding out his hand. ``But everyone
calls me Lex. Now, tell me about this flying man, and I'll see what to
think of the matter.''

``Dimitri Vladkov,'' said the man. He seemed shaken, but it was
difficult to tell whether that was from the personal attention of Lex
Luthor or from whatever delusions he was apparently suffering. ``There
was this car crash down near 1st Ave and 30th Street, you could see it
comin', but all of a sudden this guy swoops down from the sky. He was
wearing a funny costume, blue tights and a red cape, with a big `S' on
his chest, and he stops these two cars from hittin' each other, moving
fast as lightnin'.''

``I see,'' said Lex with an arched eyebrow. ``And how did he fly? Did he
flap his wings, like a bird? Did he use engines, like a plane?''

``Well, I didn't see him come down Mr. Luthor, but I saw when he left,
and he just stuck his hands up in the air and lifted off, like he was
being pulled up by strings.'' He looked at Lex's face. ``Only there
weren't no strings, not that I could see, and I looked for 'em.''

``And when he stopped this car crash from happening,'' said Lex in a
calm and steady voice, ``He did so with only his hands? How would he
even have known that there was going to be a car crash?''

``I don't know how he knew it was gonna happen,'' said Dimitri. ``But he
landed right between `em and put out his hands, like he wasn't afraid of
gettin' squished, and sure enough he touched 'em like they were barely
there, slowed 'em right down, an put a dent in each of 'em.''

``So not only can this man fly, he has incredible speed and strength as
well, if what you're saying is true?'' asked Lex. His smile was so sharp
it could have cut glass.

``Well, yeah Mr. Luthor,'' said Dimitri. ``He didn't say nothin'
afterwards, just looked to make sure that the drivers were alright and
then flew away, fast as a racehorse but straight up into the sky. We had
all sorts of questions, but he didn't answer none of 'em.''

``Thank you,'' said Lex. He signaled for the barman to get Dimitri a
drink, then went back to sit in his customary chair and think.

Heavier than air flight had been given its first practical demonstration
almost two decades ago, but to do it without the assistance of a machine
was physically impossible. It was highly probable that Dimitri Vladkov
had hallucinated, or that he was simply lying for attention. Lex also
entertained the notion that the trick of flight had been accomplished by
smoke and mirrors, and that Dimitri had merely been fooled, but he
couldn't see what the point of that would be.

Lex frowned, and returned to other thoughts. Yet even as he tried to
decide what to do about Nikola Tesla, who was staying in the Hotel
Metropolis on LexCorp's dime, his thoughts returned to the improbable
story about a man who could fly. With a twitch of his fingers, he
signaled for Mercy Graves, his indispensable secretary and the only
woman allowed in the Skylight Club before the sun went down.

``A pencil and paper,'' said Lex. Mercy nodded, and from a large purse
she kept at her side produced both for him, nearly before he had asked
for them. He'd won Mercy's service in a poker game three years before,
and often wondered how he had managed without her.

It was a simple physics problem, with a high number of variables
involved, but Lex was nothing if not quick to attack a problem. He had
always liked numbers. After a few minutes of working at it, he had an
upper and lower bound estimate on the amount of force that would be
required to stop two cars from hitting each other, and another estimate
for what it would take to raise a man into the air ``as fast as a
racehorse''. He frowned at the answers.

``Mercy darling, there was very nearly a car accident near 1st Avenue
and 30th Street. Be a dear and see if you can't find me some
eyewitnesses to speak with.'' He knew it was foolish, but if money and
power didn't allow you to chase down the things that piqued your
curiosity, Lex didn't know what they were good for.

Three hours later, Lex stood at the intersection himself, looking
around. Mercy had gotten corroboration from eight witnesses, which only
raised further questions. Talking to more people would be useless,
especially since their stories had begun to contradict one another
fairly quickly. Lex took this as evidence that this wasn't some
elaborate ruse, or at least that if it was a ruse it had been
constructed by someone sufficiently intelligent. Eyewitness accounts
were notoriously unreliable, but most people didn't know this, and so
someone running a confidence scheme of some sort would likely have had
the confederates agree on a story. Lex looked down at the street, which
showed some patches of rubber where the cars had skidded, then up at the
sky. It might have been possible to do it with ropes and wires, though
nearly impossible to hide.

There was a type of elastic rope known as a bungee, and if you could
time it absolutely perfectly, you might be able to drop down and appear
right between two cars, touching the ground just as you reached zero
velocity. From there, you could use a harness and carabiners to clip on
to something while everyone was distracted by an elaborate costume. That
would make ``flying'' as simple as unclipping again and allowing the
elastic to propel you skywards. It would be delicate work, and
incredibly dangerous, but Lex had seen enough Charlie Chaplin films to
know that sometimes people did delicate and dangerous things simply for
the benefit of an audience and a small amount of money. That left the
question of hiding the ropes themselves, which would be no easy task,
even if the ropes were quite small, and you would also need a large
number of people to be complicit, which would further complicate things,
and all to accomplish what? It reminded Lex of a magic trick, and Lex
hated magic tricks, at least until he figured out how they were done.
After ten minutes of looking around, Lex grit his teeth. Mercy, standing
just behind him, politely coughed.

``You're right,'' said Lex. ``Enough time wasted on this distraction.''
He forced a smile. ``Put out the proper feelers. If someone tries this
stunt again, I want to know about it.''

\begin{center}\rule{0.5\linewidth}{0.5pt}\end{center}

``He came outta nowhere,'' said Little Tony.

Lex gave the man a sympathetic nod. ``Tell me about him.''

``We was robbin' the jewelry store,'' said Little Tony. ``An he came
outta nowhere, left the door spinnin' behind him.'' Little Tony was a
giant of a man, ironically named by his fellow thugs who considered that
the height of wit.

Lex Luthor had gone legitimate five years ago. Oh, he hired goons from
time to time for various bits of dirty work and still maintained
contacts in the criminal underworld, as well as receiving cash into a
slush fund from enterprises that he'd set in motion long ago~---
whorehouses, fighting rings, smuggling operations, and things of that
nature~--- but the fund was never touched by him, and existed only in
case there was an emergency. But for the most part, the crimes that Lex
Luthor was guilty of were white collar crimes, the kind that it would
take a forensic accountant or highly trained lawyer to unravel, and even
those he didn't do too often. Lex didn't see the need to run underground
gambling dens when he could get a special piece of legislation passed
that would allow an exclusive permit for a casino on the outskirts of
Metropolis. There was no need to be a criminal when you could get the
law to work for you.

Little Tony worked for Willie Calhoun, one of the largest crime bosses
in Metropolis and a former mentor to Lex Luthor. Lex and Willie had
parted ways amicably around the time that Lex was picking up his first
doctorate, but they'd always kept in touch, and occasionally they would
call in favors. Lex was talking to Little Tony in a small room lit by a
bare bulb as the result of one of these favors, as well as a promise to
pay for Little Tony's legal expenses. The large man was currently out on
bail.

``Leroy spun around and shot at him,'' said Little Tony. ``But the guy
moved as fast as lightning, and had his hand around tha barrel of tha
shotgun before it even went off.''

``Was there a thunderclap?'' asked Lex.

Little Tony scratched his head.

``Was there a loud sound that accompanied his movements?'' Lex asked. He
was out of practice in dealing with people like Little Tony. That had
once been his whole life.

``Nah,'' said Little Tony. ``Just like a little breeze, you know?'' Not
nearly as fast as lightning then, just a turn of phrase that people
seemed to like using.

``Leroy missed then?'' asked Lex.

``No,'' said Little Tony, shaking his head. ``Hit him square inna chest
with the full load.''

Lex frowned.

``I'm tellin' a truth,'' said Little Tony. ``Buckshot bounced offa him
like it were nothin'.''

``Did it tear the costume?'' asked Lex.

``The suit?'' asked Little Tony. Lex nodded. ``Yeah, tore it right up,
ripped it good. How'd you know?'' Lex hadn't known, he'd just been
asking, but it wouldn't do to tell Little Tony that.

``Continue on,'' said Lex.

``Well, this guy takes Leroy's shotgun right outta his hands, bends it
in two, and drops it to the floor. Guy just got shot in the chest an
acts like he didn't even notice. He looks at me and says that we're
under arrest, an' I tell him he ain't no cop, an' he says somethin'
about a citizen's arrest. An' as he's goin' on 'bout how he's got a
legal power or whatever, I rush him. That weren't too smart, because
before I know it I'm on my back.'' Little Tony rubbed the back of his
head and let out a sigh.

``Could you see him move?'' asked Lex. ``When he fought back?''

``Sure,'' said Little Tony. ``An' he didn't hurt me none. It was like I
was a little kid to him. All of a sudden I was flipped around and laid
out on the ground, gentle like he was worried about hurtin' me. Then we
just waited for the police to come, since Leroy and I didn't wanna try
our luck again. This guy, he gives the police a salute, didn't talk
about nothin' but the robbery even though they had all sorts of
questions for him, and then flew off.''

Lex frowned. ``One final question. How did he know about the robbery?''

``He musta seen us go in,'' said Little Tony, scratching his head.

``Do you know how rare crime is in Metropolis, all things considered?
The idea that he would just happen to be in the neighborhood and spot
you go into the jewelry store is~--- well, not inconceivable, but
significantly unlikely enough that I'm not willing to credit it as
plausible.'' Especially not given the other reports that were coming in.
Lex stood up from the table. ``Thank you for your time. A lawyer will be
in touch.''

Willie Calhoun was waiting outside the room. ``If it isn't my favorite
egghead,'' he said with a smile. Willie was in his late fifties now, and
had grown fat and soft. He was no longer the terrifyingly muscular man
that had trained Lex to fight, cheat, and steal.

``Willie,'' said Lex. They spoke as equals now, which both considered a
mark of respect for the other.

``What are we dealing with here?'' Willie asked. ``Some guy shows up in
tights and starts hassling my boys?''

``This is bigger than you,'' said Lex. ``Bigger than Metropolis even.
Lay low for the time being. Call off any jobs you have planned.''

``I've got mouths to feed,'' said Willie. ``I can't just slam on the
brakes.''

``He's stronger and faster than anything the world has seen before,''
said Lex. ``He can fly. And unless you've been unusually sloppy, he has
some way to learn about crimes as they're happening. Stop everything
until you know more. I've tracked five separate instances today, and you
can be damned sure there will be more.''

``You're with us on this one?'' asked Willie. ``We need your brains.''

``No,'' said Lex. ``Like I said, this is big. Bigger than the city.
Maybe the biggest thing that's ever happened in the history of the human
race. I may call in a few favors trying to get a handle on it, but rest
assured even if I'm not with you, we're working towards the same goal
for the time being.''

\begin{center}\rule{0.5\linewidth}{0.5pt}\end{center}

The interview came out the next week.

``It makes no sense,'' said Lex Luthor, setting down the paper.

``Why not?'' asked Mercy from her desk.

Lex pointed directly at the offending line. ``\,`Superman told me that
he was an alien from the planet Krypton, the last of his kind.'\,'' read
Lex. ``He's an alien, or so he claims, yet he looks just like a human.''
Lex lifted the paper to show to Mercy. On the front of The Daily Planet
was a picture, with Superman standing right next to Lois Lane. The
headline read ``Exclusive Interview With The Man of Steel''.

``Now then,'' said Lex, ``I will admit that a degree of convergence is
implied by Darwin's theory of evolution, but not to such an extent.
These features, a strong jaw and brown hair, blue eyes and his
ridiculous musculature, well, I can accept that the marsupials of
Australia bear more than a passing resemblance to the more traditional
mammals of North America despite being separated by millions of years,
but this beggars belief. And why does he even need legs if he can fly?
What evolutionary reason would there even be for that? And not only does
he look human, but he looks like an attractive human at that!''

``Do you think he's lying?'' asked Mercy. Boredom was apparent in her
voice. She was possessed of a disinterested character, one that Lex
found quite pleasing. He never had to worry about what Mercy thought,
and never had to engage her in unwanted conversation. She was there for
him to bounce his thoughts off of, and she knew it, which was what made
the whole arrangement work. Before she'd come under his employ, Lex had
muttered to himself, which didn't feel nearly so good as speaking aloud
to someone. It helped that Mercy was one of the few people that Lex
could treat as trusted.

``I don't know,'' Lex answered. ``I need more data. Certainly there are
things he says that are inconsistent with reality as I knew it, yet if
you had asked me a year ago whether I would see a man like Superman who
can bend steel with his hands and fly through the air, I would have been
tempted to fire you for expressing such sheer stupidity. Obviously
something that I thought was true about the universe is not.'' He looked
back down at the newspaper.

``And here!'' he practically shouted, pointing at another offending
sentence. ``Here he claims that he can hear a gunshot from across the
city. It's ludicrous, sound doesn't travel that far, and even if his
ears were as sensitive as his muscles are powerful, a gunshot would fade
in with the background noise. He's not only claiming that he can hear
things from two dozen miles away, but that he can further distinguish
those sounds from all the other sounds happening in the city at any
given moment. And yet how else can we explain what's been observed? He
really does dart across the city at just below the speed of sound,
flying through the air at these incredible velocities, going right where
he thinks he's needed. And it's not just that he can hear things he
shouldn't be able to, it says here that he can see through walls and
watch for criminal activities from miles away. It should be literally
incredible~--- not worthy of credit. And yet based on what we can
observe of him, he seems to be telling the truth, at least about his
abilities if not his origin.'' He turned to look at Mercy. ``How do you
feel about him?'' asked Lex.

``Feel, sir?'' asked Mercy, stirring slightly in her seat but by no
means looking uncomfortable. Lex had never seen Mercy look
uncomfortable.

``If he's telling the truth, he can hear everything that we're saying
right now. He can watch us as we speak. When you change your clothes or
take a bath, he can look in on you.'' Mercy was nearly as beautiful as
she was competent, though she kept up a rather severe look most of the
time, with her hair tied back in a neat bun and her skirts with perfect
pleats that fell well below the knee.

``He's doing good,'' said Mercy. She always gave her honest opinion when
he asked it of her, without hesitation, which was another reason she was
so valuable to him. ``I imagine that he's too much of a hero to watch
me.''

``He's a hero,'' said Lex. ``For now.'' He looked down at a pad of
paper, where he'd been making revisions to his estimates. ``I've run the
numbers. Even using the lower bounds for his strength and speed, if he
ever decided that he wasn't a hero anymore, he could demolish this city
in the space of three hours, down to the last man, woman and child. If
we're just talking about the central business district, he could do it
in three minutes. He\==='' Lex stopped. ``He can hear everything that we
say. He can watch us. He can read the files that are sitting in my
drawers. Before anything else, I think it's time to clean house.''


%%%%%%%%%%%%%%%%%%% NEW CHAPTER %%%%%%%%%%%%%%%%%%%%%%%%

\hypertarget{dissemination-of-information}{%
\chapter{Dissemination of
Information}\label{dissemination-of-information}}

Lois Lane was undeniably at the top of her game.

There had been grumbling from some of the other reporters that it had
been dumb luck, but obviously Superman had chosen her for a reason, and
obviously that reason had been her reporting. That's what she kept
telling herself, anyway. Lois worked twice as hard as any other reporter
at \emph{The Daily Planet}, and put out three times as many stories.
She'd started there at the age of sixteen, after she'd sent in an
anonymous letter to the editor that had so impressed Perry White that
he'd put out an advert asking for her to identify herself. From there it
had been a quick climb to the top of the heap, with no real challengers
until Clark Kent had come along.

``Listen to this one,'' Lois said to him. ``\,`Superman is not Christ
Reborn but the Herald of the Apocalypse, a False Prophet that Presages
the End Times'.'' She set down the paper with a laugh and looked at
Clark, who was hammering away at his typewriter. ``Where on earth do
people come up with this stuff?''

Clark paused and looked at her through his thick glasses, apparently not
seeing the humor. ``It's from the book of Revelation,'' he said.
``\,`And he doeth great wonders, so that he maketh fire come down from
heaven on the earth in the sight of men, And deceiveth them that dwell
on the earth by the means of those miracles which he had power to do in
the sight of the beast.' They're thinking that Superman is capable of
these wonders and must be an agent of the Antichrist. Some others are
arguing that he's the reincarnation of Christ.''

Lois laughed, and Clark frowned, just as she knew he would.

``It's not right to make fun of people's legitimately held beliefs,''
said Clark. Lois was a Roman Catholic, in that she attended mass twice a
year on Christmas and Easter. Clark was a Lutheran and a bit more
serious about his faith. She enjoyed needling him about it, more to
annoy him than because she had any strong theological opinions. More
often than not, Clark would bring forth some bit of folksy wisdom from
his father~--- his ``pa''. By all rights Clark should have been chewed
up and spit out by Metropolis two weeks after he'd arrived, but he'd
clung on for a solid five months. Still, Lois didn't think he'd last too
much longer. He didn't have that core of steel a reporter needed in the
big city.

``Do you know how much ad space costs in the paper right now?'' asked
Lois. ``People know that Superman reads \emph{The Daily Planet}, and
that's their only way of communicating with him short of calling out for
him and hoping that he stops by, which we know doesn't work and probably
just pisses him off.''

``Superman doesn't get upset,'' said Clark with a sigh.

``Maybe he does, maybe he doesn't,'' said Lois. ``He doesn't show it,
sure, but that doesn't mean he doesn't feel it. You think that he's just
got a heart of stone when he interrupts a rape in progress?'' Clark
flinched at that. ``I met the man, and talked with him more than anyone
else since he got here, and I don't buy it. He may not be human, but he
still feels. Anyway, my point is that the paper is the only way that
they have any hope of getting across a message, and so ad prices have
skyrocketed since the interview came out. Don't you have any interest in
the kinds of crazy crap that people are putting in our pages? It's all
the more crazy knowing that they're paying top dollar for it.'' She
looked down at the paper. ``This ad only makes sense to people who
already buy into what it's selling, so what's the point of it?''

``I don't know,'' said Clark.

``You okay Smallville?'' asked Lois. That was the name of the town that
Clark was from, no joke. When Lois had found out she'd laughed for five
solid minutes. She'd looked it up on a map later, and hadn't been
surprised to see that it was almost precisely in the middle of nowhere.
``Usually you at least pretend to be enchanted by my wit.''

``I've got a lot on my mind, sorry,'' said Clark. He looked genuinely
apologetic, and turned to engage her in proper conversation. ``Did you
know Superman is being sued?''

``Had to happen eventually,'' said Lois. ``No surprise it's sooner
rather than later. What's the complaint?''

``One of the alleged perpetrators of a jewelry store robbery is claiming
that Superman broke his wrist,'' said Clark.

``Obvious bullshit or legitimate claim?'' asked Lois. ``That'll make the
difference between the front page and the back page.''

``It's obviously untrue,'' said Clark. He didn't swear, which Lois found
unaccountably annoying, like many things about him. ``Superman will
still have to go to court to have his say though.''

``If he wants to get involved in the police business, he'll need to get
used to courtroom appearances,'' said Lois. ``God those are boring. It's
too bad Superman sells. I don't look forward to being asked to cover
those.''

``Me either,'' said Clark. He looked uncharacteristically glum. ``Perry
wants to see you, by the way.''

``You couldn't have mentioned that twenty minutes ago?'' asked Lois. She
walked off to their editor's office without waiting for an answer. Clark
Kent was notoriously unreliable, and if it hadn't been for his uncanny
ability to get stories, Lois was certain that he would have been sacked
a few times over. The newspaper was supposed to be a meritocracy, and
Clark seemed to skate by on something like luck.

``I've got a story for you,'' Perry White said with a grimace. He was a
thick man, with white at his temples and an invariably neat crew cut.
Lois could usually tell what kind of day it had been by how far down
Perry had unbuttoned his shirt. Today was a two‐button day.

``That bad?'' she asked.

``A special request from upstairs,'' said Perry. ``There's a man by the
name of Lex Luthor that very much wants you to write a story about the
orphanage he's building in Suicide Slums.''

``Me specifically?'' she asked.

``In person,'' replied Perry with a nod.

``I could do it over the phone with no problem,'' said Lois. ``So my
guess is that this Luthor character has some ulterior motive?''

``One might be tempted to assume that,'' said Perry. ``But this is an
order from high above, and so I want you to play it straight. You're
going to his house to get an interview about the orphanage. Don't assume
anything more. If something else comes up, play it by ear, but he's got
the attention of the boss and that means he's probably a bad man to piss
off.''

``I'll be on my best behavior,'' said Lois as she rolled her eyes.
``Scout's honor.''

Perry gave her a warning look, but Lois merely smiled. She'd known Perry
for a full decade now, and could read him better than anyone else. He
was just as annoyed with the request as she was. She went off to do some
prep work for the interview. Orphanages were soft news, the kind you
kept in your back pocket to fill in some space on a slow news day. Lois
could only hope that whatever Luthor was really after would make for a
good article.

\begin{center}\rule{0.5\linewidth}{0.5pt}\end{center}

Lex Luthor had become a different man. The key to maintaining a
long‐term deception was to find a lie as close as possible to the truth,
so that it would be more difficult to get caught out. There were
perfectly benevolent reasons for a multimillionaire to seek out every
scrap of information he could get about Superman. He could only hope
that the gathering of information wouldn't attract much notice from
Superman, though he could hardly be the only one trying to get answers.
So far he'd done nothing illegal, simply paid people for their accounts
of meeting with Superman. The most important witness was still ahead.

``Welcome, Miss Lane,'' said Lex with a pleasant, practiced smile. He
led her into the smoking room of his mansion, walking with a light and
graceful step. Lois wore a blue skirt and a white blouse, showing some
of her figure. She was pretty enough, but Lex had other concerns.

``Pleased to meet you, Mr. Luthor,'' said Lois.

``My father was Mr. Luthor,'' said Lex with a smile. '' You can call me
Lex.''

``I'm sure you're a busy man, so I'll try to keep this brief,'' said
Lois. ``I just need a few quotes for the newspaper about the orphanage
you're building in, ah, Southside.''

``Suicide Slums,'' said Lex. ``No need to stand on formality, I grew up
there. Southside is what the city planners called it. It's how the area
is talked about by the politicians. But to those who live there, it's
always called Suicide Slums.''

``You grew up there?'' asked Lois with a raised eyebrow. He watched her
take a quick glance around the smoking room. It was about as far away
from Suicide Slums as you could get.

``I did,'' said Lex. ``If you're thinking that I'm building an orphanage
because I was an orphan myself, I can assure you that wasn't the case.
My mother and father were poor, but they were at least present. The
orphanage is for those children who aren't quite so fortunate. My
adolescence was decidedly unfortunate, and it was only through sheer
luck that I was able to get out.''

``Luck,'' said Lois Lane. ``I did some research Mr. Luthor. You have
three PhDs and run the largest private corporation in Metropolis. There
are half a hundred patents to your name, and you're the discoverer of
something called Luthorian bonding that I couldn't make heads or tails
of.''

``It allows for a more efficient form of industrial lubrication,'' said
Lex.

``What I'm saying is that your success seems to go a bit beyond luck.''
She stuck the end of her pencil in her mouth and bit it. ``Yet prior to
a week ago, you stayed in the shadows. On paper, LexCorp is enormous,
but I'd wager that most people in Metropolis have never heard of it,
even if they work for one of its subsidiaries. So far as I can tell,
\emph{The Daily Planet} hasn't filed a single story about you. And now
here you are, stepping out from behind the curtain to set up an
orphanage in Suicide Slums~--- one of a few grand charitable gestures
you've been making. I have to wonder why.''

``I don't suppose that a person ever really wakes up one day and decides
to be a better person,'' said Lex with a laugh. ``That certainly wasn't
the case for me. No, it was the influence of a man that I believe you're
well familiar with. Superman.''

``You know Superman?'' asked Lois. He could hear the skepticism her
voice.

``No, of course not,'' said Lex. ``I merely said that I was influenced
by him. There's something quite heroic about seeing an alien with such
marvelous powers using them exclusively for the greater good. In fact, I
had a few questions that I'd like to ask you about him, if you don't
mind.''

Lois raised an eyebrow. ``So that's your game,'' she said. She sat back
in her chair and smiled. ``I should let you know that as a matter of
journalistic ethics I don't divulge information about the people that I
interview. For high profile subjects who aren't the subject of
controversy, I let them look over what I've written in case I've gotten
something wrong or let slip something that wasn't supposed to be on the
record. Superman is about as high profile as it gets. I suspect you know
all that, and I'm guessing that's why you asked me here under false
pretenses, thinking you could convince me otherwise. I should also let
you know that as a matter of personal taste, I hate deception.''

``It's only a minor deception,'' said Lex with a friendly smile. ``I
really am building an orphanage in Suicide Slums, and I really do think
that there's a story in it. I have my own burning curiosities about
Superman, and would like more information than can be found in the
paper, but strictly speaking I haven't lied to you. Building an
orphanage to get a chance to talk with you is hardly the worst thing in
the world.''

``All the same, I see no reason to continue this line of conversation,''
said Lois. ``A journalist is only as good as their reputation, even
without the watchful eyes of the big guy.'' She looked towards the
ceiling, where Lex could easily imagine Superman was looking down on
them.

``I'm not asking for you to reveal any implicit or explicit secrets that
Superman might have shared with you. Nothing that was off the record.''
Lex waved his hand. ``All I want are the small details that you might
not have considered interesting enough to print.''

``No,'' said Lois with a sharp smile. ``I can't be bought.'' All the
same, she hadn't moved.

``Miss Lane, it's been my experience that people who say that
underestimate what money can buy,'' said Lex. He watched her carefully,
as though measuring her, but he'd done some research of his own, and
already knew what to offer her. ``I can get the Equal Rights Amendment
passed.''

Lois showed not even the slightest reaction, which in itself was
telling. ``It's been dead for a decade.''

``Introduced every session and bottled up in committee,'' said Lex with
a nod. ``I can get it to the floor, and I can help to ensure it has the
votes. I have the ear of powerful men.''

``You're talking about bribes,'' said Lois. She glanced towards the
ceiling, and Lex couldn't help but smile. Finally, here was another
person who saw what Superman's abilities really meant. No conversation
could be presumed private.

``Not bribes Miss Lane, influence. If I were to mention to the right men
that my companies would be preferentially hiring women, and that I would
make election day a paid holiday throughout my workforce, do you think
they could ignore that? Do you imagine that a man who won his seat with
a margin of half a percent could afford not to change his stance in
response?'' Lex smiled. ``No bribes. No money changing hands. When
you're responsible for the employment of a quarter of a million people,
politicians listen.''

``All that for what amounts to scraps of information from me?'' asked
Lois.

``I won't pretend that our political interests are unaligned,'' said
Lex. ``I've often considered myself something of a feminist. The world
is set to undergo a transition away from manual labor, and everything
I've read indicates that women are just as capable as men in the
intellectual fields, limited only by a lack of education imposed by the
existing social structures.'' That language could have been lifted
straight from one of Lois Lane's inflammatory articles on equal rights.
Lex watched her carefully to make sure that he hadn't said things too
perfectly. She was trying hard not to respond, but a faint trace of
quirk of her lips betrayed her excitement.

``I'm supposed to just take your word for it?'' she asked with excessive
nonchalance.

``As you said, reputation is worth its weight in gold,'' said Lex. ``If
you've done your homework, you should know that you can trust me. I
pride myself in my fair dealings.'' Lex had managed to avoid any messy
lawsuits that would be a matter of public record, and many of the more
unsavory aspects of his life had been scrubbed away in the past week.
There were perhaps a dozen people who could connect him to any ongoing
criminal acts, and he had a long story of redemption if any of his
adolescent crimes surfaced. He had no criminal record to speak of. He
also knew that Lois Lane couldn't possibly have done enough
investigation to unearth anything in the twelve hours since he'd called
in the favor, but she would be swayed by the mere appearance of
openness. Lex was a comfortable liar.

``And what about Superman?'' asked Lois. ``You know I can't risk losing
the next interview, if there is one.''

``Do you think this arrangement would upset him?'' asked Lex. ``He acts
very much like a man with nothing to hide, and I somewhat doubt that he
exposed you to anything that he didn't want known to the world, even if
he does have secrets. There's no personal gain for you, it's purely an
altruistic act, and if Superman has a problem with equal rights for
women I doubt he would have chosen you in the first place. You win, I
win, and Superman at the least loses nothing.''

Lois sat and thought it over. Lex was in no rush. ``Alright. I have one
condition,'' she said slowly.

``Go on,'' Lex replied.

``Tell me why,'' said Lois. ``Superman seems to be the only story in
town these days, but everyone's got some angle on it, some reason that
they're curious. Some people think he's got something to do with
religion, that he's Christ or Antichrist, some people are envious of his
power, and more than a few women are rather keen on him.''

``Including yourself?'' asked Lex.

``My interests are more professional,'' said Lois. ``But go on, tell me
what you're in it for.''

``You've heard of humanism?'' he asked. Lois nodded. ``I read the
manifesto, and I'm sure that if the Humanist Society of Metropolis had
known of my inclinations that they would have asked me to sign. I would
have declined, for a number of reasons, not least of which is their
rejection of profit‐seeking, which is perhaps the most efficient method
of incentivizing useful work yet known to man. In many ways I'm in
agreement with them though. The Industrial Revolution and its
consequences have been a boon for the human race. I can think of no
better path forward for humanity than a pursuit of further revolutions
through applied reason. When I look to Superman, I can only imagine the
eons of dead ends our scientists could skip, and the advancements that
could be had if he could be convinced to give us his knowledge. It would
be an end to disease, and an end to death.'' Lex poured himself a glass
of whiskey. ``I'm doing my best to investigate Superman, because I want
to persuade him to do the most good.''

``Alright,'' said Lois, seemingly satisfied with his answer. She started
talking.

\begin{center}\rule{0.5\linewidth}{0.5pt}\end{center}

It began with a note on her desk. She'd come back from the break room
holding half a sandwich in one hand, a bottle of soda water in the
other, and a cigarette between her lips. Sitting on top of her
typewriter was a small envelope which simply said ``Miss Lane''.

Here Luthor stopped her, and asked her about the specifics, and Lois
tried her best to remember. The envelope was delicate and white, the
kind you could get from any corner drugstore. The words on the envelope
and in the note itself were typewritten. Lois had saved it, and it was
somewhere in her desk drawer. When Luthor had said that she should leave
nothing out, she hadn't realized how literal he was being about it.
Being a reporter was about being concise. You had to pay attention to
the details, but only say those things that were actually important.
Luthor just wanted a raw stream of consciousness. Luthor then asked
whether she would part with the note, and she'd said that she would let
him take a look at it if she could find it.

The note had said to meet him on the roof of The Daily Planet Building,
and it was signed ``Superman'' in swirling cursive. She'd asked around,
and no one had seen who had left it, so Lois had taken the elevator to
the top floor, and then taken another flight of stairs up beyond where
any offices were to reach the roof. She'd thought it was going to be a
prank of some sort, but when she'd opened the door she'd seen Superman
standing on the very edge of the roof, looking out over the city. His
cape flowed behind him in the wind.

``Hello Lois,'' he said as he turned around.

He had a body like a strongman. The material of his suit clung to his
skin, exposing every muscle to the world, each of them perfectly
defined. He was undeniably handsome, with a curl of hair hanging down
that made him look almost roguish.

Luthor stopped her again, and asked questions about the costume, and
about Superman's hair. The suit was red and blue, but had no visible
seams, not even any that were hidden. Lois wouldn't have expected any
either, not with the way that the material clung to his body. It was
obviously made of some fabric unknown to the scientists of Earth. The
costume could be damaged, but Lois had scattered reports about it
repairing itself over time, knitting back together. Luthor had made
special note of that. Superman's hair was styled with some kind of gel,
though from what Lois could remember it always looked like that, even
when he was flying. She was about to bring up the possibility that he
used a gel from the planet Krypton when Luthor waved his hand for her to
continue on.

``I'm afraid my etiquette classes didn't prepare me for this,'' Lois had
said. ``What do I call you? Superman was just the name I made up, I
hope\===''

``Superman is fine. It's fitting,'' said Superman with a smile.

``Perry was worried people would connect it to German, to Nietzsche's
Ubermensch,'' said Lois, ``And I told him to remember who our audience
was.''

``It's fine,'' said Superman again. He had a certain gentleness to him,
a patient understanding that was so palpable that Lois could instantly
understand how people compared him to Christ. ``I thought that perhaps
some people had questions that they'd like answered.''

``Sure,'' said Lois. She cursed herself for not taking the note
seriously and fumbled for her pencil and paper. ``Alright, let's start
with where you came from.''

``I'm an alien,'' said Superman. ``From the planet Krypton.''

Lois wrote this down as though it weren't utterly insane. She could
decide whether it should be spelled Krypton or Crypton and decided to go
with a K because it looked more foreign. ``And does everyone from your
planet have your abilities?'' she asked.

``No,'' said Superman. His turned somber. ``Just me. Krypton was a dying
planet, and my parents were able to fashion a spaceship that could hold
only one. They sent me here just as the planet imploded. It wasn't until
I got to this planet that my abilities began to manifest. I'm the last
of my kind.''

Lois didn't know how to respond to that. ``Why did you choose Earth?''
she eventually managed.

``I didn't,'' said Superman. ``The coordinates were locked before I knew
what was happening. If I could speak with my father, I would ask him
that same question. I suspect he chose this planet because he thought I
would be able to blend in.''

``There are other aliens then?'' asked Lois. ``Other planets with
intelligent life?''

``Yes,'' said Superman. ``But it's not my place to spoil the secrets
that await humanity when they reach for the stars.''

Lois had frowned at that, but continued on all the same. ``You're out
saving people and stopping crimes every day, and many of us are
wondering why.''

``I think it says something profoundly sad about your species that you
have to ask that question,'' said Superman. ``Helping people is its own
reward. If you were given the same power, wouldn't you do the same?
Wouldn't you put out fires and stop muggings?''

``But you do it for free,'' said Lois. ``You don't ask for anything in
return, and half the time don't even stick around long enough for people
to thank you. Most of us might stop crimes, but we might ask for a
little money from it, or at least get official police sanction. They're
calling you a vigilante.''

``I've looked over your law books,'' said Superman. ``I'm acting within
my rights as a resident alien. It's important for me to have my
independence from human society. I don't want to disturb things too
much.''

She'd let that go, and afterwards had hated herself for it. Couldn't he
see that he was throwing a wrench into human society just by being
there? Everyone from Washington to the Vatican was clamoring for a
sit‐down conversation with him, and he was already a celebrity whether
he liked it or not. He'd be in the movies, on television, and spread all
through the culture of the world. Just the news of extraterrestrial life
would have caused an immense, irreversible change in how humans saw the
world. Superman was so much more than that.

But she hadn't properly prepared, because she'd thought that it was a
joke, and so she was too out of sorts to press him on it.

``We know you can fly, and there are reports that you can stop bullets
with your chest, but what else can you do?'' she'd asked.

``Would you like a demonstration?'' he'd asked with a grin.

She'd nodded, and he'd swept towards her in a rush. Before she knew it,
he'd hooked an arm under her legs and swept her off her feet. Seconds
later they were flying to the air, and she had her arms around his neck.

``This wasn't in the paper,'' said Luthor.

``It wasn't important to the story,'' said Lois with a dismissive wave
of her hand.

In truth, her heart still raced when she thought about it, and not in
the good way. Superman had been presumptuous in touching her, and
reminded her of one too many boyfriends who had tried to take their own
liberties. Superman had scooped her up like he had known her, like it
was some grand flirtatious lark that they were both enjoying. When they
were past the roof of The Daily Planet Building she had looked down for
only a moment before burying her head in his shoulder and closing her
eyes tight. Lois wasn't afraid of heights, and had even flown as a
passenger with Amelia Earhart once, but this was different. Her life was
entirely in Superman's hands. If he'd stopped and turned her head
towards his own, had tried to kiss her, what choice would she have but
to kiss him back?

She hadn't put it in the article, both because it was a sour note and
because she didn't need people implying that she and Superman were an
item. She'd already caught Clark using the phrase ``Superman's
girlfriend Lois Lane'' in a different article and she'd pitched a fit to
Perry until he'd taken it out. She didn't want to live in someone else's
shadow. None of her inner thoughts could be revealed to Luthor of
course, especially since Superman might be listening in. Lois wanted
that second interview, even if she didn't have any particular fondness
for Superman.

She skipped ahead to when they'd landed on a beach north of Metropolis.
The flight had seemed to take an eternity, but Lois had kept her eyes
closed the whole time so it was tough to say. Superman had begun a
demonstration of his powers, and Luthor quizzed her on each of these,
though she'd already included all of that in the article. Superman could
crush a rock to dust with his bare hands. He was faster than a speeding
bullet and more powerful than a locomotive. He could see straight
through walls and read newsprint from miles away. He could hear the
faintest whisper while the ocean roared around them. Luthor asked for
details about all of these.

``He called it x‐ray vision?'' asked Luthor. ``Or was that an invention
of your own to describe the phenomenon that you observed?''

``That's what he said,'' Lois replied. ``I know it's probably not how it
works.''

``No,'' said Luthor with a frown. ``It's not.''

After the demonstration, Superman had asked her if she wanted to return
or if she had further questions. She had almost said that she would get
a taxi, but she wasn't sure where she was and didn't want to offend him.
She had already been thinking about the next interview, even then. He'd
scooped her up and flown her back to the top of the building, and she'd
tucked her face into the crook of his shoulder to protect herself from
the wind and so that she wouldn't have to be sick from the view below.
She felt him lean his head towards her, pressing his cheek against her
hair, but he hadn't tried to kiss her.

She'd thought about that often afterwards. Superman was untouchable. If
he'd wanted to act against her, there was nothing that would stop him,
and no retribution that could be enacted against him. Lois had been
trained by her father in hand‐to‐hand combat and carried a pistol nearly
everywhere she went, but both would be useless against the Man of Steel.
It was frightening simply on the face of it, to know that you were
completely at the mercy of another person. It was worse knowing that he
could watch everything that you did and hear everything that you said.
The whole thing was hopelessly complicated of course. Superman was
attractive, there was no questioning that, and he was the most perfectly
good and selfless man in the whole damned city, but there was an extreme
imbalance of power between the two of them and questions of what it
would mean for her career. She didn't even know whether she liked him,
though she suspected that she didn't.

Luthor was staring at her, and she realized she hadn't said anything for
a while.

``Wait right here,'' she'd said when they got back. Her legs were shaky
on the roof, but if Superman noticed he didn't seem to take it as
anything more than the effects of the flight. ``I need a picture or no
one will ever believe me.'' She'd rushed downstairs and grabbed Jimmy
Olsen, the first photographer she'd seen. She'd half expected Superman
to be gone when she came back up to the roof, but he was still standing
there, looking out over the city. It had taken Jimmy three tries to get
the photograph that ran on the front page and in all the extra editions,
since he was nearly as out of sorts as she was. And after that, Superman
had shook their hands and flown away. Lois had typed the story up right
away, not wanting to risk someone else getting the jump on her since she
stupidly hadn't confirmed that the interview was an exclusive. She'd
left the article on her desk for three hours along with a note to
Superman asking his blessing on the article even though that wasn't
strictly necessary and she had no idea whether he would even spare her a
glance. He hadn't stopped by to make any comments, and she hadn't spoken
to him since.

``And that was it?'' asked Luthor. His eyes were cold and piercing, the
earlier warmth forgotten. He had listened intently the entire time,
drilling down into the details of the interview, the minutiae that
surely didn't have any bearing on anything. He seemed to remember
himself, and the veil of friendly concern lifted back into place.

``That was it,'' said Lois. ``You'll keep your end of the bargain?''

``Of course,'' said Luthor. ``I'm a man of my word. And if you do get
that second interview, I'd be very interested in talking to you about
it. In fact, I have a few questions I'd like to suggest, if you don't
mind\ldots{}''

\begin{center}\rule{0.5\linewidth}{0.5pt}\end{center}

Lex wrote down his findings in a notebook. The language he used was of
his own devising, one that he'd invented a decade ago specifically so
that he could write down what he was thinking without the risk of anyone
reading it. The book that defined the grammar and vocabulary had been
burned in a fireplace shortly after he'd felt confident that he knew it
all. Lex was reasonably certain that Superman didn't have a universal
translator of some sort~--- he'd heard a few reports from the immigrant
neighborhoods of Superman having difficulty communicating. It could be a
feigned weakness, but Lex thought the odds were that it wasn't.

Superman had access to a typewriter. He either had money to buy a card
or he stole one. If Lex could get his hands on the card itself, he might
be able to divine something about the typewriter that had been used to
make the letters, some pattern of offset keys that would give some clue
to its origins. Most likely it was a dead end, and the typewriter would
prove to be Miss Lane's own, but it was something to look into.
Furthermore, it might be possible to lift fingerprints from the note
itself. If not, Lex would look into getting some from the crime scenes.
He wasn't sure what use that would be, but it never hurt to get more
data.

There were a number of points of curiosity in the story as it had been
relayed to him. The first was the distance. Superman had said that
Krypton was millions of miles away, but it had to be trillions at the
least. A hundred million miles would get you to the Sun but not much
further. It was either a revealing mistake or a simplification that the
alien had used for a non‐technical audience. Second, he had called his
vision x‐ray vision, which was plainly false. X‐ray photography worked
by placing an object between the source of the x‐rays and the x‐ray
film. If Superman could actually see on the x‐ray spectrum, everything
would be too dim, and wouldn't have much of an advantage over visible
light. If his eyes were emitting x‐rays, they'd have to output an
enormous amount that would have to be reflected back mostly by chance,
and if that were the case he'd likely be killing people simply by
looking at them. It was possible that the term ``x‐ray'' was another
colloquialism, but to Lex it suggested that Superman didn't know how his
powers worked.

There was no easy way to know how much if any of Superman's story was
true. A civilization capable of interstellar travel being destroyed so
utterly that there was only a lone, ignorant survivor, who somehow made
it across unimaginably vast distances to land on a planet filled with
people who looked exactly identical to him? Some part of it had to be a
lie. Lex could think of a hundred ways in which the story would start to
make sense. If Superman's race had the technology to travel between
stars, then perhaps they had the ability to alter their form at will,
and the perfectly chiseled features of Superman were merely a mask laid
over something tentacled and many‐limbed. Superman could be an exile or
a narcissist who had chosen to leave of his own volition or been forced
out by his peers or elders. Lex could think of a thousand variations on
the story that would make it more plausible, but it was an exercise in
futility. None of it could be trusted in the first place. Lex was
reasonably certain that Superman was an extraterrestrial, because for
him to be a product of human ingenuity would require a vast network of
scientists and engineers operating in secret and working toward some
unfathomable goal. It was even less plausible than the entire Krypton
race being at once capable of sending a ship to Earth and being utterly
wiped out in a single planetary event.

In the face of such uncertainty, lesser men might have simply given up.
Lex Luthor believed that very few problems were unsolvable if you put
your mind to it. Examinations of the evidence that Superman left behind
could only go so far though. It was time to escalate.

\begin{center}\rule{0.5\linewidth}{0.5pt}\end{center}

\emph{Author's Note: This is a ``bonus'' chapter~--- I'm still planning
to update on Sunday, I just didn't want to have too long a chapter and
this stuff was relatively easy to finish up and post now. As always, I
appreciate corrections / feedback / reviews.}

\emph{Lois Lane getting hired as a teenager because of an anonymous
letter to the editor is a detail pulled from the life of Nellie Bly, a
female reporter of roughly the same era (who Golden Age Lois was based
on).}


%%%%%%%%%%%%%%%%%%% NEW CHAPTER %%%%%%%%%%%%%%%%%%%%%%%%

\hypertarget{the-allseeing-eye}{%
\chapter{The All‐Seeing Eye}\label{the-allseeing-eye}}

Lex Luthor wasn't the only one gathering information. As the days
passed, people began to make their observations, and a few things began
to become known.

Superman would show up at misdemeanors in downtown Metropolis, felonies
in the greater metropolitan area, and large disasters in the continental
United States. Those who had done the math would point out that Superman
could reach any point on the planet within an hour, but he only rarely
seemed to use this ability; he went to a mine collapse in Peru, a
landslide in Bangladesh, and an earthquake in China, but he seemed
inconsistent in his ranging.

He prioritized crimes against people above crimes against property.
Murder and forcible rape were almost sure to bring a response, while
burglaries often went unstopped. He avoided controversy and grey areas,
and tended to stay away from incidents where both parties were at fault.
He tended to avoid crimes committed by people in the immigrant
neighborhoods, and there was some question about whether this was the
result of a language barrier or because Superman harbored some ideas
about class or racial purity. There were some members of the Eugenics
Society of Metropolis that pointed out that Superman was white.

Superman didn't participate in any foreign wars, despite repeated
requests. There was a civil war in China, and a war between Bolivia and
Paraguay in South America. Thousands died, and Superman did nothing,
presumably because of his claimed neutrality. It was unknown whether
Superman would side with the United States if they once again went to
war. In Germany, the National Socialists had risen to power and
repudiated the Treaty of Versailles, which was generally agreed to be a
worrying development. When the Nazis killed eighty‐three people in a
political purge, there was much discussion about whether Superman's
absence from Germany had been a calculated effort to avoid becoming
embroiled in global politics, a tacit endorsement of their politics, or
whether he simply hadn't known about it until it was too late.

Certainly Superman wasn't active all the time, and he'd proved to be far
from omniscient. Even with him going on patrol and being visible high
above the city, murders still happened with some frequency. The United
States was slowly creeping its way out of the Great Depression, with
Metropolis as the vanguard. Where there had been three murders per day
before his arrival, there was now an average of one. Some people
grumbled that he should do more.

Superman was in the news on a regular basis. He pulled Pretty Boy Floyd
out of a rathole hotel in Gotham City, and requested that the reward be
donated to charity. When the SS Morro Castle caught fire and burned on
the way up from Havana, Superman swooped in and saved the lives of
hundreds. He stopped a tornado in Kansas, and a hurricane moving towards
Florida. He was undeniably a hero.

Through it all, the lawsuits began to pile up. A good number of
criminals came forward with complaints of brutality, and some had the
injuries to prove that they'd at least taken a hit to make their story
plausible. There were accusations of rape that no one believed. Not
every legal issue was so spurious. Superman was sued for theft after
taking steel girders off the back of a truck to shore up a collapsing
factory. He was subpoenaed as a witness to all manner of man‐made
disasters. The case of Shoe v. New York was working its way towards the
Supreme Court. At issue was whether Superman's x‐ray vision could be
used to obtain a warrant for arrest or whether that unreasonably
infringed upon the right to liberty guaranteed by the Fourteenth
Amendment. Most of the court watchers predicted that a half dozen cases
would end up going to the Supreme Court in the coming year. It was a
wonderful time for those with an interest in jurisprudence.

Lex Luthor existed in the background. In public, he was a champion for
Superman, arguing in favor of the stances he believed Superman to favor
and heading the first Conference on Extraterrestrial Science which of
course had Superman as its sole focus. In private, he was the world's
most cautious puppet‐master.

\begin{center}\rule{0.5\linewidth}{0.5pt}\end{center}

``You sure we should be doing this?'' asked Ted. ``It's not exactly
acting.''

``It's acting,'' said Claire defensively. ``We're pretending at being
different people for an audience.''

That Ted had landed a bit part in a doomed production of The Stationary
Man wouldn't have been worthy of note if not for the fact that this made
him more successful than Claire. It was a constant source of tension
between them, and the subtext of nearly all of their conversations.

``Easy for you to say,'' Ted replied. ``You're not the one who's going
to go to jail.''

``Oh hush,'' said Claire. ``The pay is good enough.''

``We probably shouldn't be talking about this where he can hear,'' said
Ted. He fidgeted with the gun tucked into the waistband of his pants. It
wasn't loaded, and he was thankful for that. Guns made him nervous.

``There's nowhere Superman can't hear, the papers said so,'' said
Claire. ``Now come on, I'm ready to go.''

``You'll drop the charges?'' he asked.

``Who on earth do you think I am?'' asked Claire. ``Of course I'll drop
the charges. This whole thing is going to last a single night, tops.
Maybe he won't even show up and we can get paid to do this again.''

``Fine,'' said Ted. He pulled the ski mask down over his head and
whipped out the gun. ``Gimme your goddamned money and you don't get
hurt.''

Claire glanced nervously from side to side. ``Please, I need that money
to feed my baby sister.''

``Hand over the dough,'' said Ted. ``Just hand over the goddamned dough
or I swear to God I will shoot you right in your pretty little mouth and
steal the money off your warm corpse.''

``Superman!'' screamed Claire at the top of her lungs. ``Superman, save
me!''

``Shut your mouth, bitch,'' said Ted with what he hoped was a convincing
sneer. But then he saw Claire's face when he said the b‐word, and
instantly regretted it. He was about to break character and tell her he
was sorry when Superman appeared between them. Neither had seen him
arrive. He was simply there with a rush of air.

``What seems to be the problem?'' asked Superman with half a grin on his
face. He plucked the gun from Ted's hand.

``This bastard was trying to mug me,'' said Claire.

``I wasn't,'' said Ted. He didn't have to feign the fear in his voice.
He'd never realized how tall Superman was before. Odd that it would have
such an effect, when that was the least impressive thing about him.

``Ted and I will be going to the police station,'' said Superman. Ted
felt his stomach tie into a nervous knot at Superman saying his name
before realizing that Superman had probably just read it off of one of
the cards in his wallet. ``If you'd make a statement it would help to
put this man behind bars.''

As Claire looked at him, Ted felt another jolt of honest fear run
through him. She looked like she was going to agree to it. But at the
last second, her face softened, and she shook her head.

``I need to get home to my baby sister,'' said Claire. ``I'll file
something with the police in the morning.''

``Very well,'' said Superman. ``Have a good day.'' Then he flew up into
the air, carrying Ted with him.

A homeless man watched from a distance, and wrote something in his
notebook in an extremely neat script. The next day, a curious personal
ad appeared in The Daily Planet. Lex Luthor made a point of reading
through both of Metropolis's daily newspapers each morning, and so even
if Superman had been watching, there would be nothing suspicious about
the way that Luthor's eyes flickered over the page. There was no copy of
the key to be found anywhere on Lex's person~--- it had been committed
entirely to memory. The actors had been hired by an intermediary who had
no knowledge of Lex Luthor, and the man who'd watched them received
payment from a slush fund that Luthor had cut his connection to years
ago.

\begin{center}\rule{0.5\linewidth}{0.5pt}\end{center}

Leroy Barnes pulled his mask down over his face and hefted his tommy
gun, then charged straight in through the revolving doors of the
Commerce Bank of Metropolis. He used the butt of his gun to smack the
security guard hard in the nose as Sean ``Moustache'' Murphy and Big
Paul Castellano followed closely behind him. Leroy fired off five rounds
into the ceiling, bringing plaster down on the customers. They scrambled
to the floor without having to be told, men in fine suits and women in
glitzy dresses pressing themselves up against the immaculate marble of
the Big Apricot's most prestigious bank.

``God dammit Leroy,'' said Murphy, ``We were supposed to do this
clean.'' Murphy picked up the guard's gun and stuffed it into the burlap
sack they'd be using to carry the money.

``This is a robbery!'' yelled Big Paul, a short man who had once worked
as a jockey down at the Apricot City Racetrack before he'd broken his
leg. He limped, but it didn't slow him down much. ``Get down on the
floor! We don't wanna bump off nobody, so no funny business and we'll be
through this caper in a flash!''

The three of them walked towards the cash registers, guns held out in
front of them, trying their best to cover the whole room. The idea was
to get in and out before the cops had a chance to show up. There was the
question of the Big Blue rearing his ugly head, but that was what
contingencies were for.

``Empty the cash register sweetheart,'' Big Paul said to one of the
cashiers. He was careful not to point his gun straight at her, just in
her general direction. He'd found that people panicked with a gun to
their head. It was better to hold the gun like you didn't want to use
it, instead of like you were seconds away from killing them. ``Throw it
all in this sack and we won't have any trouble.''

``There's no need for that,'' said a voice from the front of the room.
Everyone turned to look at Superman. He'd entered the bank silently, and
stood with his cape hanging down behind him. The revolving door spun
around behind him. Superman looked the same as he ever did, a god
striding among men.

``Stop right there,'' said Murphy. ``We planned for this, ya see?
There's hostages, planted all around the city, and you can stop us or
save them, but not both.''

``I can do both,'' said Superman. ``And I don't negotiate.''

Superman glanced rapidly between the three robbers and closed the
distance to Murphy in the space of a heartbeat. He bent the barrel of
the tommy gun with one hand, and reached into Murphy's jacket with the
other. Murphy dropped the gun and tried to beat against Superman, but it
was like slamming his fists into granite. Superman pulled out a thin
metal case from Murphy's pocket and stared at it with a frown. It was
locked shut, but Superman pried it open with ease and pulled out a slip
of paper. He let the note flutter to the ground after reading it, then
moved forward and tied up both of Murphy's hands with the sleeves of the
man's own jacket.

Leroy and Big Paul had started running away as soon as Superman had
grabbed Murphy, the promise of money forgotten. Big Paul, with his limp,
was falling behind. Superman came at them from behind as they ran,
ripping the guns from their hands and setting both men on the ground.

``You gonna kill us?'' Leroy spat at him. ``Or are you some kind of
pussy?'' Superman turned his implacable gaze towards the criminal, and
Leroy lost his bravado at once, like a balloon being popped.

``No,'' said Superman. He seemed about to say more, but tied them up and
dashed through the revolving door of the bank, leaving it spinning
behind him. On the floor of the bank, huddled among the other customers,
Lex Luthor smiled.

\begin{center}\rule{0.5\linewidth}{0.5pt}\end{center}

Watching the robbery had been a risk, but Lex Luthor had wanted to see
the Man of Steel at least once in person, just in case it would stir
something loose within his mind. Lex stopped by the Commerce Bank three
times a week at the same time of day, and so there was little unusual
about him being there when the robbers arrived. There was nothing that
Superman could use to trace the robbery back to Lex, unless Superman had
been watching as Lex planned it. Even then it was unlikely given the
precautions that Lex had taken.

In his home, Lex Luthor had built a keyboard which connected to the
phone lines. Many nights he could be seen pressing the keys while
staring at his coded notebook, with no apparent output. When he hammered
down the keys to, they didn't produce the normal solid clack of metal
levers pressing up against a ribbon of ink. Though it looked much like a
typewriter, the keys were attached to an electrical mechanism which
translated each press of a key into a tone, which was in turn sent down
the phone lines.

Someone watching Lex Luthor's hands from above might try to observe what
he was typing, but that would be a useless exercise since Lex Luthor was
typing in a crude code on keys that were completely unmarked. Someone
with absurdly superior hearing might find the terminus to the phone
connection at an office building in downtown Metropolis, where the tones
were magnetically recorded on a steel wire and later translated into a
still‐encrypted paper copy by a somewhat bewildered secretary. The paper
copies were filed away, and from time to time Lex Luthor could be seen
stopping by to leaf through them, seemingly able to decode them without
need for a cipher.

The line was split of course, and the terminus in LexCorp offices was a
decoy. The coded messages that filled the cabinets were nonsense, the
letters randomized past the point of recovery, not that Superman had
shown himself to be much of a code‐breaker. The real coded message was
received by a small office out in Star City, California, where it was
decoded into a set of instructions, with a header in English and the
rest in some other language. The people who worked at the office knew
little about who they worked for or what purpose their work served. The
English portion of the message was for them, and told them who to send
mail to, or occasionally who to call, while the second part was for
their recipient, and invariably in a language that the people at this
small office didn't speak~--- an additional protection against Superman,
though it was really more of a minor inconvenience than good security.
The people at the office assumed that their secret master was the United
States government.

This circuitous route was a bit paranoid, even given Superman's
demonstrated surveillance capabilities. Superman had repeatedly been
shown to need to focus on stimulus, and it was Lex's working theory that
Superman's brain filtered out the vast majority of the input that it
received from his ears and eyes. Superman could prime himself to listen
for a gunshot, or the sounds of shouting, but he didn't have total
information processing. For this very reason, most murders in Metropolis
were now surprise attacks using melee weapons that would eliminate the
victim's ability to produce sound. A gunshot was distinctive, while the
sound of a knife slicing flesh was not. In a way, Superman's arrival had
made the underworld a more brutal place.

It was likely that Lex could have skated by on lesser security
precautions than he took, but he'd woken up to nightmares of having his
skull crushed between Superman's hands too many times. In the dream he
was just one in a long line of people that stretched out on either side
of him, an endless number of people waiting to be killed by Superman.
The alien did the work calmly and cleanly, and Lex was the only one who
was trying to fight back. Precautions were the order of the day.

``Mustache'' Murphy hadn't known why he'd been asked to rob the bank.
The jeweler on 4th St and 16th Ave hadn't known why he'd been asked to
make a small case lined with lead. Leroy Barnes hadn't known why he'd
been asked to fire off his gun towards the ceiling. All these men knew
was that they were being paid. Strings had been pulled and messages had
been sent.

The end result had been that Lex Luthor discovered that Superman
couldn't see through lead. More than anything, he was upset that
something so stupid had worked.

It was what you would try if you knew a little bit about x‐rays. Lead
was used to block x‐ray radiation, even people who didn't have a clue
what x‐rays were knew that, so it made sense that Superman's vision
could be blocked by it. Yet Superman's x‐ray vision fairly conclusively
did not use x‐rays. That was obvious just from thinking about it, and of
course Lex had tested it by having patsies carry around sealed strips of
x‐ray film and subject themselves to his gaze. Furthermore, Superman was
able to distinguish colors using his x‐ray vision, and in all respects
treated it simply as ``the ability to see through objects'' instead of
something that made any sense. Yet lead blocked it all the same. It was
a victory to learn that, but utterly infuriating. Lead was used to block
x‐rays because it was dense, and yet it was apparent that any amount of
lead stopped Superman's vision, even a few centimeters. If lead blocked
Superman's vision in the same way that it blocked x‐rays, Superman
shouldn't have been able to see through a solid foot of steel or three
feet of concrete either. Thinking of new physical laws which would
explain this behavior made Lex Luthor frustrated, though this wasn't
terribly unusual where Superman's powers were concerned.

Lex had to wonder whether Superman realized he'd given something away by
reaching to grab the case instead of getting the information through
some other means. Lex had other plans in place~--- when Superman
eventually followed the trail of clues that started with the piece of
paper in the case, he would be confronted with a number of challenges to
his x‐ray vision, and he would be forced to give up a bit of information
at each one. The clues would lead to three locations; a diving bell
beneath a hundred feet of water, a large Faraday cage, and a steel vault
in a closed down bank. There were no hostages to speak of, and either
Superman would use his so‐called x‐ray vision to confirm this or be seen
by spotters engaging in a rescue for someone that wasn't there. Either
would give information.

As the day passed, the reports came back from the spotters. Superman
wasn't seen at any of the locations he should have been led to. If lead
was the only thing that would stop Superman's vision, then it would have
to be lead that Lex would use.

\begin{center}\rule{0.5\linewidth}{0.5pt}\end{center}

Simply lining a room with lead would be of limited use, since it would
give Superman the incentive to pry in precisely the places that his
attention was least wanted. It would be like erecting a sign that said
``Don't look here''. The only way around that was to make lead shielding
so common that Superman wouldn't be able to keep track of them all, and
for that Lex developed a plan.

A scientific paper was mailed out to a number of universities and
businessmen with the cryptic title ``Non‐Röntgenian Vision; An
Exploration from Inference''. The paper used complicated words where
simple ones would do, and meandered over twenty pages when its findings
could properly be summed up in two. There were numerous digressions and
spelling errors, and the author identified himself as a former professor
of physics living in a cabin in the Adirondacks who had been exiled from
Harvard some decades earlier due to indiscretions which the author
implied were fabricated by his jealous colleagues. It was for the most
part scientifically sound, but so mired in authorial problems that it
had no hopes of being properly published in any journal of note. You
would have to read it three times before understanding that it was
talking about Superman.

The disgraced professor had died some years earlier in a Prohibition
speakeasy that had been owned by Lex Luthor. The professor's body had
been dumped in the river and never identified, and his death was known
by very few. The paper's true author was Lex Luthor, who had crafted it
carefully using information made available to members of the public
through the police and the newspapers. The original incidents which
demonstrated Superman's inability to see through lead had been
engineered by Lex himself, both the first one at the bank and a host of
others used to confirm the finding. Taken on its own, the conclusions
were tenuous, but it was enough to get the ball rolling.

The paper was mailed to the office of Thomas Nivas, a Dutch businessman
with no obvious connection to Lex Luthor, and he made a show of reading
it carefully. Where others would dismiss the professor as a crank, Nivas
would take a gamble and begin immediately buying up shares in the
handful of companies that mined or traded in lead. Within two weeks,
Nivas would announce to the world that lead conclusively stopped
Superman's vision, and publicly challenged the Man of Steel to
demonstrate otherwise. Superman never showed up, and though that proved
little, Nivas began to see a trickle of customers. One of the first of
these was Lex Luthor.

It was a happy bit of serendipity when Lois Lane scheduled a second
interview.

\begin{center}\rule{0.5\linewidth}{0.5pt}\end{center}

``Well, of course I trust Superman,'' said Lex. Lois Lane sat across
from him in one of his leather chairs. Across the hallway, the sounds of
construction could be heard, as his study was ripped apart in
anticipation of lead lining on all the walls, the floor, and the
ceiling. When the sheets of lead were in place, the fine woodwork would
be replaced and the room would look exactly like it was before.

Lois Lane had apparently asked Nivas for the name of one of his clients,
and Nivas had mentioned Lex Luthor. It was a minor betrayal of
confidence, but Lex guessed that Nivas had given up Lex's name because
of the conversation they'd had wherein Lex had put forth what he
believed was the most cogent possible argument in favor of a perfectly
innocent man obtaining protection from the eyes of a watchful and
seemingly benevolent god. Nivas didn't know that Lex was the one behind
the funding, nor the author of the paper he'd been mailed.

``I trust Superman,'' said Lex, ``But do you believe that Superman is
perfectly good?''

``Perfectly?'' asked Lois. ``That's a high standard. But he's as damned
close as we're going to get. He's been here four months now, and he's
saved hundreds if not thousands of lives. He doesn't act as a law unto
himself, he just flies through the air and helps people like it was the
most natural thing in the world. He hasn't killed anyone, and despite
what people might allege, I don't believe that he's ever seriously
injured anyone either.''

``All true,'' said Lex with a smile. ``But given that he isn't perfect,
do you think that it's unreasonable to take precautions against the
possibility that he one day acts in some unconscionable way?''

``Is it really worth however many hundreds or thousands of dollars this
renovation is costing you?'' she asked.

``There are a number of factors that go into determining that,'' said
Lex. ``I have enough money that the expense is somewhat trivial to me,
and I have enough intellectual property that having it stolen would be
quite damaging to me~--- patents, ideas, formulas, processes, and half a
hundred other things. Beyond that, there is a value to me in not being
watched, even when I'm not doing anything of note. It brings me peace of
mind, which is worth something even when the actual risk is low. I
suspect much of the sales of this shielding will go to husbands who want
to know that their wives aren't being spied on in the bath.''

``\,`Humans have an intrinsic right to privacy','' said Lois. ``Navis
told me that, and I suspect that he heard it from you.''

``I believe I said something like that, yes,'' said Lex. ``It's one of
the great flaws of our Constitution that a right to privacy is not among
those enumerated. It's funny, isn't it? No one would begrudge you from
having frosted windows in the bathroom or drawing your curtains when
company is over, but as soon as Superman enters the picture many people
think that such measures are somehow indicative of criminality, or
morally wrong in and of themselves.''

``I didn't bring up crime,'' said Lois.

``But you will, in the article?'' asked Lex.

``Of course,'' Lois nodded.

``Then I have a further argument for you,'' said Lex. ``Perhaps you
perfectly trust Superman not to look at you while you change, or perhaps
you have no secrets you'd rather he not be privy to, but do you believe
that Superman will always be the only one with his abilities? We can
infer that there are other aliens out there, and here on Earth there are
plenty of scientists~--- myself included~--- who are working to
reverse‐engineer the things they see him do. If tomorrow my rivals in
business can see through my walls, they'll find my defenses already in
place, which is only prudent.''

``I suppose,'' said Lois. She looked down at her notebook. ``I think I
have everything I need. More than I need, actually. The article isn't
going to be particularly long.''

``You can admit that you enjoy talking to me,'' said Lex.

``It's stimulating, I'll give it that,'' said Lois. ``But I also came
here to thank you. The ERA passed the Senate and moved onto the House,
and even if it fails there I'll consider you to have held up your end of
the bargain.''

``I'm a man of my word Miss Lane,'' said Lex. ``Though I have to warn
you that prospects are bleak. The Eighteenth Amendment has made people
shy of modifying our founding document.''

``All the same,'' said Lois.

There was a moment where perhaps Lex could have asked her to dinner, but
he let it pass by. Lois was tenacious and decisive, intelligent and
principled, and in another time he might have tried to see whether she
could sustain his interest in the long‐term. Now was a time of action,
and the threat of Superman was too great to permit for such idle
distractions. Later perhaps, when Superman lay dead in the street, Lex
would go on the pursuit.

After they'd said their goodbyes, Lex sat in his smoking room and
thought about explosives. The actual designs would have to wait until
his study had been coated in lead, but until that time he could refine
his plans within his head. He would need to find someone to carry out
his will, someone without a strong moral compass, but he thought that he
had just the right person in mind.


%%%%%%%%%%%%%%%%%%% NEW CHAPTER %%%%%%%%%%%%%%%%%%%%%%%%

\hypertarget{like-clockwork}{%
\chapter{Like Clockwork}\label{like-clockwork}}

Harry Kramer loved explosives. He loved the danger of working with them
and the thrill of watching them go off. A properly made bomb was an
amazing piece of engineering, a compact device of wires, springs, and
explosives all set up in a very precisely and ordered way. When the bomb
went off, all that hard work evaporated in a single transformative
moment. It was like taking a piece of fine crystal and hurling it
against the side of a brick wall, and how could someone not feel joy at
that? How could someone not see that there was something magical that
only existed in that single solitary moment when the product of labor
and a thoughtful mind became nothing more than garbage? Though there
wasn't anything sexual about it, the best word that Harry had found for
it was orgasmic.

A thick letter came in the mail for him. He ran a few simple tests to
see whether it might contain a bomb, sniffing at it and hefting it
carefully. Letter bombs were tricky to do, because you couldn't reliably
set them on a timer unless you knew for certain when they'd be opened.
The letter also had to make it through the postal service without
detonating or being discovered, which was a challenge in and of itself.
The most common way to make a letter bomb was to fill an envelope with
two chemicals that were explosive when mixed, separating them with
layers of paper. Another chemical trigger was placed along the top where
the paper was going to be ripped. The chemicals would mix when the
letter was opened, and the bomb would explode, but that was often messy
because people didn't always open their own mail. It was easier to make
a larger package that would explode, because then you didn't need to
worry about the bomb being bent or squeezed, but there was a very clear
distinction between a ``letter bomb'' and a ``mail bomb'' owing to the
restrictions on construction.

Harry had a recurring fantasy about being sent a letter bomb. In the
fantasy he would smell the metallic powders and carefully disarm the
bomb in his workshop, pulling it apart to expose its secrets. Written
inside the letter bomb would be words of congratulations for showing
caution, and a coy invitation to begin using his skills in earnest. In
the fantasy he and the other bomber would engage in a conversation
written across the city in explosive force, needing nothing more than
concussive blasts to speak to each other. There was something raw and
primal about destroying the ordered world of the city. Eventually Harry
would prove himself the superior of the two, and she would reveal
herself to him, and declare her undying love for him. It was always a
woman, of course. They would exchange hot, hungry kisses on the rooftop
of his apartment as Harry's bombs leveled the city.

The letter he'd received wasn't a bomb. Instead, there was an offer of
employment. Beneath that, the bulk of the envelope containing crisp
twenty‐dollar bills, enough to pay for his apartment for two full years.
The letter was concerning, because it meant that someone knew about him,
but it was exciting, because it meant that he was going to get to do
something that he loved. It wasn't some simple job that required only a
simple demolition or death, it was finally a chance to be unchained and
fully funded. No longer would he have to cobble something together from
bits and pieces. He was going to make something beautiful.

\begin{center}\rule{0.5\linewidth}{0.5pt}\end{center}

``What makes a person do a thing like this?'' asked Clark.

Lois rolled her eyes.

Clark was a heavy man, thick without really seeming muscular, though you
could tell from a glance that he'd never learned to buy clothes that
fit. He had terrible posture, his hair was messy, and he wore glasses so
thick you could hardly see his eyes through them. He seemed to get sick
constantly, and he was so out of shape that whenever they had to move
quickly he could be seen gasping for air afterwards. He had the desk
right next to Lois's, and so she'd had time to examine each and every
one of his faults~--- that was just a small sampling of the physical
problems with Clark. Much to her consternation, he was somehow the
second best reporter at \emph{The Daily Planet}. They were often paired
together for the big stories, since it allowed Perry to run a companion
story to a front‐pager. More often than not, Lois found that being
around Clark tried her patience. It was made worse by the fact that he'd
quite obviously developed an infatuation with her from nearly the day
that he started working at the Planet. He'd asked her out during his
second week, and she'd politely but firmly told him no, but he was still
hung up on her. One of the only good things about Clark was that he was
as transparent as glass. His crush was more sad than annoying, most days
anyway.

Lois and Clark were standing outside the remains of an apartment
building. It had exploded earlier in the day at around noon, sending
bricks, wood, and personal belongings in every direction and shattering
a number of windows all around the block. Two people had died, and a lot
more had been seriously injured. The apartment was still standing, but
three of the upper floors were now just a gaping hole, and it was likely
that there was enough structural damage that the building was a total
loss. Everyone talked about how much worse it could have been. It was
front page material for sure.

``Some people are just evil,'' said Lois.

``I don't think a person is born a certain way,'' said Clark. ``People
make choices, for good or evil. Free will is part of God's design. I
just can't understand why someone would make this choice.''

Lois tried to stop herself from rolling her eyes again. ``Some design,''
she said, as she spotted a severed arm in the rubble that no one seemed
to have picked up yet.

Lois and Clark had done their interviews, talking to the victims,
police, firefighters, and neighbors. There was little question that the
explosion had been deliberate. The police were already chasing down some
promising leads, though Lois knew that half the time they only said that
to keep people reassured.

They'd been back at the Daily Planet Building working late when the
second bomb had gone off, exactly six hours after the first. This one
was at a sales office downtown. Most of the staff had gone home, but the
rescue workers had pulled a few corpses from the wreckage. She overheard
one of the onlookers say that it was a tragedy that people had died
because they'd stayed late to work. She made sure to put that in her
article.

The third bomb exploded in Superman's face. He'd found it in the freezer
of a grocery store, and got people out of the way before he'd tried
moving it, which was when it had blown up. Lots of people reported
seeing a gaping hole torn right in the center of his costume. Superman
had spoken directly with the chief of police, giving him as much
information as possible. Lois had come back into the office late at
night in order to write about it, and found that Clark was already there
in a wrinkled shirt, looking for all the world like he'd never stopped
working when she'd left at eight. Though he finished his article before
her, she came up with the better moniker~--- the Clockwork Bomber. Perry
groused about them being too competitive and wasting effort writing the
same story, then decided to run Lois's article in the morning edition
with the headline ``Clockwork Bomber Strikes Midnight!''. The long hours
were worth it just for the forlorn look on Clark's face.

Lois set her alarm for five in the morning. The first bomb had been at
noon, the second at six, and the third Superman had detonated just
before midnight. The pattern was obvious to anyone with half a brain.
Ten minutes before six o'clock in the morning she heard a distant rumble
from across the city, and she was ready to trek off towards it in her
most sensible shoes. Clark was nowhere to be seen, and despite being
tired as hell, Lois felt a warm glow of satisfaction that she'd beat him
to the punch.

The mayor and the chief of police held a press conference, where they
promised that they would find the man or men responsible. No one made
any demands, and no one claimed credit. Everyone braced themselves for
another bomb at noon, but it didn't come. Four bombs had claimed the
lives of six people, and there didn't seem to be a point to it. The
casualties had been much lower than they could have been, given the time
of day that the bombs had gone off and the locations that they'd been
placed, but it was anyone's guess what that said about the bomber.

A few days passed, and eventually things began to settle down again.

Lois was surprised when she found a second letter on her desk, addressed
to Miss Lane and requesting to meet her in the same place as before. She
was ready this time, and grabbed a sheet of paper with a series of
questions from inside her desk. She stopped by Perry's desk to tell him
where she'd be going, just in case something happened. Perry looked
ecstatic, but Lois felt her nerves getting the best of her.

She prided herself on being utterly fearless. She'd stood on the spire
of the Emperor Building as the first airship came in, strapped in with
what amounted to a thick belt. She'd hunted big game with Hemingway over
a memorable summer in Kenya. She'd braved storms while sailing the North
Atlantic in a yacht, the closest she'd ever come to actually dying. She
found these adventures exhilarating instead of terrifying. Yet there was
something about Superman that tickled some animal part of her brain. She
did her best to ignore it, and made the trek up to the rooftop where the
Man of Steel was waiting.

\begin{center}\rule{0.5\linewidth}{0.5pt}\end{center}

``Hello Lois,'' he said as he turned around. His smile was gentle, but
it didn't help her nerves. Luthor had said that Superman moved faster
than muscles alone would dictate, but that didn't make the muscles look
any less impressive. It was impossible for her to look at him and not
think about the fact that he could cross the distance between them
faster than she could blink.

``Hello Superman,'' she replied. ``I've got some questions for you.''

``I know,'' he said.

Lois immediately imagined him staring through the walls, looking over
the questions she'd prepared for him and composing answers. It felt
utterly invasive~--- she would never allow an interview subject to look
over the questions like that, not at this stage in her career. She
really should have gotten one of those lead‐lined drawers. Of course,
maybe he'd just meant that he knew she had questions because everyone
had questions. She found herself unwilling to give him the benefit of
the doubt.

``Go on,'' said Superman. ``But I can't answer everything.''

``Where is your ship?'' she asked.

``It burnt up over the Atlantic on my way in,'' Superman replied.

``Could you find the wreckage?'' she asked.

``There wouldn't be anything left,'' said Superman. ``Even if there
were, I wouldn't hand it over. If humanity were able to work backwards
and figure out how it was made, I fear the results would be disastrous.
It would be like giving a gun to a baby.''

Lois frowned. ``And you're the final arbiter of what's good for
humanity, what we can and can't handle?''

``I am the arbiter of myself,'' said Superman. ``I can only do what I
think is best, and hope that humanity gives the same consideration to
their own actions.''

``Okay,'' said Lois. ``But are you really doing the most good? I mean,
I've seen proposals for what other people would be doing with your
powers, digging canals or generating power, searching out veins of ore,
the amount of money\===''

``I don't need money,'' said Superman. He interrupted her so delicately
that she momentarily lost her train of thought.

``You don't,'' she replied slowly. ``But the rest of us do. These are
lucrative jobs that could bring in millions, and with that you could
fund orphanages, women's shelters, homeless shelters, or whatever
charitable organizations you wanted. We could set up a trust. It
wouldn't matter that you were using your powers for a profit, because
that profit would be directly translated into good works that would
overwhelm positive effects of the crime fighting and general heroism you
do now.'' Lex Luthor's words were coming out of her mouth. ``And if you
embraced the celebrity that you already have you could charge enormous
amounts of money for the use of your image. People are already making
lunchboxes and trading cards with your emblem, and I've heard that
they're making two different movies about you. These things are going to
happen whether you're involved with them or not, and you could at least
make some money that you could use for good causes.''

``Saving people from violent crime is an unambiguous good,'' said
Superman. ``Bringing money into it isn't, and I don't know that I should
be supplying humanity with a brawn that it doesn't and shouldn't have
yet. I've tried my best to confine myself to acting only when there is a
clear good to be done. I'm trying not to bend the course of human
history, or force my morality on anyone else. I do that by operating
within the laws of the country and avoiding controversy as much as
possible. I have as few points of interference with a citizen's daily
life as possible.''

``You think that an avoidance of controversy is part of the greater
good?'' asked Lois. ``Do you think that the laws of this country are
anywhere close to just?'' She pointed across the city to the docks, and
the channel where ships were streaming in and out of the harbor. ``A
hundred years ago there were slaves being sold here. If you'd shown up
then would you have stopped slavemasters from beating their slaves? Do
the laws of men mean that much to you that you'd actually let such an
injustice stand?''

``You're losing your cool,'' said Superman.

Lois looked down at her notebook. She hadn't asked him a question from
it for quite some time. ``You're right,'' she said. ``I'm sorry. It's
just that sometimes I think about what I would do if I had your powers,
and in comparison you seem so\ldots{}''

``Reluctant?'' asked Superman.

``Yes,'' Lois replied.

``During Prohibition, as part of an effort to stop people from drinking
industrial alcohol, it was denatured and methyl alcohol was added,
making it toxic. They thought that people would change their behavior.
The end result was that the United States government killed ten thousand
of its own citizens.''

``I wrote an article about that,'' said Lois. ``It never made it to
print.''

``I know,'' said Superman. He looked out towards the city in quiet
contemplation. ``I believe that the people who poured their poisons into
the vats truly believed that they were doing good. They just couldn't
see what the end result would be. Even with the work I've been doing,
there have been unwanted side effects.'' He pursed his lips. ``I get the
distinct impression that people are less cautious with their lives now
that they have me around. People shout for me to save them instead of
taking action. There was a fire in an apartment building three days ago,
and half the occupants ran up to the roof and screamed for me to come
pick them up. If I'd been dealing with some other more serious disaster
at the time, those people would have died. These are the things that
happen on even a small scale when humanity is saved from their own
mistakes and steered away from forging their own path. I'm sure you
could think of half a dozen other examples of the unintended
consequences.''

She could. The budgets for the police and fire department in Metropolis
were up for review, and both looked like they were going to be cut by a
large percent, because the city saw no point in paying the same amount
for services when Superman had taken up much of their duties. Those
elements of the underworld with sufficient mobility were moving to
Gotham City, causing a crime wave there the likes of which hadn't been
seen in a decade. The ones that stayed in Metropolis were more organized
than before, with a higher propensity towards subterfuge, trickery, and
crimes which didn't make a sound. Superman didn't speak anything but
English, and so there had been an explosion in language learning. That
was above and beyond the general insanity that came from having a man
that flew through the air, and the world's first extraterrestrial.

There were many things that Lois wanted to say, but she was worried
she'd get too wrapped up in argument again. A good reporter pressed
their subject, but didn't get heated. If she were speaking to him
outside of her role as a professional, she might have called a policy of
non‐intervention the definition of moral laziness. She might have told
him that he had the most inconsistent moral system she'd ever had the
displeasure of encountering. The truth was, she didn't like Superman.
They'd both read the various proposals and the pleas for aid. There were
so many things that he could do, and he simply refused to do them. It
might have been one thing if he'd engaged in reasoned debate, but
Superman had acted unilaterally, thinking that he knew what was best for
humanity. Her thoughts returned again to when he'd scooped her up like a
child. Superman was a man~--- an alien~--- of presumptions.

But Lois Lane was a good reporter, and so resisted the urge to berate
him.

``How long were you on the planet before you began intervening?'' asked
Lois.

``Two weeks,'' said Superman. ``I learned English on the way over from
your radio signals and spent a good deal of time watching from above and
getting a more in‐depth understanding of your culture and the ways of
your people, as well as the relevant laws.''

``And did you anticipate what followed?'' asked Lois.

``For the most part,'' said Superman. ``Celebrity, shock, awe, analysis
--- that was predictable. What I hadn't counted on was the cruelty or
organization of the attempts to kill me.''

Lois furrowed her brow. ``You're talking about the people trying to
shoot you?''

``No,'' replied Superman. ``That I expected. The criminal element was
bound to try. I let them sometimes, just to prove how useless it is to
stand against me, but most of them attack me like it's going to do some
good. I stopped a mugging three weeks ago, and the man kept stabbing my
eyes. It didn't do anything more than dull his knife, and eventually he
ran out of steam. Sometimes they shoot me and look at their guns like
they're shocked that it didn't work. Maybe some people don't really
believe the stories until they see it for themselves. No, all that I
expected. I'm talking about the bombs. That's why I came to speak with
you today.''

``The Clockwork Bomber,'' said Lois.

``Yes,'' said Superman. ``All the bombs were meant for me. They were
encased in lead and had mechanisms inside to prevent me from doing
anything with them. I think someone was making an effort to kill me.''

``It seems obvious that wouldn't work,'' said Lois. ``Even on the face
of it.''

``The bombs were special,'' said Superman. ``They used focused blasts
and a variety of different materials. I think one was an attempt to
blind me. They're probing for a weakness.''

``But it didn't work,'' said Lois.

``No,'' said Superman. ``I've been looking over the city and trying to
connect the dots. Whoever set the bombs up is going to try again. I need
you to warn the people of Metropolis. If I'm right, next time it's going
to be worse.''

\begin{center}\rule{0.5\linewidth}{0.5pt}\end{center}

Ninety‐nine percent of the time, ripping a handful of wires out of a
bomb will safely defuse it, either by removing the fuse from the
detonator or the detonator from the explosive material. Most people who
made bombs were unsophisticated, and most bombs were designed not to be
found until after they had detonated. There wasn't much point in making
them particularly hard to defuse or move, and there weren't many people
with the technical skill to do it.

The bombs that Harry designed were complex, above and beyond the
complexity designed into them by his benefactor. They had to be, because
their target was Superman.

Many things could be made fail‐safe. The railways used air brakes, in
which a piston was held up by compressed air. To apply the brake, some
air was let out of the system, causing the piston to lower and the brake
to be applied. If any of the components of the system failed, the brake
would be engaged by the loss of pressure, stopping the railcar and
preventing it from going out of control. Fail‐safe design was becoming
more and more important as a method of stopping machines from
self‐destruction.

The bombs Harry made were fail‐deadly. The detonator was connected to a
timer, but the timer didn't cause the bomb to explode~--- it prevented
the explosion from happening. Removal of the timer would collapse a
circuit and cause the bomb to explode. Removal of the detonator would
cause a circuit to collapse and trigger a secondary hidden detonator.
Several small glass tubes were filled with beads of mercury which were
part of the circuit, and if the bomb was tilted too far in any direction
a circuit would complete and cause the bomb to explode. No one would
ever be able to see this hard work, not even Superman, because the whole
thing was encased in lead shielding. Wires were affixed to the interior
of the casing, and if the lead shielding was removed the bomb would
detonate.

Most bomb makers didn't make their bombs this complex. It was more work,
and with the work came a higher risk of accidental detonation. With the
amount of explosives that Harry was using, it wasn't really a concern
for him. What he feared was a small explosion that left him limbless and
bleeding out, but given the number of pounds of cyclonite he was working
with, an accident would leave him vaporized. It didn't seem like such a
bad way to go. In truth, Harry liked the heightened sense of reality
that came from being one mistake away from utter destruction. The
benefactor had designed the bombs to be dangerous things, and Harry had
modified them to be nearly reckless.

``Be careful with that,'' said Harry as the workmen took the first bomb
out of the workshop that had been rented for him. ``It's fragile.''

They hadn't smiled at his joke, but then they didn't know what was in
the crate they were carrying out. The circuit with the mercury switches
was on a separate timer to ensure that the bomb wouldn't blow up in
transit, but there was still more risk than most people would want to
take. Harry had no idea where the workmen had come from. Like many
things, the benefactor took care of it.

He also had no idea where the bomb was headed, but he couldn't help
smiling as his bomb ventured off into the world. He'd headed back into
his shop to make some variations on the theme.

\begin{center}\rule{0.5\linewidth}{0.5pt}\end{center}

Lex had tried doing things cleanly. The Conference on Extraterrestrial
Science had put out a plea to Superman, asking him to attend a meeting
of minds so that they might make a cultural bridge between human and
Kryptonian science. Superman could have come forward and simply spoken
to them about what the true limits of his powers were, but he hadn't
even responded to them. The invitation carried nearly every important
name among the scientific elite, and the lack of response couldn't be
seen as anything but an insult. Lex had put forward a mountain of plans
and proposals that would allow him to get close to Superman, and almost
all of them would allow for an advancement in what most people would
consider to be the common good. Superman hadn't responded to any of it.

The bombing campaign served multiple goals, as any good plan did.

Superman was an extinction level event waiting to happen, and where
those were concerned there were no second chances. If Superman ever
decided to kill everyone, there would be no stopping him, and so it
stood to reason that humanity should take every possible precaution to
prevent that from happening. The most direct path would be through
killing Superman. Lex had written multiple letters to the editor under
various pseudonyms, but none had ever been published, and his point of
view seemed entirely unpopular. It was always one that he voiced from a
position of anonymity, because in public he was playing the role of
Superman's champion.

People were bad at estimating the risk that an extinction posed, because
no one had ever lived through one. People were also quite bad at
imagining a catastrophe so large. A woman might weep when you mentioned
the possibility of her child dying from consumption, but the total
obliteration of Earth‐originating life would produce only a shrug. It
was too vast for people to think about rationally. Worse, they assumed
that ``Superman is the greatest threat to humanity'' was a shorthand for
some decision on Superman's part, when in truth that was only a part of
it.

Many people accepted Superman's story at face value; the last son of a
dying planet, the only one of his kind to exhibit such incredible
powers, with little aid from technology save for the ship that had
provided him with a trip through the stars. There were many parts of the
story that Lex was skeptical of, but he found it most terrifying to
think that the story was true, namely because of what it suggested about
Kryptonian science.

Huntington's disease was a hereditary degenerative disease with
cognitive and psychiatric symptoms, one of which was psychosis.
Huntington's was seen in perhaps one in eight thousand people, and
psychosis was seen in perhaps one in ten of those. If a randomly
selected human of Superman's apparent age were to obtain Superman's
powers, there would be a one in eighty thousand chance that they would
both have Huntington's disease and symptoms of psychosis, the result of
which would probably be casualties that would dwarf the Great War by a
large margin. If Superman was telling the truth about the culture that
he came from, his society wasn't much further advanced than humanity,
and so likely hadn't grown past degenerative diseases and hereditary
defects. Even if Superman were perfectly good in some abstract sense,
the onset of a mental disease might be just around the corner.

Worse, if Superman's powers weren't the result of engineering and
carefully controlled science (a hard pill to swallow) then no one had
made sure that they were safe, and perhaps some day something internal
to him would simply unravel, unleashing enough energy to destroy an
entire hemisphere. If Superman was to be believed, his powers had come
from seemingly nowhere, and yet everyone simply trusted them as though
it were the most natural thing in the world.

Estimates were difficult to make, given Superman's silence. His second
interview with Lois Lane had provided little illumination. Nevertheless,
numbers could be pulled from thin air in order to get a sense of things.
There was the possibility that something would happen that was
completely outside of Superman's control which would result in Superman
destroying the Earth. There was the possibility that Superman could
simply have a bad day and decide to kill a large number of people, which
many people seemed to think was absurd. There were also failure modes
which didn't involve the destruction of humanity but would nevertheless
result in an effective end to humanity as Lex Luthor knew it, the most
probable of which seemed to be that Superman would turn into a tyrant.
When these probabilities were multiplied together, the final very rough
estimate was that Superman had a one in ten chance of bringing about a
global scale human catastrophe of some kind in the next thirty years.
Even if the odds had been one in a hundred, Lex would have taken a
similarly extreme course of action.

The collateral damage caused by the bombs was negligible in comparison
to the threat of Superman.

But of course the bombs were unlikely to kill Superman. The first four
had been for calibration, built with a small device which gave a series
of loud chirps prior to detonation to allow Superman time to get to it
before it exploded. The next series of bombs would introduce more exotic
methods of harm which hadn't yet been conclusively ruled out, but the
prospects looked grim.

The secondary goal was to probe for a weakness. Lex had it on good
authority that Superman had taken the equivalent of a direct hit from
navy artillery to his chest when the third bomb exploded. He'd simply
looked surprised that he'd set it off. The magnesium and phosphorus
compounds had done nothing to blind him, and he'd been talking with the
police soon afterwards with no ill effects. Lex had suspected as much,
but perhaps something would be found that could harm him but not kill
him, or otherwise give Lex an advantage. Lead had been a boon, and
allowed Lex a level of freedom that was gratifying until he remembered
how free he'd been before Superman's arrival.

The third objective was testing Superman's limits. Lex kept a detailed
log of Superman's movements in his study, as well as a large map of
Metropolis which was covered in small color‐coded labels that
corresponded to Superman sightings or activities. Superman's patterns
had been mapped against the general patterns of crime and emergency in
Metropolis, and Lex had not been all that surprised to find that the
patterns didn't quite match one another, even taking into account
Superman's preferences for certain crimes and emergencies over others.
There were two lulls, one during the daytime that seemed to start around
eight in the morning and end around five in the afternoon, and one in
the dead of night from three in the morning to five in the morning. Lex
had no idea what to make of it, but kept the information safely locked
away behind lead walls. Perhaps Superman needed to sleep, or needed to
recharge in some other way, but sustained and consistent bombings would
allow for information to be gathered.

The fourth objective was to identify the place that Superman retired to
when he wasn't flying around the city, since Superman demonstrably
didn't spend all of his time on heroics. Lex strongly suspected that the
ship hadn't broken up over the Atlantic, and was in fact located
somewhere in or near Metropolis. Depending on the size it would be
difficult to hide, but Superman could surely lift the craft up and move
it at will, which meant that it could be nearly anywhere. All that was
under the assumption that Superman was an alien~--- there was still an
outside possibility that there was some other explanation. If the
spaceship existed, finding it was of utmost importance. Lex had already
hired a team of private investigators to see if they could find some
trace of a ruined ship in the Atlantic, though without eyewitness
accounts of where the spaceship had burned up it would be impossible.
With them it would merely be very, very difficult. Still, it was worth
trying.

The next wave of bombs would be planted in two weeks time. Perhaps Lex
would get lucky and Superman would prove to have a weakness.

\begin{center}\rule{0.5\linewidth}{0.5pt}\end{center}

\emph{Author's Note: This chapter was getting too long, so I split it in
half. The next half will come at the regularly scheduled time on Sunday
night.}

\emph{Any numbers that Lex or anyone else gives is their own best guess
based on what might have been knowable in the 1930s before the age of
the internet. I don't guarantee that these are at all period accurate,
and obviously we're dealing with an alternate universe where a city
named Metropolis exists.}

\emph{A note on geography: I'm writing on the assumption that Metropolis
replaces New York City and Gotham City replaces Chicago, though with
different city layouts and some changes to small scale geography of the
region.}

\emph{Two historical notes: First, the American eugenics movement was
still alive and well at this time, so if you see references to it pop up
here and there, just remember that this was an opinion you could voice
without anyone really raising an eyebrow. Second, the United States
really did denature alcohol, which wound up killing more people than
Superman probably saves in a decade. The more you know!}

\emph{As always, I appreciate any corrections, comments, or general
feedback, and thanks for reading.}


%%%%%%%%%%%%%%%%%%% NEW CHAPTER %%%%%%%%%%%%%%%%%%%%%%%%

\hypertarget{a-stopped-clock}{%
\chapter{A Stopped Clock}\label{a-stopped-clock}}

Lois Lane had been walking down 15th Avenue looking for a place to eat
breakfast when she'd heard the bang from the next block over. She'd
started running towards it seconds after she'd heard it, while everyone
else on the street was looking around like they'd missed something. If
they'd read the paper, they'd know that Superman had predicted that the
Clockwork Bomber would be back. Two weeks had passed and whatever leads
Superman had been following, they hadn't led anywhere, since no arrests
had been made. Lois looked down at her wristwatch as she ran~--- it was
almost exactly nine in the morning.

Lois hadn't been close enough to the other bombs to get there in time;
when she or Clark showed up, the whole thing was already over, and the
mad panic and confusion that followed a bombing had given way to shock
and grief. This one was different, a chance to be close enough that she
would be one of the first on the scene. As she turned the corner and ran
past an appliance store, she could see the debris strewn out over the
street and the broken windows. People were still picking themselves up,
and a few were bleeding, but it didn't look nearly as bad as the other
bombs had. She was fishing a pencil and paper out of her purse when the
second bomb went off. It was small and subdued, much softer than the
first, and there were shouts of surprise but few of pain.

Lois started forward, just as Superman arrived on the scene.

He moved at speed, darting into the damaged storefront so fast he was
little more than a blur, and leaving minutes later. He was carrying what
looked like a box, and trailing yellow‐brown smoke behind him. Lois
tried to follow his movements, but after only a few seconds she'd lost
him. He was back half a minute later, and landed right in front of Lois.

``Can you smell anything?'' he asked.

``Horseradish,'' said Lois. The pieces clicked into place. ``It's
mustard gas.''

Superman nodded, and was back in the fray in moments. Lois sprang into
action, calling out to people to get away from the site of the
explosion. If she could smell the mustard gas, that meant that she was
too close. She tried to remember what the medics used on people who'd
been exposed mustard gas. Her father was a general in the army, and had
fought in the Great War, but it was too far before Lois's time. The most
she could do was to get people away from the gas, so that they wouldn't
suffer from exposure. Mustard gas was an insidious poison mostly because
it took a while to take effect, and if you didn't know what the odor
meant you wouldn't think to take action until long after it was too
late. Lois concentrated on getting people to safety, and yelling out
instructions. It caused blisters, not just where it touched exposed skin
but in the nose and throat as well. It could damage the eyes so badly
that you'd go blind. If you weren't killed by the swelling of the
throat, you could still be made mute. She ripped at her blouse and
fashioned a crude mask for herself, and helped others to do the same.

After everyone was clear of the gas, or at least in an area where they
could no longer smell it, Superman landed next to her.

``Call the radio stations, tell people to stay inside and keep their
windows closed. If I'm right, the next one will come in six hours.'' She
nodded, but didn't really need to be told what to do. She could keep a
cool head under pressure. Superman was crouched down and ready to launch
into the air when a buried thought surfaced.

``Wait!'' she called, worried that he would be a mile away by the time
the words left her lips. But he stood up and looked at her, puzzlement
on his face. She took a deep breath. ``You said that the bomber was
trying to get to you,'' she said. ``Why not let him have you? We could
send a message out over the radio, and make a deal. Even if we can't
disarm the bombs it'll let us evacuate people.''

Superman hesitated. She could swear she saw his eyes blur as they moved
around to take in the crowd in a fraction of a second. ``Lois, I don't
know whether or not these things will kill me. I don't even know if the
mustard gas is going to have an effect on me. I breathed in more than
anyone else before I realized what it was. He wants to kill me, that's
the only kind of deal I think he'd listen to.''

``But you'd rush in to save people anyway,'' said Lois. ``We're just
cutting out the possibility of collateral damage. You'll be fine.''

Superman stared at her, and she was sure they were both painfully aware
that everyone around them was listening in on their conversation. Some
were outright staring.

``No,'' said Superman. ``I don't negotiate. And if the bomber wants me
at the site of these bombs, I'm not going to play into his hands.''

He hurled himself into the sky and flew away before she could figure out
how to respond to that. Had Superman just said that he was handing
Metropolis over to the bomber?

\begin{center}\rule{0.5\linewidth}{0.5pt}\end{center}

``An officer Kennedy for you, sir,'' said Mercy.

``Thank you dear, I'll take it now if you please,'' replied Lex. He
calmed himself, and got into character, becoming a man who knew nothing
about what was happening across town. ``Officer Kennedy?'' he asked with
a pleasant voice. ``What is this regarding?''

``Ah, Mr. Luthor we really appreciated your donation to the Policeman's
Ball this year,'' said the policeman, ``And the chief was saying how you
wanted updates on anything real important having to do with Superman, so
he just thought I should give you a call to keep you up to date.'' It
wasn't a bribe per se, just a mutually beneficial friendship.

``Has something happened?'' asked Lex. He allowed some genuine‐sounding
concern creep into his voice.

``Well sir, it seems like the Clockwork Bomber is back, and he's working
with some nasty stuff. The boys say that it's mustard gas, like from the
trenches of the Great War?''

``I'm familiar, yes,'' said Lex. ``Superman came in to save the day?''

``Well, here's the thing,'' said Kennedy. ``He came down and pulled
people out of the gas, and said it was the Bomber come back, but then
that lady reporter told him that he should try to make some kind of deal
with the Bomber, because she seemed to think that the Bomber was trying
to kill him and that maybe Supes could save everyone a lot of trouble by
letting him try, for all the good it would do because he's invincible
right?''

``I see,'' Lex replied. ``And his response?''

``He said he wouldn't negotiate,'' said Kennedy. ``And then he just flew
off like he didn't want to hear any more about any bombs. Like he was
done helping out with them.''

``He's abandoning us?'' asked Lex.

``I don't know sir,'' said Kennedy. ``But it sort of sounded like it.''

``Thank you for the update, officer,'' said Lex. ``Tell the captain that
the police of this city continually reaffirm my faith in them.'' Lex set
down the phone without waiting to hear a response. He steepled his
fingers for a moment before remembering that he needed to keep up with
the role he was playing. Someone might notice if Lex responded to news
of an attack on Metropolis with only a look of quiet contemplation.

``Mercy, it appears that the Clockwork Bomber is back, and using
different tactics. I believe chemical agents were mentioned,'' he said.
Radioactive and biological too, though the person he was pretending to
be didn't know that. ``We should be safe in this building, but I want
you to start on calling the managers and telling them to follow the
drills and keep tuned to the radio. If it's like last time, the next
bomb will be in six hours.''

``Yes sir,'' said Mercy, picking up the phone before he was even done
speaking. She was invaluable, and likely could have handled the entire
crisis on her own without his instruction. She didn't know the full
extent of Lex's plans, but she knew enough to implicate him in a vast
number of crimes, and that was a mark of the extreme faith he placed in
her. Lex turned on the radio in his office. It was there more as
camouflage than to provide any information. Superman's movements and
actions were the most important thing right now, and he was skeptical of
the radio's ability to provide that information. Lex's other channels
were slower but more reliable, and there were enough of them that any
unreliability in one could be compensated for in the others.

Lex had a contingency plan in place. There were two couriers waiting by
phones in separate locations within the city. A message could quickly be
relayed to them which would send them to the nearest police station. One
courier had an encoded message, while the other had the one‐time pad
needed to decode it. Both pieces were in envelopes lined with lead. Once
decoded, the message would give the locations of all of the bombs, and
the buildings could be evacuated, saving lives and likely preventing
property damage. The only question was whether this was the proper time
to deploy that plan.

Lex had killed for the first time when he'd been fifteen. Willie Calhoun
had entered him in a bare knuckle boxing fight, and Lex had landed a
lucky punch that burst an artery in his opponent's neck. He'd been
rewarded with a twenty dollar bill and a slap on the back. He'd
committed his first actual murder later that year, when a shop owner had
gotten wind of the plan for a nighttime robbery and decided that the
best course of action was to lay in wait with a revolver and the lights
turned off. That shootout was the closest that Lex had ever come to
dying. His hands had been trembling when he shot the shopkeeper in the
face. He'd been less hardened then.

Lex took no special pleasure in murder. It stirred no passion in him to
see the life leave a man's eyes. It gave him no glee to hear about the
people who died or were injured by the bombs, just a certain sense of
sadness that he imagined other people might feel more keenly. He
certainly didn't feel any guilt. Lex sat back and looked at his watch.
The next bomb would be going off soon. He tried to make a careful
consideration of the possibilities.

It was possible that Superman had a weakness to biological, chemical, or
radiological attacks, as it was one of the only vectors of attack that
hadn't yet been tried. Numerous witnesses had reported seeing Superman
breathing, and none had specifically noted that he wasn't drawing
breath. Though Superman had never been seen coughing or sneezing, and
had surely endured smoke inhalation on an absurd level while engaging in
fire rescue, it was still reasonable to assume that he had a biology of
sorts, and that this biology could be disrupted in some way. He had not
yet been observed eating, drinking, or sleeping, but that might have
been something done during what Lex thought of as the quiet periods when
Superman was less active.

Superman might be afraid of dying to the bombs. If true, this would be
incredibly valuable information, and assuming that Superman didn't
radically alter his modus operandi in response to these attempts, it
would be fairly simple (in the scheme of plans that had to work around
Superman's powers) to stage some event to attract his attention and
deliver the poison while Superman suspected nothing. The reason that Lex
hadn't done it this way in the first place was the enormous amount of
planning, expense, and exposure that would have to go into doing that
for each of the thirty candidate attacks that he had planned. Mustard
gas, phosgene, chlorine, contact poisons, pesticides, polonium; it saved
an enormous amount of time to simply allow Superman to know that someone
was trying to kill him and put him in a position where he would either
expose himself or expose an unwillingness to intervene.

The second possibility was that Superman was thinking of the future.
Superman had routinely refused to make deals with criminals who had
taken hostages, presumably because he knew that if he did, more
criminals would begin taking hostages in order to put themselves in a
position to strike a bargain. Similarly, if the bombs were only being
placed because the bomb‐maker expected Superman to show up, then
Superman's best course of action to prevent bombs from being placed was
to stop showing up. Of course, that would only lead to a change in
tactics, and not one that would likely result in better outcomes from
Superman's perspective. Lex had dozens of ideas on how to administer the
poisons if Superman refused to touch the bombs. But perhaps Superman was
under the delusion that his unseen enemy would stop trying over a little
thing like changing strategies.

Ultimately, Lex decided against using the contingency plan, at least in
the short term. The message from the police officer had been too vague,
and even if Superman had directly stated his intent to leave the rest of
the bombs untouched, it was possible that the alien was bluffing. Lex
didn't particularly like the prospect of martial law being declared, nor
the unfortunate economic impacts of a sustained series of bombs in the
largest city in America, but the dice were already cast. If Superman had
anticipated the bombings and outright stated his refusal, perhaps Lex
would never have spent the time and money going down that path, but with
everything set up the majority of the costs had already been sunk.

\begin{center}\rule{0.5\linewidth}{0.5pt}\end{center}

``Where the hell have you been?'' asked Lois. She'd been seen by a
doctor, gone home to shower, changed clothes, and then gone back to
work. It turned out that there wasn't all that much you could do about
mustard gas, and while the doctor had wanted her to wait it out to see
whether she would develop any symptoms, she was pretty confident that
she'd had a low enough dose, so she'd slipped out the door when he was
seeing to someone else. No way was Lois Lane sitting on her ass when
there was news being made. From what the doctor had said, people didn't
get worse all at once, so if it got bad she'd go in. She'd called Perry
to let him know she was alright, and then kept calling Clark because she
wanted updates.

Clark sat at his desk, typing up an article. He typed with both his
index fingers, punching the keys down one at a time. As she watched he
took a glance at the keyboard to see which key was which. Lois could
type so fast she very nearly hit the mechanical limits on her Underwood.
Speed didn't matter all that much for a reporter, but it was still
grating to watch him do such a poor job of something so basic.

``Clark?'' she asked. ``Where have you been?''

``Sorry,'' he said. He pointed to the typewriter in front of him. ``I
got a call in from the Midwest, apparently there was a Superman
sighting.'' He hadn't answered her question, but then again, Clark was
hopeless. ``You're taking point on the return of the bomber?'' He always
said ``bomber'', not ``Clockwork Bomber'', which Lois felt was a bit
petty. He was sore that he hadn't been the one to name him.

``I am,'' said Lois. ``Superman's flown the coop. He's said he's not
going to help out with the Clockwork Bomber.''

Clark turned to look at her. ``Why not?''

Lois shrugged. ``I don't know. I guess Superman wasn't sure whether it
would affect him or not.''

``Makes sense,'' said Clark.

Lois raised an eyebrow.

``I mean, let's say that you walked down a dark alley and got shot, only
to find out that the bullet didn't do much more than tear up your
blouse,'' said Clark. ``You might try shooting yourself again to see
whether you really were bulletproof, but maybe you'd be too scared that
you'd just end up with a gunshot wound. And you certainly wouldn't go
drink some poison, because maybe it would kill you.''

``I understand that,'' said Lois. ``Even if I buy that maybe Superman
doesn't know the full extent of how he's protected, he's still supposed
to be a hero. It doesn't take a whole lot of courage to walk up to a guy
with a gun when you know that his gun can't hurt you. Superman says he
wants to be a symbol and then runs away the first time he might get
hurt? That's what I don't get.''

``I guess,'' said Clark. He frowned. ``With Superman's powers, isn't it
better for him to stay alive and saving people instead of risking death?
I mean, how many people does he save in a week?''

Lois shook her head, and pulled a cigarette out. ``Clark, you're not
thinking in the long term. Superman might think that there's some risk
of dying, right? And he's got a general stance that he doesn't negotiate
with criminals, for the obvious reasons. But let's assume that this
bomber's got huge amounts of money, no morals, and an honest desire to
kill Superman, all of which I think are probably true. If Superman's
going to stay away from the chemical end of things out of a sense of
self‐preservation, then assuming Superman still intends to operate
within Metropolis that means that the bomber is just going to resort to
tricks. He's going to\ldots{} I don't know, cause a train derailment and
vent pesticides over the area. Superman shows up thinking it's a
legitimate threat, and then bam~--- poison right in his face.''

``Because Superman can't figure out whether or not there's going to be a
trap,'' said Clark slowly.

Clark Kent wasn't as dumb as he looked. It had taken Lois a long time to
figure him out, but she was pretty sure that she knew what games he
played. Clark Kent wanted to be underestimated, because it would make it
easier for him to exceed expectations. People clapped with delight when
Clever Hans had done math, not because the math was impressive, but
because it was impressive for a horse. It was the same way with Clark.
You saw this four‐eyed Midwestern guy in the middle of Metropolis,
looking for all the world like he'd taken a wrong turn leaving the farm,
and then when he actually put out a competent story you couldn't help
but feel like he'd done something amazing~--- like he was a horse that
could do math. But the thing was, if you were actually good at math you
wouldn't need anyone to think that you were a horse. There was more to
Clark than met the eye, but once you'd lived and worked with him for
long enough and recalibrated your expectations of him, Clark was simply
below average in every way that really mattered to Lois. He typed with
two fingers for Christ's sake.

``Superman's got a problem either way,'' said Lois. ``That problem is
that someone with means, motive, and intellect is trying to kill him. If
he doesn't deal with the bombs, it's going to be something else,
something that he won't see coming, I'm pretty sure of that. Making a
deal isn't ideal, but it would at least help for him to actually be the
symbol he talks about being.''

Clark looked to the ceiling, which was quickly becoming a universal sign
that powerful ears might be listening. ``Are we having this conversation
for his benefit?'' asked Clark.

Lois shrugged, which meant yes. She knew that Superman could hear her.
It would be better for Metropolis to not have a war between Superman and
whoever was behind the bombings~--- and she had a few ideas of who that
might be. She was about to add to her argument when Perry's door slammed
open.

``There's been another bombing,'' he shouted.

``But it's too early,'' said Lois. ``Last time there was six hours
between bombs.''

``Either the Clockwork Bomber screwed the pooch, or the schedule's been
stepped up,'' said Perry.

``Let's go,'' said Lois as she turned towards Clark, but Clark was
already gone.

\begin{center}\rule{0.5\linewidth}{0.5pt}\end{center}

Sal Maroni was a Superman spotter, which really just meant that he sat
on a rooftop with a notebook and drank beers while looking out over the
skyline. He listened to the radio, usually some kind of music, and
smoked like a chimney. Spotting didn't pay all that well, but there
wasn't an easier job in the entire city. Sal had worked as a security
guard once, and this was just like that except there wasn't ever the
slightest amount of danger. In addition to the radio, the smokes, and
the beer, he had a comfortable chair he'd pulled up from his apartment
on the fifth floor and a parasol he'd bought at a flea market to block
out the worst of the sun. On an average day he'd see Superman half a
dozen times, and he would faithfully write down his best guess of
Superman's location, speed, and direction of travel. On a few occasions
Sal had been tempted to just take a nap and then make things up, but
he'd been told that his observations would be checked against what the
other spotters put down. He could see a few of them on other rooftops.

He'd heard the sirens earlier, and WGBS has switched from \emph{The
Adventures of Lolly Lemon} to reporting on the return of the Clockwork
Bomber. It was about two hours after that when Superman rose up from
near The Daily Planet Building, moving so fast that Sal might have
missed it if he hadn't been paying attention. In his notebook, Sal wrote
down the details, making some best guesses. There was a man named Lonnie
who sat at Grecco's Cafe. He took in the notes from the spotters once
night fell, and had taught them how to make the most accurate
estimations of speed, distance, and direction.

Sal enjoyed being a spotter. It was boring, most of the time, but boring
was the same as relaxing if you looked at it the right way. Another perk
of the job was getting to see the news in the making. Sal had seen
Superman go in for a slow landing on top of Daily Planet Building, and
then the next day he'd read the interview in the paper. It was nice, to
be able to see Superman flying and connect the dots later on. Sal would
read the newspaper and be able to make sense of what his notes actually
meant. More often than not, the crimes he stopped were small or private,
but sometimes something big would happen in Metropolis, and Sal would
get a glimpse of it.

When the radio started talking about bombs, Sal cracked open another
beer and settled in. Today would be a busy day for spotting.

\begin{center}\rule{0.5\linewidth}{0.5pt}\end{center}

Superman responded to the second bomb, and Lex felt a sense of relief.
There was no way to know whether it had been a bluff or simple
indecision, or maybe even poor information, but for whatever reason
Superman had decided to stick his neck out. Lex would have to arrange
another interview with Lois Lane in order to find out what Superman had
really said to her, but it would have to wait. That Superman hadn't
tried to make a deal with the bomber was not wholly surprising.

The selection of attacks to try had taken careful consideration.
Anything that caused a death throes had to be avoided, and Lex put a
preference towards those agents which would cause weakness or paralysis
in humans. There was no way of knowing whether Kryptonian biology was
similar, of course. Lex had considered the possibility that in
attempting to destroy Superman he might unintentionally cause the
disaster he wished to avert, but Lex was certainly not the only player
in this game, and their plots were far more dangerous than his. All the
more reason to take minor risks to kill Superman, when the other players
sometimes seemed to be doing nothing more than trying to piss him off.

There were forty‐eight bombs, spread out over four days, one every two
hours.

After the third bomb had gone off, he'd sent all his employees besides
Mercy home for the day and sequestered himself in his office. He had
adequate food and water, a set of fresh suits hanging in his closet, and
a private bathroom. It was more or less everything that he needed.
During this time of crisis, Lex would play things safely, and do nothing
too terribly out of character. He would offer a reward for information
leading to the bomber, he would offer to help the police in any way that
he could, and he would listen to the reports as they came in. The facts
could be collected afterwards, when the whole ordeal was over, but Lex
didn't think that the man he was pretending at being would apply harsh
scrutiny during a time of crisis.

There would be immense scrutiny. If the bombs simply stopped, the police
would go on the hunt. Lex made the call that would tie up the loose ends
and divert attention away from him. He was extremely skeptical that a
path could be drawn back to him, but Lex Luthor was cautious, and so a
false trail had been laid instead.

\begin{center}\rule{0.5\linewidth}{0.5pt}\end{center}

Officers Milheiser and Kennedy walked up the stairs, sweating in the
summer heat. They'd been working back to back shifts ever since the day
before when the bombings had started up again, as had most of the police
and firefighters in the greater Metropolis area. The mayor had briefly
talked about instituting martial law, but no one was keen on that. The
compromise was double shifts. The elevator in the building was out, and
it was just their luck that the apartment was on the tenth floor. It was
more or less how the last few days had been going for them.

``Any reason the captain wants us chasing this down?'' asked Milheiser.

``He said an anonymous tip is more trustworthy,'' said Kennedy.

Milheiser nearly stopped. ``How does that figure?''

``Well, there's a big reward out for information, right?'' asked
Kennedy. ``More tips have been flooding in than we could ever take a
look at, because there's no penalty for making stuff up and maybe if you
get lucky you get a little piece of the pie. So we got people sending us
all sorts of crap, gossip about their neighbors, reports about people
that they just don't like, paranoid fantasies and all that. Ten thousand
dollars is in the pot right now thanks to Luthor, and that's enough to
attract all kinds of crazies.''

``So the captain thinks that an anonymous tip is more trustworthy,
because no one stands to gain from it?'' asked Milheiser.

``You got it,'' said Kennedy with a strained smile. The heat was getting
to him.

``And the captain didn't stop to think maybe someone else would figure
that and send the pair of us to a building with no working elevator so
we'd have to sweat our asses off climbing to the top?'' asked Milheiser.

Kennedy had no response to that. He might have said that no one would do
that in a time of crisis, but he knew Metropolis well enough to know
that wasn't the case. He'd seen enough rioting and looting to come to
the conclusion that people were bastards.

When they got to the tenth floor, they knocked on the door, and found
that it swung in to the touch. Kennedy and Milheiser shared a glance and
drew their revolvers. It occurred to both of them that perhaps the
Clockwork Bomber had lured them there, just to make a point, but they
entered anyway.

In the center of the apartment, a young man was hanging from the rafters
by his belt. He'd been dead for hours, and the smell was utterly
offensive. Milheiser rushed to the bathroom to throw up, while Kennedy
made sure the place was cleared. It was a pretty cut and dry suicide,
with a kicked out chair beneath the young man, but Kennedy went through
the motions anyway. He stood the chair up and made sure that the hanged
man would actually have been able to stand on it, since he'd heard that
sometimes people would stage a murder to look like a suicide but forget
the details. He was vaguely disappointed when the chair was the right
height.

Kennedy had moved on to a small workshop area by the time Milheiser
walked out of the bathroom, wiping his mouth with the back of his
sleeve.

``Looks like our guy was a tinker, at least,'' said Kennedy. He leafed
through a set of schematics, pulling some out from the bits of
electrical wire and springs, trying to make heads or tails of it. There
were copious notes and detailed drawings, but it didn't snap into focus
until Milheiser unearthed a book titled ``The Manufacture of
Explosives''.

``It's really him,'' said Milheiser with a shake of his head as Kennedy
began laying out the papers. The body was in the other room, and would
have to be dealt with, but neither of them relished the thought of going
up and down the stairs again, which they'd surely have to do a few times
before the day was out.

``Let's call it in,'' said Kennedy. ``Looks like there's an address
here, might be the place where the bombs were made.''

\begin{center}\rule{0.5\linewidth}{0.5pt}\end{center}

Lex Luthor was a people person. People told him their problems, and he
found solutions. It had been that way ever since his childhood on the
streets of Suicide Slums, the worst neighborhood that Metropolis had to
offer. So far as anyone besides Mercy knew, Lex had gone legitimate. The
vast majority of his criminal enterprises were run through various
intermediaries, who knew him only by codewords over the phone. Since
Superman's arrival, Lex had let much of that go to rot. It was easy
enough to make money in perfectly legitimate ways if you had a mind as
keen as Lex's. Instead, he used his network of slush funds and discreet
contacts in order to facilitate his private war against Superman.

Harry Kramer had been a piece of serendipity. He'd been an expert in
explosives by the age of sixteen, thanks in part to a father who had
done demolition work at a mine upstate before losing his life to a
faulty detonator. Kramer liked to blow things up, and got involved in
professional fireworks before he was discharged after an incident that
lost his boss the use of two fingers. It was when Kramer got hired on to
do a bank job that he came to the attention of Lex. The job had been an
abject failure, though it was through no fault of the explosives, which
had worked perfectly. Kramer had been willing to hire himself out again,
but he was difficult to work with, and there wasn't much call for an
explosives expert in the criminal underworld of Metropolis. Harry had
been working as a grocery bagger until Lex needed his expertise. Lex
could design the bombs easily enough, but wasn't willing to put himself
in a position where he could be seen making or delivering them. He'd
given Harry a new apartment and a workshop, along with a large amount of
freedom.

A careful examination of the evidence would reveal a hidden hand behind
the Clockwork Bomber. Harry Kramer had received a large amount of money
from an uncle down in Georgia, and if that thread were tracked down the
sham would be revealed, and point back to Metropolis. This was part of
Lex's design.

\begin{center}\rule{0.5\linewidth}{0.5pt}\end{center}

There were forty‐eight bombs in total. Thirteen were found by Superman
prior to detonation, and he managed the evacuation and the removal or
controlled detonation of the bomb. Any hesitance he'd displayed in front
of Lois was completely gone, and over the course of the extended
bombing, the enactment of martial law, and everything else, he'd proven
himself to be a complete hero in every way. When he wasn't helping with
rescue efforts or stopping the bombs, he could regularly be seen
watching over the city.

``You look like shit, Clark,'' said Lois when they got back to work.
Most of the businesses had temporarily closed after the second day;
\emph{The Daily Planet} had closed on the fourth, when some people were
saying that the bombs would keep going off forever.

``I didn't get much sleep,'' he replied with a yawn. ``I kept worrying
that my apartment was going to explode out from under me and I'd die
choking.''

Lois had escaped the mustard gas with only a small blister on her left
hand and a light cough. She considered herself lucky. No one had died
from the mustard gas, but it was one of the tamest things that had come
out of the bombs. She'd spent the days off from work pacing back and
forth, sleeping heavily, and using her home phone to try to get a break
in the story, though the phones were nearly as useless as the radio.

``Who do you think did it?'' asked Lois.

``They caught him Lois,'' said Clark.

``One man, working alone, and you believe that?'' she asked.

``He came into a lot of money,'' said Clark. ``He was smart and
deranged. Everyone who knew him thought that it made sense after the
fact, and some of them had even reported him to the police. If he hadn't
switched apartments they'd have got him.''

``Sure,'' said Lois. ``And if you buy that I've got a bridge to sell
you. The police are investigating it all. They've found a few of the
guys that planted the bombs, and a couple of places that delivered the
materials used for construction. I don't know anything about making
bombs, but I can believe that a single person might be able to make as
many as he did, if given enough time. But add on all the logistics on
top of that, all the scoping out of locations and arrangements for
delivery? No, no way he was acting alone. I'm not saying that we can
solve it from our desks, but think about it Clark.'' She looked at him.
``Someone intelligent, resourceful, wealthy, with deep criminal
connections and a strong desire to see Superman dead. There's one guy
head and shoulders above everyone else on that list.''

``William Calhoun,'' said Clark.

``The last crime boss of Metropolis,'' said Lois with a nod. ``If you
could follow the trails well enough, I have no doubt that they'd lead
back to him.''

\begin{center}\rule{0.5\linewidth}{0.5pt}\end{center}

William Calhoun was fifty‐eight years old, which was ancient for a crime
boss. When Superman had come along, organized crime had to either
toughen up or flee the city, and Willie seemed to be one of the only
ones willing to toughen up. Boss Moxie had continued on like nothing was
different, and now he was sitting in Sing‐Sing. Johnny Stitches and Toby
Whale had left for Gotham City, while Angelo Baretti simply evaporated
like mist. And that left Willie as a big fish in the biggest pond in the
world, with the only problem being that the pond was being shot full of
holes by a nut with a tommy gun. Willie had been working on the metaphor
for a while, and it still wasn't quite right.

Willie was looking over the books in his lead‐lined office, and trying
to figure out a way to get people to pay their bookies when there was a
commotion downstairs in the bar. Not really having any enthusiasm for
the drudgery of what he'd been looking at, Calhoun wandered down the
stairs. His two guards followed.

Superman stood in the middle of the Elephant Club, with everyone around
him giving him a wide berth. Superman was staring at Willie from the
moment he started walking down the stairs, and maybe even before that.
He could see through walls, the prick.

``Hello William,'' said Superman. His voice was calm and gentle as a
breeze.

Willie put on his most casual demeanor. He kept telling the boys that
they had nothing to be worried about when it came to Superman. Sure,
Superman would foil crimes and get them locked up, but he never hurt
anyone, not even in the process of arresting them. Micky Fingers had
stabbed Superman in the eyes and Superman had just stood there like a
statue. But it was hard not to think about what the man could do.

``You're trespassing,'' said Willie. He tried to keep his voice light.

``This establishment is open to the public,'' said Superman.

``Well you're blacklisted then,'' said Willie. ``I'll have to put up a
sign that says `No Supermen'.'' This brought a round of nervous chuckles
from the crowd.

``I'll be leaving soon,'' said Superman. ``I just wanted to let you know
that I'm watching you. You've been careful, but not careful enough.
There's nowhere that you can hide from me. There's nothing that I won't
do to bring you to justice.''

``Oh really?'' asked Willie, striding towards the Superman with a
confidence that he almost felt. ``Anything? Then I've got a deal for
you. Tear off one of my arms, and I'd be in so much pain I'd give you a
full confession for whatever it is you think I did. Go on, do it.''

Supeman didn't move. ``I'm not a monster,'' he said evenly.

``No, you're a monster alright, you just don't want people thinking that
you are. You don't want to get your hands dirty,'' said Willie. ``I've
heard from a bunch of guys that you're nothing but a big fat pussy, and
standing here looking at you I can see it's ab‐so‐lutely true.'' Willie
could feel his blood pumping in his ears. Months of frustrations at the
hands of Superman were coming to a head. Willie had tried to stay low,
but his organization could only stay starved of cashflow for so long.
Willie'd been funding lawsuits against Superman, false accusations and
red tape, along with whatever else he could think of. Some of the guys
talked about killing Superman, but that was a fool's errand~--- the
bombs had proven that. Willie just wanted him to leave, to go bother
Gotham City or Blüdhaven.

``No one likes you,'' said Willie. ``No one wants you here. Get that
through your thick alien skull. You think the government doesn't have
plans to kill you? Hell, you think that they haven't tried?'' That was
Willie's best guess as to who was behind the bombings after talking it
out with Luthor. ``You do whatever the fuck you feel like doing and
expect us to praise you. Well I got news for you, it's not going to
happen. Eventually someone is going to find a way to kill you, and I'll
be first in line to piss on your grave.'' Willie spat at Superman, and
watched as the glob of phlegm hit him in the cheek. Superman could have
dodged it, probably could have reached across the room and grabbed a mug
to catch it in, but he'd just let it hit him.

``I just wanted to let you know that I know,'' said Superman. ``In
everything that you do, be aware that I'm watching you. When you're
arrested, it will be completely by the books. When you're convicted to
life in prison, I hope that they're able to rehabilitate you.'' Superman
didn't touch the spit on his cheek. He just turned and walked out the
door. The bar exploded into conversation, and Willie went back upstairs
to think about what it was that Superman had actually known.

\begin{center}\rule{0.5\linewidth}{0.5pt}\end{center}

Forty‐eight bombs, and not so much as a cough or a sneeze from Superman.

In his lead‐lined study was a large map of Metropolis, five feet to a
side, which took up a place of prominence on one wall. Stuck into this
map were pins with small flags on them, each of them a recorded Superman
sighting. The information had been collected from various sources, from
newspaper reports to eyewitnesses. Lex had dozens of people around the
city who worked as Superman watchers, and they would sit atop tall
buildings and make notes of the lone figure flying through the sky
whenever they could.

Lex was looking for patterns. Which directions did Superman come from?
Which directions did he go? What crimes did he tend to respond to, and
which did he ignore? What were his hours of activity? Lex had long
hypothesized that Superman had a base of operations somewhere, likely
the same place that his spaceship was stashed. Finding it would be a
godsend. The arrival of the Clockwork Bomber had provided a wealth of
data. Lex sat down to do some math.

Each arrival and departure could be defined by a vector, and these were
represented on the map by small lines drawn moving away from the pins in
different colors. Lex compared the times and directions, and began by
throwing out all of those vectors with known destinations. When he was
done, he was left with one‐thousand eight‐hundred sixty‐one vectors to
manipulate. He began mapping them in different ways, to see whether
Superman favored one side of the city over the other, or whether he
consistently came into the city from one direction. He found a slight
eastward inclination to arrivals and westward inclination to departures,
though given that the entire United States was to the west of
Metropolis, that might have just been because Superman often responded
to large‐scale crises outside of the city. Following that middling
success, Lex did some complicated math to make another map that showed
where vectors converged. He eventually circled ten square blocks in the
center of Metropolis. It was there that Superman kept going towards,
though that might have simply been because Superman spent his time
waiting in the center of the city.

It was close to a futile exercise. The data was bad. It was cobbled
together from too many sources, and too many of those sources were
unreliable. There were certainly data points that were lies told by
people who wished they had more interesting lives. Lex couldn't properly
trust the data, and so couldn't properly trust the conclusions that he
drew from the data. Worse, Superman was aware that people were watching
him. Still, it was better to grasp at straws than to simply give up.

Lex began segmenting the vectors into blocks of time. Even with
unreliable data, it was well‐established that Superman was less active
during working hours, and so perhaps it might be that paring down the
data would help to reveal something more. The big problem there was that
there was that the data became thinner, and even less reliable.
Nevertheless, Lex continued on. There were other plates spinning that
wouldn't need to be touched for a while, and in the meantime Lex could
pretend that he was getting somewhere. The math was somewhere between
difficult and tedious, and not at all pleasant.

When he was done, Lex frowned at the result. He circled four city blocks
on the map, slightly away from the direct center of downtown. He turned
to look at Mercy, who sat in a padded chair drinking tea and reading a
book.

``Mercy darling, my brain is failing me,'' said Lex.

``Sorry to hear that sir,'' replied Mercy, not bothering to look up.

``I've been staring too closely at this for far too long,'' said Lex.
``Eight o'clock in the morning to five o'clock at night. I can feel
something refusing to spring to mind there, something that's not quite
clicking.''

``It's standard working hours for most of downtown,'' said Mercy.

Lex turned back to the map. He stared at it. There was something he was
missing, some piece of the puzzle. Nine to five, but not on weekends. It
was fuzzy, painfully fuzzy, but the data was clear and the correlations
were real. Lex was on the verge of a breakthrough, if only he could~---

``Son of a bitch,'' said Lex softly.

\begin{center}\rule{0.5\linewidth}{0.5pt}\end{center}

\emph{Author's Note: As always, thanks for the reviews / favorites /
follows, which are always a pleasure to see. Thanks to those people
who've pointed out typos; you're making the story better for people who
read it after you.}


%%%%%%%%%%%%%%%%%%% NEW CHAPTER %%%%%%%%%%%%%%%%%%%%%%%%

\hypertarget{private-wars}{%
\chapter{Private Wars}\label{private-wars}}

\emph{Superman has a day job.}

It was just a joke, the kind of thing that the brain coughs up when it's
trying to match a pattern. Kant said that humor was expectation strained
until it suddenly dissolved into nothingness. Lex had been making maps
and doing complex math for weeks on end, and if that was a joke, it made
sense that the punchline was simply that Superman walked the streets of
Metropolis as a human. The very thought of the most powerful entity in
the world choosing to work a nine to five job in downtown Metropolis
should have caused any right thinking person to burst into laughter. But
as Lex turned the idea over in his head, the humor faded. And once the
idea had presented itself, it refused to leave.

``Son of a bitch.''

People liked to believe that brilliance was a matter of sudden insight.
Isaac Newton was sitting beneath an apple tree and just happened to be
struck on the head with an apple, which led to him developing the theory
of universal gravitation. Archimedes sat in the bathtub and realized
that an object displaces its equal volume of water. Friedrich August
Kekule realized the structure of the benzene molecule after having a
dream of a snake eating its own tail. These were the stories that people
liked to tell, because it made thinking seem like magic, and no matter
that the stories weren't true. Even where there was a grain of truth
behind the story of the insight, there were hundreds of hours of thought
and study before it, and another hundred hours of proving it afterward.
Another thing that was never mentioned was how often a startling insight
proved to be rubbish.

Some years ago, he'd spent days trying to make what he called a
battlesuit a practical reality. It was going to be a callback to the
knights in shining armor, creating a solitary soldier encased in
impenetrable armor and capable of advancing on enemy lines with
impunity, mounted machine guns firing away the whole time while a diesel
engine belched smoke. He'd drawn up schematics and eventually began
stripping parts away, replacing those things that thrilled the
imagination with those that would work practically and reliably. The
steel legs were replaced with treads. The arms were removed in favor of
a larger cockpit with buttons and levers. The center of gravity was
lowered, until the cockpit sat between or just on top of the treads. He
still remembered the feeling of looking down at his design and realizing
he'd done nothing more than make a better tank. LexCorp now owned two
factories that made them, building up a stockpile to sell to the
European powers when the next inevitable war broke out. Still, the whole
project had been borne out of a vision he'd had, of diesel powered
mechanized armor striding across the battlefields of the next war. The
fact that he'd spent so much time pursuing that vision was a source of
embarrassment. It had been a valuable lesson in critically examining
those ideas that came to him suddenly and struck him on some emotional
level.

What Lex needed was someone who would ask some pointed questions and act
as a foil to his enthusiasm, a devil's advocate. He made a quick
calculation of the risks of speaking out loud, and found them
acceptable. If he was right, Superman engaged in surveillance far less
than he had supposed, and if he was wrong, there was no harm in it.
There was only one person that he trusted enough to discuss the idea
with, and conveniently she was sitting in the same room as him, drinking
tea and reading a book.

``Mercy, your attention for a moment?'' asked Lex. He used French, a
language that they both shared, as a weak form of security.

``Of course sir,'' she said as she put down her book with a finger
resting between the pages.

``Convince me that Superman doesn't have a secret identity,'' he said.

``A secret identity?'' she asked, as though she had never heard of the
concept. On the long list of wonderful things about Mercy Graves was her
ability to effortlessly take the role of the ignorant in their dialogues
when it was required of her. Lex found being forced to define himself
quite helpful.

``Like a spy,'' said Lex. ``Or a philanderer, I suppose. Superman leads
a double life, and in the second one he doesn't wear the costume.''

Mercy took a sip from her tea. ``And what does he do in this second
life?'' she asked.

``I don't know,'' said Lex. ``I'd have to guess at motivations, and if
he has an alter ego I know less of his psychology then I had thought.''
Lex ran a hand across his hairless head. ``What does a man need? Food,
water, sleep, shelter. Superman has never displayed any need for those.
Perhaps he eats and drinks in secret, but playing at being human would
be the least efficient way to go about satisfying those needs. Sex or
family\ldots{} it's possible. He'd have no trouble convincing women to
sleep with him or bear his children as his costumed self though. So it
must be something more ethereal, something that he can't get as
Superman. True, honest friendship untainted by his brute strength and
speed, not to mention his celebrity? Or perhaps just the thrill of
deception? There's some historical precedent for it. Tsar Peter of
Russia used to dress up like a workman and go among his people.''

``Peter the Great was six foot eight in a time when the people of Russia
were starving,'' said Mercy. ``It was because he was tsar that no one
dared broach the subject, but surely they knew the man by his height
alone. It's the same with Superman. They'd recognize him.''

``Perhaps,'' said Lex. ``But when people look at Superman, what are they
really seeing? They see the emblem on his chest, the bright colors of
his costume, and brilliant smile and the curl of hair that hangs down
just so. If you saw Superman in the street wearing a suit and tie, would
you recognize him in that new context?''

``Most likely,'' replied Mercy.

``Photographs of him are surprisingly rare,'' said Lex. ``When people
think of Superman, they don't think of him as he really exists, they
think of Norman Rockwell's painting of him on the cover of \emph{The
Saturday Evening Post}. Superman has posed for a single photograph, the
one that showed him and Miss Lane, and all the rest are of the man doing
some impossible thing, lifting cars above his head or flying through the
air, and the focus is seldom on his face. He keeps his interactions with
people short. The photographs from the courthouse, at least the ones
I've seen published, are always from a distance, the better to take in
his full appearance. They emphasize the muscles and the costume, not the
face. And they're published by the newspapers in terrible quality.
Perhaps putting him in a suit and tie wouldn't be enough, but if you
added a hat, an overcoat to hide his bulk\ldots'' Lex scratched his
chin. ``A change of mannerisms, a slouch, makeup, prosthetics, wigs, a
false moustache or beard, glasses, speaking in a low or high voice, or a
false accent, well, there are a large number of ways he could disguise
himself and go unnoticed. Charlie Chaplin once lost a
look‐alike‐contest, or at least that's what he told me. Very rarely do
people distinguish faces by their component parts, they look at
demeanor, gait, gestures, that sort of thing. They think in
caricatures.''

``You're getting dangerously close to pure ex post facto rationalization
for something you want to believe is true,'' said Mercy.

``I am,'' said Lex after a moment. Mercy could cut straight to the heart
of matters like few other people. ``I find it attractive because it
would reveal something hitherto unknown about the man. I've run into
failure after failure in trying to understand Superman, and this is the
first theory that might actually lead somewhere. Even if the probability
is low, we have to pursue it. Can we at least agree that Superman might
stand to gain something from having an alter ego and that he might be
able to pull off a long running disguise?''

``I can accept that perhaps he would be able to walk into a deli and
purchase something to eat without arousing any suspicion,'' said Mercy.
``But you're suggesting a sustained deception.''

Lex nodded. ``The quiet period, when he's less active, lines up too
nicely with working hours, and not just because there are fewer crimes
around that time. His movements point to a specific location that he
keeps going to or coming from. That data is fuzzy enough that it
suggests to me an inexpert attempt to hide the pattern, or perhaps just
an attempt by someone who wasn't clear on what methods could be used to
reveal the truth. Superman doesn't strike me as a mathematician.''

``They would still know,'' said Mercy. ``If they ever saw Superman in
the flesh, they would see his alter ego for what it is.''

``Perhaps not,'' replied Lex. ``No one is looking around for Superman in
disguise, because the concept is nearly unthinkable to them. No one
believes that they would work a day job if they had his powers. They
would become filthy rich and live a life of celebrity and hedonism.
Perhaps it occurs to his coworkers that the man they work with looks
like Superman, but they wouldn't immediately make the leap to thinking
that he actually \emph{was} Superman. Maybe they would make jokes, but
he would deflect them, or play along. Maybe he even has a few people in
his confidence. Think about it. Superman doesn't wear a mask. If he wore
one, people would wonder what was behind it. Many people have thought
that Superman was hiding something, but they think it's his spaceship,
or invasion plans, when all along it's just so\ldots{} mundane.''

``You've made up your mind,'' said Mercy. It wasn't a question, and
wasn't said with any trace of disapproval. She was simply informing him
of what she had observed, and as usual, she was right.

``Thank you Mercy,'' said Lex. In French this was rendered as ``Merci
Mercy'', a minor bit of wordplay that nevertheless brought a rare smile
to Mercy's lips. ``I believe that this lead is worth the resources
required to pursue it. Even if the odds of it being true are somewhat
low. The only question is the methodology.'' He smiled. ``Perhaps an
investment in the arts.''

``If you find him, will you expose him?'' asked Mercy. She asked without
any real curiosity or concern, and Lex was certain it was only intended
him to get him thinking about the answer before he walked too far down
the path. Mercy could convey quite a bit of information with a flat
affect.

``Lord no,'' Lex replied.

Lex Luthor saw antagonizing Superman in and of itself as having no
value, or more likely negative value. If Lex Luthor and Superman were
the only actors on the stage, Lex might even have refrained from using
the bombs, and instead relied solely on those methods that revealed no
foul play at all. It would have been more difficult, but on balance
probably worth it. Unfortunately, the stage was crowded with actors, and
some of them seemed to find great sport in trying to take Superman down
a peg. In Lex Luthor's public role as Superman's champion, he'd done
everything from funding legal efforts to defend Superman to penning
articles in favor of Superman's ridiculous moral stances. In the context
of the other actors, antagonism became a more acceptable risk only
because it would blend into the background.

Superman's presumed secret identity was a vector of attack, but not one
that Lex had any intention of using against him. The people who thought
they had something to gain from disrupting Superman's emotional state
were fools.

\begin{center}\rule{0.5\linewidth}{0.5pt}\end{center}

``The judge is dropping the case,'' said Clark as he laid his phone in
its cradle. He was visibly upset, which was rare for him. He pouted in a
way that might have looked adorable on a small child but just didn't fit
a grown man.

``There wasn't enough evidence,'' said Lois. ``It shouldn't have even
made it to the judge in the first place.''

``Calhoun is guilty,'' said Clark. ``I know he is.''

``You think he is,'' corrected Lois. ``And even if he's got to be guilty
of something, there's no guarantee that he's actually guilty of
manipulating Kramer. I know this story is near and dear to your heart,
but maybe it's time to let it go.''

``It's an injustice,'' Clark insisted.

``I should introduce you to my friend Vicki Vale,'' said Lois. ``She
works for The Gotham Gazette and I'm sure she could regale you with some
stories about real injustice. Actually, you might like her, I think
she's your type.''

After she'd said it she realized that it sounded like a bit of a low
blow instead of an olive branch. Lois knew Clark still had a crush on
her, and to him it might have sounded like she was making fun of him and
saying that he had a thing for female reporters. But Vicki Vale really
would be his type, and she really could set them up the next time that
Vicki came to town. Lois was never actively cruel to Clark, she just
liked to push his buttons. She liked to see him get all uncomfortable
when she swore around him. She would watch his face while she sucked
back a cigarette or took a glass of whiskey at her desk, neither of
which Clark approved of. These were small, harmless pleasures. Clark was
like a puppy dog in a lot of ways. She didn't want to hurt him.

``I didn't mean it like that,'' said Lois.

``Mean it like what?'' asked Clark. Clark had always seemed like the
kind of guy who would blush at the drop of a hat, but Lois hadn't seen
it once. He would get visibly embarrassed, but even after all this time
she kept looking for a hint of red in his cheeks or ears.
Disappointingly, it was never there.

``Nothing,'' said Lois. ``I was just thinking that she would like you.''

Clark gave her one of his familiar grins. Lois worried she'd gone too
far in rolling back what she'd said, but turned back to her typewriter.
She wasn't in charge of Clark Kent. And if Clark got his feelings hurt
because he misinterpreted something she said, well, it wasn't the end of
the world.

\begin{center}\rule{0.5\linewidth}{0.5pt}\end{center}

William Calhoun should have felt relieved that the judge had dropped the
case, but instead he just felt angry. He'd been accused of being in
cahoots with that bomber on charges so paper thin it would almost be
laughable. Willie had spent five of his fifty‐eight years in prison
though, and he never laughed about time in the clink. He'd sat down with
his lawyer and looked through the evidence himself, and could admit that
there was an implication there, but it wasn't even firm enough that he
could say he'd been framed. Even if it was just coincidence, it pissed
Willie off to get called out on something he'd had no part in,
especially considering all the things that he was actually guilty of.

It was Superman's fault. Superman had barged into Willie's bar and
announced as much, and it must have been Superman who whispered in the
right ears to get the case moving forward. Superman was a prick of the
highest order. Worse, people listened to him.

Luckily, Willie's schemes were paying off. The barrage of lawsuits had
mostly been a nuisance to keep Superman tied up in court, but some of
them had been taken further than he'd expected. Three decisions were due
to come down from the Supreme Court, and if Preethi v. New York went the
right way, Superman would be bound by all sorts of rules. Superman had
already agreed to abide by the rulings no matter what they were, and so
far the man had never broken his word. It made him predictable, and
Willie hoped he could use that against him.

One of Willie's early tactics had been to have people accuse Superman of
everything under the sun, to try to smear the alien's name if not
actually get him in trouble with the police. Willie had paid a young
girl to claim rape, and a few other people as witnesses. No one had
believed it though, and the girl had crumbled after a confrontation with
Superman on the steps of the courthouse where he'd been kind, courteous,
and forgiving. After that it was tough to find people to make false
allegations, and though Willie had searched, he'd never found someone
with a real criminal complaint. It occurred to him that Superman was
becoming so universally loved that even if Superman did do something
truly evil most people wouldn't believe it.

Slander and libel weren't working, and Willie was being bled dry.
Business had been brought to a near halt. There had to be a way to turn
the tide against Superman, and Willie was willing to do anything to
figure out what.

\begin{center}\rule{0.5\linewidth}{0.5pt}\end{center}

Hershel Whitman had become governor of New York when Franklin Delano
Roosevelt had won the Presidency in '32, and he was in it for the long
haul. The state of New York was most famous for Metropolis, its crown
jewel, and nearly half of the people in the state hailed from that city
or the surrounding greater metropolitan area. Ever since Superman had
shown up from out of deep space, politicians had been clamoring to be
seen as associating with Superman, and Whitman was certainly no
exception.

From a politician's perspective, Superman was perfect. He didn't upset
the apple cart, he didn't hold public opinions, he'd had nothing but
positive effects on the rate of crime in Metropolis. As the incumbent,
it would be nearly impossible for Whitman to lose his next election if
the people were happy, nevermind that he hadn't had all that much to do
with Superman. Most of the hard work of governance was in building roads
and bridges, passing funding measures, and wrestling with the other
parts of state government to hammer out laws. The vast majority of
people didn't place their votes because of anything sensible like the
actual work that was done, they would see Superman flying through the
air and think ``governor Whitman must be doing something right''. The
bombings had been a black mark, but the city was recovering better than
anyone could have hoped for, and thankfully the bomber had hung himself
and spared everyone the ordeal of a lengthy trial. Whitman hated the
inevitable appeal for clemency from death row inmates.

Whitman would have taken a meeting with Lex Luthor no matter what it was
about. The man was a billionaire after all. When Luthor had asked to
discuss a public‐private partnership of the arts, Whitman couldn't help
but feel that someone up there was looking out for him. Whitman was a
strong supporter of the New Deal policies, and there could be no
downside to adding in a billionaire's funds.

``There's much discussion about you, you know,'' said Whitman with a
smile. Prohibition had been brought to an end, thankfully, which meant
that a man could enjoy a martini on his veranda without having to worry
about scandal. A hot summer had made way for a cool autumn, and
Whitman's two children played in the yard.

Luthor shared the smile. ``I'm sure that tongues will wag. What do they
say, I wonder? That I came up from nowhere?''

``Things of that nature,'' said Whitman. ``I dare say there's a risk
you'll be named Metropolis's most eligible bachelor. There's a mystery
about you people quite like. You were born in Southside, if I recall
correctly, and the charitable work you've been doing there has been
admirable. Yet prior to Superman's arrival, you were known only inside
the world of business, and then more as a name than a man.''

Luthor shrugged, and looked out at the yard at the children. They'd
invented some game that involved ever more elaborate cartwheels. ``I've
never wanted fame,'' said Luthor. ``For a time I wanted money, but I
think I have enough of it to last me for a good long while. No, now is
the time for giving back. Superman has shown me that. And that's
precisely what I'm here to talk to you about.''

``You have my full attention,'' said Whitman.

``Simply put, I would like to fund the arts. I'm not an artist myself, I
can acknowledge that, but I have certain ideas that I think would help
towards increasing the beauty of our beloved city and showing off its
character. Now I'm aware that the Public Works of Art Program has run
its course, but I was just speaking with Harry Hopkins over the phone,
and he suggested that a pilot program might be just the thing. They're
getting close to putting together a second New Deal, which they hope to
include some arts in, and I think we might be finished with what I had
in mind before the bill goes through Congress. It might help grease some
votes, as it were.'' Luthor sipped at his martini. ``I would put in a
good deal of the funding of course, but I was thinking that perhaps
working jointly with the state could be mutually beneficial. That sort
of partnership isn't unheard of.''

``Of course,'' said Whitman quickly. Lex Luthor was becoming known as
quite the philanthropist, and the photo opportunities would help in an
election year. There were vague rumors about something criminal in
Luthor's past, but the man had been born in Suicide Slums and if
anything he was stronger for the narrative of reform.

``I have three in mind,'' said Luthor. ``The first is a statue, that I
think would look nice in Fort Hob's Park, though of course that's
negotiable. Not one of Superman, but something close I think, clearly
inspired by him. The idealized man, cast in bronze and standing tall, a
reminder for each of us to be the best person that we can possibly be. I
believe this is the lesson that Superman intends for us to take. It
would capture the zeitgeist, don't you think?''

``I do like the sound of that,'' replied Whitman. There would be an
unveiling ceremony, and Whitman would be standing in front of the statue
holding a pair of oversized scissors. He rather enjoyed the mental
image.

``The second is a large mural that will grace the length of Gerald
Ordway Drive, along a length of the West River between the Queensland
Bridge and Dockside,'' said Luthor. ``I have no specific vision there
beyond it showing a progression of the city from its humble origins to
the future we're striving for, perhaps something in mosaic. Metropolis
is the City of Tomorrow after all, and I think it would be nice to pay
some tribute to our roots as well as our aspirations.''

``Very doable,'' said Whitman. ``I'll need to speak to the mayor and the
city council about it, but very doable indeed.''

``Of course,'' replied Luthor. ``I'll be sure to put in a few words as
well.''

``And the third?'' asked Whitman.

``For the third, I want a photography exhibit,'' said Luthor. ``Sharp,
candid photographs of the people of the city. As I picture it, we'd hire
some photographers and park them downtown, to get a full sampling of the
lifeblood of Metropolis and the rhythm of workers coming and going. When
we're finished, we'll gather these photographs together and display them
in a gallery~--- I have just the one in mind~--- packed from wall to
wall in order to show the full breadth of humanity from the immigrant
populations to the high financiers. I believe it would be a marvelous
demonstration of both our similarities and our differences. More than
that, people who aren't normally interested in the arts might stop by to
try to find their own photo, or the photos of their friends. My
provisional title is `Faces of Metropolis'. I'd like some creative
control over that one, since the artistry will be in how we compare and
contrast the people we capture rather than the photographs themselves.''

Whitman nodded, still thinking about the political opportunities. He was
up for re‐election in November, and while there was little chance that
the projects could be completed by then, he'd be able to use this deal
with Luthor in his stump speech. He could spin it to sound like his own
idea, a melding of business and government for the improvement of the
lives of the citizens of the state. The project would surely create
jobs, but more importantly it would be a highly visible way of creating
jobs.

His children ran towards the house and poured themselves tall glasses of
lemonade before dashing back off into the yard again. June was eleven
and Robert was nine, and a father couldn't ask for better.

``I enjoy children,'' said Luthor. ``I've thought about having a few
from time to time. But more and more I find myself thinking of
Metropolis as my child. I want nothing more than to help her grow, to
protect her from harm, and to make her into the best city she can
possibly be.''

Whitman nodded. He found himself quite liking Lex Luthor.

\begin{center}\rule{0.5\linewidth}{0.5pt}\end{center}

``Calhoun just got arrested again,'' said Clark with a smile.

``What are the charges?'' asked Lois. ``Something solid this time?''

``Racketeering, murder, conspiracy to commit murder, loansharking,
illegal gambling, obstruction of justice, bribery, and tax evasion,''
said Clark.

Lois let out a low whistle. ``That's a long list. Any of them that will
stick?''

``All of them,'' said Clark with confidence.

``You're too close to the story, Clark,'' said Lois. ``And it's back
page material anyway. If Superman's involved it might be one of the
first cases that hinges on the outcome of whatever the Supreme Court is
doing, but that only bumps it up to page four or five.'' She looked him
up and down. Usually Clark wasn't so happy. The bombings had begun to
fade into the background, but Lois had found that it affected people in
different ways. She'd gone drinking in one of the clubs, and the
conversation had dropped into awkward silence when someone mentioned
that they'd had a friend who died in one of the blasts. Clark seemed
certain Calhoun was behind it, and Lois didn't think he'd get his
closure until Calhoun was in jail or dead. ``Look Clark, take your mind
off this. Justice takes time. Write up the story and then just forget
about it until the verdict comes in. Perry's not going to want to devote
too much space to it.''

``Alright,'' said Clark, but Lois didn't miss the pleased look on his
face as he pecked away at his typewriter.

\begin{center}\rule{0.5\linewidth}{0.5pt}\end{center}

A dozen photographers were sent downtown, where they snapped picture
after picture of people going to or leaving from work. They had cards to
hand out, and by and large most people were game. Pictures were taken
even of the ones that didn't seem too keen on the idea. The shots ranged
from candid to posed, with some being simple headshots and others taken
from a balcony or second story to capture everyone on the streets.
Ideally, Superman would be hiding somewhere among them. Of course, it
was possible that Superman would see the photographers and simply turn
the other way to avoid them, but Lex had been trying to work out the
alien's psychology for a while now, and felt that it was unlikely. If
Superman really did have a secret identity, it was probable that he
enjoyed being a normal human, and what could appeal to the alien more
than being simply one of many, a face in a sea of faces? Besides that,
Superman wouldn't want to be seen avoiding the cameras, because that
would be just as conspicuous.

There were too many people to photograph them all. The Emperor Building
and the Daily Planet Building were each within the four block area, and
the Emperor Building alone had 10,000 workers. Still, a good number of
people could be photographed, and if Lex was right, Superman himself
would be attracted to the photographers, no matter how ill‐advised that
would be. If the plan failed to work, there were other, more risky
plans. Private investigators could be set to work, company payrolls
could be combed through, and hard data could be examined. The trick was
to find out who Superman's secret identity was~--- if he had one~---
without tipping him off.

It was late November by the time Lex and Mercy sat together in his
lead‐lined cabin some distance from Metropolis and sorted through the
photographs.

``Dark haired white male, likely above six feet tall,'' Lex had said
when they'd first begun. ``Superman is six feet and four inches tall,
when he's actually got his feet on the ground. We can't rule out that
the identity we're looking for has a slouch, or an affected limp, but
there'd be no changing his physical size, not unless there's some power
we haven't seen yet. We can't rule out that he wears a wig in his daily
life either, so set aside all those photographs with tall blonde men as
well.''

``Yes sir,'' said Mercy. She worked with quiet efficiency, sorting
photographs into various piles with Lex. It was boring work, and quite
slow, especially as the faces and people all began to meld together. It
was in the second day of this that Mercy found a picture of Lois Lane.
When she slid it across for him to look at, Lex saw Superman standing
next to her.

``It's him,'' said Lex, and Mercy moved around to look over his
shoulder.

``Are you sure?'' asked Mercy. ``I would have put it in the pile for
later review, but I'm less immediately convinced than you are.''

``He's the perfect mockery of humanity,'' said Lex.

The man clearly didn't want to be there. Lois Lane was as feral and
energetic as ever, staring directly into the camera with a winsome
smile, but the man was looking slightly off to the side. He was tall and
large, and looked slightly disheveled. Your eyes were attracted to the
notepad he tucked into his jacket pocket, then to the glasses that were
so thick you could barely see his eyes through them. Almost immediately
you'd peg the man as an oaf. He was so unlike Superman that it had to be
him.

``Superman always holds his head high, with his jaw thrust out,'' said
Lex. ``This man spends most of his time looking down, with his chin
tucked in. It disrupts the lines of his face, makes him less noticeable.
But the nose, you can tell from the nose it's the same man. It's him.
It's Superman.'' Lex flipped over the photograph. The idea had been for
the photographers to capture essential information from their subjects
wherever possible, but from the sampling so far it was clear that not
all of them had been so diligent. In this particular case, Lex Luthor
got lucky, and a number of nascent schemes for manipulating Lois Lane
into giving up information were quickly put to rest.

\emph{Lois Lane and Clark Kent, reporters, outside Daily Planet bldg.}

\begin{center}\rule{0.5\linewidth}{0.5pt}\end{center}

\emph{Author's Note: This chapter once again grew too long, so again I'm
splitting it up into what I think works best for the story breakdown.
Ten thousand words seems a little bit long for a chapter, and that's
what I was approaching. Chapter 7 will be posted on Sunday.}


%%%%%%%%%%%%%%%%%%% NEW CHAPTER %%%%%%%%%%%%%%%%%%%%%%%%

\hypertarget{choices}{%
\chapter{Choices}\label{choices}}

From \emph{Preethi v New York} 293 U.S. 367 (1934):

The State of New York has provided such significant encouragement, both
overt and covert, that the actions of Superman must be judged to be that
of the State. {[}\ldots{]} It is this Court's considered opinion that
there would not be much use to Constitutional protections if the State
could do an end run around those protections through the use of private
parties. By engaging in the same type of work as the Metropolis police
department, and with their cooperation and approval, Superman may fairly
be described as a state actor.

\begin{center}\rule{0.5\linewidth}{0.5pt}\end{center}

From \emph{Shoe v New York} 293 U.S. 377 (1934):

Obtaining by enhanced senses any information regarding the interior of
the home that could not otherwise have been obtained without physical
intrusion into a constitutionally protected area constitutes a search.
{[}\ldots{]} In permitting the use of this evidence upon trial, we
believe prejudicial error was committed.

\begin{center}\rule{0.5\linewidth}{0.5pt}\end{center}

From \emph{The Daily Planet}, anonymous letter to the editor, December
19th, 1934:

Taken together, there can be no question that these rulings severely
curtail Superman's ability to effectively conduct law enforcement within
the United States. In the coming months, dozens if not hundreds of
appeals will be filed on the premise that Superman has engaged in
procedural error, in which the Metropolis police department and others
were complicit. The Fourth Amendment of the Constitution has been
incorporated against the states, which many see as a worrying expansion
of federal power. Yet while people argue over what the right legal
structure for dealing with Superman is, what they seem to miss is that
Superman only obeys the laws because he chooses to. He has already
graciously said that he will abide by these rulings, yet one has to
wonder what the Man of Steel actually thinks of them. All too often, we
forget the enormity of his powers and treat him like a constant, but
what man can exist without change?

\begin{center}\rule{0.5\linewidth}{0.5pt}\end{center}

Lois and Clark stood outside the Metropolis Courthouse with the other
reporters, waiting for the verdict. Calhoun's trial had been sped
through, and there was little doubt that Superman had used pressure of
some sort to make that happen. The early portion of the trial had been
marked by an enormous amount of evidence being thrown out, with the
judge citing the new Supreme Court rulings. A number of the charges had
been dropped after that, though it was still enough to put Calhoun away
for the rest of his life. Bail had eventually been set at one hundred
thousand dollars, which Calhoun had happily paid as though it were chump
change to him. Clark no longer smiled when the topic of the case came
up. He'd submitted an article to Perry about corruption in the case. It
alleged witness intimidation, jury tampering, and juror misconduct, but
his sources were shaky and couldn't be verified to Perry's satisfaction.

``Not guilty!'' came a shout from within the courthouse. The reporters
began to crowd around, to get a picture of Calhoun or shout a question
out to him as he walked out. Lois went with the pack, but Clark stayed
behind. He had a defeated look on his face, like he'd known that it was
coming but hoped he was wrong. Lois got her comment, and Clark wrote up
an article about how Superman was nearly useless in the face of
organized crime with the laws the way they were.

A week later, someone began setting fire to the homes of known or
suspected abortionists. Superman stopped them, which caused a
significant controversy. So far as Lois could tell, that was the whole
point.

``Why do you think Superman doesn't stop abortions from happening?''
asked Lois. It was a question that many of her fellow Catholics had
asked for a long time. She'd been practically mobbed by the other
churchgoers when she'd gone to Christmas Mass, since people seemed to
think that she and Superman were as close as two peas in a pod. In their
defense, \emph{The Daily Planet} hadn't been quick to correct that view.

``He used the term unambiguous good, didn't he?'' asked Clark. Lois had
predicted that Metropolis would eventually break him, but she hadn't
thought it would be such a long, slow decline.

``Well that's the whole idea,'' said Lois. ``If Superman isn't stopping
the abortions, then that means he doesn't seem to think stopping them is
an unambiguous good.''

``He wants to avoid the controversy,'' said Clark. It was clear that his
heart wasn't in the conversation.

``Avoiding controversy outweighs unambiguous goods?'' asked Lois.

``I don't know,'' said Clark. ``The world is complicated. I'd really
rather not talk about this.''

Early on, Clark had been eager to engage her. He'd liked having her
attention of course, but he'd also been more sure about himself then,
more convinced that he could get her to come around to his way of
thinking. It wasn't just that she'd worn him down though, everything
about him had started to become so\ldots{} mechanical. It hadn't
affected his work, and if anything he had been increasing his output.
But the spark that was Clark Kent was dimming, and Lois wondered if
there was anything she could do about that. She and Clark were more
colleagues than friends, but she spent more of her time with him than
anyone else, at least when they weren't out in the city chasing down
stories.

``Do you want to go see the mural after work?'' asked Lois.

``Sullivan already covered that,'' said Clark without looking up from
his typewriter.

``I said after work. I meant more as something to look at,'' said Lois.
``For entertainment. Which I think is the point of it.'' Clark looked at
her. ``Not a date or anything like that, just friends. And maybe
afterwards we'll get a bite to eat somewhere?''

A slow, cautious smile crept onto Clark's face. ``Sure, I'd like that.''

When the mural was finished it would stretch for three city blocks, but
so far only two blocks of it had been completed. It was a mosaic made up
of small tiles, each about the size of a fingertip, visible as a
coherent image only from a few steps back. They started walking it from
the end that was supposed to represent the past, when the island that
Metropolis was built on was home to the Lenape Indians.

``It's white‐washed,'' said Clark. ``But I don't suppose anyone expected
anything else. None of the subjugation or slavery that marks the actual
history of the city. There should be men in collars somewhere
around\ldots{} there.''

``Clark, I know you're still a bit raw about Calhoun getting off,'' said
Lois. ``But you've got to snap out of it eventually.''

``It's not just him,'' said Clark. ``It's all the rest that are just
like him. Do you know how many guilty men go free?''

``Better for ten guilty men to go free than one innocent man rot in
jail,'' said Lois.

``Why that number?'' asked Clark. ``Why ten and not five?''

``It's not meant to be literal,'' said Lois.

``I'm just curious,'' said Clark. ``It's in the Bible, did you know
that? Genesis 18:23, `And Abraham drew near, and said, Wilt thou also
destroy the righteous with the wicked?' The numbers were different
though. God said that if he could find ten innocent men in the whole of
Sodom and Gomorrah he would refrain from raining down brimstone and
fire.''

``That's kind of gruesome,'' said Lois. They walked past a colonial
scene of men planting crops and raising cattle. It was unimaginable that
land in Metropolis had once been cheap enough that you could farm it.

``In the end, God destroyed Sodom and Gomorrah,'' said Clark. ``Because
it was a place of evil. But he saved the only innocents in it first,
because God is perfect, and that was within his power.''

``Unfortunately,'' said Lois. ``The justice system is run by men.
There's a distinct lack of perfection. Are you just figuring this out
now?''

``No,'' said Clark. ``Believe me, I know how imperfect people can be.''
He bit at his lip. ``I don't know, maybe I just never studied history as
closely as I should have. It's easy to forget that slavery ever
happened, you know? And there are crimes against humanity that are just
swept under the rug, forgotten by everyone, though you could still find
the mass graves if you looked hard enough.''

``Jesus Clark,'' said Lois. ``You really know how to show a girl a good
time.''

Clark was silent after that, but she could tell he was still thinking
along the same lines as before and just not saying anything out loud.
She wished that the final part of the mural had been finished, so that
they could talk about something more pleasant. She'd heard that it was
going to be like something out of science fiction, with spaceships going
to the moon and robots serving people dinner. Lex Luthor was the man
behind the project, and he'd proven himself an optimist. It was somewhat
comforting that the future history of the world was going to be written
by men like him.

``Do you think that Superman should have just killed Calhoun?'' asked
Clark.

``No,'' said Lois. ``Can you imagine the panic that would have caused?''

``No one would have to know,'' said Clark. ``Superman could just abduct
him and drop him in the middle of the Pacific to drown.''

``Superman wouldn't be that cruel,'' said Lois. ``Even the state tries
to keep their executions as clean and painless as possible. And that's
all a moot point. Superman doesn't kill, everyone knows that. Your
average criminal would rather be arrested by Superman than the cops,
because Superman is gentle.''

``You're right,'' sighed Clark. ``They take him for granted. The whole
trial with Calhoun proved that. No one feared what Superman would do
when the verdict came down. They didn't think it was suicidal to
challenge Superman's will. And they were right.''

It was time for a change of tactics. ``Clark, can you talk to me about
life in Smallville?'' asked Lois.

``You hate Smallville,'' said Clark.

``I was a military brat, and I grew up all over the country,'' said
Lois. ``I lived in a couple places like Smallville, and I was always
bored. But I think maybe I've been projecting my own experiences onto
what I've been imagining. So come on, I promise not to make fun. You
never talk about it anymore.''

Lois had been right. Smallville seemed to be just the trick. There was
no more talk about mass graves or killing the innocent along with the
guilty. Maybe it was just because she'd spent so long around Clark, but
where she'd rolled her eyes at his stories about small town Kansas
before, now she was almost interested. As he talked, he grew more
animated, until his mood had visibly improved. From there it wasn't that
difficult to keep him upbeat, and after a long talk about the
possibilities of the future in front of the unfinished section of the
mural, they'd gone out for dinner and then drinks, though Clark only had
soda water. Lois wasn't sure whether he wasn't so bad as she'd thought
he was, or whether she'd just been worn down by his constant presence.
Either way, Operation Cheer Up Clark had been a rousing success, and
when he came into work the next day he was nearly back to his old self.

Everything started to fall apart two weeks later when the governor's
children were kidnapped.

\begin{center}\rule{0.5\linewidth}{0.5pt}\end{center}

Lex Luthor was slow and careful.

He never said the name ``Clark Kent'' out loud. There were hundreds of
Clarks in Metropolis, and hundreds of Kents, but so far as Lex could
find, there was only a single Clark Kent. It wasn't inconceivable that
every time he heard his full name his super hearing kicked. Everyone
chattered about Superman all day, but surely very few people talked
about Clark Kent. He was a reporter, and his name appeared in nearly
every issue of \emph{The Daily Planet}, so perhaps there was some cover
there, but Lex wasn't about to risk it. He had Superman's secret, and it
was the most precious thing in the world.

Getting records was difficult. Lex had set himself up as one of
Superman's champions, a man inspired by a zeal for the alien that few
others had. He was the chair of the Conference on Extraterrestrial
Science and two other organizations, and somewhat noted as a collector
of information. Now this was working against him, because any connection
he formed with Clark Kent would be immediately suspect. If Lex had
simply remained an anonymous businessman, there would be nothing too
surprising about him purchasing \emph{The Daily Planet} and looking
through its files. But for Lex Luthor the Superman scholar to do it~---
well, there was no way that Superman wouldn't suspect something.

Lex was moving slowly, and the other players in the game were getting
creative. He was certain that Willie Calhoun was one of them, but didn't
know what intent would explain the actions. There were smear campaigns
and contrived moral quandaries~--- attempts to put Superman in a
position where his values would be challenged. Thankfully, none of it
seemed to affect the alien. Lex would have killed Calhoun if he could
have seen a way to do it. It would have been worth it just to stop the
plots. There were so many contacts and lines of communication that had
been burned in the last few months though, and so few ways of getting
dirty work done. Worse, a failure might alert Calhoun. Lex could only
hope that he would figure something out about Clark Kent before Calhoun
or someone else made Superman angry.

\begin{center}\rule{0.5\linewidth}{0.5pt}\end{center}

Willie Calhoun was losing.

He'd won in court, but everything else was in a shambles. Crime was
dropping in Metropolis all over the place, and loyalty seemed to be a
thing of the past as more and more people moved away. The ones that were
left were animals, idiots without the proper restraints. Willie had once
had money, and a nice house, but he was in debt to the banks now with no
way he could see of getting out. He had no real skills he could use in
the real world, and no real nose for legitimate business like Luthor. He
was getting old, and this was the end of the road.

``Fuck Superman,'' said Willie to his empty office. He hoped the alien
would hear. There was hardly a day that went by without some new fantasy
of what he'd do to Superman if the alien weren't invulnerable. It was
comforting, thinking of ripping into that impenetrable flesh.

Superman had cast a spell over the city, one that grew with every
passing day. The last time that people had really doubted him was during
the bombings, when they'd wondered why it was that he wasn't doing more.
What Willie needed to do was to replicate that feeling. If the people
stopped believing in Superman, maybe he'd finally fuck off and fly away.
All the worst psychopaths of Metropolis had been left in Willie's
employ, and it was time to use them.

\begin{center}\rule{0.5\linewidth}{0.5pt}\end{center}

The governor's two children were abducted on their way home from private
school. The abductors had used chloroform on both the driver and the
children. The operation must have been carried out in nearly complete
silence to prevent Superman from hearing, but this was par for the
course in Metropolis. The driver was found laid down in the front seat
with his throat slit. By the time Superman had arrived at the governor's
mansion, an hour had passed and the kidnappers were long gone. No ransom
note ever came. The radio and newspapers latched on to the story, and
someone from somewhere had dug up a picture of June and Robert Whitman
waving at Superman as he flew through the air, which only added fuel to
the fire.

It was five days later that Lois found another letter perched on her
desk, again requesting that she come up to the rooftop. She grabbed her
pencil and notepad, then made the trek up.

``Hello Lois,'' said Superman. He stood with his back to her, looking
out over the city. His cape flowed out behind him. Even after all this
time, Lois couldn't help but see him as anything but a god.

``Superman,'' she replied. ``What brings you to my neck of the woods?''

``I found the governor's children,'' said Superman. He didn't turn
around to face her.

``And are they alright?'' she asked.

``No,'' replied Superman.

Lois was quiet for a moment. She'd been covering the story double time,
since Clark was out with the flu. She'd been hoping it wasn't the
Lindbergh baby all over again. ``Were they\===''

``Off the record?'' asked Superman.

Lois hesitated for a moment, then tucked her pencil behind her ear.
``Sure.''

``I found them in a farmhouse forty miles outside of Metropolis. They
had June in the kitchen on a table,'' said Superman. ``Laid out on her
back. Only eleven years old and they were\==='' Superman stopped. ``I
barely recognized her. They were taking turns with her.''

Lois felt her stomach churn. She didn't want to be hearing this.

``Robert had been put into the refrigerator,'' Superman continued.
``Nine years old, and they'd used a hatchet to get him into small enough
pieces that he'd fit on the shelves.''

Superman kept clenching and unclenching his fists, and Lois could only
think about how much power he was exerting when his knuckles went white.
Enough to turn coal into diamonds, probably.

``There were three men there,'' said Superman. ``Three men, and they
were~--- animals. Monsters. June had a gag in her mouth, and she was
screaming around it.'' He took a breath. ``I flew in as fast as I could.
I pulled her out of there and flew her to the nearest hospital. She beat
against my chest the whole time, crying and shouting. Either she didn't
realize who I was or~--- or she realized, and she hated me for being too
late.'' He swallowed hard. ``And then I went back for the men.''

Lois wanted to say something, but the words were stuck in her throat.

``Do you know what I did to them?'' asked Superman.

Lois took an involuntary step back. She couldn't help herself. She could
see the anger radiating off of him now, barely kept in check. It had
been there the whole time, as plain as day, she just hadn't thought to
look for it. The muscles on his neck were strained and his teeth were
clenched. ``What did you do?'' she asked in a soft, small voice.

``I arrested them,'' said Superman.

``You\ldots{} what?'' asked Lois.

``It would have been so easy to kill them,'' said Superman. ``No one's
seen the upper limits of my strength. I could have just snapped my
fingers and~\==='' He did just that, and there was a thunderclap. It left
Lois's ears ringing. ``--- like that. Dead. I could have pushed my
fingers straight into their brains, faster than a speeding bullet. It
would have been better than they deserved. They deserved to be chained
up in the deepest, darkest cell I could make for them and slowly starved
to death.''

``Superman,'' said Lois, but there wasn't any set of words that could
come after that to make everything okay.

``I can't keep doing this,'' he said. He finally turned around, and she
could see tears in his eyes. ``I can't keep pretending that I'm someone
that I'm not~--- some paragon of truth and justice. I'm just~\==='' he
seemed to start to say something but changed his mind. ``Just an alien
from the planet Krypton. I'm not perfect.''

``No one is asking you to be,'' said Lois, but she knew that wasn't
true. Millions of people were clamoring for Superman to be a million
different things. They assumed he was perfect, they just thought he was
perfect in the wrong way. ``They just want you to try your best.''

``My best? I can hear everything going on in the world right now,'' said
Superman. ``No one thinks about what that means.'' He pointed to the
north. ``Just there, six miles away, a house is on fire. The family has
evacuated, but their possessions are burning. A little girl is crying
because she left her doll behind, and I can see it melting. She's
calling out for me to do something. Over there, two miles down the road,
a man just punched his wife in the mouth, and shouldn't I be going to
stop him from doing it again?'' He pointed east. ``There was a flash
flood in China a handful of minutes ago. I can hear three women choking
to death. If I left now, I might be able to save them.'' He pointed to
the south. ``There was a car accident near Atlanta, eight seconds ago.
When the windshield shattered it sliced a man across his neck. If I left
now, I might be able to get him to the hospital before he bleeds out.''
He shook his head. ``But I'm not doing anything to help anyone. I'm
standing here on this rooftop, talking to you.''

Superman stared out over the city, unmoving. Lois watched him.

``It's not selfish to take time for yourself,'' said Lois. She tried to
keep her hands from shaking. She was scared of him, and she wondered
whether he could tell. ``If that's what keeps you sane, there's no shame
stopping to take a breath.''

``Of course there is,'' said Superman. ``Do you know why I wanted to
kill those men? It wasn't just because of what they'd done. It's because
I didn't do enough. I was busy taking time for myself. Those men were
monsters, but I'm a monster for not doing more. I'm a fraud.''

He was silent for a long moment, staring out into space while he
listened to people die. ``I really should be going.'' Lois tried to
think of something to say, but Superman stepped backwards off the roof
and plummeted downwards. The last thing she saw was his cape fluttering
behind him.

Her heart was hammering away in her chest. Her palms were sweaty. There
was no force in the world that could stop Superman. He was being pushed
harder than he could handle, and she was the only one that knew. He'd
revealed himself to her in confidence, but what she now knew was bigger
than any promise. Superman was unstable. She had no idea what to do
about that.

\begin{center}\rule{0.5\linewidth}{0.5pt}\end{center}

Lex Luthor had done some quick, sloppy math.

Superman spent a minimum of four hours a day as Clark Kent. He didn't
spend the entire day in the office, and was often out in the field
reporting on something or another, which gave him some time to be
Superman. Lex Luthor had read every article written by Clark Kent over
the past year, and there were some trends that suggested to him that
much of the information was gathered through the use of x‐ray vision and
super‐hearing. Clark Kent rarely used direct quotes, and rarely claimed
that he'd asked someone a question. He also had a tendency towards
unnamed sources. So call it four out of every eight hours of every
workday as Clark Kent. Forget for a moment that Superman went about his
do‐goodery in an incredibly inefficient way and just crunch the numbers
with best guesses about the variables and probabilities.

The existence of Clark Kent cost four people their lives in the average
day. A human life was worth less to Superman than the ability to sit at
a desk for an hour. And that was just actual death. If you included
rape, assault, property damage, and theft, it became even more
atrocious. Lex immediately revised his estimate of the existential risk
posed by Superman upwards by a substantial amount.

Lex had investigated the Clark Kent issue as much as he could from as
remote a distance as possible. There were a number of troubling aspects
to it, aside from what it implied about Superman's psychology and the
value that Superman placed on human life.

Clark Kent's first byline for The Daily Planet had preceded Superman's
arrival by three months. Superman had claimed to study the world for two
weeks before intervening in human affairs, but that was clearly a lie.
And where had Clark Kent come from? You couldn't just get hired without
paperwork and references. It was admittedly possible that a number of
people were in on the deception, but Lex thought it unlikely. He'd
spoken to Lois Lane in person on a number of occasions, and she hadn't
let even the smallest false note slip. Even if she were a masterful
liar, now that Lex knew the truth he should have been able to spot
something in retrospect. He would speak to her again to make sure, but
if Superman's interviewer weren't in on the secret, Lex couldn't imagine
anyone else would be either.

No, the signs pointed to Clark Kent existing in some respect prior to
his arrival in Metropolis, and this buried past was where Lex needed to
be looking. He hired out a private investigator to strike up a
conversation with a photographer at The Daily Planet named Jimmy Olsen,
and when the topic of a recent article came up, Jimmy was all too ready
to spill the beans on Clark Kent. He'd been obliging enough to provide a
location: Smallville, Kansas.


%%%%%%%%%%%%%%%%%%% NEW CHAPTER %%%%%%%%%%%%%%%%%%%%%%%%

\hypertarget{peeling-back-the-veil}{%
\chapter{Peeling Back the Veil}\label{peeling-back-the-veil}}

Jimmy Olsen sat at the bar, gulping back his fourth beer. It was
possible to forget, for brief moments.

Lois Lane had come over to him, shaking slightly, and said that they
needed to take a trip out into the country. He'd grabbed his camera and
plenty of film, then raced downstairs where he'd had to wait in the car
for nearly ten minutes while Lois made some calls and tried to figure
out where exactly they were going.

Lois drove. Her knuckles were nearly white on the steering wheel.

``Where are we headed?'' asked Jimmy.

``A farmhouse near Bott's Pond,'' said Lois. ``Superman found the
kids.''

``Thank God that's over,'' said Jimmy, and Lois had shot him a look that
shut him up for the rest of the trip.

There were two cop cars outside the place when they arrived. The
kidnappers had been taken away an hour ago, but he and Lois were the
first reporters on the scene. Jimmy would have been fine just getting a
shot of the farmhouse with the cop cars in front of it, but Lois had
loudly insisted to the police that Superman himself had sent them to get
pictures of the interior so they could document the actual crime scene.
Jimmy had no idea whether that was true or not, but the police seemed to
believe her. He'd nearly thrown up when he'd seen the body parts stacked
like cordwood. Lois had just frowned and stared at the scene with an
intensity that scared him.

Jimmy looked up from his beer the second time he was tapped on the
shoulder.

``Hi,'' said a cute redhead in a willowy dress. She held out her hand
towards him. ``I'm Eleanor.''

``Jimmy,'' he replied. Her handshake was firm.

``Our hair matches,'' she said with a laugh.

``I guess so,'' he said.

``Rough day at the office?'' she asked. She raised her eyebrows and bit
her lip, like she couldn't wait for his answer.

``I'm a photographer,'' said Jimmy. He'd wanted to continue, to explain
the things he'd seen, but couldn't find the words. And on second
thought, maybe it was better not to inflict that on anyone. The worst of
the photographs wouldn't make it to print. Perry would pick out
something that was suggestive of horror but didn't actually show
anything. To Jimmy, it was almost worse to only catch a glimpse. He was
sure that he would be a better photographer if he could understand why
the small puddle of blood on the edge of the kitchen table was somehow
worse than directly seeing the dismembered corpse.

``What kind of photographer?'' asked Eleanor.

``I work for the newspaper,'' said Jimmy. ``For \emph{The Daily
Planet}.'' He paused. There had to be something that he could say that
wouldn't ruin her evening. ``You know that picture of Lois Lane standing
next to Superman? I took that.''

Eleanor placed a hand on his arm. ``Oh, I read \emph{The Daily Planet}
every day. I wonder how many of your pictures I've seen?'' She had an
easy, pleasant smile, and Jimmy slowly began to take notice of her.

``Lots, probably,'' said Jimmy. ``People look at the bylines, not the
photo credit. Most of them probably don't even look at the bylines.''

``I look at the bylines,'' said Eleanor happily. ``Clark Kent and Lois
Lane, right? Do you work with them?''

``Yeah,'' said Jimmy.

``Say, what does Clark Kent look like? I've seen photos of Lois,
obviously, but I've sometimes read the name of Clark Kent and wondered
what he was like.''

``Clark?'' asked Jimmy. He swallowed down the last of his beer and
signaled for another. ``He's a big guy. Sort of a hunched over
gorilla.''

Eleanor laughed. She was still touching his arm. Jimmy felt his cheeks
warming, and it wasn't just the alcohol. ``That's not how I pictured him
at all. In my head he was tall and upright, very dapper. Like Clark
Gable.''

``No,'' said Jimmy. ``Not like that at all.'' Between Eleanor's
questions, the beer, and the images from the farmhouse swimming around
his brain, Jimmy was beginning to feel out of sorts.

``Where's he from?'' asked Eleanor.

``What?'' asked Jimmy. He'd been distracted by her eyes.

``Clark Kent, is he from the city or somewhere else? I pictured
somewhere on the East Coast, but the city itself,'' said Eleanor.

``Kansas,'' said Jimmy.

``Really?'' asked Eleanor. Her eyes lit up. ``I'm from Kansas too! Which
part?''

``Smallville,'' said Jimmy.

``Yes, I think I've heard of it,'' said Eleanor. She looked over at the
clock above the bar. ``Well I have to go, but it was nice talking to
you.''

``You're not staying?'' asked Jimmy. He tried to keep the hurt from his
voice.

``You didn't seem all that interested in talking to me,'' said Eleanor
with a frown. She gestured towards his beer. ``And I don't know how many
of those you've had, but I think it's probably been too many.''

``Today's the worst day of my life,'' said Jimmy. ``Worst so far anyway.
There might be other days that are even worse than this one. I've got a
feeling that's the case. I just need someone to be by me. Please?''

She seemed about to brush him off, to offer some excuse and leave, but
she must have seen something in his face because she just nodded and
stayed with him.

They got to talking, actually talking, and eventually Jimmy felt like
the world wasn't about to come crashing down on him. Eleanor had a
certain brightness to her that made the world seem less grim. She'd come
to the bar alone, and after an hour had passed, he'd offered to walk her
home. When they got to her place, she must have sensed how desperately
he wanted not to go back to his cold, cramped apartment. She invited him
up.

Her apartment was just as small as his was. He sat on her bed while she
put on a kettle of tea, and that was when he started crying. He felt
embarrassed and ashamed, but she sat down next to him, ran her fingers
through his hair, and made comforting noises. They laid down side by
side on her bed. She didn't seem surprised or upset. It must have been
around two in the morning that she started telling him about her father.
He'd come home from the Great War with shell shock, and killed himself
with a shotgun when she was six. Jimmy didn't know how to respond to
that. He hoped it was enough that he had listened. Eventually she fell
asleep, and he followed suit soon after.

In the morning he'd thought that there would be sheepish looks and
awkward goodbyes, but she'd made them breakfast in her tiny kitchen and
didn't show an ounce of shame.

``I need to change out of these clothes and get dressed for work,'' said
Eleanor. Her voice was soft and gentle. ``But if you ever need someone
to talk to, you know where I live. There's a communal telephone on this
floor, I can give you the number.''

``I'd like that,'' said Jimmy. ``I never even asked what you do. We
talked about me too much. I feel like a lout.''

Eleanor looked at him for a moment before answering. ``I work for a
private detective agency. And I really do need to get going, I'm
sorry.''

Jimmy said his goodbyes and left for \emph{The Daily Planet}. He felt
better, more at peace with what he'd seen the day before. He couldn't
imagine spending that night alone.

\begin{center}\rule{0.5\linewidth}{0.5pt}\end{center}

``Are you okay?'' asked Clark. Of course he got better right after the
biggest news story since the bombings was already on the page. It was
typical of him.

``Peachy,'' replied Lois. She'd barely slept the night before. She would
have gotten drunk, but she'd done some thinking about alcohol on the
ride out to the farmhouse. She'd become too entangled with Superman for
loose lips. So far, she'd been making up for it by smoking more, but
that didn't seem to be helping her nerves at all.

``Sorry I wasn't here,'' said Clark. ``Sorry you had to see that.''

``See what?'' she asked.

``The body,'' said Clark. ``The blood. I read your article and looked at
Jimmy's photos, the ones that didn't make it to print. It was
gruesome.''

Lois waved her hand. ``That was nothing,'' she said. ``I mean, not
nothing, but there are hundreds of millions of children in the world,
and you've got to figure that hundreds of them die every day, right?
Maybe thousands? Lots of little girls get raped. Lots of little boys get
chopped up. The only reason this is front‐page news is that they were
rich and white with a famous father, and because Superman didn't quite
get there in time.'' Clark watched her. She tried to concentrate on her
typewriter, but she couldn't even remember what she was supposed to be
typing up.

``Perry told me that Superman talked to you. What exactly did he say?''
asked Clark.

``That's between me and Superman,'' said Lois. She was being too harsh
with Clark, she could tell, but it would have taken more effort than she
was willing to spend to make her words come out nice.

``Lois, if you need someone to talk to, I'm here for you,'' said Clark.
``And I don't mean any offense, but it seems like you've got something
you need to get out.''

``Possibly,'' said Lois. She stopped for a moment to think through her
wording. Superman had come to her of all the people in the whole world
to get things off his chest, and that meant that she was important to
him. She had to assume that he was listening and watching, so talking
about Superman became a matter of framing him in the best possible
light. ``In general terms, he explained to me that being Superman can be
difficult sometimes.'' There, that didn't sound so bad as it really was.
``He said that he can't do everything.''

``And that upset you?'' asked Clark. He had a look of serious and
heartfelt concern, like she were some delicate doll that he was worried
would break under stress. She hated that. She'd had more adventures in
her life so far than Clark Kent could ever dream of, and to him it was
like she was made of glass.

``It made me think about how right he is,'' said Lois.

``Lois, look, I don't know what it was he said, but I'm sure he didn't
mean to upset you,'' said Clark.

Lois nodded. ``I agree, it wasn't his intent. But he opened my eyes up,
and if my reaction to that is to be upset with the world, then so be
it.''

Clark kept staring at her, and she kept avoiding his eyes. ``Do you know
what I think?''

Lois didn't answer, because she wasn't confident that she could speak
without snapping at him.

``My pa was in prison for a while, I told you that,'' said Clark. ``And
for a long time he never really talked about it, but I knew it was bad.
And I think that maybe talking about it would have made it less bad for
him, you know?''

``You're saying I should talk to you,'' said Lois.

``No,'' said Clark. ``I'm saying that maybe whatever Superman said to
you, he just said because he was having a bad day. Maybe he just\ldots{}
needed someone to talk to, and talking to you made whatever difficulties
he was having easier to bear.''

Lois found this far from comforting.

Superman was holding back, in nearly everything that he did. He didn't
hurt people, and certainly didn't kill people. He could fly at twenty
times the speed of sound, maybe even more, but he almost never did. He
worked quickly and efficiently towards his objectives, and most of the
time if you showed up after he'd gone there'd hardly be any evidence
that he was there at all. Everyone thought that was just who Superman
was. He was so totally and completely in control of himself that he
would never do anything truly wrong. He firmly followed the doctrine of
unambiguous goods.

It wasn't true though. People thought that Superman did everything
effortlessly, and maybe as far as the physical realm went that was true.
Inside his head though, he wasn't much more than a man. She'd heard that
Superman had walked into Calhoun's bar and let himself be hit in the
face with a gob of spit. She'd believed that Superman had been
unbothered by that, but now it was clear that Superman was human enough
to have felt something there. Superman's ideals weren't innate to him,
they took conscious effort on his part. And what would happen when
Superman had a day so bad that he decided that his ideals weren't worth
keeping?

\begin{center}\rule{0.5\linewidth}{0.5pt}\end{center}

Picture a circle. Next, picture a point outside that circle, call it
\emph{O}. Draw a line from the point such that it pierces the circle in
two places~--- a secant~--- and call those two points \emph{A} and
\emph{B}. Draw another line originating from that \emph{O} such that it
intersects the edge of the circle in only one place~--- a tangent~---
and call that point \emph{C}. The secant‐tangent theorem states that
\emph{OA} times \emph{OB} is equal to \emph{OC} squared.

If the circle is Earth and the point outside it is Superman, then that
tangent defines how far Superman can see before his vision starts to
clip the crust of the Earth. To find that distance, take the diameter of
the Earth (roughly eight thousand miles) plus Superman's distance from
the Earth (rarely seen to be more than ten miles), then multiply that by
Superman's distance from the Earth, then take the square root of that.
The result was 280 miles, the distance that Superman could see to the
geometric horizon from the height that he stayed within ninety‐nine
percent of the time.

There were 1,127 miles between Smallville, Kansas and Metropolis, New
York.

Of course, Superman had x‐ray vision, but that was stopped by lead. Lex
Luthor had consulted a book of geological science and found that the
estimated abundance of lead in the Earth's crust was one thousandth of
one percent, which meant that for every mile of earth that Superman
looked through, he was looking through sixteen millimeters of lead.
Based on Lex's calculations, it was safe to assume that it only took a
centimeter of lead to stop Superman's x‐ray vision. The upshot was that
Superman could not see what went on in Smallville unless he specifically
moved himself into a position to do so.

It allowed for a comparatively enormous amount of breathing room.

It was imperative that he get someone there as quickly as possible. What
records he could pull showed that Clark Kent at least existed on paper,
and a quick call done through layers of intermediaries confirmed that
the \emph{Smallville Ledger} had once employed him, or at least claimed
to have employed him. Lex was starting to once again doubt that Superman
was an alien, since it very much seemed that Clark Kent's backstory was
solid, but he kept digging all the same. Learning about the existence of
Clark Kent had produced numerous threads to pull on.

He needed someone in Smallville, but the constraints on hiring were
immense. He needed someone intelligent, prone to following orders,
trained in espionage, and willing to go into deep cover for an extended
period of time. He would need to instruct them to take precautions above
and beyond what any covert operation had required in the history of
spycraft, a constant cover that remained unbroken for weeks or even
months at a time. The list of people that fit the bill was very, very
short. Lex was in the middle of trying to figure out whether it would be
possible to put someone in deep cover and still keep them in the dark
about the connection between Clark Kent and Superman when the doorbell
rang.

A few minutes later, Mercy stood in the doorway of the study. ``Miss
Lane is here, requesting a moment of your time,'' she said.

``Send her in,'' said Lex.

She looked different, though Lex couldn't say exactly how. Did she know
that Superman was Clark Kent? If so, it wasn't obvious from her face.
Lex was wearing the outermost layer of his personas, the one where he
was a simple enthusiast and advocate for Superman with no knowledge of
the alien he wouldn't willingly share with the world. He mentally
prepared himself for Lois Lane to peel back the personas one at a time.
He'd been careful, but part of being careful was preparing for your
carefulness to fail you. He had stories prepared that would justify his
actions.

``Miss Lane,'' Lex said with a smile.

``Mister Luthor,'' replied Lois. He pinpointed what was different about
her; she was tense. ``I called your office and they said you were
here.''

``The businesses mostly run themselves,'' said Lex. ``I have a knack for
hiring competent people, and that's left me with the free time to pursue
my passions.''

``Superman,'' she said. She began to dig a pencil and notepad from her
purse.

``Just so,'' replied Lex.

``I've read your proposals,'' said Lois. ``What would you do, if you
were Superman?'' She began writing in the notepad.

``A common question,'' said Lex. He was about to continue on when Lois
turned her notepad around to face him. It said \emph{Can Superman be
stopped?} Lex's eyes moved to the door, to make sure it was closed. They
were encased in a hidden layer of lead. Lois had been over when the
shielding was being installed, and knew they were behind it. She was
being cautious.

``A common enough question,'' repeated Lex. ``For many it's the perfect
fantasy. People talk about setting foot on the surface of the Moon, or
going to the Olympics and dominating in every sport. They talk about
standing up to their various oppressors. My companies have been picking
up quite a few Jewish immigrants from Germany of late, and I feel that
many of them would like nothing better than to fly down and put a hole
in Hitler's face.'' He turned to look at her. ``Superman can't be
stopped. It's frightening to think what might happen if his power fell
into the hands of someone without such a strong moral compass. For
myself, I'm not sure that I would want the power. I'd use it for good as
best I could, I suppose. No flashy displays, no material wealth, just
the betterment of mankind.''

``I was wondering whether you could help me,'' said Lois, pointing to
her notepad, where the words were still written.

Lex watched her carefully. Lois Lane could easily be working for
Superman. Even if she didn't know that he was Clark Kent, she could have
been sent in to get some admission of guilt. He couldn't trust her. But
perhaps he didn't have to. ``Help you with what?'' he asked, not missing
a beat.

``I've written two books,'' said Lois Lane, ``One on the radium girls
and another on the role of women in the World War.''

``I know,'' said Lex. He pointed to his bookshelf. ``I've read them.''

Lois seemed momentarily taken aback by this, but of course he had read
them. He'd read The Daily Planet every single day for the past year, and
after he'd learned that Clark Kent was Superman he'd gone and read every
issue again. Earlier that morning, when he'd learned that Clark Kent had
once written for the \emph{Smallville Ledger}, he'd immediately started
thinking up possible methods of getting back issues of it to his home or
office without immediately allowing Superman to connect the dots.

``My new book will be about Superman,'' said Lois as she wrote in her
notebook. ``And as you and I have something of a working relationship, I
was wondering whether you would be willing to contribute.'' She flipped
the notebook towards him again. \emph{S is losing faith in us}.

``What sort of contribution?'' asked Lex.

``You're the preeminent scholar of him, and one of the greatest examples
that his efforts to be a symbol actually work,'' said Lois. She pointed
at the notepad and raised her eyebrows.

All Lex could think was that it was a trap. She would have to be a
masterful liar for that to be true, but that was certainly possible. If
he'd been willing to admit that Superman was using the disguise of Clark
Kent and lying through his teeth to everyone he interacted with on any
given day, then surely he had to admit that the same might be true of
the woman that sat next to him every day. The idea of Lois Lane turned
to his side was seductive though. And though he was well aware that the
best traps didn't look like traps until they'd been sprung, it truly
didn't look like a trap.

``I'm afraid I'm a busy man,'' said Lex. ``Though I admit that sharing
my thoughts on Superman with a wider audience appeals to me. What
precisely would be the nature of this arrangement?''

Lois wrote in the notebook. Superman could surely hear that, if he were
listening. Lex couldn't decide whether he was being too paranoid in
thinking that Superman would find it suspicious. ``I'd like you to write
two chapters,'' she said. ``They can be short. There will be a chapter
on the science that I'd like you for contribute to, and another chapter
on how he's changed the people of the city.'' She held up the notepad
again. \emph{S is more human than he lets on, might turn on us}. ``Does
that sound reasonable?''

``Let me think on it for a moment,'' replied Lex. ``In the meantime,
feel free to peruse my library, I'd be happy to give you any book that
you have an interest in. Give me five minutes, by the clock?''

Lois looked unhappy, but she nodded all the same.

Lex closed his eyes, relaxed his body, and thought.

There was too much unknown information. He could make all sorts of
educated guesses about what Lois Lane and Superman knew, but there was
so little information available that these guesses were barely worth
anything. There were dozens of configurations of truth which fit the
data as he saw it, and in some of those possible worlds it would be
correct to allow himself a partnership with Lois Lane, and in others it
would throw not just his operations but the fate of the entire planet
into jeopardy. Lex Luthor had set himself up as a follower of Superman,
highly visible and shining like a beacon. If Superman really was losing
his faith in humanity, what would happen if he learned that Lex Luthor
was responsible for the deaths of dozens, nevermind that it had been the
correct decision given the information he'd had available at the time?

He looked to Lois. If she were telling the truth, why had she chosen to
confide in him? Well, he was a billionaire with an active interest in
the betterment of humanity, the premiere scholar on everything related
to Superman, and likely one of the few people she knew who had a room
lined with lead and the sense not to immediately blurt out a strangled
``What?'' when shown a secret message. On top of that, they had an
established relationship. It made a certain sort of sense. The more he
thought about it, the more he thought it plausible that she really had
come to him in good faith.

He walked over to her and took the notepad and pencil from her hands.
She had a hopeful look.

``I've decided that I'll do my best to help,'' said Lex. He pointed to
where she'd written \emph{might turn on us}, then began to write
something of his own. ``I'm a busy man, but a partnership could benefit
us both.'' He turned the notepad towards her. \emph{Tell me everything
you think you know about Superman.} ``I have a number of things coming
up in the near future, so it would be good to get this done quickly.''

``Agreed,'' replied Lois. She grabbed the notepad from him. ``I should
warn you that I don't have a publisher lined up just yet, but it
shouldn't be a particularly hard sell.'' \emph{He can't know I'm telling
you.}

``A problem to be dealt with in due time,'' said Lex. ``If you're free
tomorrow, we could meet here? There are a few things that I'd like to
think over first. I'll try to have some initial thoughts ready.''

Lois watched him for a moment, then nodded.

\begin{center}\rule{0.5\linewidth}{0.5pt}\end{center}

The next day, Lois Lane picked up the piece of paper from Lex Luthor's
desk as he said unimportant things for the benefit of Superman.

\emph{I'm not saying that I believe you, Miss Lane. But if you think
that Superman is losing his faith in us, then that's something that
needs to be discussed, and I can only hope that if he finds out, he'll
understand that the discussion couldn't happen in front of him, as it
were. You have more exposure to the man than anyone on the planet, so
far as I know. You're the only one he's really talked to. If you have
concerns, I need to hear them, no matter how outlandish.}

``There much to the science of Superman,'' said Lex. ``His x‐ray vision,
for example, doesn't use actual x‐rays. The current best theory is that
there's an exotic type of particle which is as yet undetectable to us.
It permeates the planet, with lead atoms being the only thing that can
stop it for reasons that possibly relate to its atomic weight, electron
density, or some other property. But there's so much unknown, as with
much about Superman. I've been working on it for a year, and I still
don't have the faintest understanding of how his hearing works. I want
to make it clear that much of what I say about the science of Superman
is on the cutting edge, and not to be taken as gospel.

``I've done the liberty of typing up a very rough draft, and would be
pleased if you could take a look,'' he said. He handed her a blank sheet
of paper and a pencil. She was about to object that if they really
wanted to be secretive she'd need to leave his study with some actual
papers, but he pulled out a number of typewritten pages, already marked
up with a few corrections and notes, and set it beside her. She began to
give her account.

From time to time, she would ask Lex an inane question to keep up
appearances, and he would respond with inane answers. To Superman it
would sound like they were simply working on a book together. She wasn't
sure whether she could trust Luthor, but he was by far the most capable
man in the city, and she hoped that the worst he would do would be to
burn her notes and refuse to see her without letting Superman know what
she thought. She tried to use the strongest, most persuasive language
she could, and hoped that Superman would never learn what she really
thought of him.

Still, she left some things out. She didn't mention the possessive way
that Superman had touched her when he'd picked her up and flown her
through the air. She'd interacted with Superman on a number of
occasions, and he always seemed so familiar with her. So far as she
knew, she was his only friend, but she was also something more to him.
She could feel his eyes on her while she undressed sometimes. She could
feel him staring at her while she tried to sleep. With every
conversation she had, she imagined Superman listening in. This feeling
had grown in intensity since their last meeting. She hoped it was just
paranoia on her part. But either way, Lex Luthor didn't need to know.

\begin{center}\rule{0.5\linewidth}{0.5pt}\end{center}

The picture Lois Lane printed was a grim one.

He was now reasonably confident that she knew nothing of Superman's
alter ego. Her account of Superman was vivid and unflinching.

\emph{He can hear everything that's happening in the world, and it's
driving him to despair. I think he can shut down his hearing and tune it
all out, but that's almost worse in a way, because he still knows all of
the pain and suffering that's happening, and turning away from it
doesn't make it disappear. He sounded like a martyr to me, forcing
himself to bear witness not just to the evils but to the vast but simple
indifference of the world.}

Yet that was very different from the picture that Lex had been forming.
Superman spent time as Clark Kent, which implied a certain apathy
towards suffering. What did Superman get from maintaining the Clark Kent
persona? From what Lex's various sources could tell him, Clark Kent
didn't seem to take very many pleasures from life. He didn't drink or
smoke, and he had no romantic relationships to speak of. It seemed
unbearably dull to Lex. Even in his work life, Clark Kent was only
second best, and he didn't seem to leverage the full force of his
powers.

The first possibility was that everything Superman had said to Lois was
a ruse. Superman was an abject liar, he'd already proven as much by
spending an entire year pretending at being someone he was not. It was
possible that he was manipulating Lois Lane towards some end, though Lex
could only make the vaguest guesses as to what end. Superman should have
no need for a reporter, since he already was one. If it was
manipulation, Lex suspected that it was in pursuit of inflicting some
mental or emotional harm, but it was also possible that he had some
delusions about Lois. Lois hadn't mentioned Clark at all, and Lex hadn't
thought it prudent to bring him up.

The second possibility had taken some time to see. Lex had been under
the assumption that the persona of Clark Kent had been invented as a
cover for Superman, but it was distinctly possible that Superman was a
cover for Clark Kent. The solidity of his background information
suggested as much. Lex had told Mercy that Clark was a mockery of
humanity, but perhaps the outward appearance of Clark Kent matched his
inner feelings. Lex Luthor could almost imagine Clark Kent as a simple
man who wanted nothing from life but to be left alone, burdened by
powers that he didn't understand or desire, donning a costume and flying
through the air because the guilt of sitting at his desk simply became
too much sometimes. It was almost sad, until you remembered that he was
the most dangerous man on the planet.

If there were answers, they would be found in Smallville.

\begin{center}\rule{0.5\linewidth}{0.5pt}\end{center}

Joseph and Loretta Greene bought one of the town's two general stores.
They moved into a small house on Cherry Street, and quickly made friends
throughout the community. Joseph was always ready to ask about the
history of Smallville, a town which he seemed to have adopted as his
own, and Loretta was relentlessly social. They attended church every
Sunday at the Zion Lutheran Church. Though they didn't have any
children, they often spoke of it as an eventuality. If you could see
straight through Loretta's clothes, you would see a scar running at a
diagonal from the side of her left breast to just above her navel. If
you could see straight through Joseph's dress shirt, you would find
three puckered marks that were unmistakably bullet wounds. Joseph and
Loretta had stories ready in case anyone ever saw and asked. Those were
the only marks of their former lives.

As it turned out, Clark Kent was somewhat famous in Smallville. His name
had come up on the very first day that Loretta and Joseph had come to
town, when the previous owner of the general store had told them that
they should carry The Daily Planet, even though it would be at least two
days old by the time it arrived. Though he hadn't been especially
popular or well‐known growing up, Clark Kent had become the nearest
thing that Smallville had to a celebrity, and the people of Smallville
often talked about what Clark was up to in the big city.

Every few days, Loretta would write a letter to her family back in
Gotham City. She wrote an enormous amount, even when there wasn't much
to say, and often included some of Joseph's historical research about
the town and its residents. Joseph took to Smallville like a fish to
water, and some days could be seen two doors down talking to the small
group of men that worked at the Smallville Ledger, a once weekly
newspaper that served as the main source of news for the county.
Anything and everything of interest he learned there went into the
letters to Gotham.

From time to time, a letter would come back.

The player piano had effectively died out in 1929 with the stock market
crash, and few of the things were produced anymore, since radio had
effectively taken its place. Player pianos worked through pneumatic
action to play music, and the different songs were recorded on sheets of
perforated paper. Joseph and Loretta had brought a player piano with
them when they moved in, and a very careful observer might note that it
routinely seemed to break down just after one of these letters from
Gotham City came in. Joseph would take the perforated sheet of paper
with the music out of the machine and go to work repairing whatever was
wrong, and Loretta would lay the sheet on top of the letter. The
typewritten letter would perfectly line up with perforated sheet music,
revealing a scattering of letters that formed a message. Those brief
seconds were the only time that someone watching through the walls from
hundreds of miles away would know that they were something more than
just rural shopkeepers.

``Do you think we'll ever know?'' asked Loretta one night over dinner.

``No,'' said Joseph.

``How much longer, do you think?'' she asked.

``No idea,'' said Joseph. He leaned over and kissed her on the cheek.
``Let's not talk about these things.''

Five thousand dollars were deposited into a Kansas City bank account
every week for each of them, courtesy of a trust that had been set up
according to the will of Joseph's fictitious uncle. They had no idea who
their employer was, only that he was fanatically paranoid. Joseph and
Loretta weren't their real names, but all the proper records were in
place if anyone went looking. If asked about the money, they would
confess that they simply liked the small and quiet life of a small town
and didn't want to complicate things.

\begin{center}\rule{0.5\linewidth}{0.5pt}\end{center}

Hershel Whitman sat on the veranda of the governor's mansion. It was
early in March, and too cold for the veranda, but he didn't like to be
inside the house anymore. He'd have never thought that so soon after
winning an election he would feel like leaving his office. People had
offered their condolences and paid their respects, but it had been more
than a month now, and mostly all that was left were awkward glances and
sad looks. June was shut up in her room, and Robert was buried in the
Oakwood Cemetery.

Superman landed in the yard and started walking towards the house.
Hershel tried not to react. Early on he'd wanted to yell at Superman for
failing to save his children. He had yelled, in fact. Late at night,
after June had been brought back and Robert hadn't, when Hershel
couldn't sleep, he would walk a mile or so from the mansion and scream
at the sky. He didn't know if Superman had listened, or if Superman
cared. He felt somewhat guilty about that now. If it hadn't been for
Superman, June might not have come back at all.

``Superman,'' said Hershel. His voice caught.

``Governor Whitman,'' Superman replied. ``I never said how sorry I
was.''

``No,'' replied Hershel. ``You didn't.''

``I came here to ask a favor,'' said Superman. ``Thirteen minutes ago
Francis Pasqua spoke with his lawyer about getting immunity. He named
William Calhoun as the man who gave the orders.''

``Immunity,'' said Hershel. ``You want me to give him immunity in
exchange for testimony.''

``No,'' said Superman. ``I need to know what June heard them talk about,
and if it's enough, I need her to testify.''

``Just kill him,'' said Hershel. His voice was barely a whisper. ``Just
fly in and kill him. No one would stop you, no one could stop you. Hell,
use a gun and no one would even think of you. There are a dozen people
with cause to kill Willie Calhoun. You want my daughter to take the
stand against him, to say that his name was thrown around by those men?
Calhoun would have the right to face his accuser, and that means cross
examination. No. I won't put her through that.''

``He needs to be brought to justice,'' said Superman.

``Do you know why it didn't happen in the last trial?'' asked Hershel.
He'd had two whiskeys before Superman had shown up, and swayed slightly
as he stood. ``It's because you let him. The criminals don't care about
you. They know you won't hurt them. They know how to hide from you.
Ronald Oakes. That was the name of the man driving my children, and
everyone forgets about him. They slit his throat because they knew that
if they didn't he would call for you. You're not making them stop,
you're just making them adapt.''

``Crime has dropped ninety percent since I've come to Metropolis,'' said
Superman. ``You can ask the chief of police. I know you're angry, but if
we don't have the rule of law, we don't have anything.''

Hershel crumpled into his chair. Arguing was no use. ``If June agrees,''
said Hershel. ``If June agrees to talk, and she knows enough to convince
the district attorney, and the jury listens to her and then they say
he's not guilty, if all that happens\ldots{} you'll just let him go?''

``No,'' said Superman.

``No?'' asked Hershel.

``No,'' replied Superman.

\begin{center}\rule{0.5\linewidth}{0.5pt}\end{center}

In 1911, a baby boy was left in the hallway of a tenement in Metropolis.
He was taken to the Metropolis Foundling Hospital and from there became
part of the Orphan Train program. In Metropolis the abandonment of
children was a continual problem, while in the Midwest there was a
continual shortage of labor. The inventive solution to these twin
problems was for the children and babies to be delivered to the
heartland of America by railway. At every stop the children would be
taken out and displayed before the gathered crowd, sometimes having
their muscles felt and teeth checked. Some would be selected for
indentured servitude and possibly adoption, while others would be put
back on the train and sent to the next stop. When the orphan train
stopped in Oskaloosa, the foundling, Clark, was selected by Martha and
Jonathan Kent. They adopted him a few years later.

So far as Lex could tell, that was the official story that was believed
by the residents of Smallville. Though the orphan trains had fallen out
of favor, the Metropolis Foundling Hospital was still standing. As Lex
Luthor was funding five different orphanages in Metropolis, it wasn't
terribly hard for him to get the records from the Foundling Hospital,
and more importantly, it wouldn't look too terribly suspicious,
especially when it was known that Lex Luthor was looking to expand his
charitable giving. It had taken only a day of looking through the
records to see that they contained no mention of a boy named Clark
leaving the train at Oskaloosa, and no record of the Kents as sponsors
for a child.

This in itself was nothing too out of the ordinary. Lex had found that
few people took record keeping seriously. Ownership of the records
changed, people developed new formats, and sometimes entire years worth
of data were destroyed by insects, acids in the paper, or an excess of
humidity. Yet it still felt suspicious to Lex. If you were trying to
hide someone's parentage, you couldn't do much better than the orphan
trains. Clark Kent had the perfect excuse for not having a birth
certificate.

According to the reports he received from his two agents, Martha Kent
owned a farmhouse outside of town, which she shared with a live‐in
farmhand named Elias Clayton. His agents had spoken to her, and remarked
only that she was a nice woman who went to church every Sunday and spent
most of her time on the farm. Jonathan Kent had died a year before Clark
had come to Metropolis, and if that was a deception, someone had at
least given him a gravestone.

Clark Kent had grown up in Smallville. There were dozens of people who
could recall him as a boy. His worn and faded initials were carved into
desktops and trees. The evidence of his existence was so utterly
convincing that it couldn't be denied. There were aberrant incidents in
and around Smallville that suggested the powers characteristic of
Superman extending back to the time that Clark Kent was eleven years
old. Superman had not actually arrived in a spaceship, he had grown up
on a farm in the middle of Kansas. Even if Lex believed this, it didn't
help to clear up anything. The power had to have come from somewhere.

The solution had to be on the Kent farm.

\begin{center}\rule{0.5\linewidth}{0.5pt}\end{center}

Floyd Lawton had come into Smallville as a drifter looking for room and
board with barely a dime in his pocket. He'd walked down the dusty dirt
roads, going door to door looking for work, until finally he'd happened
upon a small house that belonged to a greying old lady. He'd gone down
the path and up the steps to the front porch, then knocked with a ready
smile on his face.

``Missus Kent?'' Floyd had asked as she came to the door.

``Yes? Do I know you?'' she'd asked. She was in her sixties, maybe even
older, with white hair tied up in a loose bun. Her dress was simple and
blue.

``No ma'am, sorry, the name was on the mailbox. Name's Floyd Lawton.''
He took off his hat and clutched it to his chest. ``Sorry to trouble you
on this fine day, but I've been on the road a long while and I'm looking
to settle down for a spell of work. If you have something that needs
doing, or if you know some neighbors that need some work, I'd do it just
for room and board, whatever's asked of me.''

Martha Kent gave him a warm smile. ``Why you know, I had a live‐in
farmhand up until just two days ago, Elias Clayton. He was a strong and
able man, helped with the few animals I still keep, the garden, and the
maintenance on the old barn. We made enough to keep ourselves afloat,
along with the money brought in by leasing out the land to the Parkers,
and I paid him a good wage. Well Elias had aspirations, you see, but he
was a black and so work didn't come too easy, especially not the kind of
work that he was keen on doing, which was acting. Then just a week ago a
director of movies came out to Smallville, right out of the blue. He
said that he was going to make the great American movie, and said that
Smallville would make the perfect location for it. Well now, Elias took
the day off to go speak with that director. I thought nothing of it of
course, until Elias came back and told me that he'd been discovered. He
said it happens all the time, if you can believe that, so I said to him
that he wasn't to leave until he'd finished putting up new chicken wire
around the coop. I was thinking it might be I'd try taking this year by
myself for a change, but if you're looking for work, then boy do I have
some.''

Floyd nodded through all this, a slightly desperate grin on his face
like he thought a real drifter would have. Martha mostly seemed happy to
have someone to talk to though, and they'd moved the conversation
inside. They'd come to an agreement over homemade lemonade that had too
much pulp in it for Floyd's liking.

Later that day, Floyd had picked up his meager belongings from the
Greene house in Smallville, where he'd rented out a room for the night.
He had a rifle slung over his shoulder, and two pistols in a wooden box
that draw Martha's attention.

``There's not much use for pistols out here,'' said Martha with a frown.
``We have a shotgun, and a few rifles for dealing with the coyotes and
wolves, or for bringing in more meat.''

``They were my father's,'' said Floyd with a smile. ``Hand‐crafted and
fine quality pieces, and I'm only thankful that I've never had to sell
them.''

``My husband Jonathan, may he rest in peace, he abhorred pistols,'' said
Martha. ``He was pacifist and an absolutist, and thought every war was a
crime against God's own will.''

``He's lucky he didn't get drafted then,'' said Floyd with a smile.

Martha's face became very serious. ``Oh, my Jonathan was drafted
alright. He'd applied to be a conscientious objector. When I say he was
a pacifist, I don't mean that he thought it was better not to kill, I
mean he believed with every fiber of his being that it was simply
something a good person doesn't do, no matter the circumstances. He went
to prison for his beliefs.''

``Ma'am, if you don't want me bringing pistols into your \\ home\==='' Floyd
began.

``No, no,'' said Martha. ``There were more than a few things that
Jonathan and I didn't see eye to eye on. You don't use those pistols
lightly though. If someone tries to steal from our farm, I'd rather just
let them take what they came for. It's not worth killing a man over a
pair of chickens.''

Floyd breathed a silent sigh of relief. He loved his pistols. He liked
to use both at once, feeling them kick in tandem. He'd once cleared out
an entire poker den with those two pistols, killing thirteen men with
twelve bullets and earning him the nickname ``Deadshot''. He was handy
with a rifle too, and had been briefly trained in sharpshooting by the
military before a dishonorable discharge that had left him perfectly
positioned to become an assassin. He was very explicit on that term, and
had maimed more than one thug who called him a mere hitman.

He'd met men who didn't want to kill before. Hell, most men didn't want
to kill. But he'd never met a man who'd prefer jail over being in the
army, except perhaps those cowards that only wanted to stay out of the
fighting because they were afraid for their own safety. In his opinion,
Jonathan Kent was probably just a slacker, but he held his tongue.

He settled into a routine at the Kent house. He would listen to Martha
Kent yap away during an early morning breakfast, go out and do whatever
work needed to be done until lunchtime, take a break during which he'd
work on composing a letter to his completely fictitious sister, and then
keep working on the farm until nearly sunset, when he'd go into town,
grab a copy of the Smallville Ledger, and on occasion mail off his
letter for the week.

``Why is there a lock on the storm cellar?'' asked Floyd.

``Oh, that old thing,'' said Martha. ``It kept blowing open, so I put a
lock on it a while back and somehow forgot the key.''

``I could cut the lock,'' said Floyd. ``I wouldn't want to get caught in
a tornado without a storm cellar.''

``It's rusted shut anyway, I think,'' said Martha. ``And there's a small
basement room we can go to if the storms ever get too bad. I wouldn't
worry about it dear.''

Floyd had gone back and looked at the doors to the storm cellar more
closely. They were made of metal, and when he looked closely at the
seams, he could see that the whole thing had been welded shut. It was
hard to make out with all the rust, but the storm cellar had been sealed
shut as tightly as possible.

So far as he could guess, whatever was down there was the entire reason
for his being on the Kent farm. He made sure to mention the storm cellar
in his letters to his fictitious sister, cloaking the information in
long paragraphs about how he was afraid of tornadoes. Hopefully his
employer was smart enough to read between the lines.

\begin{center}\rule{0.5\linewidth}{0.5pt}\end{center}

\emph{Author's Note: Orphan trains were a real thing. Whether this was
slavery by another name or an ingenious solution to the societal
problems of abandoned children and a lack of cheap labor is left as an
exercise to the reader.}

\emph{If you have an interest in reading more about the treatment of
conscientious objectors in WWI, search out ``Armed with Prayer in an
Alcatraz Dungeon'', which does a lot more justice to the topic than I
can do here. It's interesting reading even if you disagree with the
moral philosophy of it. My grandfather was a conscientious objector in
WWII. One of my strongest memories of him was when he told me about how
he was routinely spit on while building bridges and roads around the
Midwest by people who thought that sticking to his beliefs was somehow
the height of cowardice.}

\emph{As always, I appreciate the favorites / follows / reviews /
recommendations. A special thanks to my wife Alyssa for being my beta
reader.}


%%%%%%%%%%%%%%%%%%% NEW CHAPTER %%%%%%%%%%%%%%%%%%%%%%%%

\hypertarget{dust-to-dust}{%
\chapter{Dust to Dust}\label{dust-to-dust}}

Lex needed to know what was in the storm cellar. It was a matter of
practical necessity, but there was an emotional component as well. He'd
spent nearly a year of his life in pursuit of what lay there, slowly
working his way backwards from Superman to Clark Kent and Clark Kent to
Smallville. He had three agents in Smallville, one of whom was living on
the farm itself, so close that it ached. In Lex Luthor's fantasy, he
stood in a clean, pressed suit and watched over a workman using an
oxy‐fuel cutting torch. When the doors were opened, he would stride down
into the cellar and find whatever was hidden there.

It couldn't be a secret laboratory. If it were, there was no way that
Lex would have been able to insert his agent onto the farm so easily. It
was possible that the storm cellar was a decoy of some kind, but Lex
found that doubtful. The game wouldn't be at this stage if Superman were
such a supremely paranoid person. More likely, the storm cellar was
booby‐trapped, or simply impassible by human means. Those metal doors
could hide explosive devices or three solid feet of steel. And contained
within the cellar could be anything. All of that made planning a mission
difficult.

Lex Luthor had involved himself in a number of thefts, especially in his
youth. Stealing an unknown object from behind unknown defenses with a
guard that had a nearly unlimited surveillance ability would be
challenging but not, strictly speaking, impossible. Removing Martha Kent
from the farm for the day would be easiest part. Superman could be
distracted by a disaster of some sort, or more likely a series of them.
Getting the proper equipment into place would be trivial, and the
thieves themselves already had their cover identities. It might even be
possible to break into the storm cellar, retrieve whatever was in there,
and then weld it back shut without Superman even knowing a theft had
taken place until he flew over Kansas and used his x‐ray vision to
check. Whether Superman ever checked at all was an open question, but
Lex found it unlikely. Neither Clark Kent nor Superman had been seen in
or around Smallville in the time that his agents had been there.
Superman could watch from high up in outer space, but from what Lex knew
of his psychology, this too was unlikely~--- though not so unlikely that
the theft could be done without precautions.

Judging how Superman might react to the theft was more difficult.
Superman would find out that someone knew his secret identity, and he
would know that someone had whatever was in the storm cellar. Obviously
that was far from ideal, but it might be worth it if the cellar
contained the means to defeat, depower, or contain him. Lex Luthor laid
his plans.

\begin{center}\rule{0.5\linewidth}{0.5pt}\end{center}

The letter arrived on April 7th. The date was written at the top of it
was ``4/3/35'' rather than ``April 3rd, 1935'', which was a prearranged
indication that it contained a coded message. The code was fairly
simple, as these things went, and it was solvable without the use of
pencil and paper so long as you knew that the variety of salutation
defined which of the six codes was being used. In this case, ``Dearest
Floyd,'' meant to take the last letter of every word and put it into a
four by four grid which was then read from bottom to top and right to
left. Floyd deciphered it quickly. \emph{Go to Greene shop and get
tickets. April 14th leave from church and take Martha with you. Keep her
away until after the show. Top priority.}

Floyd went down to the grocery store owned by Joseph and Loretta Greene.
He had no idea what their level of involvement in this scheme was, just
as he had no idea what the goal of the scheme itself was. So far as he
could tell, they were either patsies with no real knowledge of what they
were doing, or very skilled deep cover agents. Sometimes Floyd thought
he could see something hard and dangerous behind Joseph Greene's smiles,
but he might have just been imagining things. If they were something
more than store owners, their employer wanted to keep them
compartmentalized, since he'd never been told much about them. When he'd
stayed with them, they'd acted as nothing more than shopkeepers looking
to help out a traveler.

``Floyd!'' called Loretta. ``Good to see you. Would you like to buy a
raffle ticket?''

Floyd smiled at her. She had pretty, blue‐grey eyes. He could imagine
her as a killer, if he tried, but it wasn't clear on her face. ``Well
that depends now, what's the raffle for?''

``Two tickets to see \emph{Anything Goes} in Wichita,'' said Loretta.
She smiled with her eyes. ``A nickel to enter, though a few folks around
here have bought a few entries to increase their chances.''

``Well that sounds lovely,'' said Floyd. ``I think I have a nickel on
me, as a matter of fact.''

It was no surprise to hear that he'd won a week later.

``All the way in Wichita?'' asked Martha when he'd asked her to come
with him.

``We can leave from the church, have lunch in the city, and then see the
show,'' said Floyd. ``I have some money saved up, and I wouldn't mind
spending some of it to show you a nice time. It'll be good to see the
big city.''

``I suppose you're right,'' said Martha. ``I haven't been outside of
Smallville since Jonathan passed.''

In truth, the two of them didn't get along that well. Martha was clearly
lonely. Her husband was dead and her only son was two days away in
Metropolis. In the first week she'd shown him everything that was
required of him, but even after that, she would sit down on a tree stump
with a glass of lemonade and talk at length while he worked on mucking
out the chicken coop or tended to the small garden. Floyd tried to smile
and encourage her. Listening to her stories was half of the reason he
had been hired, but she had a way of rambling on that irked him. She
quite proudly held opinions that might have made her outspoken among the
people of rural Kansas, but were practically pedestrian by the standards
of the people Floyd had met throughout his life. Martha talked about the
exodusters coming to town when she was a little girl, her involvement in
the radical temperance movement, and working the farm with Jonathan
through tornadoes, blizzards, hail, floods, grasshoppers, and droughts.
Floyd tried his best to pretend to be interested, and most of what she
said went into the letters.

Her distaste for pistols aside, the first time they'd really locked
horns was when he brought home a small jar of moonshine. While the rest
of the nation might have recognized Prohibition for the folly that it
was, Kansas had laws against alcohol long before the amendment was
passed, and had kept them in place after it was repealed. The ban was
mostly thanks to little old ladies like Martha Kent. She'd shamed him
for bringing moonshine into the house, told him it was against the law
as if he didn't already know that, and then made him dump it out on the
ground just beside the front steps. The only reason he hadn't gotten his
pistols and shot her three times in the head was the enormous and
ever‐growing amount of money waiting for him once his work was done.

Floyd Lawton was a professional, but the job was getting to him. In the
normal course of his work he would get a job and then spend some time
doing the homework and employing a wide range of skills in things like
lockpicking, disguise, forgery, and so on. The actual murder itself took
a day at the most, and then he'd make his getaway and spend his newfound
wealth on women and booze. His life consisted of long periods of
debauchery punctuated by razor‐sharp focus on a task that had been set
before him. This particular job upset that natural rhythm. For the last
few weeks, he'd been working as a farmhand with no clear end in sight.

When he'd initially arranged for the job, he'd been told that the term
of employment was indefinite, but he hadn't really thought that it would
be so long, especially with the amount that was being put into his
account on a daily basis. Sitting in the woods with a rifle trained on a
cabin for three days was easy for Floyd; this required a different kind
of patience that he wasn't sure he had.

\begin{center}\rule{0.5\linewidth}{0.5pt}\end{center}

The SS \emph{Excelsior} caught fire at nine in the morning. It was a
cruise ship which had taken a recent turn as a ritzy floating restaurant
in order to drum up business for its next voyage. Three times a day it
would pull into the harbor and exchange passengers, giving a large
number of people the chance to experience what a life of luxury on the
seas was like. Sunday was Lois's day off, but her work as a journalist
was never far from her mind, and she didn't have a real affection for
personal time. There was a story somewhere on the \emph{Excelsior},
something that went beyond just the glitz and glamour of it. If there
wasn't a story, then a day of eating fine foods on a fancy ship was a
small price to pay. Luthor was a part owner of the ship, and had paid
her way.

Lois was first alerted to the fire when a crewman hurried across the
dining room. She'd set her fork down and rushed after him, and it was
when she heard the panic in their voices that she began to smile. It
wasn't many days that she got to be so close to a story as it developed.
She was an excellent swimmer, and in the worst case scenario could tread
water for long enough to get rescued, if not outright swim to shore. The
water wasn't too cold, and hypothermia wouldn't be an issue. All in all,
it was a pleasant enough time and place to be on a sinking boat.

It took a full fifteen minutes for Superman to show up, by which point
the electrical cables and hydraulic lines had both been burned through,
leaving the ship adrift and without radio. The \emph{Excelsior} had been
at its furthest distance from Metropolis when the fire started, and just
making the turn back towards the city. If it weren't a Sunday, the ports
would have been busier, but as it was the effort to provide them a
rescue was looking pitiful.

Superman moved low to the water as he came in, splashing up waves behind
him, and entered straight through the side of the burning ship. The fire
was out within half a minute, though smoke and steam still rose around
the ship. The ship was listing to one side, and Lois held firm to the
railing. A number of the lifeboats had been lowered into the water, and
the women and children were being put onto them. Someone had tried to
grab Lois's arm and lead her away, but of course she was having none of
it. She felt a lurch from the keel of the ship, followed by a loud
snapping sound.

``The ship is too damaged for me to move,'' said Superman from just
beyond the side of the ship. He'd moved there so quickly she couldn't be
sure he hadn't been there all along. He stood in mid‐air with his feet
pointed down, and talked clearly and loudly with a rich baritone. Lois
doubted that there was a person on the ship who couldn't hear him.
``Everyone stay calm, the fire has been put out and you're in no
danger.''

The evacuation was neat and orderly, and done with a minimum of fuss. A
small boy laughed and jumped into the water, and Superman pulled him out
and put him on a lifeboat with a stern admonition not to engage in
foolishness. With Superman there, no one really feared for their lives.
Lois heard a man say that it was impossible to die when Superman was
standing next to you.

He landed on the deck next to Lois. ``Do you need assistance Miss
Lane?'' he asked with a half grin, as though nothing had ever passed
between the two of them. Worse, he said it like there weren't hundreds
of people dying all over the world with every passing minute. His mask
was so complete that she almost believed it.

``I can make it to the lifeboat by myself,'' said Lois. She'd been
thinking what about she and Luthor had been talking about of late, and
forced the next words out. ``But if you're heading back into Metropolis
anyway, I wouldn't mind a direct flight.'' She smiled, and could feel
herself showing too much teeth, but Superman smiled back and returned to
helping people into their life boats. When everyone had been evacuated
from the ship and the Coast Guard were on their way, Superman once again
landed beside her and held out an arm towards her. Trying not to think
about it too much, she stepped towards him and allowed herself to be
swept up in his arms.

She'd been sitting in Lex's study two weeks prior ago he'd brought up
the idea.

\emph{You're one of the anchors holding Superman in place}, wrote Lex.
\emph{You need to bind yourself tighter to him, so that he'll listen to
you. He's attracted to you. Use that.}

She made the hand signal for \emph{No}. She and Lex had some two dozen
signals that they used for messages that were too short for paper, a
sign of how long they'd kept up their charade. The book was nearing
completion, with Lex as a full co‐author, and there was nothing close to
a solution for the Superman problem. He'd suggested that they begin work
together on a new book after the one on Superman was done, but Lois
wasn't sure that there was a point in continuing.

\emph{Why?} he signed back.

Lois sighed and started writing a message. \emph{He would know that I
was lying. His senses are too sharp for me to fool him. And I could only
keep it up for so long before he would figure it out.} She paused with
her pencil poised over the page. \emph{It would increase the scrutiny on
me. And I don't like him. He's too powerful.}

\emph{You've criticized me for not doing enough}, Lex wrote back.
\emph{This is a good plan. Scrutiny we can deal with. I understand that
you don't like him, but if you're truly worried about him going rogue,
this is one of the best ways to stop it from happening.}

She and Luthor had gotten to know each other well over the course of
their two person conspiracy, but she still wasn't entirely sure that he
took what she'd told him seriously. He'd expanded his charitable efforts
and began contributing to various legal efforts on Superman's behalf,
but it never felt quite as concrete as she might have hoped. Luthor
wanted to deal with Superman on an ideological or psychological level,
and when she'd told him that wouldn't be enough, he'd quirked an eyebrow
and asked what more they could possibly do. Superman was invincible,
everyone knew that.

She couldn't argue with the logic of providing an anchor for Superman,
but the thought of courting him made her skin crawl. He was strong,
handsome, popular, and powerful, but she hadn't been able to shake the
sense of danger she felt on their first meeting, and after his breakdown
she'd stopped trying to see him in a more favorable light. He was an
alien pretending at being a moral exemplar when really he was much
closer to an ordinary man. Who knew what personality lay in wait behind
the mask he wore? Lois had never had anything resembling a lasting
relationship, but she'd gone on dozens if not hundreds of dates. Some of
the men were creeps right off the bat. With others it didn't become
clear until the third drink, when she'd already begun thinking about the
next date. And just once, the guy she'd been dating was arrested for
beating a woman to death. She'd been dating him for two weeks at that
point, and wouldn't have believed he was actually guilty except for the
fact that she had contacts within the police department who'd shared the
evidence with her. It had taken a long time for her to actually want to
spend time in the company of a man after that.

\emph{It's fine if you don't want to do it}, wrote Lex. \emph{But it's
important to make the distinction between you having a personal distaste
for your involvement and the plan actually being a poor one.}

Lois thought about her objections. Superman could use his incredible
senses to watch a person's breathing and listen to their pulse, but so
far he hadn't shown any real ability to translate that into an ability
to see whether someone was lying. She didn't want to be his girlfriend
or anything else, but it was difficult to argue that humanity as a whole
would be in a better position if Superman had someone that he actually
listened to. Superman already cared about her in some way, and she
already had to assume that he was watching her. There was a risk that
Superman would discover that she was trying to play him, but that came
down to whether Lois was good enough to keep it up. She would just have
to become a better liar.

She didn't give Lex an answer, but had started preparing for the next
time her path crossed with Superman's all the same. And that was how she
ended up in his arms, flying over the Lower Metropolis Bay.

It wasn't so bad as before. He kept the speed gentle and stayed close to
the water, so that if he dropped her it would only be unpleasant and not
instantly fatal. Lois had her arms wrapped around his neck, and pressed
her face against his chest to keep it out of the wind. So far as she
could tell, it was exactly what he wanted. Her fear was still present,
but if any of it showed perhaps he would mistake it for something else.

He set her down gently, near the stretch of river where Luthor's long
mural stood.

``Thank you,'' said Lois. She placed a hand against his chest, and stood
close to him. ``For everything.'' She tried to ignore the people
watching them.

Superman seemed about to say something, then cocked his head to the
side. ``There's a chemical spill down in Dockside,'' he said. ``If
you're ever in need, just call my name.''

And with that he was off, flying through the air towards some new
disaster. Lois's hands were trembling slightly, but it had gone better
than she'd thought it would.

\begin{center}\rule{0.5\linewidth}{0.5pt}\end{center}

``Why doesn't Superman do something about this drought?'' asked Bill
Parker.

Martha Kent always made it a point to go to church early, and Floyd sat
with her. Attending the Zion Lutheran Church was more about community
than religious fulfillment, and Martha never missed a chance to chime
in, no matter the topic of conversation.

``And how would he do that Bill?'' asked Martha. ``He can fly, not
control the weather.''

``Well,'' said Bill. ``Well he could spin around a bunch and pull some
water to us.'' He spun his finger around in front of him to demonstrate.

``He'd be liable to flood our farms if he tried that, and where on Earth
would he get the water to do it?'' asked Martha.

``Lake Superior,'' replied Bill. ``Fresh water, more than we'd ever
need, and he could just funnel it up like that. Five hundred miles or so
ain't nothing to him. And there are waterspouts, ain't there? Same
thing.''

``It wouldn't work,'' said Martha with her arms crossed in front of her.

``Then a canal, say,'' replied Bill. ``We can't take much more of these
dust storms.''

``Sit back and enjoy the clear day,'' said Martha.

``Superman doesn't do hard labor,'' said Pete Ross, who ran the auto
repair place.

``Well we could use a canal,'' said Bill. ``I don't care how we get
it.''

Floyd tried to resist rolling his eyes, and settled in for another
sermon. The pastor was young, and his lessons were obvious by the time
he was three sentences in. Floyd was far from being a religious man, but
he'd always thought that the true meaning of what was said shouldn't be
revealed until near the end, when it all came together and made itself
clear.

After the sermon was over, Floyd waited next to the truck for Martha.
The musical was showing at two in the afternoon, which left them just
enough time to have some lunch in the city. The skies were clear and
blue. Martha liked to talk to the other church goers for a good long
while, and if Floyd owned a watch he would have been looking at it every
few seconds. There was no real hurry though. The whole point of the
operation was for him to keep Martha away from the farm for as long as
possible, and it didn't matter whether she was talking to friends or on
the road.

``The barometer's dropping fast,'' said Martha as she walked towards the
truck. ``There's going to be a dust storm.''

``Skies look clear and blue to me,'' said Floyd with a ready smile.
There wasn't a cloud in sight. ``I'm sure if there's a storm we can take
cover in Wichita better than on the farm.''

Martha shook her head. ``No, the radio says it's going to be bad, and we
can't be out on the open road. Besides that, we need to prepare the farm
to weather it as best we can.''

Floyd thought on that. The Greenes hadn't been in church, which probably
meant that they were already on the farm~--- probably cracking open the
storm cellar. He couldn't very well go back to the farmhouse with Martha
and come across them in some incriminating position.

``Please Missus Kent, I'm sure we'll be fine. Worse comes to worst we
pull over and take shelter in someone's cellar. Folk in Kansas are nice,
I can't imagine that anyone would turn us away.'' Floyd smiled, and
hoped he didn't seem to desperate. ``I've never seen a musical before,
and if we miss this one I think I might never.''

``I know you had your heart set on it,'' said Martha. ``But we have to
go home. If we don't seal those windows the house will be full of dirt,
to say nothing of what's going to happen to the chickens.''

``Alright,'' said Floyd. ``Maybe someday I'll save up enough to go see a
musical all on my own.''

``If it's as bad as I think it is, the theater would be closed anyway,''
said Martha. ``Now let's get going.''

The next step was sabotage. Floyd could choke the engine and then
disconnect some vital part when he popped the hood of the truck to see
what was wrong. He was just about to do this when Martha spoke.

``I've heard some unpleasant rumors, Floyd,'' said Martha. Floyd spared
a glance at her and saw a frown on her face.

``Rumors?'' he asked, though he could guess right away what they were.

``You and that Betty Graber,'' said Martha. ``There's some talk that the
two of you are an item, and I can't say that I could tolerate you living
with me if that's true.''

Betty Graber had made eyes at him from nearly the moment he'd set foot
into town. She was nearly sixteen, and naive enough to think that there
was something romantic about a drifter. More likely than not she thought
she could change him, but better women had tried and failed at that. She
would chat with him whenever their paths crossed, and pretend to be
going in the same direction as he was so they could walk the two blocks
that made up Smallville's downtown together. If he'd been smart, he
would have avoided her, but Smallville had little to offer in the way of
entertainment and booze was prohibited. He'd taken her virginity in a
grassy field, and she'd cried the whole time. Afterwards, she followed
him around like a puppy that was particularly desperate for affection.
It was the very definition of trouble, and it was only after he'd been
with her that he could see that with any clarity.

``There's no truth to it,'' said Floyd. ``She's keen on me, I can tell,
but I would never take advantage.''

Martha said nothing, and Floyd risked a glance over at her. She looked
upset. He couldn't be sure how much she had heard from the gossipmongers
at church, but there was a serious risk that his room, board, and dollar
a week were about to disappear, and that meant in turn that his enormous
salary was going to disappear too. Depending on what Betty had let slip,
there might be some way to salvage things. He was mulling this over when
he realized that he could see the Kent farm ahead of him. He choked the
engine, and popped out of the car.

``I'll see what's wrong,'' he said quickly.

``I'll just walk the rest of the way,'' said Martha. She pointed back
behind her. ``I can see the dust storm on the horizon already.'' And in
fact she was right, to the north the horizon was muddled and blackish
brown. The storm was moving fast.

Floyd could see a large truck on the farm from where they were, and it
was just a matter of time until Martha noticed it too. The mission was
blown, and now it was a question of what his employer would want. The
problem was, he just hadn't been given enough information, because he
wasn't supposed to be anywhere near the operation. The storm cellar had
to be the target. It was a question of whether it would be better to let
the operation be discovered by Martha or better for Floyd to lose his
cover. The fact that Martha was threatening to kick him out made the
decision easy.

Floyd Lawton pulled out his gun. He caught up with Martha in a few short
strides, and smacked her in the head with the butt of the pistol as she
turned to look at him. From there it was just a matter of half a minute
to pick up her light and frail body and set it in the back of the truck.
With some quick work with ropes and a handkerchief he had her bounded
and gagged. He restarted the truck and drove towards the house.
Hopefully his employer would understand.

``What the hell are you doing here?'' asked Joseph Greene as Floyd
pulled up.

``Change of plans,'' said Floyd. ``I'm here to help you two out. There's
a dust storm coming and we need to get out ahead of it if we can.'' That
was when he spotted Loretta beside the large delivery truck they had,
aiming a rifle at his chest. Either he was getting rusty or she was
well‐trained.

Joseph stared at him. ``Where's Missus Kent?''

``Knocked out,'' said Floyd. ``In the back of the truck.''

Joseph swore. ``Alright, we need to move then, quickly. Come this way.''

The moving truck was backed up towards the storm cellar, which had its
doors cast wide open. Beside it was an array of cutting tools. Joseph
stepped down into cellar, and Floyd walked behind him. A small lantern
cast light on the object.

``What in the hell is that?'' asked Floyd.

It looked like a kite had swallowed an enormous egg. There was hardly a
straight angle on it, save for the tips of the wings which were set two
thirds of the way back, and it was easily six feet wide. The metal was
gleaming a dull gold where the dust had been wiped away. There were no
openings or protrusions of any kind, just pleasantly sweeping curves.
When Floyd looked closely at the area that had been cleaned of dust, he
could see that it was tiled in an intricate pattern.

``No idea,'' said Joseph. ``Now come on, no talking, we need to get
moving. Now.''

The three of them heaved at it, and eventually the two men got their
shoulders beneath the two stubby wings and managed to lift it up enough
to start moving it up the wooden steps. They stopped to rest once it was
outside, then with another burst of effort got it up into the back of
the truck. Floyd collapsed against the side of the truck. He'd initially
thought that the object would be unmovable without wrapping ropes around
it and using the truck to pull it, but instead it was just obscenely
heavy.

``Storm's coming,'' said Loretta. A black cloud stretched from one end
of the horizon to the other, hanging low. She walked over to Floyd's
truck and turned to look at him with a frown. ``Take that truck and
drive as far away from here as you can. Keep Martha with you.'' The
Greenes moved swiftly, and were already on the move by the time that
Floyd had gotten the truck started up again. In his rear view mirror he
could see Martha Kent, folded up like a doll. After a moment of looking
at her, he realized that she didn't seem to be breathing. Floyd swore
and hopped out of the truck, but as he reached down to check for a pulse
he could see that he was far too late. Blood had trickled out from her
nostrils and dried in place, and her eyes had gone milky. He swore
again, and got back in the truck.

The storm was a godsend, so far as Floyd was concerned. It would cover
up both the death of Martha Kent and his disappearance from Smallville.
Dust storms didn't usually kill unless they caught you by surprise and
choked you out, but if this one was bad, maybe that's what people would
assume happened. He and Martha would both be missing, along with the
truck, and surely the police would draw their own conclusions. The empty
storm cellar with its doors blown open would only contribute to that.
Floyd's money was held in a bank in Kansas City, and he'd make a
withdrawal before anyone knew what had happened.

Floyd was a dozen miles away from Kansas City when the storm front
caught up with him. Visibility dropped down to nothing, and he kept
going more through the feel of the road than because he could see what
was in front of him. A strong gust of wind hit the truck, nearly sending
it sliding sideways. When Floyd looked back, Martha's body was gone.

The driver's side door flew away in a tumble of twisted metal and broken
glass, and Floyd was wrenched from his seat and flung into the dirt. He
closed his eyes tight and spat out a mouthful of blackened soil. Half a
second later the wind whipped him hard, pulling him up into the air. He
fell, twisting in the wind, for what seemed like a long time. He was
stopped when his shirt snagged on something, suspending him off the
ground. He wiped at his eyes, trying to clear the dirt away. It was only
slowly that he realized he was being held by a man. The dirt wasn't
blowing anymore, because the clouds were now below them, sweeping over
the Midwest like a horde of black demons. He had been thrown up into the
sky and caught by a god.

Superman~--- for it could only be Superman~--- was covered in the same
fine soil that Floyd was. His hair was a mess and his face was caked
with dirt, save for just below his eyes where there were twin streaks of
pink flesh. He was crying. Floyd didn't move, and didn't say anything.
His employer had been taking precautions against the arrival of
Superman, and now Superman was here. The only thing to strive for was
getting out of this alive, and the only way to do that was to convince
Superman to bring him back down to the ground. Superman didn't kill
people, but he wasn't supposed to cry either. Floyd was being held up by
the cheap, dirty fabric of his shirt, which was pressing uncomfortably
against his armpits.

``You killed her,'' said Superman in a voice filled with cold fury.

``It was an accident,'' said Floyd. His voice was hoarse. He must have
swallowed quite a bit of dirt on his way up. ``I meant to knock her out,
not kill her. I just hit her too hard.''

``An accident,'' spat Superman. ``I spend my every waking second
treading lightly, trying not to go too fast, trying not to break your
fragile little bodies. Do you understand how careful I was in bringing
you up out of the storm? How easily I could have broken your bones, or
liquified your muscles? Do you think I have one single \emph{ounce} of
sympathy for you?'' Superman let out a raw and primal scream that left
Floyd momentarily deaf. It was so loud his very bones had vibrated. And
even then, he could tell that Superman had been holding back.

``I'm sorry,'' said Floyd, barely able to hear his own words.

``She was my mother,'' replied Superman.

Floyd had a sudden moment of clarity. He'd had a dozen conversations
with Martha Kent about her son, and all of them had been given their
context. His employer's paranoia now seemed reasonable. There were
pictures of Clark throughout the house, and as he stared at the dirty
and distraught face in front of him, he realized the truth.

``Listen Clark,'' said Floyd quickly.

One of Superman's hands flickered forward and wrapped around Floyd's
throat, stopping the attempt at persuasion before it could even begin.
The pressure was firm but gentle. If not for the other hand still
twisted around and grabbing Floyd's shirt, he'd be choking to death.

``Don't call me that,'' said Superman. He stared at Floyd with hatred in
his eyes for a long moment. Floyd wondered whether this was the end.
Surely Superman wouldn't let him live with the knowledge of his second
identity.

``Is there a point to your life?'' asked Superman. ``Did God have any
purpose behind your creation other than to test me?''

Floyd tried his best to nod. Slowly, Superman released his throat.

``My employer,'' said Floyd. ``I can help you get to him. He never
showed his face, but we have ways of communicating, and there's a bank
account he puts money into.''

Superman nodded. ``Talk.''

\begin{center}\rule{0.5\linewidth}{0.5pt}\end{center}

\emph{We were able to remove the foreign contaminant from the lab's
water supply. The source of it was a large, singular deposit beneath the
surface, which has now been safely separated out. The origin of the
contaminant is unknown, but initial tests have shown it to be somewhat
exotic. In other news, our biological research is going well, but
unfortunately our prized test subject has been injured, perhaps
mortally. We suspect mishandling by one of the other workers in the lab.
While that experiment was originally going to be a double‐blind, we now
believe that some bias may have crept in. With that said, we're proud to
report that our total cycle time is down to just an hour and a half.}

Lex stared at the after‐action report. The storm cellar had contained a
spaceship~--- or something similar enough to it~--- and was now housed
three hundred feet below the ground in a lead mine near Pleasanton,
Kansas. In the next part of the plan it would be encased in a quantity
of refined lead, and from there shipped out to an atomic research
laboratory in Hub City which had been set up far in advance. The man and
woman posing as Joseph and Loretta Greene were long gone, and their
usefulness was at an end, given that Superman might have seen their
faces.

There had been no word from Floyd Lawton.

That Martha Kent was injured and probably dead was troubling. Superman
had tethers to the world, and she was one of them. From what Lex had
been able to find out, Clark Kent had few friends, and none that
extended beyond his employment at \emph{The Daily Planet}. In all
likelihood, Superman now knew that his secret identity was compromised,
which was another point of worry. Events were not yet spiraling out of
control, but if the plan had followed the happy path, managing Superman
would have been much easier.

With the spaceship in Lex's possession, hopefully a solution could be
found before Superman broke free of his moral constraints.

\begin{center}\rule{0.5\linewidth}{0.5pt}\end{center}

\emph{Author's Note: ``Black Sunday'' was the worst dust storm of the
era, and shortly afterwards the term ``Dust Bowl'' was coined.}


%%%%%%%%%%%%%%%%%%% NEW CHAPTER %%%%%%%%%%%%%%%%%%%%%%%%

\hypertarget{a-vast-and-terrifying-enemy}{%
\chapter{A Vast and Terrifying
Enemy}\label{a-vast-and-terrifying-enemy}}

William Calhoun sat in the jail cell, staring blankly at the wall. Ten
years ago, or even five, he might have been trying to plot his escape.
Now he was simply old, fat, and broke. What was left of his money was
going to pay for the lawyers for this current case, but even if he won,
there would be nothing to go home to. His empire had crumbled, and the
last crime boss of Metropolis was soon going to be finished. It had
taken Superman a year.

``I did it,'' said Willie softly. ``I told those men to kidnap the
children, gave them instructions on how to get away without you finding
out, and found a place for them to lay low.'' Superman could hear
everything. Willie found himself talking to Superman often, sometimes
just trying to goad him, but other times confessing his sins. The
Whitman thing turned out worse than he'd thought. He'd picked ruthless,
violent men, and though he hadn't told them what to do, he'd had a vague
idea of what would happen. It didn't sit right with him, now that it was
over. Willie had been responsible for a number of kidnappings, and had
never had any real problems with coercing a man by using his family
against him. This was different. The children were picked because they
would make the news, not because of anything that their father had done
beyond the usual political dickery about being tough on crime.

Willie didn't exactly live by a code, but he had a notion that people
were responsible for their own actions. If a shopkeeper didn't pay
protection money, he got a brick through his window. If a boxer didn't
take the fall when he was supposed to, he got his legs broken. If people
made him angry, they got hurt. Maybe he would have felt differently if
Superman had actually done as he was supposed to and shown his true
colors by killing the kidnappers. As Willie sat in his cell and stared
at the wall, he couldn't help but think that he'd simply done an evil
for no purpose at all. He'd never really thought of himself as a monster
before.

``I did it,'' Willie repeated. ``I ordered those men to do what they
did. People say that you weren't fast enough, but we both know that's
not true. The reason the whole thing happened was that you didn't kill
me when you should have. You're a chicken‐shit, and people are dying
because of it.''

There was no response, but Willie hadn't expected any. If Superman was
listening, he didn't show it.

\begin{center}\rule{0.5\linewidth}{0.5pt}\end{center}

The atomic research facility in Hub City had been built shortly after
Lex had discovered that Superman's x‐ray vision couldn't penetrate lead.
Lead was heavily used for radioactive shielding, and so it made sense
for thick plates of the stuff to be nearly everywhere in the facility.
Of late, Lex had put together a team of scientists to work on creating
an atomic super weapon, though the room in the basement was off‐limits
even to them. The culture of paranoia, suspicion, and obedience to the
rules had been deliberately cultivated.

Lex often considered what it would have been like if Superman had shown
up twenty or thirty years later. This entire operation would have been
done through television, ideally with robotic arms of some kind. It
would have allowed for a degree of anonymity that strongly appealed to
Lex. Unfortunately, the technologies had not progressed to such an
extent that it was feasible, and so Lex was left with two choices; he
could hire a scientist or group of scientists to conduct research which
would be overseen by Lex at a great distance through the usual means, or
he could investigate in person. Given the baffling death of Martha Kent,
the choice was clear. Lex had lost much of his confidence in the ability
of outside parties to carry out their assigned tasks.

He locked the door behind him after he entered the secret room, and
looked carefully at the large crate in front of him. It sat at the
bottom of the lead mine for nearly two days, and spent another two on
the road. If Superman had a way to track it beyond his usual methods, he
had made no effort to steal it back. Lex half suspected that Superman
would come crashing down through the roof to kill him at any moment, but
he was nearly certain that he was just being overly paranoid. He steeled
his resolve and took a crowbar to the crate, opening it up and revealing
what looked to be thick sheets of lead. A small catch at the bottom was
enough to start unfolding the leaden container and reveal the spaceship
inside.

Lex deliberately avoided looking at the spaceship, and instead grabbed
the notes that were strapped to the interior of one of the lead walls.
The agents posing as John and Loretta Greene had been instructed to
leave a more detailed report, one not constrained by the need for codes
and limited in length. He read through it carefully, frowning as he
went. Floyd Lawton was clearly the problem, but it still wasn't clear
what specifically had gone wrong. Floyd had seemingly disobeyed his
orders for some reason that would likely remain unknown. Four days had
passed, and there had still been no word from Floyd, though if he was
using the system Lex had set up the expected time for a message to reach
Metropolis would be nearly that long. Floyd was immaterial either way.
The only thing that a letter might do was illuminate the root of the
problem, but Lex had half a dozen ideas about what he might have done
differently already. He'd picked agents with reputations for being cold,
calm, and meticulous, but apparently that hadn't been enough.

Lex turned his attention to the spaceship. It still wasn't entirely
clear that the term ``spaceship'' accurately described it, given that it
was missing the vents and exhaust ports that Lex would expect to see,
but it certainly made some pretensions towards being aerodynamic, and it
very clearly had stubby wings. The ship was curiously aesthetic in
design, and hewed to the golden ratio wherever possible, which surely
said something profound about the people who had made it. He was eager
to open the ship up, but it would be at least a few days before he felt
satisfied that it was safe to touch, no matter that it had been
physically carried by three people with no special equipment.

The ship emitted nothing that Lex could detect. There was no radiation,
no radio waves, no light, no sound, and nothing else that Lex was
capable of sensing with his various tools. He took a large number of
photographs of the spaceship, and thanks to equipment that had been
ordered long ago, more than half of them were x‐ray photographs that
would allow him to look inside the ship before he did anything all that
dangerous with it. Lex had learned much when drawing up the plans for
Harry Kramer, and the safest way to approach the ship was no doubt to
treat it as a live bomb, despite the fact that it had probably been
sitting in a Kansas storm cellar for twenty years.

The x‐rays weren't powerful enough to pierce through the ship entirely,
but they gave some idea of how its internals were arranged. The skin of
the craft looked utterly seamless, but there were latches and hatches
that had been crafted with ridiculously advanced engineering that left
them invisible from the surface and completely flush with the rest of
the ship. In the center, where the spaceship had a bulge, was a pocket
of complicated engineering surrounding empty space. Presumably this was
where the baby had been pulled from, though it wasn't obvious how the
Kents would have known what they were looking at. Towards the back of
the ship, where an engine would traditionally sit, the x‐ray came back
completely white, blocked by what had to be some absorptive material.

It was only after two full days of looking at it from every angle that
Lex Luthor decided he could get no further without actually touching the
thing. He put on some gloves and began opening up the machine.

\begin{center}\rule{0.5\linewidth}{0.5pt}\end{center}

The hole was easily three hundred feet deep, and even if he could
escape, he'd be hundreds of miles away from civilization. It widened out
at the bottom, which made getting a handhold difficult, but it could be
accomplished by standing on top of the tin cans that held his food and
jumping up to scramble at the rock. Worse, even if he got out, Superman
would simply find him again. He'd get dumped right back in the hole,
with the walls smoothed down more than they already were. Superman had
dug the hole in a handful of minutes, and it would be little trouble for
him to change it.

Floyd opened a can of baked beans and settled in for what he assumed was
either breakfast or lunch. They were far enough north that the small
amount of light coming in through the top of the hole was a constant
twilight, making it nearly impossible to track the time. He was halfway
through his meal when the light dimmed briefly. Superman stood in front
of Floyd, as though he'd been there all along.

``You can't keep me here forever,'' said Floyd.

``Why?'' asked Superman.

Floyd had spent the long hours with nothing to do trying to calculate
the best thing to say. So far as he could figure, Superman really could
keep him there forever. Still, it was worth a shot. ``It's illegal,''
said Floyd. ``You care about laws, right?''

``Less and less every day,'' said Superman. ``I went to visit your
sister in Florida.''

``Look, I told you everything I know,'' said Floyd.

``I'd thought that I'd done my due diligence when you first came to the
farm,'' said Superman. ``There's a real woman living at the address you
sent your letters to, living a mundane life. When she got your most
recent letter, she read it carefully and put it in a pile with other
papers. And that's as far as I watched when you first came to the farm,
because I wasn't paranoid enough.''

``I didn't set any of that up,'' said Floyd, ``I was just given
instructions.''

``My mother was trusting,'' said Superman. ``She had a kind heart. I
told her that I could help her with anything that needed doing on the
farm, but she always liked taking in strays. I watched you, Floyd. In
the first week you were on the farm, I read every letter you sent or
recieved. I watched all of your movements. And you didn't act like
anything other than a drifter. I thought I'd been sufficiently careful,
and eventually you just became a fixture of the farm. I turned my eyes
back towards the city, and very nearly forgot about you.''

Floyd was silent.

``Your sister dropped the letter off at the law office she works for,''
said Superman. ``From there it was translated into a code of random
letters and numbers through the use of a one‐time pad behind lead walls.
Even if I'd been watching closely before I might not have caught it.
They copied it and sent it out to the seven largest cities in the United
States, and from there was transmitted out into the open by radio.''
Superman stared off into the distance past Floyd, but it wasn't clear
whether he was looking through the rocks or just thinking about
something.

``Are you gonna keep me here forever?'' asked Floyd.

``Maybe,'' said Superman.

``You can't be my jailer,'' said Floyd. ``I need food, water, showers,
some actual damned light, and something other than a bucket to relieve
myself in. I swear to god, hand me over to the police and I won't say a
single thing about the other guy.'' Superman had nearly throttled Floyd
the last time the name of Clark Kent came up, but it was difficult to
talk around. Floyd had information that Superman didn't want made
public, and it didn't seem to matter to Superman that Floyd's employer
already knew.

``Do you know why I came here?'' asked Superman.

Floyd shook his head.

``I want to kill you,'' said Superman. ``I want it with every fiber of
my being. I came here because I thought it was important to test myself,
to prove to myself that I wouldn't ever do it because I let my emotions
overwhelm me. And if I slipped up here, no one would have to know. You'd
just be a red smear across the wall in an anonymous hole in the middle
of the Alaskan wilderness. To take a life is evil, but maybe, if it's
necessary, I think it might also be good.''

Floyd watched him closely. ``Are you going to kill me then?''

``I'm still not sure,'' replied Superman. ``I'll let you know when I
figure it out.''

There was a blast of air as Superman launched himself up and away. Floyd
looked up at the rough rock walls that Superman had carved out by hand,
and decided that even if escape was useless, it was better than waiting
for death.

\begin{center}\rule{0.5\linewidth}{0.5pt}\end{center}

The city was unexpectedly quiet, but for once that suited Lois. She sat
on her balcony in one of the wooden chairs, sipping at a glass of wine
and waiting. She was wearing her most alluring dress, a blue one that
clung to her hips. Hopefully it would draw his attention away from how
nervous she was, if he even showed up.

Clark Kent had gotten word that his mother died on Monday, and broke
down crying at his desk. He'd be gone for five days as he went to Kansas
to settle his parents affairs and go to the funeral. At the same time,
Lex Luthor was off at some scientific conference in Hub City. A year
ago, she would have thrown herself into her work, but she was a changed
woman now, and being the star reporter of the world's largest newspaper
just wasn't enough. So she'd done her best to arrange a date with
Superman, because at least that was something.

\emph{Superman,}

\emph{I've been wondering whether you would like to join me for dinner
on Thursday. I live in an apartment building on the corner of 13th St.
and 33nd Ave. E. You should be able to land on my balcony, where I'll
have everything set up. I'll be eating at seven whether you're there or
not, but I'd be pleased if you would join me.}

\emph{Lois}

She'd gone over the letter a half dozen times trying to get the wording
right, and was never quite happy with it. She wasn't terribly good at
turning on the charm, at least not with someone she wasn't actually
attracted to. She'd had any number of brief relationships over the
years, but she didn't fully understand what it was that attracted men to
her. She knew those qualities that she found attractive in herself, but
had no real idea which of them were cause for attraction in others.
She'd tried her best to play up what she thought that Superman would
like. If he didn't show up, at least she would have made an effort.

He made a deft landing on her balcony ten minutes before seven.

``Miss Lane,'' he said with a gentle smile.

``Please, call me Lois,'' she said with what she hoped was a flirtatious
smile. ``I hope you don't think I was too forward inviting you over, but
you don't exactly have a mailing address.''

``I've been meaning to set something up with the post office,'' said
Superman. ``Though of course I think there's some benefit in keeping out
of reach, and I don't think I'd have the patience to keep up with the
flood of mail. You're only lucky that I've been keeping a special watch
over people close to me.''

Lois couldn't help but feel a pang of discomfort at that. She and
Superman were far from close. ``I'm making spaghetti for dinner,'' said
Lois. ``I have no idea whether you eat or not, but I can make enough for
two.''

``I eat,'' said Superman with a smile.

Lois walked into her apartment. It was small, which was the price she'd
paid for having a balcony on the top floor. The place was littered with
souvenirs and photographs, along with a number of framed headlines that
she was particularly proud of. She'd been a reporter for eight years,
and that was enough time to do a great many things and see a great many
places. She had cleaned the night before, for the first time in a very
long time, and was almost proud of how neat the apartment looked. If
Superman had been spying on her he would know how she lived from
day‐to‐day, but she hoped he would take the effort as a compliment all
the same.

She didn't mention the fact that Superman was taking time off from
saving the world to spend time with her, and he didn't bring it up. She
desperately hoped that he had some way to turn his super hearing off,
because the thought of him listening to every single death in the world
with a grin on his face was almost enough to make her physically ill.
Shutting off his hearing wasn't really a solution either though. On days
when she was in a particularly bad mood, she could imagine that she
could hear all of the pain and suffering happening at any given time.
Superman had opened her mind to it, and now it was hard to ignore, even
if she'd never actually heard what it was like firsthand.

``Spaghetti is the only thing I know how to cook,'' said Lois. She had a
pot of sauce and noodles in boiling water, all ready to go. She'd had
more than one man tease her about her lack of domestic skill, and she'd
let them think that she was simply an independent woman of the new mold
rather than let them know she had an actual, unintentional deficit of
skill.

``That's fine,'' said Superman.

Lois served up two plates, and took them back out to the balcony, where
they sat down together. She didn't think she was in any actual danger,
but felt an uneasy tension all the same.

``So when you say that you eat,'' said Lois. ``It's voluntary for you?''

``No,'' said Superman. ``I get hungry, just the same as anyone else. I
can just go longer.'' He dug into his spaghetti and Lois couldn't help
but think that he looked ridiculous in that costume. It was all well and
good to wear a skintight red and blue outfit with a long flowing cape
while you were saving lives, but it just looked silly while he was doing
something so mundane. He looked too human.

``Where do you eat?'' asked Lois. ``You could eat for free at any
restaurant in the city, but so far as I know you never have. And you
disdain money.''

``Just to be clear, this isn't an interview?'' asked Superman with a
raised eyebrow.

``No, just~--- just a date,'' said Lois. She could hear how strained her
voice sounded, but either Superman didn't notice, or he didn't care.
Maybe he just thought she was nervous, which was at least true. At the
same time, she was worried that he would contradict her and gently tell
her that he had no interest in her, which would have been humiliating
given that she was actually trying to use her femininity for once in her
life.

``Distance isn't really a factor for me,'' said Superman. ``I've shared
stew with Mongolian nomads and African tribesmen, and I can hunt and
forage with ease. But I don't really get any weaker from not having
food, it's just a nagging irritation. I once went three weeks without
food just to see if I could, and it didn't seem to make any difference
in terms of strength. The same goes with sleep. I sleep for two hours
most nights, but I could stay up for a month without any real trouble if
I had to.''

Lois took a small bite of her spaghetti, but wasn't really hungry.
``Tell me about your life.''

``My life?'' asked Superman. ``What about it?''

``You're a mystery,'' said Lois. ``Deliberately so, it seems. I just
want to know what it's like to be you.''

``I'm surprisingly boring,'' said Superman with a laugh. ``I wake up at
five in the morning, circle the planet once to make sure that there's
not anything major happening that needs my attention, and then patrol
Metropolis looking for places that I can do good.''

``The city's gotten a lot better with you patrolling,'' said Lois.

``For the most part,'' replied Superman. His face darkened slightly.

``How's the spaghetti?'' asked Lois.

``Good,'' replied Superman. A smile returned to his face. ``It's very
good. Thank you for making it.''

``If you eat\ldots{} I can understand why you have a non‐interventionist
policy, but I don't really understand why you wouldn't take in a free
meal at a nice restaurant,'' said Lois. ``You're basically the patron
saint of the city, but you've never eaten at all of the best places. I
could show you around.'' This was one of Lex's ideas, a way to get
Superman more invested in the city.

``I look ridiculous in the costume,'' said Superman. He laughed as he
watched her expression. ``Come on, you know you were thinking it.''

``A little bit,'' admitted Lois. ``And that's the only reason?''

``I can't go anywhere without people looking at me,'' said Superman. ``I
wouldn't be able to eat in peace. People would come up to me and thank
me for what I've done, or tell me what I should be doing differently, or
try to touch me just so they could tell their friends that they had. And
I'd have to grin and bear it, or calmly explain how I just want to be
left alone. That's not all. It would drive business to the restaurant,
and might have an impact on the other restaurants across the street, so
I'd have to figure out some way of dividing up my meals between the
restaurants that's equitable. And if there was a scandal of some sort,
like the owner of Paulucci's getting arrested for dealing drugs, it
might tarnish my public image and stir up all sorts of controversy that
detracts from the message I'm trying to send. On top of all that, not
everyone would be as understanding as you've been about the fact that I
need some time to myself. They'd draw up charts to show that while I was
eating hundreds of people were dying. You can imagine the headlines.''

Lois nodded, though of course she didn't really understand. She would
have run herself ragged trying to improve the world if she had
Superman's powers. Hell, she had no special powers at all and still
spent nearly all her time working, or thinking about work. But of course
this whole exercise wasn't about what Lois thought, it was about keeping
Superman happy.

``I could arrange something discreet for you,'' said Lois. ``We could
take lunch together on top of the Daily Planet Building. I eat at my
desk or out in the street anyway.''

``I'd like that,'' smiled Superman. He moved his hand across the table
to cup hers, and it was only because she'd been expecting it that she
was able to smile back at him.

\begin{center}\rule{0.5\linewidth}{0.5pt}\end{center}

The trial went quickly.

The last crime boss of Metropolis took on a serious, concerned look in
the courtroom. Of course it was a terrible thing that happened to June
Whitman and her tragically deceased brother, but she was a confused
young girl coerced into testimony by Superman, the alien god who had a
personal vendetta against Calhoun. On the third day of the trial, June
took the stand. She broke down under cross‐examination, and was ushered
out of the courthouse by her father, who shot a withering look at both
Calhoun and Superman himself. Willie didn't figure that he had much of a
shot of winning the trial, but hell if he wasn't going to go down
fighting. If he lost, he had a revolver ready to shove into his mouth.
At his age, with all the enemies he'd made over the years and hardly a
dime to his name, it seemed like a better option than ten years in Sing
Sing.

At night, he said his prayers to Superman. Willie confessed to every
single crime he'd ever committed, and a number of them he hadn't. He
described in vivid detail the things he'd do to Superman if only he
could, and when that got old he moved on to anything else he could think
of. There had to be some way of provoking the alien, something that
Willie could say that would get some reaction. There had to be
something, some set of words that would get the alien's calm stoicism to
crack. His prayers were greeted only by silence.

\begin{center}\rule{0.5\linewidth}{0.5pt}\end{center}

Someone knocked on the window when Clarence had just gotten himself
ready for bed. He nearly jumped out of his skin when he turned to the
side and saw Superman standing on the fire escape. He slowly padded over
and opened the window. He'd seen Superman in court, but up close he was
much more impressive, and more threatening.

``Two days ago you received three hundred dollars to help sway your
position,'' said Superman.

Clarence didn't trust himself to speak. The woman had walked beside him,
and told him to help deliver a guilty verdict. The money had been in his
hands shortly afterwards, without him even agreeing to anything. It was
more than he made in a month.

``I was going to tell the judge,'' said Clarence. ``I was going to
explain to him.''

``No, you weren't,'' said Superman. ``I don't care. You need the money,
I can accept that. But when you go into deliberation, don't let it sway
you. Think about what you've heard in court, and make up your own mind.
Decide the case on its merits.''

``You\ldots{} you're helping Calhoun?'' asked Clarence. ``You hate
him.''

``I do,'' said Superman, not even trying to deny it. ``But when he's
convicted, it needs to be by the books. I promised him that. There are
forces working against him, powerful people with their own agendas, and
if he goes to jail because people with money and influence wanted him
there, that's just as bad as if he stayed out of jail because he
intimidated witnesses and tampered with the jury.''

Clarence nodded along. He would have nodded along to anything that
Superman said at that moment.

``I'm not saying whether you should find him guilty or not guilty,''
repeated Superman. ``I'm saying that your verdict needs to be true to
the laws as they stand.''

Clarence nodded once more, and Superman stepped back from the window.

``Clarence?'' asked Superman.

Clarence choked on his words, and simply nodded once.

``I was never here,'' said Superman. He flew away, a quiet as a whisper.

It was clear now that Clarence should have ducked out on jury duty.
Tomorrow he'd have to go into a room with all the other jurors and
deliberate, knowing full well that Superman was listening to every word
they said. He wondered how many of them were getting visits from
Superman in the middle of the night. It was a long time before he got to
sleep.

\begin{center}\rule{0.5\linewidth}{0.5pt}\end{center}

``The standard of proof that we're hewing to is `reasonable doubt',''
said Clarence.

``If it's reasonable doubt, then we have to return a verdict of not
guilty, simple as that,'' said Louis.

``He's gotta be guilty of \emph{something},'' said Frank with a drawn
out sigh. ``I feel like before today, we were in agreement here. Calhoun
is guilty as sin, it's written on his face. Superman's been cleaning up
town, and Calhoun just wanted to hurt him however he could.''

``It didn't come up in the courtroom,'' said Clarence. ``And we're not
supposed to be reading the papers.''

``Sure,'' said Frank. ``But I don't understand why we have to throw out
things that we know. Sure as shit Superman knows things that he's not
allowed to say, but you can't look at him sitting opposite Calhoun and
possibly think that Superman is making a mistake.''

``We have to do this by the books,'' said Stewart. ``Could a reasonable
person doubt that Willie Calhoun was guilty of these specific crimes
relating to what happened to June and Robert Whitman? Seems to me that
the answer is yes. The whole case rests on June, and I think it's damned
reasonable to question her testimony.''

``She's eleven,'' said Frank in disgust. ``You're calling her a liar
after what she's been through?''

Arlo coughed into his fist. ``Not a liar,'' he said. ``We're spinning in
circles here. The question isn't about the crime, it's about who ordered
the crime, and the evidence doesn't seem to go past the point of
reasonable doubt. I'm not saying the girl is a liar, I'm saying that
maybe she misheard something, or maybe she got confused, both well
possible.''

Frank sat back in his seat and sighed. ``Alright, you fellas want to
take another vote and see whether we're coming to an agreement?''

\begin{center}\rule{0.5\linewidth}{0.5pt}\end{center}

He descended from the heavens like a golden god. There were no strings
or wires to hold him aloft, no jets or boosters, only a simple power of
flight that seemed to defy the laws of physics. The reporters cleared a
space around him as he touched down with perfect grace. His brown hair
was perfectly styled with a curl at the front, as always. Instead of the
trademark half‐grin, Superman wore a scowl.

``Not guilty,'' he murmured only seconds before the doors to the
courthouse opened wide and people began to spill out. The crowd of
reporters around him shoved their bulky microphones in his face as they
heard the news. Lois stood towards the back, not bothering to hide the
worry she felt on hearing the news. Luthor was supposed to take care of
this kind of thing, dammit.

``Superman! How do you feel about Calhoun getting off again?''

``Are you going to catch him again, Superman?''

``What's the point in putting bad guys away if you can't make it
stick?''

William Calhoun strolled out of the courthouse, surrounded by a flock of
reporters all of his own. He wore a brown suit with a bright red tie,
and smiled for the cameras as the flashbulbs went off around him. It
would be headline news. Calhoun spotted Superman only moments after he
stepped outside, and casually walked over.

``Pleasure to be out for a stroll on this fine day, ain't it Supes?''
said Calhoun with a grin. He was only a handful of feet away from the
alien, and the reporters had backed off enough to get a good photo of
the two standing together. The wind picked up, causing Superman's cape
to billow out behind him, and the flashbulbs started going off in
earnest.

``You'll pay for your crimes,'' said Superman. If you ignored the cape,
the skintight suit, and the oversized muscles, you could almost imagine
him as a teacher ready to haul a student out of the classroom by his
ear.

``I'm sure you've heard with those marvelous ears of yours,'' said
Calhoun, his shit‐eating grin never leaving his face. ``I'm innocent.''

``The justice system isn't perfect,'' said Superman. ``I think that's
been made clear today.''

``I bet it just eats you up,'' said Calhoun. ``To know that you got it
wrong, once again. Three times you hauled me in, and three times I
walked free. What is it you got against me? Is it 'cause I'm Irish?'' He
was posing for the cameras in subtle ways. Calhoun, with a smile and a
strut, and Superman, with his hands folded across his chest.

``You're a murderer,'' said Superman. ``A rapist, a pimp, a liar, and a
crook. You are everything wrong with humanity, and they let you go.''

Calhoun put on his widest smile and leaned in close, close enough that
Superman surely could have smelled the man's breath. ``I'm not guilty in
the eyes of the law,'' he declared. ``Chalk one up for truth, justice,
and the American way.''

It happened faster than anyone could see. They said that Superman could
react to lightning before seeing the flash. They said that he could
catch the bullets from a dozen guns at once. He was, by any fair
accounting, the single fastest thing to have ever been on Earth. The
time between when he decided to do it and the time it was already done
could have been measured in milliseconds. Later in the day, one lucky
photographer would develop a picture of the exact moment that Superman
landed his punch, so fast that it was a blur.

One moment Calhoun was taunting Superman, and the next Superman stood
with a single fist held straight out in front of him. It was covered in
blood. Calhoun's head was spread out over the crowd, covering the
reporters with bone and gore, and Calhoun's body fell to the ground with
a soft thud. Superman lowered his fist and then rose up into the sky,
flying away from the shouted questions and the flashes of cameras.

\begin{center}\rule{0.5\linewidth}{0.5pt}\end{center}

Lex worked carefully to pull the large tube from inside the spaceship.
He'd first thought that it must be some special alloy, like the skin of
the ship seemed to be, but the reason that it so effectively blocked the
x‐rays was that it was nothing more than simple lead.

When he was finished, a tube of lead sat on the floor of the workshop.
So far as he could guess, this was a power source of some sort. He'd
already identified the engine analog, though he had no idea how it
worked. The thick cables internal to the ship all seemed to terminate at
the leaden tube, and in another portion of the ship that Lex had only
vaguely guessed the purpose of. The tube of lead was attractive though,
above and beyond anything else. There were few reasons to use a material
like lead, and one of them was shielding.

Lex set up a containment area for the tube, which consisted of little
more than layers of lead to surround it and a Geiger counter that was
wired through to where Lex could read it. The mechanical apparatus took
some time to make, but eventually he was able to rig the whole thing up
so that he would be able to see whether there was any lethal radiation
once the tube had been opened without having to expose himself to it.
Lex worked slowly and carefully, and was eventually satisfied that he
wouldn't get a lethal dose of radiation poisoning. He took away the
layers of lead, and peered closely at what he had uncovered.

The immense promise of this particular part of the ship was that whoever
had built the thing saw fit to include shielding in the first place. If
it was a threat to the infant alien, then perhaps it would be a threat
to the adult alien as well. When Lex looked down at the green glow of
the central core, he could only smile. There was an immense amount of
work still left to do in order to determine the precise nature of the
threat it might pose to Superman and how best to capitalize on that, but
it held definite hope for the future.

It would need a name, of course. Kryptonite had a nice ring to it.


%%%%%%%%%%%%%%%%%%% NEW CHAPTER %%%%%%%%%%%%%%%%%%%%%%%%

\hypertarget{actions-and-consequences}{%
\chapter{Actions and Consequences}\label{actions-and-consequences}}

``He what?'' asked Lex. He gripped the phone closer to his ear, though
that wouldn't help with the poor connection. It was moments like this
that made him want to revolutionize the entire telecommunications
industry. An investment of a million dollars would surely be enough to
get clear audio between Metropolis and Hub City. Of course, the world
was filled with such problems waiting for the right solutions, and
pushing things along too quickly was a waste of money more often than
not.

``He killed William Calhoun, sir,'' repeated Mercy.

``Who knows?'' asked Lex.

``Everyone, sir,'' replied Mercy. ``It happened just outside the
courthouse after the not guilty verdict was handed down.''

``I had hoped they would find him guilty,'' said Lex. His voice was
tight. Mercy was supposed to take care of things in Metropolis while he
was in Hub City. This was the very first time in their long association
that she had failed him, and either that meant she was slipping or
someone powerful was working against her. Both were worrying.

``I know sir,'' replied Mercy. ``I'm still trying to find out what
happened.''

Lex thought for a moment. ``I'll fly home later today,'' said Lex.
``Whatever is happening there needs my attention.''

He hung up without waiting for a response. Early on he'd made a
classification scheme for the most probable scenarios involving
Superman, and so far as he could tell, this was somewhere between class
C and class E. Superman had killed in a public way, which might signal
any number of things: a simple rash decision, a campaign of lethality
against the criminal element, or the opening moves of tyranny. Superman
had given no indication that he knew what Lex was doing, and Lex was
highly unlikely to get caught in the cross‐fire, which was the important
thing for now.

The estimated deaths from a class E scenario were in the thousands, and
while there would be severe economic effects, it was nothing that
couldn't be weathered in the short term. Hopefully the short term would
be all that Lex would need. In fact, a class C scenario might be of some
benefit. If Superman had only killed because he had momentarily snapped,
it was possible it would make the other scenarios less likely, depending
on which model of his psychology was correct. People were hard to
predict though, especially those with alien psychologies and a penchant
for lies. Scenarios of class J and higher involved the effective
obliteration of the human race in some way, but so long as Lex Luthor,
his stores of knowledge, and the spaceship were all safe, it was still
possible that Superman might yet be killed, which meant that anything up
to the murder of hundreds of millions might still allow humanity to
survive.

He still didn't know how kryptonite worked, or what, precisely, it did.
The use of lead as a shield implied radiation of some kind, in addition
to the radiation of green light that gave it the distinct glow. He'd
taken copious notes and photographs as he'd taken apart the ship, and
while the kryptonite definitely seemed to be a power source, it wasn't
clear how that power was generated or harnessed. Lex had made no attempt
to activate the ship, and had no real plans to do so until after
attempts at using kryptonite as a weapon had failed, and then only after
careful consideration of the risks and dangers. The piece of the ship
that seemed to be an engine would be left alone for the foreseeable
future; anything with the power to exceed the speed of light or even
achieve a reasonable fraction thereof was a de facto weapon of
unconscionable power.

The kryptonite was in a solid block that must have weighed nearly twenty
kilograms, which was wholly inconvenient. Lex was hesitant to split it
up into smaller pieces, in the event that doing so would interfere with
its use as a power source for the ship, though at least the lack of
internal padding and shock absorbers suggested that this wouldn't be
dangerous. It was possible that kryptonite by itself held no harm at all
for Superman, and that kryptonite was only dangerous when the ship was
powered on and using the no‐doubt immense amounts of power that
interstellar travel required. Lex had settled for doing experiments on
the exposed surface of the core of kryptonite.

The lead tube that contained the kryptonite was itself surrounded by a
box of lead that Lex had constructed to provide for shielding. He
carefully lifted a cage out from the box, and peered at the rat that had
been living on top of the kryptonite for the past three days. A quick
dissection confirmed what a physical examination had suggested; the rat
was no worse for the extended exposure. When Lex was finished, he tossed
the corpse into a wastebasket with a frown.

Whatever kryptonite was emitting besides light was essentially invisible
to every tool that Lex possessed, but Superman's amazing powers
suggested that there were many aspects of physics that humanity had not
yet discovered. Lex made a snap decision. He put on a pair of thick
gloves and pulled the leaden tube from the box he'd built, and then
carefully pulled the block of kryptonite out of the tube. It came free
on the first attempt with a slight click. The spaceship had proven
remarkably easy to take apart once Lex had gotten to know the tricks to
its design, and he was confident in his ability to put it back together
again. Whoever had built it was an engineer of the highest caliber who
had designed it with serviceability in mind. He was being more risky
than he would normally have been, but time was not on his side.

It cleaved cleanly when he tapped at it with a hammer and chisel.

\begin{center}\rule{0.5\linewidth}{0.5pt}\end{center}

Floyd had a bedroll, a pillow, a bucket filled with his excrement, a can
opener, a large amount of tinned food, and a rain barrel that Superman
refilled every few days. It wasn't much to fashion an escape with. The
hole was three hundred feet down, and curved slightly at the top to keep
rain or snow from getting in. The rock had been smashed through by
Superman, leaving what looked like easy handholds, but a single slip
even halfway up the hole would surely result in death. It was,
unfortunately, wide enough that Floyd couldn't brace himself against
both sides without stretching, which meant it would be difficult to get
a real rest. He looked up the hole for the third time in as many
minutes, trying to plot out a route and not think about how dangerous
and futile the climb was going to be. After he escaped out the hole, a
jaunt through the wilderness and certain recapture would be waiting for
him. He'd just about worked up the nerve to make the first jump up when
Superman came down through the hole, moving at speed.

The blast of wind flung everything into the air, including Floyd. Before
he even had time to react to his meager possessions being slammed
against the walls of the room, a solid hand was against his throat,
pinning him in place. Superman's eyes blazed with anger, and he pulled
back a fist. Floyd flinched back, which under Superman's hand amounted
to little more than turning his head a half inch to the side. When no
impact came, he opened his eyes back up. Superman was breathing hard.
His face was still a mask of fury, and his fist still poised for the
punch. A long moment passed.

``Ykr frkr,'' Floyd tried to say.

A single tear rolled down Superman's cheek. Superman flung Floyd to the
side, and he landed in a heap with what felt like a number of broken
ribs. He coughed, not just because of the hit against the wall, but
because the bucket had been knocked over and suffused the air with a
foul smell. When Floyd looked up, Superman was gone again. Floyd had
been trying to say \emph{Your father}, a last ditch effort at saving his
own life. He had no clue whether it had made the difference. Superman
had wanted to kill him, but hadn't been able to bring himself to throw
the punch.

\begin{center}\rule{0.5\linewidth}{0.5pt}\end{center}

Strangely enough, Lois felt better about Superman now that he was off
the reservation. The anticipation had been the worst part of it all, and
now that he had finally snapped, she found herself calm and focused. The
time for subtle manipulation and walking on eggshells had passed, and
that came as a relief. Actually dealing with a disaster was something
she could handle; it was worrying about the possibility of disaster that
had been destroying her. Or perhaps she was simply too numb to properly
feel dread anymore.

``You're telling me that Superman killed the last crime boss of
Metropolis in the middle of broad daylight, and you didn't get a picture
of it?'' asked Perry White. He leaned over his desk and laid his hands
down on either side of it, looking for all the world like he was about
to vault over the mess of paperwork and personally throw Jimmy Olsen out
of the building.

``I couldn't!'' said Jimmy. ``He was~--- he did it too fast! I took a
picture just before, and the cops started movin' people away just after
that, I swear!''

``Feh,'' said Perry. ``We'll have to send a runner to one of the other
papers and pay out the nose for a picture if we can, because I'm sure as
hell not going to be the only guy printing off extras without the blood
and guts, obscenity laws be damned.'' He turned to Lois. ``You're
getting this story written, right?''

Lois wore a skirt that hung down just past her knees and a white blouse.
Both were splattered with blood on the left side, marking a perfect
silhouette where she'd been standing behind another reporter. She'd
cleaned most of the blood and gore off her face with the sleeve of her
blouse during the taxi ride over, and she'd had to tip the guy extra for
the mess she'd left behind in his backseat, but that was so far down the
list of things to worry about that it might as well not have happened.

``Just tell me how many words,'' said Lois. ``I got the opening portion
of it written on the way over. `Local businessman William Calhoun was
murdered shortly after his not guilty verdict by none other than
---'\,''

``Change that to `Alien vigilante Superman murdered local businessman'.
Maybe add `allegedly' though I don't know how he'd contest that,'' said
Perry.

``We don't need to say alien, everyone knows he's an alien,'' replied
Lois. ``And vigilante is true but harsh, even given what happened. I
don't want to give people whiplash by shifting our position too
quickly.''

``You write it, I'll mark it up and get it to print,'' said Perry.
``You're both dismissed, this is a steaming pile of shit that's not
going to wrap itself up anytime soon. Clark is supposed to come back
today, and he should be able to take some of this off your hands
whenever his train gets in.''

Lois practically ran back to her desk and started typing away, getting
all of her thoughts out before doing a second pass to reduce it down to
something that people would actually want to read. The headline was the
most important part, and the picture after that, but Perry would be in
charge of both of those, so they didn't bear thinking about. Her
typewriter was her steadfast friend, and it clattered loudly as she
jammed down the keys. Lois often felt like she belonged behind a
typewriter. She would take what she'd heard and seen and turn it into a
narrative that people would consume, and eventually that would become
the version of events that people told themselves.

``You okay?'' asked Jimmy. He stood next to her desk, shifting side to
side uncomfortably.

``I'm fine kid,'' said Lois, her fingers never leaving the keys. She
paused a moment and looked down at her clothes. ``My favorite blouse got
ruined.'' She chose her words carefully, not implying any agency. She'd
been training herself to be a better liar and a more careful speaker,
and she could tell that today was going to be a test of that. ``If you
want to help me, and Perry's got nothing better for you, go grab me a
change of clothing from Hudson's. I've gotta be back out on the street
as soon as this is done.'' Lex was in Hub City, and though it wasn't
safe to speak on the phone, she at least needed to touch base with his
valkyrie of an assistant.

She went back to typing, just as fast as before. The big story was
Superman, not Calhoun, but she'd spent the whole day preparing to write
about the outcome of the trial and couldn't help but sprinkle in more
about the man who'd died. Calhoun had no doubt deserved it, especially
if he was the mastermind behind the bombings, but Lois wasn't going to
position herself as Superman's cheerleader. Luthor seemed to want her as
something of a sycophant, but he hadn't yet been able to bring her
around to his way of thinking. Instead, she planned to tell Superman he
was wrong in as persuasive and gentle a way as possible. Luthor could
have words with her later.

``Okay,'' said Jimmy. ``But~\===''

``Jesus Christ kid, are you still here? Go!'' said Lois. She shook her
head as he scurried away. Some people just didn't have what it took to
make it in the news business. The Daily Planet needed a photographer
that could stare down mutilated children and burned out homes. Jimmy
Olsen was a few months away from dropping out, by her estimation, though
she'd thought the same of Clark almost from the time he'd signed on.

She turned back to her article, and tried to focus on the facts. The
verdict of the trial had been a big surprise, and the article she'd been
expecting to write was about Calhoun's slow decline, peppered with his
personal history and an overview of the utter destruction of organized
crime in Metropolis. Calhoun's death~--- his murder~--- changed all
that. Now the story was about Superman, and his failure to live up to
the impossible ideals he had set for himself. It was a story that she'd
been wanting to write for a long time, but she tempered her language.
Superman wasn't going to get a free pass from her, but she would imply
disappointment rather than outrage. Hopefully she'd be able to have some
influence on what the people of Metropolis were talking about tomorrow
(and thus what Superman was hearing), though no doubt the radio was
already having its say.

She dashed off her second draft as quickly as she'd ever done in her
life, and ran it back to Perry's office. Most of the blood on her
clothes had dried from a bright red to a dark brown. She'd felt parts of
Calhoun's skull hitting her face, and thought that she might have a cut,
but the story was done, and that was what mattered. The bone that hit
her had stung, like when a car kicks up gravel that hits your shins,
only in this case the gravel was bone that had been crushed into tiny
pieces. A brief image came to her mind of Superman killing every
criminal in Metropolis, littering the streets with their bones, no piece
left larger than a key on her typewriter. The momentary imagery was
unwelcome. She turned her thoughts back to the matter at hand and
slapped the article down on Perry's desk.

``It's no use,'' he said with a frown. ``We got a gag order.''

Lois grimaced. ``From who?''

``\,`From whom', darling,'' said Perry. ``It came down from on high, in
fucking triplicate. First a call from the chief of police, then a call
from the President of the United States himself, then the only man I
really give a fuck about, our employer.''

``Who the hell do they think they're kidding?'' asked Lois. ``There were
a hundred witnesses, there were cameras all over the place, it must have
gone out over the radio almost the instant after it happened. They think
they're going to keep this quiet?''

``The radio stations went to dead air in about two seconds flat,'' said
Perry. ``Someone had a plan in place for this, or something like it. I
don't think they're trying to contain this thing, just to manage it in
the short term. I've got no idea what they're going to say to make it
better, but for now they're just trying to keep a small bit of control
on the situation. I don't really blame them.''

``This is bullshit,'' said Lois. ``Complete and utter bullshit.'' In the
back of her mind, she wondered whether Lex was behind it. So far as she
could tell, he enjoyed his grand gestures. Giant statues in the park,
vast murals along the side of the road~--- shutting down mass media in
Metropolis would be just his style. It was almost reasonable too, if it
would prevent panic in the streets of Metropolis.

``No argument there,'' said Perry. He leaned back in his chair and lit
up a cigar. ``I'll edit for you, and if they lift the embargo we'll run
the story quick as can be, but until then, maybe you better get yourself
home and cleaned up.''

Lois again looked down at her clothes. ``I already asked Jimmy to go run
and get some for me,'' she replied. ``I'll keep on writing in
preparation for when we're allowed to talk about it.''

When she got back to her desk, there was a note waiting for her, one
that she'd been expecting for the last half hour. She read it twice,
then steeled herself and headed on up.

\begin{center}\rule{0.5\linewidth}{0.5pt}\end{center}

``Why'd you do it?'' asked Lois the moment she stepped onto the roof. Of
course Superman could hear her coming, and she could have asked the
question at any point during the walk up. No doubt if Superman was
coming to speak with her, he'd been watching and listening to her from
the moment that he laid the letter on her desk, and probably from the
moment Calhoun's body slumped to the ground.

Superman didn't respond to her immediately. He stood at the very edge of
the roof, and his cape twirled and billowed in the wind behind him. She
wondered whether he had planned it that way, to look more impressive. He
had a flair for the dramatic, and an eye for looking impressive.

``I was angry,'' said Superman.

Lois watched him. He kept his back to her, so that all she could do was
listen to his voice. ``You've been angry before,'' she said. The image
of Superman standing in a room with the three men who had taken the
Whitman children, waiting for the police to arrive, was so real to her
that she could almost believe she had been there. ``What made this time
different?''

``I've been thinking too much,'' said Superman. ``I've been angry too
often. He was saying all those hateful things, I just\ldots{} it wasn't
that I snapped, really. I didn't lose control. If I'd actually punched
him as hard as I could have, Metropolis would be a smoking crater. I was
standing there, hating him, and thinking about how much better the world
would be if he were dead. Not by my hand, necessarily, but if he'd had a
stroke right on the steps of the courthouse the world would instantly
have been a better place. And to be honest, I was thinking about how
satisfying it would be to kill him.''

Superman kept staring out over the city. Lois waited him out. ``I can
slow down time,'' he said finally. ``Not that, exactly, but my
perception of time can change when I need it to. I wouldn't be able to
catch bullets otherwise. Moving fast isn't enough, you have to think
fast and see fast in order to really make use of the power. When I
really push it, the world dims down and sounds are happening so slowly
that it's nothing more than a persistent drone. I can live out a day in
the pause between words when someone is talking. The world goes so black
I can't even see my nose in front of me. First the sounds become too
long and stretched out to make sense of, and then they stop
altogether.''

His cape flapped behind him. ``I must have spent three days thinking
about Calhoun while we stood there. `Truth, justice, and the American
way', those were his last words. I meditated on them. The State of New
York executed seventeen men last year, and I had a hand in catching
eleven of them. I had vowed not to kill, you understand. The first one
was William Vogel, who was convicted of murder. I watched him spasm in
the electric chair, and I felt like a coward. I could have killed him
faster and more humanely. When I killed Calhoun, he probably didn't even
have enough time to register that I was moving before his brain was a
thick paste.

``I once believed in redemption. I believed in the justice of the legal
system, and when it failed, as it often did, I would tell myself that it
wasn't my place to rush in headlong. I didn't want to be a shepherd of
sheep, I wanted humanity to stand on its own two feet. But when law and
morality contradicted each other, as they often did, I was left with the
cruel alternative of losing just a bit of my moral center or losing just
a bit of my respect for the law. It's like you said; if I had come to
America when slaves were being sold on those very docks, would I have
respected the law then?''

``I didn't mean~\==='' Lois began.

``No, I know,'' said Superman. ``I read the article you just finished
writing. I deserve worse than what you said. I'm just trying to
explain\ldots{} to explain how I got to where I am now.'' He took a
breath. ``I used to think that good was something that was defined by
actions. Don't lie, don't steal, don't murder\ldots{} I thought that if
you worked at it hard enough, you could make up a set of rules to
follow, and that would make you a good person. I think that's what my
father thought. But eventually I moved away from thinking like he did,
and tried to live my life by his values instead. Maybe it was okay to
lie, if it was for the greater good. Maybe it was okay to break the law,
if breaking the law resulted in the greatest good for the greatest
number of people. The change was so slow I barely even noticed it, and I
don't think I was fully aware that I was thinking any differently until
a few days ago. But it didn't really matter, because for the most part I
was living my life the same way. Even if I no longer believed in an
absolute prohibition against killing~--- and how could I when I was
sending men to their death at the hands of the state~--- I still
wouldn't kill because of what it would say to people. I wanted to be a
symbol for them.

``But as I was sitting in the pitch black of slowed down time, I kept
thinking about truth and justice. I'd ensured that Calhoun would have a
fair trial, because I'd promised him he would. What would it say though,
to have him walk free? Not just from that trial, but from the botched
trial in December? He couldn't simply be immune to consequences. He had
to pay for his crimes. That was justice. I kept weighing these things
until I came to my decision, and it was only when my fist was halfway
through his skull that I realized my emotions had their thumb on the
scale.''

``What now?'' asked Lois after a long stretch of silence.

``I need to take time off,'' said Superman. He turned to face her. ``I
know you'll think I'm a monster for it~\===''

``I don't~\===''

``Lois, I know you better than you think I do,'' said Superman. ``I'm
going to take some time off from listening to the vast suffering that I
can only make a small dent in. I'll take time for myself and think on
what I really want to be. I fully expect that you'll hate me for it, but
I can't rush things and make mistakes, not when I have the power to
level mountains.''

Lois stared at him, then nodded. ``I won't hate you for it, but I can't
claim that I understand. If you think that taking some time off from
being Superman is what's best\ldots{}'' She sighed. ``I'll be here for
you when you come back. We all will.''

Superman stepped forward and kissed her on the cheek. He was off in the
air before she could formulate a response to that. He'd said he knew her
better than she thought, but she didn't know what the hell that was
supposed to mean.

He didn't want to be a shepard to the sheep. There was something about
that phrase that bothered her. It sounded familiar. She'd retained
enough knowledge of the Bible to recognize it as an allusion. It was
another way in which Superman showed a solid grasp of the culture he'd
married himself to, not just idiomatically correct English far beyond
what any foreigner learned, but cultural allegory. Certainly Superman
had been compared to Christ on enough occasions for him to recognize it,
but there was something else that made it stick out. It was like
something Clark would say.

She was halfway down the utilitarian stairwell that led down into the
building proper when it struck her. It wasn't just the turn of phrase,
it was the entire conversation. Clark's father was a pacifist, she
remembered him saying that over drinks one time. In fact, save for the
fact that Superman hadn't once brought up religion, all those words
could have come from Clark's mouth instead. A long, slow turn from faith
in the goodness of humanity~--- that was the story of Clark's time in
Metropolis.

She remarked on the physical similarities a few times. It was almost
always a joke at Clark's expense. ``Hey Clark, you spend some time on
athletics and you might rival Superman.'' But you couldn't look at Clark
and actually think that he was much more than an oaf. He had hidden
depths, but those depths weren't nearly so deep that he could actually
be~--- that he could have~---

Lois sat down in the stairwell and put her hands on her knees. She was
trembling slightly. ``Some time off from being Superman,'' she'd said.
They had the same eye color, the same hair color, and close to the same
height. They had a similar infatuation with her, and Superman treated
her as familiar because\ldots{} because she was familiar to him. They'd
sat side by side for months before Superman had shown up. They'd talked
about almost everything under the sun while putting together their
stories, and they'd certainly read almost everything the other had
written.

She needed to make notes, but Superman could watch her. Even now,
sitting in the stairwell, he was probably watching her and wondering
what she was thinking. Some distant part of her brain was telling her to
think up a cover, so she ran her hands through her hair and muttered
``He really did deserve it.'' Calhoun would provide a cover while she
tried to work through every conversation she'd ever had with Clark Kent.

She remembered Clark flinching when she'd said something\ldots{}
something about Superman not being totally emotionless when he came
across a scene of brutal violence. Clark hadn't been flinching because
he was a naive Midwestern farm boy, he'd been flinching because he'd
been \emph{remembering.} He was unreliable because he had other
obligations. He wasn't lucky, he was able to see through walls and
listen in on conversations that happened on the other side of the city.
He used a lot of unnamed sources. He covered Superman's trials. The span
of the deception that it would have required was breathtaking, shocking
enough that Lois had to remind herself to breathe. But it was true.

Clark Kent was Superman.

A thousand small details came sliding into context, and a hundred
questions followed in their wake. Lois clenched her hands into tight
fists. She could feel tears in her eyes. The biggest argument against
the theory was that she was \emph{smarter than that}, dammit, and if you
refused to believe something because it would mean that you were the
biggest idiot in the world, well, that alone said something about how
smart you really were. She'd been played. Clark Kent had lied to her
face for a year and a half, over and over. And Superman had done the
same.

``Calhoun deserved it,'' some small part of Lois remembered to say.
Superman was watching, always watching. And if he really was Clark, if
she hadn't simply gone insane\ldots{}

A few weeks after Clark Kent had first shown up in Metropolis, Lois had
taken a rare break from twelve hour days and gone out drinking. She'd
met a sailor at one of the dockside bars, and taken him back to her
place to do a few things that good Catholic women weren't supposed to do
outside of marriage. In the morning, she'd shoved him out the door and
gone into work. It would have been very hard to miss the fact that Clark
was in a bad mood, and that this bad mood was directed towards her.
She'd thought perhaps he'd seen her in the club while she was hanging
off the sailor's arm, and had simply pretended that she couldn't tell
what was bothering Clark. Now she had to wonder whether Clark was
watching her the entire time, or listening to that particular night of
passion. She'd imagined Superman's eyes on her frequently, and it was
never a pleasant thought, but if it was Clark watching her undress,
watching her with other men\ldots{} she could feel tears streaming down
her face.

There remained the question of why he would do it~--- why, if he had the
power to fly through the air and could crush coal into diamonds, he
would ever want to spend a single solitary second as Clark Kent. The
answer had to be that he was a monster. She flitted through her memories
of Clark Kent the reporter, and found one of him hunched over the
paperwork for his taxes. People had been dying, and he'd been
\emph{filing his fucking taxes}. He'd sat in on boring meetings about
style standards while people literally burned to death. And he'd lied to
her, over and over, every single day that they'd worked together. He'd
cheated his way to the top without any remorse. A cold fury ran through
her veins.

She stood up, wiped away her tears, and looked at herself in her compact
to assess the damage. Her hair still had blood in it, and that did
nothing for the effect. Lois did her best to fix what she could, then
smoothed out her skirt, took a deep breath, and walked out into the
newsroom.

Clark Kent was standing there, speaking with Perry. If she'd been less
angry, she might have run across the room to start beating him with her
fists, but she was far beyond that now. Instead, she gave him a weak
smile and sat down at her desk. There was no hiding that she had been
crying, but she could keep up a front for now.

``Are you okay?'' asked Clark as he sat down at his desk.

``Fine,'' said Lois. ``There's been a lot going on around here while you
were away. How was Smallville?''

``Small,'' replied Clark. ``I'd forgotten how small. My mother will be
missed. Listen, are you okay? I heard what happened, and I know you were
in the thick of it.''

``No, I'm not okay,'' said Lois. ``Are you back to work now? Because I
think I need to lay down for a while.''

``Sure,'' said Clark. He gave her a gentle smile that made her want to
stab him through the throat. He was Clark the deceiver, with a
sympathetic smile like he hadn't been the one to ruin her blouse with
Calhoun's blood.

Later, when she was safely at home, she took a shower and tried to keep
from imagining Clark staring at her with an infatuation that had once
been merely annoying. Clark Kent and Superman were both lies, and put
together they were an abomination of a person that pretended at
humanity. Lois could almost understand the anger that Calhoun must have
felt, and the desire to hurt the creature's feelings in lieu of being
able to damage his physical form. Lois wouldn't be so stupid as to give
into that temptation.

Instead, she would have to persuade Lex that something needed to be
done.

\begin{center}\rule{0.5\linewidth}{0.5pt}\end{center}

Lex sat in his study, trying to keep himself as current as possible with
the newest developments in atomic research. His cover story for working
with kryptonite was that he was doing experimental work on a potential
weapon, and the best way to ensure that a deception was believable was
to make it real. Atomic weapons were certainly coming, and a worrying
prospect in their own right depending on how much power they would prove
capable of harnessing and how easy it would be to refine the necessary
materials. Lex had been quietly buying up uranium mines for a few years
now, but if the technology developed he had no doubt that he would end
up having to lock horns with the various governments that might lay
claim to them.

``Lois Lane to see you, sir,'' said Mercy.

``Wonderful,'' said Lex. ``Send her in, if you please.'' The study had
been cleaned of anything remotely incriminating before he left for Hub
City, and the atomic research was nothing that Lois would be able to
make sense of.

``It looks like we may have to cancel the book,'' said Lex as Lois
walked in. There was an actual book, with actual chapters, but it mostly
served as a plausible cover for passing notes to each other. Lex's
supposed role in the authoring of the book had grown as the months
dragged on, and now he was a full co‐author. Half of the time when they
spoke of the book out loud, it was in code. In this case `cancel the
book' had an equivalent meaning to `stop our covert attempts to
manipulate Superman's mental states'.

``Or at least write a new chapter,'' said Lois with a nod. ``Did you
read the paper this morning?''

``You spoke with Superman following his\ldots{} unfortunate decision,''
said Lex. ``And he'll be leaving us for a while.'' He wasn't sure how
much he believed it. Superman was a meddler by nature.

``Yes,'' replied Lois. She pulled a notebook from her purse, and began
writing on it. ``The book was always intended as a living book, but of
course we won't be able to release it with things as they are now, and
we don't want it to be obsolete the moment it hits shelves. I do have
some thoughts for what we'll need to change in light of this new
information though.'' She turned the notebook towards him. \emph{Is
there still no way to stop him?}

He watched her carefully. He had not, as yet, given her any rope to hang
him with. So far as she knew, she was the one in the lead, and Lex had
only used his immense resources in ways that conformed to the moral
standards of society.

``The science of Superman hasn't changed one bit,'' said Lex. ``Those
chapters will need the smallest amount of work, I should think. Public
reaction will need to be rewritten entirely.'' He made the hand signal
for \emph{No} in order to underscore his point, then the hand signal for
\emph{Why?}

``True,'' said Lois. She began to write again. ``I'd still like your
help, if you have the time. Even if Superman stumbles from time to time,
we can still use him as an example to live our lives by. Do you agree?''
She turned the pad towards him again, not missing a beat. \emph{Superman
has a secret identity that I am in close contact with. I need your help
in figuring out a way to stop him before he kills again.}

Lex very nearly froze. It wasn't what he had been expecting. He had long
thought that Superman or Clark Kent would eventually reveal the truth to
her, perhaps after she had taken the courtship with Superman far enough.
Lex had almost told her himself in order to prepare her, but it would
have opened up too many questions about how much he had known and for
how long.

``Certainly,'' said Lex. ``Though I'm not entirely sure that it's within
my area of expertise.'' He gave the hand signal for \emph{Tell me more}.
It was still too early to take any concrete actions, especially when
events were in flux, but he was already planning how he'd use Lois to
slip Superman the kryptonite.

\begin{center}\rule{0.5\linewidth}{0.5pt}\end{center}

\emph{Author's Note: Sorry about the delay. The next chapter should land
on July 19th, and will be the final chapter of this story.}

\emph{It usually goes without saying, but these characters have their
own views and biases which are distinct from my own.}

\emph{As always, I appreciate the favorites / follows / reviews.}


%%%%%%%%%%%%%%%%%%% NEW CHAPTER %%%%%%%%%%%%%%%%%%%%%%%%

\hypertarget{finale-part-1}{%
\chapter{Finale, Part 1}\label{finale-part-1}}

``And he's just gone forever?'' asked Jimmy. His girlfriend Eleanor sat
beside him, opposite Lois and Clark. It was somewhat emphatically not a
double date. Lois was trying her best to shift her position on Clark in
a way that he would actually believe. Eventually she would pretend to
see the light, or give him a chance, and they would presumably have a
relationship built on a foundation of lies. It left a sour taste in her
mouth, but since Superman was impervious to physical damage, he needed
to be anchored to the mortal world.

``It won't be forever,'' said Lois. ``He'll probably come back, once
he's figured some things out.''

``It's been a week,'' said Clark. ``Maybe what he'll figure out is that
he just doesn't want to help people anymore.''

Lois looked at him. A full week had passed, and it was still hard not to
marvel at how completely she'd been duped by him. Her pride was only
slightly salved by how much effort Clark seemed to put into it. He'd
changed everything about himself to put on the Clark Kent persona, and
there were a thousand subtleties to the performance that she hadn't been
consciously aware of seeing. Everything about Clark was a lie, only
there to fool people.

``He still cares,'' said Lois. ``He still considers himself an American,
I think. And there's been a lot less backlash than there could have
been.'' That had been thanks in part to the media embargo that had let
Lois get a head start on influencing public opinion. Superman had
powerful friends too, not least of which was the governor himself. There
had been no attempt to put out a warrant for Superman's arrest, and so
far as Lois could tell, no one was seriously considering trying to stop
him aside from her and Lex. A fair number of people even seemed to think
that Superman had done the right thing.

``You're the expert,'' said Clark. He shot her a smile, reveling in a
joke that he thought only he could understand. When she thought back
through all their conversations, she could see that he peppered in these
winks and nods to the truth, though he never said anything that
Clark‐the‐ordinary‐reporter‐with‐no‐secrets wouldn't say. It reminded
her of playing the game of double meanings with Lex. That was different,
because at least they were both in on it, and there was a point to it
other than gloating.

It did cross her mind that she was being uncharitable to Clark. Of
course she couldn't actually tell what his smiles meant, and it was just
as likely that he felt a fondness towards her extensive reporting on
him. But since Superman thankfully couldn't read minds, and since Lois
had to keep up a front at all times, thinking mean thoughts about Clark
was a form of private rebellion, and helped her to keep her sanity.

After dinner had wrapped up, the couples went their separate ways. Jimmy
had been dating the same girl for a while now, and things were getting
serious between the two of them. It made her unexpectedly sad, since all
the future seemed to hold for her was a sham of a relationship with
Clark, and abstinence from any meaningful~--- or even meaningless~---
romances for fear of how he'd react.

``Can I ask a delicate question?'' asked Clark. They walked together
down the city streets. Even after being worn down by the city and losing
some of his innocence, he was still a gentleman, and had insisted on
walking her back to her apartment. It was questionable how real the
transformation had been in the first place, since he had been Superman
all along, and how could he have had any innocence left when he could
see how big of bastards people were to each other?

``Out with it,'' said Lois.

``When I asked you out, why did you say no?'' asked Clark.

``A delicate question indeed,'' said Lois. She let silence settle on
them while she thought about it. To his credit, he made no attempt to
rush her. ``You'd been working at the paper for two weeks,'' she finally
said. ``I'd been working at \emph{The Daily Planet} for eight years,
since I was a teenager, and you were far from the first of our coworkers
to ask me out. I care about my job, Clark. Dating someone from the
office~--- it doesn't matter who~--- would be a recipe for professional
disaster. Even outside the office I have to think about whether my
relationships are going to be kosher. If I went on a date with a
politician, people would start saying that I was sleeping with him to
get a story. I can handle the rumors that crop up just from being in the
public eye, and the way people talk when they see a woman in a position
of power or authority, but I'm not going to invite more of it on myself,
and I think I would always have a small voice in the back of my head
that said they were right if I stepped over some abstract line.''

She took a breath. All that had been true, but it wasn't all that needed
to be said to Clark. She'd been preparing for this conversation in one
way or another for the last week, and she was grateful that she hadn't
been the one to start it. ``And I didn't like you, not when you first
came on. You're different now. You've changed. I'd thought that the city
would chew you up and spit you out, but you didn't end up going back to
Smallville, you stuck with it and persevered. You're a better reporter
now too, someone who doesn't just rely on~\==='' \emph{being able to see
through walls and listen in on private conversations} ``--- luck.'' The
pause had been barely perceptible. ``You'd better not hold this over me,
but I respect you now.''

She'd expected Clark to grin, but he only nodded. ``You've changed
too,'' he said. ``Especially after Superman showed up. You said that you
didn't want to invite rumors, but with him you just set that rule
aside.''

Lois stared at him. ``Clark, you can't possibly be jealous that Superman
and I~--- no, it's ridiculous.'' Nothing had ever happened between her
and Superman, they'd gone on a single date together, and Clark
\emph{knew} all that. It was true that her attempts at playing the role
of Superman's girlfriend had been painful from a professional
standpoint, but Clark had no reason to be upset with her. They were the
same person. Unless the problem lay somewhere else, in which case Lois
thought she knew what to say. ``Clark, do you know what I liked about
Superman?''

``Past tense?'' asked Clark. ``You do know he might be listening in,
right?''

The gall it must have taken for Clark Kent to say that almost left Lois
impressed. She knew if she tried to have a conversation about Superman's
eavesdropping she'd be liable to go incandescent with rage, so she
skipped right past it.

``What I liked about Superman is that he was kind and gentle,'' said
Lois. ``He was good. And he liked me, even though I'm not very
likable.''

``You're likable,'' said Clark quickly.

``No, I'm really not,'' said Lois. ``I'm opinionated and hot‐headed, and
I like to push people's buttons. I work more than anyone really asks me
to, I stick my nose where it doesn't belong, and I turn every tragedy or
triumph into a story for consumption by the masses without really even
thinking about it anymore. People die and I think about what the
headline is going to be, and part of me knows that's just a way of
shielding myself. I know my good qualities, but likeability just isn't
one of them.''

``You're intelligent, driven, principled~\===''

``Clark, I said I know my good qualities,'' said Lois. She had to wonder
whether any of that was what had attracted him, or if he'd simply caught
a glimpse of her legs and worked backward from there. And with Clark it
was always possible that it was another lie. ``We've gotten off on a
tangent, but what I was trying to say is that he liked me, and I liked
him, and the rules I'd set down for myself seemed really arbitrary. If
Superman had picked me out of all the reporters in the world~--- hell,
all the women in the world~--- then maybe I was just being obstinate
about how I wanted to be seen by the people around me.'' The lies
spilled out easily, but the next part would be harder. ``It wasn't that
I broke the rule for Superman, it was that Superman made me see that it
was a rule worth breaking.''

They had reached her apartment building. Lois turned to look at Clark.
``Look, I don't know whether you still feel the same way about me, but I
do like you Clark. And if you asked me out again, maybe my answer would
be different.''

\begin{center}\rule{0.5\linewidth}{0.5pt}\end{center}

Creating more kryptonite proved to be a challenge.

Lex took a minor risk and shipped portions of the kryptonite to two
different facilities which both operated as part of the Scientific and
Technological Advanced Research Labs. Robert Meersman had wanted to
create a series of research laboratories which were disconnected from
any corporate or governmental interests. Lex had quickly seen that the
end result of such a philosophy would surely result in either collapse
at worst or organizational drift towards the very same set of problems
which it was trying to escape at best. A combination of money and mild
coercion had put S.T.A.R. Labs in his pocket, though few people knew the
source of their funding, and fewer still knew the primary beneficiary of
their research.

The kryptonite was given the name PU‐356. It had supposedly been found
in the core of a meteorite, and transferred for analysis shortly
afterward, all of which was backed up by a trail of falsified
documentation. It was the work of the labs to analyze the PU‐356 into
its component pieces, generate a full list of its properties, and then
attempt to make more of it. It was semi‐crystalline in nature, and after
only a few days of work it was suggested that more might be made by
introducing a shard of it into a super‐saturated solution which
contained the composite elements. It wasn't entirely clear how the
structure of the PU‐356 produced the properties that it did, but the
elements which made it up were eventually sorted out, and a multitude of
experiments were run to achieve synthesis.

This was part of the reason that Lex had taken the risk of shipping the
kryptonite outside of his immediate control; it would have taken him an
enormous amount of time to arrive at the proper solution to creating
more kryptonite, which involved enormous amounts of energy, a small
shard as a catalyst, and a wide variety of purified elements in very
precise quantities. The process was slow and costly, but more of the
kryptonite was produced with every passing day.

On those nights when he knew that Clark Kent would be occupied with Lois
Lane, Lex began a slow renovation of his house.

\begin{center}\rule{0.5\linewidth}{0.5pt}\end{center}

Before she'd known that he was Superman, Lois had imagined Clark Kent's
apartment as being relatively bare, with little more than a picture of
his mother and father and a large cross. After she'd seen the truth,
she'd imagined only a more extreme kind of minimalism; no toiletries, no
toothbrush, no food in the cupboards or any other sign that a real
person lived there, because no real person \emph{did} live there. The
world was Superman's playground, and he had no real needs beyond those
he decided to indulge in.

She had been wrong. Clark's apartment was slightly smaller than her own,
but just as packed with mementos, curios, and pictures. Where Lois had
accumulated souvenirs over a lifetime of travel, Clark had instead
pulled in pieces of Metropolis. It wasn't just the photographs that
lined the walls, there was a collection of bric‐a‐brac on top of one of
the short bookshelves; a model of the Emperor building,
three‐dimensional map of the city made with pressed tin, and a signed
baseball among others.

``Jimmy took most of the pictures,'' said Clark. He seemed nervous,
though there was no way of knowing whether that was his usual act or
whether he was actually tentative about letting her see how he lived.

Her anger was starting to fade, which was a problem. It had been five
weeks, and though she still felt hot sparks of rage, it was hard to stay
as angry as she'd been in the beginning, especially when she was wearing
a layer of deception over her feelings. Lying to Clark day in and day
out meant building up an image of how she would feel about him if she
didn't know, and almost by definition that meant some level of empathy.

She'd dealt with a number of battered women in her time, either as part
of covering a story or through one of the social programs she was part
of, and she had always found it puzzling that they would sit there with
a black eye and say that their husband or lover had done nothing wrong,
or that it wouldn't happen again. It was a lie, but it was one that they
were able to convince themselves of. She'd never thought that she would
be a woman like that, but now that she and Clark were courting, she
could see it happening to her. She would tell the lies so much that she
would start to believe them, because the alternative was making Superman
upset. And if she tried to get help, she would be laughed off and
alienated, and of course it would only make him angry. There would be no
escape.

She could imagine Clark hitting her. She could imagine his fist going
straight through her skull, pushing aside bone and flesh like it wasn't
even there, just like he'd done with Calhoun. She kept more than enough
secrets from Clark, and a few of them might set him off. She could have
stopped meeting with Lex, but that would mean giving up hope that
Superman could be brought down to mortal levels. She was willing to give
up her personal happiness if it meant keeping Clark pacified, but she
had to know for sure that there was no other way~--- some more permanent
solution. Lex had not yet declared that it was hopeless, but when he
did, Lois would focus all of her efforts on being a good girlfriend, and
eventually a good wife.

``Do you like it?'' asked Clark.

``It's not what I expected,'' said Lois. A glint of metal in the corner
caught her eye, and she walked over to stare at it. ``You can't possibly
expect me to believe that you play the saxaphone.''

``No,'' said Clark with a bashful smile. ``I bought it thinking that I
would learn, but it turns out that I don't really have an ear for
music.''

She had to wonder whether that was another bluff. Once you knew that
Clark Kent was a disguise, it called into question everything he said.
His appearance was a lie, the thick glasses most of all. His apartment
was clean, in a way that suggested that it wasn't always so pristine,
and she wondered whether that was another piece of the elaborate
deception he'd woven for her. She felt a flash of anger coming on, and
did her best to divert it.

``My father made me take harp lessons,'' she said. ``He must have
thought that it would make me more ladylike, but I hated the harp and
never practiced. After I gave up, he kept the harp in the living room,
and it was like an albatross around my neck. And we moved a lot, you
remember, so for years my father just carried the harp with us from
place to place.''

``I'd be interested to meet him,'' said Clark with a frown.

``It probably won't happen, at least for a while,'' said Lois. ``He got
pulled out of an early retirement to work on some secret military
project. He wasn't the best father, but he trained my sister and I well,
and I think we're tougher for it. He wanted boys, and didn't get them,
and on top of that he raised us alone.''

``We have that in common,'' said Clark. ``Not being raised alone, but
unconventional childhoods.'' He tapped a photo of Martha and Jonathan,
which held a place of privilege on his wall~--- the only piece of his
life in Kansas that was visible. ``They were too old to be raising a
child, by most people's estimations. Sometimes I think everyone has
their own story that's just as unique and interesting as your own, if
you could only get to know them.''

Clark made a dinner of stuffed chicken and mashed potatoes. It wasn't
really a surprise that he was a better cook than she was, since he would
almost have to be, but it still irked her just a bit. The thing was,
there wasn't really anything wrong with Clark if you could subtract out
the Superman business. If he were truly, honestly Clark, he wouldn't be
so bad, especially given the ways that he'd changed over the past year.
He had actual stories to tell now. He was kind and courteous, and he'd
left the naiveté behind him. Most of all, he treated her like an equal,
despite his infatuation. There had never been a moment in their time
working together when she felt like he was dismissive of what she was
saying, which was more than she could say for any other man that she
worked with, except perhaps for Perry. Lois could practically feel the
part of her that wanted to believe that she'd been wrong about him being
Superman. If it was all just a bad case of paranoia, and Superman was a
separate person that just looked like Clark despite all the other
evidence, it wasn't like she and Clark would live some idyllic life of
marital bliss, but at least she could see how she would find him
compelling, and possibly even attractive.

But no, Clark was an unrepentant liar. She wasn't sure whether he was an
alien that had forged a human identity for himself or a farm boy who had
developed astonishing powers, but it didn't really matter much either
way. He was cold and callous, and sat by while bad things happened in
favor of reporting on the news in the least efficient possible way. Lois
wasn't terribly religious, and much of it had to do with a conversation
she'd had with a priest when she was eleven years old about why God let
bad things happen. Most of the same arguments applied to Superman, even
if he wasn't perfectly omniscient and omnipotent. When seen through the
new lens of Clark Kent, it was possible to imagine that he'd never cared
about doing the most good at all. Being a symbol for the people
coincided with getting the highest amount of public acknowledgement, and
that seemed a little too convenient. It was easy to look at Clark as
Superman and think that it had all been about his ego all along.

After dinner they sat down on his couch together and listened to the
radio. After debating it for a few minutes, Lois yawned and then curled
up against him. It was a momentary shock to remember that he had the
same hard, defined muscles that Superman did, but she tried her best to
play the oblivious girlfriend that Clark wanted. The show that Clark had
picked out was a fanciful bit of science fiction about a man meeting
aliens on the surface of Mars, which didn't really hold her interest.
Lois slowly fell asleep against the Man of Steel.

The radio show ended and the commercials started up, which was when Lois
began to wake up. Clark leaned forward and shut it off. He turned
towards her, cupped her chin in his hand, and kissed her. His lips were
soft, and if it weren't for the thought that his hand could crush her
jawbone in a heartbeat, she might have actually enjoyed herself. He
wasn't awkward and fumbling like she had thought he would be, just calm
and tender.

When Clark backed away, he looked sad. ``How long have you known?'' he
asked.

Lois swallowed. She was still sleepy, but she knew this wasn't good.
``What are you talking about?''

``I kept waiting for you to slap me across the face,'' said Clark.
``From practically the moment I put on the suit, I was waiting for you
to figure everything out and\ldots{} I don't know what I thought that
your reaction would be. I guess I thought you'd be angry with me, but
I'd hoped that you would help to keep my secret.'' He sighed. ``Lois,
how long have you known?''

She wanted to deny it, but it was clear that wouldn't do any good. A
surge of fear was working its way through her brain, clearing up her
thoughts. ``Since just after you retired Superman,'' said Lois.

``Ah,'' replied Clark. He took off his glasses and set them on an end
table, then laid back against the couch. Some of the Clark Kent
posturing faded away. ``And that's why we're dating now.'' It wasn't a
question. Lois kept herself very still. ``I feel like I've made a mess
of everything.''

``You haven't~\===''

``Stop,'' said Clark, and so she did. ``I love you Lois. One of the
things that I always loved about you, right from the start, is that you
never held back. You said the things that other people kept to
themselves. In Smallville people talk in circles and hide barbs in their
words. My mother~\==='' His voice caught. ``My mother always disliked it.
You'd ask to borrow a cup of sugar, and they'd happily give it to you,
and then afterward they'd complain about the inconvenience. It was worse
for me, since I could hear all of the words said in private. But you
were never like that. You talked to artists, urchins, and politicians
all the same. There was an honesty to you, I guess. And then Superman
showed up, and you were different. It took me so long to see. Here was
someone that you were actually scared of, someone that you had to watch
your words around. You lied to him~--- to me. Even your affection was a
lie, because you were scared. So please, no more lies. We need to have
it out, one way or another. If you hate me, I need to know.''

Lois watched him carefully. She took a few moments to consider. Clark
already knew that she had been lying to him, and nothing short of the
truth~--- or at least \emph{a} truth~--- would convince him. ``Do you
really want that?''

Clark nodded.

``You're squandering your power,'' said Lois. ``You're invincible, and
people are \emph{dying}, and you're just\ldots{} sitting here. If I had
your powers, I wouldn't stop for a single instant. Lying to everyone
around you is one thing, and killing a man in cold blood was another,
but what I can't stand is that you're so indifferent to the suffering of
the world.'' Perhaps it was more than he wanted to hear, but he had
asked for the truth, and she hoped that he could hear it in her voice.

``You don't see the hypocrisy there?'' asked Clark. He was perfectly
calm, and it was hard to see whether that was another mask. ``People say
that all the time. They claim that if they had infinite power they would
protect the weak and heal the sick. And then they eat out at fancy
restaurants and buy expensive cigars. It's easy to say that someone else
should do something, but it's hard to do it yourself. I've been in your
apartment. I've seen how many things you could do without, if you were
really serious about doing the most good to the detriment of your own
personal satisfaction.''

``I work twelve hour days,'' said Lois. ``I work for and head up social
programs in my free time.'' She could feel her face flush. ``When I
waste an hour on something small and petty, the cost isn't measured in
terms of lives.''

Clark didn't seem the least bit hurt by this. ``The rich have a duty to
the poor. But they also have a right to do as they please with their
money, don't they? Lex Luthor engages in philanthropy, but you don't
begrudge him his mansion, or the excessive amounts of money he's spent
on lead shielding, among other things. I'm not talking about what should
be legally required of us, and I don't think you are either. I have a
moral obligation to the people of the world, as do you, but that
obligation isn't all‐encompassing. I'm not a slave.''

Lois frowned. ``I didn't say you were a slave.''

``You just think you're better than me?'' asked Clark.

``Clark, you lied to me, over and over. But even before that, you were
so powerful and so strange. You crushed rocks into dust in your hands
and you thought I would be impressed, and it seemed so hopelessly naive
to me.'' She spoke slowly, trying to find the right words. ``You lifted
me up into the air like it was nothing, and flew me out a half mile
above the city like it was second nature for my life to be in your
hands. What you can do is objectively terrifying, and anyone who doesn't
see that is just engaging in wishful thinking. I'm sorry that I tried to
pretend at being the woman you wanted me to be, the one who you could
settle into a life with, but Clark, it wasn't all an act. If things had
been different~--- hell, things \emph{are} different now, if we can be
open and honest with each other, and tear down the lies\ldots{} I'm not
promising anything, you understand, but I think we'd both like to start
over.'' There. Just the right notes of contrition, and something that
was close enough to the truth that it could pass the sniff test.

``Starting over,'' said Clark. He looked out the window at the city.
``Alright then.'' He held out his hand. ``My name is Clark Kent. I
masquerade as Superman. I can bend steel with my bare hands and move so
fast that bullets look like they're frozen in the air, among other
things.''

She shook his hand. Relief flowed through her; she'd been worried that
his outward calm was only for show. ``Lois Lane,'' she replied.
``Professional snoop. You're really from Smallville then? That wasn't
all made up when you came to Earth?''

``I was raised in Kansas,'' said Clark. ``Everything I've ever told you
about my childhood is true, but I left out all the interesting bits. My
parents found a spaceship in their field one day, and they took it as a
sign from god. I was just an unremarkable baby back then. They adopted
me without much discussion, and hid the spaceship beneath a tarp until
my father could hook a tractor up to it and stick it in the storm
cellar. I was raised like any other boy, until I started to get my
powers.'' He paused. ``How much of this do you want to know?''

``All of it,'' replied Lois. There wasn't much reason to believe it was
anything but another deception beyond her gut feeling, but he was
painting a picture for her, and either way he seemed to want to share.

``The hearing came first,'' said Clark. ``I was six years old, and I
thought I was going crazy. You can imagine my relief when I realized
that I was just hearing conversations from the next county over. It got
more powerful as the months went on, and I learned to shut it down, so
that I didn't have to listen to everything that people said or did. I
didn't tell my parents, but I thought that the hearing was what made me
special~--- what God had put me on the earth for. And then I got the
vision when I was eight. I could see straight through things. I could
count the feathers on a hawk from ten miles away. That was when I looked
inside the cellar that my parents had kept shut and saw the spaceship.''

``The spaceship that didn't burn up over the Atlantic,'' said Lois.

``It was one of the lies I told you,'' said Clark. ``Sorry.''

``Wait, this doesn't make sense,'' said Lois. ``You said that you were
baby when the spaceship came down. But the story you told me was that
you learned English from our radio waves on the way over. Was everything
about Krypton a lie then? Because if you didn't know you were an alien
until you were eight years old, I don't see how you would know anything
about the planet you came from.''

``I'm getting to that,'' said Clark. ``And I know that you're skeptical,
but you're going to have to bear with me. I asked my parents about the
spaceship, and showed them what I could see and hear. They told me
everything, and we went down into the cellar. Almost as soon as I
touched the spaceship it grabbed a hold of my mind and showed me a
vision of Krypton as it had been. The ghost of my real father was there,
and he told me about the planet as it had been.''

Lois stared at him. ``A ghost,'' she said flatly.

``Not really a ghost,'' said Clark. ``A simulacrum. A shard of my
father's personality. Krypton was a sprawling place of crystalline
spires and flying cars, and my father sat me down to explain everything
to me. He told me how my powers would grow, and tried to instruct me on
how to help avoid the fate of his planet.''

``And he said all of this in English?'' asked Lois. If Clark wanted her
as she truly was, that was what he was going to get. Skepticism as
practically second nature to her.

``I only thought to ask that later, when I was a teenager,'' said Clark.
``I'd read enough history books by that point to see that Jor‐El was
wearing a modified toga. All of the buildings and plants were inspired
by Greece, mixed with a few more artistic flourishes, but it seemed too
much like what I knew of Earth. I asked him about that, and he told me
that what I was being shown was just a representation that would make
sense to me. The real Krypton was a dark planet covered in black water,
and the real Kryptonians were something like a cross between a spider
and an eel. Before the ship landed, it mapped out human civilization and
drew in samples of humanity to examine. I'm not really a Kryptonian, I'm
something that the ship built. I actually think I was born on American
soil. Jor‐El showed me an analog of their world that I could understand,
but I think they were even further beyond us than I could imagine.''

The conversation continued on, and Clark talked about the defining
moments of his childhood. Lois listened closely, and made mental notes
for later, occasionally sharing her own anecdotes that kept him in rapt
attention even though they didn't involve godly powers and alien ghosts.
The important thing was that Clark was being honest with her now, and
his secrets were spilling out into the open. She had told him off, and
he'd called her a hypocrite, but somehow that didn't mean they couldn't
still be friends. She debated telling him about her arrangement with
Lex, but decided that was one secret to keep to herself. Nothing had
ever really come of that partnership anyway. And besides that, all the
talk with Clark hadn't really changed that much about how she felt. He
was more human to her now, but still as negligent as he'd ever been, for
all his protests. Some of the fear had left her, but not enough that she
was about to let Clark know she'd actively tried to work against him.

\begin{center}\rule{0.5\linewidth}{0.5pt}\end{center}

The first attempt involved the drinking water at the Daily Planet
Building. The kryptonite was ground into a fine powder and put into both
the water cooler and the water main connecting to the building. Lex had
run tests on it before using it on people, more because he was worried
about overplaying his hand than because he was concerned about what
effect it would have on the people. A week passed with no indication
that there had been any change, though his channels of information from
within the building were rather incomplete, especially since Lois had
cut back her visits to practically nothing. She hadn't told him
Superman's secret identity, despite his best efforts to pry it out of
her. He was working on a way to have plausibly ferreted it out without
exposing himself, but that was doubly difficult now that Superman was no
longer active.

The second attempt involved aerosolizing the kryptonite powder. Lex
thought it unlikely to work, given that the concentration would be
measured in parts per million. The kryptonite seemed to lose the
signature glow when reduced to pieces smaller than a gram, and Lex
suspected that the still‐mysterious source of the radiation required
sufficient mass in close proximity in order to continue emitting its
waves or particles. At any rate, this too seemed to have no noticeable
effect on Clark Kent or anyone else in the building.

The third attempt involved exposure to the kryptonite. A small,
thumb‐sized piece was given to a man who had only the simplest of
instructions: to walk past Clark Kent. Two spotters were put into
position to watch. Their report was typed up and broadcast in code,
which eventually made its way back to Lex. He hadn't been able to give
them full instructions for fear that they would discover too much, but
they hadn't noticed any real change in Clark's behavior, not even when
the patsy came within arm's length of him.

Brief exposure likely wasn't going to do the trick, especially not at a
distance. The fullest test of the kryptonite would be to place it
directly next to Superman for as long as possible. The spaceship's
creche had a large piece of kryptonite directly next to it, and a
relatively thin layer of lead was apparently sufficient shielding, which
said quite a bit about the danger that it posed. The kryptonite would
have to be close, nearly in range of skin contact. That meant using Lois
Lane. Unfortunately, Lois could lead Superman right back to Lex, but
that was what contingency plans were for.

\begin{center}\rule{0.5\linewidth}{0.5pt}\end{center}

Superman waited, and watched.

\begin{center}\rule{0.5\linewidth}{0.5pt}\end{center}

\emph{Author's Note: Some post‐publication edits have been made~--- if
some of the reviews don't make sense anymore, that's why.}


%%%%%%%%%%%%%%%%%%% NEW CHAPTER %%%%%%%%%%%%%%%%%%%%%%%%

\hypertarget{finale-part-2}{%
\chapter{Finale, Part 2}\label{finale-part-2}}

\emph{Dear Lois,}

\emph{I've been a longtime reader of your articles, and I have to say
that I'm quite the fan. I've been happy to note from the few photographs
I've seen of you that we seem to share a similar taste in fashion, and I
just wanted to share a tip with you. There's a jewelry store up on 18th
and 22nd called Marxhausen's, and they have just the most fantastic
pieces that would perfectly complement your outfits. Their necklaces are
so delicate and understated, just the thing for a woman like you. I
don't know if you have a special man in your life, but if you do I'm
sure he'd love to see you in it~--- and if you don't, I'm sure that it
would help attract one!}

\emph{Your loyal fan,}

\emph{Lucille Lindt}

Lois got a number of letters from the citizens of Metropolis on any
given day. Lex had arranged for this to also be a private channel of
communication. The opening sentence was one she'd memorized, and if that
weren't enough, the initials at the bottom were L.L., initials that she
and Lex shared~--- his idea of a joke, she supposed.

She went down to Marxhausen's over her lunch break, not really knowing
what to expect. She'd kept her distance from Lex ever since she and
Clark had hit the reset button on their friendship. If Clark had asked
her a direct question, she might have given up her last remaining secret
to him, but so far he hadn't shown any curiosity. That made her a bit
nervous. It was well possible that he'd already made his own deductions
on that score, especially if he'd been watching her. He knew that she
had a less than glowing opinion of him, and the book she'd written with
Luthor was damn near a hagiography, with none of the complexity that
she'd brought to her recent talks with Clark. But he hadn't asked, and
she hadn't felt like offering it up on her own.

The jewelry store was a small slice of glamour that didn't quite fit
with the rest of the block. It wasn't uncommon for the borders of the
neighborhoods to shift slightly over the years, and from what Lois could
tell, Marxhausen's had been the victim of one of these shifts. It was a
small, narrow store, staffed by a fetching woman with obscenely blonde
hair who perked up at the sound of the door opening. Lois looked around
slowly. There was nothing obvious to mark this as part of some plot.

``Can I help you?'' asked the blonde woman.

Lois took a breath, and dove right in. ``My name is Lois Lane, and I'm a
reporter for \emph{The Daily Planet}. My editor keeps asking me to write
a women's piece that's not about equal rights or social issues, and I
decided that I'd finally indulge him. So I was thinking that I would
write about jewelry. It should help pacify him, I think.'' Hopefully
Clark wouldn't think that was too suspicious. Lex had only gotten her to
the store, and she didn't have the barest outline of a script. Of
course, it would mean that she would have to actually write the article,
on top of her other work.

A sudden change came over the woman's face, her eyebrows falling and
then rising again, and when she spoke, her voice was slightly higher
than before. ``Oh, oh yes, there's so much that I could tell you about.
More and more women are buying their own jewelry these days, working
women who want to attract a husband.'' She reached beneath the counter.
``In fact, if you'd be willing to mention Marxhausen's in the article,
there's a piece I think you might like. It comes with a matching watch,
for that special man in your life.'' She had a nervous giggle.

She set two items down on the counter. One was a small golden locket,
shaped like an oval. The other was a watch, which glowed green behind
the clock face. Lois made no move to touch it.

``Is it radium?'' asked Lois.

``What?'' asked the woman with a puzzled look.

``Radium,'' Lois repeated. ``It's a metal that glows green, just like
that. Twenty years ago there was a group of factory workers~--- women
--- who painted the faces of watches with radium so they'd glow in the
dark. They licked the tip of their paint brushes to get a fine point,
and they suffered from radiation poisoning~--- anemia and bone
fractures, and then their jaws started to fall apart, disintegrating.''
She had literally written the book on it.

``I've, ah, been assured that it's safe,'' said the blonde woman.

``So were the girls who worked in the factory,'' said Lois. She wondered
how far she was deviating from what Lex had planned. Obviously she was
intended to walk out of the store with watch and the necklace. ``I'm
only curious about what makes it glow, I don't mean to be so\ldots{}
adversarial. The locket has a similar component?''

``Oh yes,'' said the saleswomen, who seemed grateful to be back on
familiar ground. She cracked the locket open, and showed a multi‐faceted
gem.

``Alright,'' said Lois. ``I'll take them. Now for this story, I have a
few questions\ldots{}''

\begin{center}\rule{0.5\linewidth}{0.5pt}\end{center}

\emph{What does it do?} wrote Lois.

Lex frowned at the notebook. He had thought that the next course of
action would be obvious to her: give Clark Kent the watch and see what
happens. He'd done his best to keep her out of the loop specifically so
that if Superman asked her, she would be able to tell the truth.
Superman had to know that there was someone plotting against him by now,
especially given the theft of the spaceship and the death of his mother,
but it was important that it appear as though Lois had been used as a
pawn, rather than the more valuable bishop or knight that she really
was. And then she'd had to go and ruin it by visiting him and asking for
answers, which would seem unacceptably suspicious.

\emph{I don't know}, Lex wrote back. \emph{I have reason to believe that
it will hurt him.}

Lois read the note and frowned at him. She tapped her pencil against the
paper for a few moments.

\emph{I'm not sure we should.}

Lex stifled a groan. He was losing Lois, that much was clear now. It was
at least gratifying to know that she hadn't taken leave of her senses in
coming to him; it was only a problem of a different sort. He could deal
with a question of loyalties, at least in the near term.

\emph{I figured out his identity,} wrote Lex. \emph{It wasn't difficult
once I started looking at the people around you.} He watched her face as
she read that, then took the notebook from her again before she had a
chance to respond. \emph{You're starting to feel sympathy towards him.
You think that you understand where he's coming from.}

Lois shrugged, then nodded. \emph{He's not perfect,} she wrote back.
\emph{But I think that I know where he's coming from now. We've been
talking a lot lately.}

\emph{He murdered a man in cold blood}, wrote Lex, but Lois was already
shaking her head.

\emph{He regrets it,} she wrote back.

\emph{What else will he come to regret?} asked Lex. \emph{We're talking
about the fate of the world. You know that there's no stopping him if he
goes rogue. Even if the chance is slim, it's a chance weighed against
the total destruction of humanity. If the odds are a thousand to one
that he'll kill us all, that's an average of two million dead. My own
estimates are higher, but you know him better than I do.}

Lois frowned. \emph{Will it depower him, or kill him?}

\emph{I don't know,} wrote Lex. He was halfway certain that she was
testing him. He would have pivoted, and claimed that he knew what the
effect was, but he'd already said that he didn't know, and couldn't take
the chance of getting caught in the lie. \emph{There is some element of
risk here, but I think it's low. The mineral in the watch emits a
radiation that I suspect will cause some interference with his power. I
need you to observe him carefully when you put the watch on him.} Lex
had done his own probability estimates, based on what he knew of
Kryptonian engineering from taking apart the ship. He strongly believed
that Superman's powers were of technological rather than biological
origin, simply given their raw power, and if kryptonite had any negative
effect at all, the engineering of the ship suggested to him that it
would have been designed to fail safely. Telling Lois that he'd weighed
the odds of catastrophe and found them acceptable would probably not
endear her to the plan though.

Lois slowly read what he'd written. She considered for a moment and
wrote back. \emph{You would be exposing yourself. Clark would know that
someone was aware of his secret identity~--- someone besides me, if he
believed I was an innocent victim of your machinations.}

\emph{A risk I'm willing to take}, wrote Lex. Superman almost certainly
already knew. That ship had sailed after the Smallville operation. But
there wasn't a convincing lie that he could tell Lois to explain to her
how he had come by that information. If she were trustworthy, this whole
conversation would have gone a lot smoother.

Lois absentmindedly bit the end of the pencil and paced around the room.
Lex didn't know how she weighed the arguments, but if she refused him,
everything got much more complicated.

\emph{You still don't actually trust him,} wrote Lex. \emph{You would
have told him about our arrangement if you did.} He handed the pad of
paper to her, and she stared at it mutely.

Some time passed, but eventually she nodded.

\begin{center}\rule{0.5\linewidth}{0.5pt}\end{center}

Clark wasn't at his desk when Lois came into the office. She put the
necklace and the watch into the lead‐lined drawer of her desk, and tried
to get some work done. She'd been put in an awkward position by Lex.
Simply talking to Clark about how she felt had done wonders, and melted
away a good deal of her stress. He was still in love with her, and that
would have to be dealt with at some point in the future, but she'd
confessed her fears and frustrations and he had been understanding. They
had their disagreements~--- deep disagreements that weren't going to go
away anytime soon~--- but they were at least talking to each other like
reasonable adults.

Clark came in, hung his coat up on a hook on the wall, and took a seat
at his desk. He smiled pleasantly at her.

``I got you a gift,'' said Lois. Her voice nearly caught. ``Not that
much of one, really, since it was free, but I thought you might
appreciate it.'' She opened the desk drawer, and heard a noise from
Clark's desk. He was standing far away from her, with a serious
expression on his face. She hadn't even seen him move.

``There's a small box in my briefcase,'' said Clark. His voice was calm.
``I want you to take it out and put both the watch and the necklace in
it.''

``Clark,'' Lois began. Something had gone horribly wrong.

``Now, please,'' said Clark. ``Be careful, the box is heavy.''

Lois did as she was instructed. The box was a crude thing. It felt
heavier than it should have been. She put both the pieces into it, and
closed it tight. When she did, Clark strode forward and picked the box
up, then sat down in his chair. No one else around them seemed to have
noticed any of this.

``It seems that we were less than perfectly honest with each other,''
said Clark. ``Old habits, I guess.''

``Clark,'' said Lois. She folded her hands into her lap, to keep them
from shaking. ``Fuck, I don't know what to say.''

``Language,'' said Clark with a mild tone. ``I'm not angry, just
disappointed. We'll have to talk this out later.'' He drummed his
fingers on top of the box. ``Just for my own personal confirmation
before I confront him, who gave you these?''

There was no way to deny it. Clark already knew. ``Lex Luthor,'' said
Lois. Her mouth felt dry.

Clark stood up from his chair, and tucked the box under one arm as
though it were weightless. ``Well, I'm off to have a talk with him.'' He
began to leave, as though nothing at all were wrong.

``Clark,'' said Lois. He stopped, and turned towards her. ``I'm sorry.''

``Well, that's a start,'' he replied. He gave her a heartbroken smile.

\begin{center}\rule{0.5\linewidth}{0.5pt}\end{center}

``Superman to see you, sir,'' said Mercy.

Lex simply stared at her. Just once he would have liked to see some
trace of emotion from her, but Mercy could announce that the world was
ending and still seem like she was bored.

``Did he say what it was regarding?'' asked Lex.

``The fate of humanity,'' said Mercy, without so much as a raised
eyebrow or a polite cough to acknowledge the absurdity of it all.

``Ah, well, send him in,'' said Lex. He looked towards the lead‐lined
drawer of his desk, where a pistol with specially prepared bullets lay
waiting. Inside each lead bullet was a small sliver of kryptonite. It
was a terrible plan, so far as they went, but at least it was there.

Superman strode into the study, looking around as he went. He was
graceful, for someone so big. The bright colors of his costume clashed
with the rich mahogany and leather upholstery of the room, but he didn't
seem to notice or care.

``I don't believe we've had the pleasure of being introduced,'' said
Lex. ``I was an innocent bystander at a bank robbery that you stopped,
though I'm sure you don't remember my face. I'm Lex Luthor. I suppose
you already know that.'' He extended a hand, and when Superman shook it,
he tried not to think about his fingers being mashed into pulp.

Superman sat down in one of the chairs, and offered Lex a pleasant
smile. ``Well, I'm not sure quite where to begin.'' He looked around the
room for a moment, perhaps contemplating the fact that he was surrounded
by lead. Then he turned and pointed to bound proof on Lex's desk, a copy
of the book he and Lois had put together about Superman. ``May I?''

``Certainly,'' said Lex. ``If there's anything that you think needs
changing, we'd be happy to\===''

Superman waved him off, and opened the sheaf of papers to somewhere in
the middle, finding what he wanted immediately. ``Ah, here we go. `The
currently accepted explanation for Superman's so‐called x‐ray vision has
nothing to do with x‐rays. Though one can be forgiven for thinking that
they have something to do with that particular form of radiation given
that both are used to peer through otherwise solid objects, as well as
the use of lead for shielding, the similarities end there.'\,'' Superman
looked up at Lex for a moment, then back down at the book. ``And so on
and so forth, and then here, this is what I wanted to point out.
`Superman's penetrative vision is thought by leading scientists to
utilize some hitherto unknown aspect of particle physics. The
hypothesized krypto particles permeate the universe and can pass cleanly
through every known element aside from lead.' And then it goes on to
talk about the difference between lead as it applies to x‐rays and lead
as it applies to krypto particles.''

Lex stared at Superman. He tried to keep calm and slow his heart rate
down. ``Are you telling me that you can see through lead?''

``No,'' said Superman. ``However, let me offer up a hypothetical. Let's
say that there's a mineral that was found in the core of a meteorite.
Two separate research facilities were sent samples of the meteorite~---
not by me, but by a third party~--- and they could find no form of
radiation using any of the instruments at their disposal. I know for a
fact that it does emit radiation, because as you seem to have guessed,
that radiation has an unsettling effect on me. As an additional piece of
information, the radiation from this mineral is blocked by lead, but
seemingly by nothing else. What do you suppose it would look like to my
x‐ray vision?''

``Bright,'' said Lex. ``Because if it emits anything, Occam's Razor
would dictate that it's krypto particles rather than some distinct
particle which shares many of the same properties. And if they're the
same, it's a matter of degree~--- the mineral emits far, far more than
you use to see by.'' He should have seen that possibility far, far
sooner. The only question remaining was how many layers of deception it
would allow Superman to peel back.

``It's as bright as a blazing sun, difficult to look at directly when
I'm using my x‐ray vision,'' said Superman. He shrugged. ``I probably
would have been able to piece it together all the same. A man walked by
me with a piece of glowing green rock in his breast pocket, and I felt
myself grow weaker. It got worse the closer he got. I have extensive
practice at faking reactions, or faking a lack of reaction, and just
when I was beginning to feel mortal, and worried that I was going to be
killed in some ignoble way, he kept on moving by. So I held myself in
check, and as soon as I got to a safe place, I looked through the walls
and watched him. I saw a piece of rock that was so bright it nearly
blinded me.''

Lex kept very still. The research facilities he'd sent the kryptonite to
hadn't had lead shielding, and he hadn't thought that they would need
it. The story he'd given for the appearance of kryptonite had been
solid, and the forged paperwork had been airtight. Experimentation and
synthesis of an unknown mineral shouldn't have been suspicious~---
except that Superman would only have had to go to space and look down at
the planet. If they shone as brightly as he claimed they did, they'd
stick out like a sore thumb.

``So,'' said Superman. ``I stole a piece of the PU‐356 from one of the
labs. I won't bore you with the details, but it would suffice to say
that I can see through walls and move as fast as I want to, which makes
me an excellent thief. I confirmed that it could hurt me, and after that
it was just a matter of being careful until I could confirm your
involvement to my satisfaction. I'm not in any real danger from the
PU‐356, though it is inconvenient. I can see it clearly from miles away
without having to try all that hard, and my superior speed means that a
bullet made of it could never hit me.'' Superman sighed. ``Of course, we
can drop the pretense of there ever being a meteorite. That would be too
big of a coincidence to swallow, if a meteorite capable of harming me
was found and put into mass synthesis just a short while after my
spaceship was stolen from me.''

Lex's mouth felt dry. ``I was blackmailed,'' he began. ``I was told that
unless I tried~\===''

Superman waved his hand. ``I don't believe you,'' he said with a half
smile. ``You made a good faith attempt to kill me, and you used Lois to
do it.''

``The military~\==='' said Lex.

``There were a few reasons that I came here,'' said Superman. ``Things
we need to discuss. First, I want you to admit to what you've done. All
of it.''

Lex's face fell. ``I don't know what it is you think you know,'' he
said. ``Or how you think you know it. I did arrange for Lois to deliver
a piece of the mineral to you, and it was an attempt to see whether you
could be disrupted in some way, but I don't know anything about a
spaceship. So far as I'm aware, it burned up on re‐entry. I felt it
prudent to have a method of dealing with you in case the worst were to
happen, and I can only hope that~\===''

``Towards the end of his life, William Calhoun talked a lot,'' said
Superman. ``He sat in a jail cell, and if prayer is an expression of
love, then he did whatever the opposite of praying is. Some of what he
said was nonsense, credit taken for crimes that he didn't commit, but I
could usually tell by how he spoke. He wasn't the guiding hand behind
Harry Kramer's bombing campaigns. If he had been, he would have brought
it up more often, instead of just in those moments that he really wanted
to twist the knife as hard as possible.''

``You think that I could possibly be behind that act of terrorism?''
asked Lex.

``It wasn't terrorism,'' said Superman. ``It was a series of attempts on
my life. Terror was only a byproduct. Given that I know you tried
earlier today, it's not unreasonable to think that you had tried
before.'' He held up a hand to forestall any objections. ``I'm less
certain about that one, and obviously I have no hard proof. Certainly
nothing that would hold up in a court of law. Still, it became clear
fairly early on that I was looking for someone who was intelligent and
possessed an enormous amount of resources. That you have lead‐lined
rooms in both your home and office, speak in languages other than
English for no good reason, and have a penchant for codes~--- well, that
helped to paint a picture. I want a confession from you, one that covers
everything you're guilty of.''

``And then you'll kill me?'' asked Lex.

``No,'' said Superman. ``I'm willing to accept your unconditional
surrender.''

``Ah,'' said Lex. ``And what does that entail?''

``Part of an unconditional surrender is that you don't get to ask that
question,'' said Superman. ``I beat you. It's over. You have exactly one
thing that can give me the slightest injury, and I can see it coming
from a mile away. I'm fairly certain I know how you think now. It's been
a learning experience, watching all of the machinations of an enemy with
nearly infinite resources and a steadfast refusal to be identified.''

``Fine,'' said Lex. It was time to change tactics, and concede some
ground in the hopes of arranging a more advantageous battlefield. ``I
confess. I was the one who figured out your inability to see through
lead, and allowed that fact to be known around the world. I arranged for
the bombs to be made and placed, knowing that innocent people would die.
I figured out your identity as Clark Kent, and inserted agents into
Smallville. Your mother's death was unintentional. I stole your
spaceship. I found a single small chink in your armor and tried my best
to use it against you. I believe that's an accurate list of my crimes.''
Almost all of the layers of deception had slid off now, with only a few
secrets still held back in reserve, more out of a faint sense of hope
than any coherent strategy. Lex felt naked.

``Why?'' asked Superman. He showed no shock or surprise.

``You are too dangerous to be allowed to live,'' said Lex. ``You cannot
be stopped after the fact, which means you must be stopped prior to
it.''

``Do you know why I killed Calhoun?'' asked Superman.

``According to Miss Lane, you were angry with him,'' said Lex. He tried
not to be bothered by the seeming non sequitur. Superman was at an
advantage in not only strength and speed, but information as well. Lois
had said he had the ability to think for long moments in the space
between blinks, and perhaps that accounted for the disjointed
conversation. Or maybe Superman had just prepared a script for himself
to follow, and was sticking to the points he wanted to hit before he
brought Lex to a messy end.

``You're close,'' said Superman. ``I was angry with him, but anger alone
wasn't enough. Instead it was a chain of thought, with each step colored
in anger. I convinced myself that it was the correct thing to do, and
that wouldn't have been possible without the anger. I decided to kill
Calhoun, and then I worked backwards to figure out all the ways that I
could make that into the single best choice.'' He paused, and stared Lex
in the eyes. ``I've been looking over what I actually believe lately,
and trying to figure out why I believe it. And do you know, I think more
than anger, my thinking has been tainted by fear. As has your own.''

``Fear is a natural response to the chance of obliteration,'' said Lex.
``It's what saved our ancestors~--- my ancestors, anyway~--- from death.
When a new predator arrives in the woods, the appropriate reaction is to
run away or fight. There was no way to run away from you. So yes, I was
and am afraid of you, but that fear had a grounding in reality. I would
have acted the same even if I didn't feel an instinctive terror at your
presence on this planet.''

``I was good,'' said Superman. ``I was a paragon of virtue. I never hurt
anyone. I never acted in a way that was contrary to humanity. I never
interfered with politics or warfare. You feared me all the same, and
made it your mission to kill me. You got this idea in your head that I
was a threat~\===''

``You were,'' said Lex softly.

``And you never stopped to reconsider whether that continued to be true
as time went on,'' finished Superman. ``I'm not a bad person. I can
understand if you had misgivings when I showed up, but as the months
passed, you never changed your mind, did you? Maybe you just didn't want
to admit that you killed all those people for nothing. You couldn't
admit you were wrong about me, because then you would be forced to think
of yourself as evil. Lex, I'm not going to destroy this planet, or
anyone on it. That's more true now than ever.''

Lex was silent. The issue wasn't whether Superman was planning to
destroy the planet, it was that he was capable of it at all. ``Knowing
what I know now, I would have done things differently,'' said Lex. ``But
up until an hour ago I thought it would be for the best if you were
dealt with, and you haven't said anything to change my mind. Regardless,
you have managed to convince me that it's not going to be possible to
accomplish that goal, so if you want my unconditional surrender, you
have it.''

``Good,'' said Superman. ``I said earlier that I had been ruled by my
own fears. They weren't fears of death or injury, for the most part.
They were fears of failing in other ways. I acted like I thought a hero
should act, and tried to be a symbol for people. I read your proposals
as they came out, and the proposals of others. Some of the ideas I'd
already thought of myself, while others were novel, but I had convinced
myself that part of being a shining symbol of hope, truth, and justice
was being static. Part of it was my father, I'm sure. He had his very
particular views about the world, and I was following his example. It
felt like I would have been turning my back on him if I'd decided that I
wanted to do things differently. And even when the evidence began to
grow that I'd been wrong~--- or at least not completely right~--- I
refused to change. I was afraid I would do something bad by trying to do
something good. I was worried that I would ruin our society, or mar
human history. I thought I would end up leading us down the same path
Krypton had traveled, letting too much happen too soon. It was logic,
tainted with the fear of failure. I had decided that I was going to keep
the world as it was, so that my responsibilities would stay small. Then
I rationalized my way towards that conclusion.''

``You're speaking in the past tense,'' said Lex.

``How much of what you've said over the past year was true?'' asked
Superman. ``How much do you want to make the world a better place?''

``I meant all of it,'' said Lex. ``I could have done much less than I
did, if I only wanted the appearance of philanthropy. I want to make the
world a better place.''

``That's what your surrender means,'' said Superman. ``You're going to
help me do the most good.''

``What's the catch?'' asked Lex. The important question of \emph{How?}
would come later.

Superman waved his hand around the room. ``No more lead. No more codes.
No more speaking in other languages. No more secrets from me. You'll
have to return my spaceship to me, and stop all of the current research
into a means of killing me. LexCorp will be turned into a machine for
generating good in the world instead of pure profit. I'll be doing large
scale labor, and you'll be managing the profits from that as well,
channeling them towards the areas where the money can do the most good.
I'll need a thorough debriefing on all of your methods of deception. It
probably goes without saying that I'll be watching you like a hawk. I
want your help in allowing me to keep my identity as Clark Kent secret,
which will likely involve buying \emph{The Daily Planet} and giving me a
list of everyone who you've told, for starters. And you're never to
speak with Lois again.''

``And if I don't want to take that option?'' asked Lex. He was already
thinking of ways to get around the restrictions that Superman was
talking about, but if the kryptonite was unworkable as a solution, it
was almost certain that the attempts on Superman's life would have to
stop for good.

``I'm going to build a prison,'' said Superman. ``You would be the
second inmate, if you refused. There would be absolutely no hope of
escape.''

``Then I'll help you, of course I will, but I'm afraid I still don't
understand,'' said Lex.

``I believe there's a goodness in you, Luthor,'' said Superman. ``I'm
still a Christian, and the story of the Bible is one of redemption. I
have nothing to fear from you, and you're in a unique position to effect
positive change. While I admit it would feel good to lock you away
forever, rehabilitation is more important than retribution. More
practically, no one knows of your crimes but me, and while I can prove
enough of it to my own satisfaction, I don't have any illusions that it
would hold up in any court of law. Making you disappear would raise
questions, and I don't know who might have the answers. I also know your
methods well enough to know that you probably have a dead man's switch
somewhere, and of course I worry about what might be in it. My existence
as Clark Kent is important to me, and I don't want to give it up unless
I have to. You're a smart man. You know I'm offering you a good deal.''

``You are,'' said Lex. He swallowed. ``It makes more sense to keep me
alive and work towards our mutual goals. You hadn't struck me as being
so level‐headed.''

``People change,'' said Superman. He blurred forward. The chair he'd
been sitting in slid backward three feet and fell over. He stood right
before the desk, towering over it. His expression was deathly serious.
``I feel like it goes without saying, but I could kill you in a
heartbeat. I don't like using the threat of force, but if you step out
of the very clearly defined lines we're going to set, I will throw you
right into a specially made cell in my jail. If you give me reason to
suspect that you're still a threat to me or anyone else around me, you
will simply vanish from the face of the Earth and never be heard from
again.''

``Understood,'' said Lex. He noted what Superman had said, and the very
specific wording the alien had used. He had not actually said that he
would commit murder, only that he could. And the threat of consequences
had been vague. Superman was back to being a pacifist, it seemed, after
a dalliance with murder. Lex could use that against him.

But then, perhaps it made the most sense to simply accept the reality of
Superman. If kryptonite shined brightly and Superman knew to look for
it, it would be nearly impossible to kill him with it. A kryptonite bomb
surrounded by lead would only work if Superman could be maneuvered
directly next to it, and as soon as he saw the casing of the bomb bowing
outwards he would be on the move. The other clear option was to get
Superman as he slept, but given what they both knew about each other,
there was a decent chance that Superman would simply stop sleeping, or
rotate through different anonymous locations~--- and that was assuming
that Superman could even be snuck up on while he slept. If Superman knew
about kryptonite, and was willing to work outside of or in opposition to
the law, the problem seemed nearly unworkable.

``Which of the proposals did you want to pursue?'' asked Lex.

Superman stood back, and brushed off his costume. ``I'm looking for pure
efficiency, which is your area of expertise. You're going to spend the
next few days tearing the lead from these walls and complying with my
demands, and then I want you to start writing a proposal for how I can
do the most good. I won't kill anyone, and I want to try to keep my
interference with governmental bodies to a minimum, but I am willing to
reshape the world in any other way.''

``I'll think on it,'' said Lex.

``Don't cross me,'' said Superman. ``I'm hoping that you can see that
this is good for both of us.''

Lex nodded stiffly. All his preparations and all his caution had been
for nothing. This wasn't the end that he wanted, but it was the best
that he could have hoped for after his masks had been taken away from
him.

``This will be the last time we see each other in person,'' said
Superman. ``Or rather, the last time that you see me. I'll be watching
you.'' He turned to leave, then stopped and stared at the door. He
glanced back at Lex with a frown on his face, then looked at the door
again. It was lead of course, just like the walls of the room, but
something had given Superman pause. ``Miss Graves, please move away from
the door.''

Instead, the door to the study began to open, and it had shifted only
the smallest fraction of an inch in the time it took for Superman to
stand behind Lex's desk. He moved quickly and efficiently, being quite
delicate with his power. By the time the door swung open, Superman had
Lex's head in his hands, one of which was gripping his jaw. Lex felt no
sensation of pain or even discomfort, only firm hands. Mercy stood at
the doorway, with a solid block of kryptonite the size of a baseball
held straight out in front of her.

``Mercy, was it?'' asked Superman. His vice grip didn't let up for a
moment.

Mercy nodded.

``You have to know that this is utterly futile,'' said Superman. ``I
shouldn't have expected you to sit idly by while Lex and I had our chat,
but we've come to an agreement of sorts. I want to leave here peacefully
and with a minimum amount of destruction or loss of life.''

Lex's jaw was held firmly in place, preventing him from speaking. He
could only hope that Mercy would understand from the look in his eyes.
She took a half step closer.

``Stop,'' said Superman. She stopped. ``I'm going to let Lex speak to
you, to try to convince you that you should leave. Lex, consider this
your first test.''

Lex's jaw was gently released. He took a breath. ``Mercy, I want you to
listen to me very carefully. Throw the kryptonite towar\===''

\begin{center}\rule{0.5\linewidth}{0.5pt}\end{center}

The grip on Lex's head vanished just as the wall behind him shattered
outwards with a rush of air. Mercy had started winding up for a throw
before Lex was halfway through his sentence, and the block of kryptonite
landed on the desk, where it slid across and fell to the floor at Lex's
feet. Lex ignored it and opened the lead‐lined drawer to pull out the
pistol.

He stepped out the hole in the side of his mansion, where it was a nice
and pleasant summer day. Superman was laying on the ground a hundred
feet away, covered by his red cape, and Lex took off towards him at a
dead sprint, trying his best to avoid the glowing green shards and bits
of lead that Superman had taken with him when he went through the wall.

When Lex was five feet away, he fired three bullets into Superman's
head. The sight of blood sent a wave of relief through him, and he
crouched down next to the body to catch his breath.

``I studded the walls with kryptonite,'' Lex said to the corpse. He kept
his eyes on the body, just in case it started moving. It as far from the
first dead person he'd seen, but the sight of it still sent a surge of
adrenaline through him. He wouldn't have been terribly surprised if he
had passed out or threw up.

After a half a minute had passed, Mercy came walking across the lawn to
join him. Her hair was in the same tight bun as always, and she
certainly didn't look like she'd just played an instrumental part in
killing a god. She carried the chunk of kryptonite in one hand.

``He had a great many options for dealing with that situation,'' said
Mercy. ``How lucky did we just get?''

``I haven't had enough time to work out the odds,'' said Lex. ``But he
didn't want to hurt either of us, even after everything I'd confessed
to. Leaving through the side of the room was probably what I would have
done. He must have scoped out the mansion before he came~--- watched me
in my study whenever you opened the door, observed our patterns. I think
it would be safe to say that we won because we were lucky. Putting
kryptonite in with the lead was at the far edge of my most paranoid
preparations, and after the brush‐by I had thought that it wasn't strong
enough. I was actually angry about wasting the money.'' He ran his hand
across his bare scalp. ``He was too dangerous to let live.''

``I know, sir,'' said Mercy. She looked across the yard. They were
separated from their neighbors by a massive expanse of lawn and thick
shrubs, but the noise wouldn't go unnoticed. ``We should figure out what
sort of story is appropriate to this situation.'' She looked at where
Superman lay. ``We should also dispose of the body.''

She was right, but Lex was having trouble focusing. He had won. It had
been damned sloppy. He should have arranged for Mercy to do what she'd
done on her own anyway. It should have been a masterstroke. If he had
lured Superman into the room and sprung a trap, he would have felt more
of a thrill of victory. But as his thoughts moved back towards the
conversation he'd had with Superman, he could tell there was another
reason that he only felt hollow. There was a small seed of doubt. Lex
had made his choice, and made that choice for all of humanity. It wasn't
unreasonable to wonder whether that choice had been the right one.
Still, the seed of doubt couldn't be allowed to grow, not after the
choice had already been made.

``I don't know how much of our conversation you overhead,'' said Lex.
``But it didn't change anything. He was just too powerful to be allowed
to exist.''

Mercy only nodded.

\begin{center}\rule{0.5\linewidth}{0.5pt}\end{center}

``Any word from Clark?'' asked Jimmy.

``No,'' said Lois. ``I wouldn't expect any letters from him.''

``He just left without saying goodbye though?'' asked Jimmy. ``I mean,
he was never very dependable, but I just didn't expect it of him.''

``He'd had too much of the city,'' said Lois. ``You read his letter of
resignation. He's back in Kansas, taking care of the farm. It wasn't
impossible to predict. You know how much he talked about Smallville.''
Lex had called her to let her know that Superman had been dealt with,
and the letter of resignation had come in the very next day. She'd been
feeling an awful pit in her stomach ever since, even as she tried to
keep Perry and Jimmy from asking too many questions. She had no idea how
Lex had done it, but she was certain that Clark was dead.

Jimmy moved closer and lowered his voice. ``The thing is, I was talking
to Eleanor. Did I ever tell you how we met?''

``At a bar?'' asked Lois.

``Right,'' said Jimmy. ``It was just after we'd gotten back from the
Whitman thing, and\ldots{} she asked me some questions about Clark. I
didn't think anything of it at the time, I was halfway to drunk and she
was~--- is~--- pretty much a goddess. Anyway, I was talking to her about
Clark's sudden retirement from the reporting business, and she broke
down and told me that it wasn't an accident that we had met each other.
I was part of a case she was working, to try to dig up some dirt on
local reporters. She works for a detective agency, and she thought that
maybe it was so that someone would be able to put pressure on him if the
wrong sort of story broke, but now\ldots{} now it seems a little
fishy.''

\emph{Luthor}, thought Lois. She should have known that after she told
him about Superman's secret identity he would try to find out more.
Maybe it had been one of the things that had tipped Clark off and let
him know that someone was on his trail. ``Wait a second, you started
dating after we covered the Whitman kidnapping?''

``Yeah, why?'' asked Jimmy.

The timeline didn't match up. She'd come to Lex months later, which
meant that either someone else was snooping into Clark's past, or Lex
had known the truth far before she had. One of those options seemed far
more likely than the other.

``Nothing,'' said Lois. ``You've just given me something to think
about.''

\begin{center}\rule{0.5\linewidth}{0.5pt}\end{center}

Just because Superman was gone didn't mean that Lex could rest easy.
There was a possibility that more aliens would arrive at some point in
the future, and if they had capabilities anything like what Superman
had, humanity needed to undergo a rapid technological advancement as
swiftly as possible. Superman's spaceship still held a wealth of
information, and there was a piece of it that Lex thought of as its
brain~--- a central component that was connected to all of the others
and likely carried signals of some sort. On top of that, there was the
brewing war in Europe to consider, along with the Sino‐Japanese
conflict. It was unfortunate that governments were more willing to spend
extreme amounts of money when there was an immediate danger to their
existence, but war~--- or at least the threat of it~--- would prove
useful.

``Miss Lane to see you,'' said Mercy. She had helped to drag Superman's
body from the wreckage and hide it in the trunk of one of his cars
before the police arrived, and as he might have predicted, the whole
experience didn't seem to have changed her at all. He was doing his best
to follow that example.

``Send her in,'' said Lex with a smile.

Lois looked different. She'd cut her hair aggressively short since the
last time they saw each other, and if she had always been a little bit
ferocious, now she seemed positively bristling.

``How much of it were you responsible for?'' asked Lois.

``I'm afraid I don't know what you're talking about,'' said Lex.

``You killed his mother, that much I'm nearly certain of,'' said Lois.
She sat down in what had been her customary chair, and stared at him
with intense eyes. ``I went out there, did you know that? I went to
Smallville, trying to find out who he had been. I'm surprised you left
so many loose ends. There was an autopsy report for Martha Kent that
didn't look right, and the day she died, when there was that big storm,
three people went missing from Smallville and never came back. They
didn't find bodies either.''

``Again, I have to insist that I don't know what you're talking about,''
said Lex. ``And if you're thinking of putting any of these thoughts to
print, I would suggest that you either have a substantial amount of
proof or a very, very good lawyer.''

``I kept thinking about the bombings,'' said Lois. ``Clark thought that
Calhoun was the man behind them, but he was wrong, wasn't he? While you
were putting out a reward for Kramer's capture in public, you were
sending him schematics and instructions in private. It wasn't possible
for a single man to have done it all, that much was obvious, so you
framed Calhoun and bombed your own properties to turn watchful eyes in
another direction. You were trying to kill Clark from the start.''

``Miss Lane, I generally make it a point to not bother refuting spurious
rumors about myself,'' said Lex. ``But given the gravity of what you're
suggesting and the fact that we were friends, once upon a time, I will
tell you completely and unequivocally that I had nothing to do with any
of that. I was a steadfast supporter of Superman~\===''

``Because you needed a cover,'' said Lois.

``I was a steadfast supporter of Superman, and I was as disheartened as
anyone when he became a murderer and fled the planet,'' said Lex.

``You killed hundreds, didn't you, without even a thought for the value
of their lives?'' asked Lois.

``I am curious about what evidence drove you towards such a wild and
unfounded conclusion,'' said Lex. There was no way that she would be
able to prove anything.

In the worst case scenario, she had found the laboratory where the
spaceship was being kept and broken through all the layers of security,
but that still wouldn't be enough to implicate him in the public eye,
let alone the court of law. After the autopsy and a collection of
samples, Superman's body had been reduced to pulp, mixed with a healthy
amount of kryptonite, encased in lead, and lowered deep into an unmarked
grave on a vast, private nature preserve in Alaska which Lex had
exclusive control of. He had used hundreds of agents in the course of
tracking down and positioning Superman, but only a very few knew enough
to implicate an unknown master in wrongdoing, and only Mercy had the
ability to implicate Lex as that mastermind. It was well possible that
Lois could or would reveal to the world that Clark Kent had been
Superman, but it would have raised all kinds of questions he was sure
she would want to avoid, and either way wasn't something that could
really be proven after the fact~--- nor would it substantially change
his plans.

``I don't have any evidence,'' said Lois. ``Believe me, if I did I would
be shouting it from the rooftops instead of coming here. You won,
Luthor. I just want to know what the hell you were thinking.''

``Well, of course I can't comment on things that I haven't done,'' said
Lex. ``If, hypothetically, I had engineered a series of heinous crimes
in pursuit of some foolish feud with Superman, I certainly would have
nothing to gain by telling you about my reasoning, especially not when
it would give more fuel to your paranoia.''

``I hated Clark for lying to me,'' said Lois. ``I hated him for living
this double life and pulling the wool over my eyes. But at least at
their core, Clark and Superman were the same person. There was a real
goodness there, even if it was clumsy and imperfect. Is there a core to
you, Lex?''

``Lois, I have a franchise of orphanages set up throughout the United
States now, headed by caring, competent people,'' said Lex. ``I am
personally spearheading a number of advancements in the sciences that
will revolutionize the world ten times over. If the United States goes
to war in the coming years, one of the reasons we will emerge victorious
is because of the vast resources that I control and the overwhelming
technological superiority that we will enjoy. I have done more to end
the Great Depression than any single other person on the planet. That is
my core.''

Lois only glared at him. She stood slowly, seeming years older than when
they'd first met. ``I wish I'd never met you,'' she said. She left
without another word.

\begin{center}\rule{0.5\linewidth}{0.5pt}\end{center}

He'd waited too long.

Superman had been stopping by with food every three days like clockwork.
The hole in the ground had been expanded, and he'd brought in more
supplies. It was gratifying to have his pitiful existence made slightly
more bearable, but at the same time every new possession in his
miserable little hole meant that his captivity became more and more
permanent.

The plan had been to wait for Superman to make his next visit (canned
foods, refill the barrel of water, empty the waste bucket) and then make
the trip up after that so that there was less of a chance of starving to
death in the woods once he made it out of the hole, not to mention that
he needed as much time to scurry away before the big blue warden came
back to tend to his only prisoner.

It was difficult to mark time, given how close to the Arctic Circle he
was. The sun dipped down to the horizon and then stayed there. But when
three days had come and gone, Floyd had waited another two days after
that, worried that Superman was simply late. He'd let his food get too
low, and was going to have to make the climb on an empty stomach.

He was thirty feet up when he slipped. He'd been trying to lunge up to a
higher handhold, and when he missed and sank back down to let his weight
rest on his feet, one of them slipped. Then he was falling.

He lay on the cold, hard floor with a broken leg. There was no chance
that he would be able to make the climb now, let alone hike through the
wilderness. He made a quick tourniquet and a splint, and hoped that
Superman would come back.

\begin{center}\rule{0.5\linewidth}{0.5pt}\end{center}

\emph{Author's Note: Thanks for reading.}


\cleardoublepage
\thispagestyle{empty}
\phantom{lol}
\cleardoublepage
\end{document}

