\chapter{Peeling Back the Veil}\label{peeling-back-the-veil}

Jimmy Olsen sat at the bar, gulping back his fourth beer. It was
possible to forget, for brief moments.

Lois Lane had come over to him, shaking slightly, and said that they
needed to take a trip out into the country. He'd grabbed his camera and
plenty of film, then raced downstairs where he'd had to wait in the car
for nearly ten minutes while Lois made some calls and tried to figure
out where exactly they were going.

Lois drove. Her knuckles were nearly white on the steering wheel.

``Where are we headed?'' asked Jimmy.

``A farmhouse near Bott's Pond,'' said Lois. ``Superman found the
kids.''

``Thank God that's over,'' said Jimmy, and Lois had shot him a look that
shut him up for the rest of the trip.

There were two cop cars outside the place when they arrived. The
kidnappers had been taken away an hour ago, but he and Lois were the
first reporters on the scene. Jimmy would have been fine just getting a
shot of the farmhouse with the cop cars in front of it, but Lois had
loudly insisted to the police that Superman himself had sent them to get
pictures of the interior so they could document the actual crime scene.
Jimmy had no idea whether that was true or not, but the police seemed to
believe her. He'd nearly thrown up when he'd seen the body parts stacked
like cordwood. Lois had just frowned and stared at the scene with an
intensity that scared him.

Jimmy looked up from his beer the second time he was tapped on the
shoulder.

``Hi,'' said a cute redhead in a willowy dress. She held out her hand
towards him. ``I'm Eleanor.''

``Jimmy,'' he replied. Her handshake was firm.

``Our hair matches,'' she said with a laugh.

``I guess so,'' he said.

``Rough day at the office?'' she asked. She raised her eyebrows and bit
her lip, like she couldn't wait for his answer.

``I'm a photographer,'' said Jimmy. He'd wanted to continue, to explain
the things he'd seen, but couldn't find the words. And on second
thought, maybe it was better not to inflict that on anyone. The worst of
the photographs wouldn't make it to print. Perry would pick out
something that was suggestive of horror but didn't actually show
anything. To Jimmy, it was almost worse to only catch a glimpse. He was
sure that he would be a better photographer if he could understand why
the small puddle of blood on the edge of the kitchen table was somehow
worse than directly seeing the dismembered corpse.

``What kind of photographer?'' asked Eleanor.

``I work for the newspaper,'' said Jimmy. ``For \emph{The Daily
Planet}.'' He paused. There had to be something that he could say that
wouldn't ruin her evening. ``You know that picture of Lois Lane standing
next to Superman? I took that.''

Eleanor placed a hand on his arm. ``Oh, I read \emph{The Daily Planet}
every day. I wonder how many of your pictures I've seen?'' She had an
easy, pleasant smile, and Jimmy slowly began to take notice of her.

``Lots, probably,'' said Jimmy. ``People look at the bylines, not the
photo credit. Most of them probably don't even look at the bylines.''

``I look at the bylines,'' said Eleanor happily. ``Clark Kent and Lois
Lane, right? Do you work with them?''

``Yeah,'' said Jimmy.

``Say, what does Clark Kent look like? I've seen photos of Lois,
obviously, but I've sometimes read the name of Clark Kent and wondered
what he was like.''

``Clark?'' asked Jimmy. He swallowed down the last of his beer and
signaled for another. ``He's a big guy. Sort of a hunched over
gorilla.''

Eleanor laughed. She was still touching his arm. Jimmy felt his cheeks
warming, and it wasn't just the alcohol. ``That's not how I pictured him
at all. In my head he was tall and upright, very dapper. Like Clark
Gable.''

``No,'' said Jimmy. ``Not like that at all.'' Between Eleanor's
questions, the beer, and the images from the farmhouse swimming around
his brain, Jimmy was beginning to feel out of sorts.

``Where's he from?'' asked Eleanor.

``What?'' asked Jimmy. He'd been distracted by her eyes.

``Clark Kent, is he from the city or somewhere else? I pictured
somewhere on the East Coast, but the city itself,'' said Eleanor.

``Kansas,'' said Jimmy.

``Really?'' asked Eleanor. Her eyes lit up. ``I'm from Kansas too! Which
part?''

``Smallville,'' said Jimmy.

``Yes, I think I've heard of it,'' said Eleanor. She looked over at the
clock above the bar. ``Well I have to go, but it was nice talking to
you.''

``You're not staying?'' asked Jimmy. He tried to keep the hurt from his
voice.

``You didn't seem all that interested in talking to me,'' said Eleanor
with a frown. She gestured towards his beer. ``And I don't know how many
of those you've had, but I think it's probably been too many.''

``Today's the worst day of my life,'' said Jimmy. ``Worst so far anyway.
There might be other days that are even worse than this one. I've got a
feeling that's the case. I just need someone to be by me. Please?''

She seemed about to brush him off, to offer some excuse and leave, but
she must have seen something in his face because she just nodded and
stayed with him.

They got to talking, actually talking, and eventually Jimmy felt like
the world wasn't about to come crashing down on him. Eleanor had a
certain brightness to her that made the world seem less grim. She'd come
to the bar alone, and after an hour had passed, he'd offered to walk her
home. When they got to her place, she must have sensed how desperately
he wanted not to go back to his cold, cramped apartment. She invited him
up.

Her apartment was just as small as his was. He sat on her bed while she
put on a kettle of tea, and that was when he started crying. He felt
embarrassed and ashamed, but she sat down next to him, ran her fingers
through his hair, and made comforting noises. They laid down side by
side on her bed. She didn't seem surprised or upset. It must have been
around two in the morning that she started telling him about her father.
He'd come home from the Great War with shell shock, and killed himself
with a shotgun when she was six. Jimmy didn't know how to respond to
that. He hoped it was enough that he had listened. Eventually she fell
asleep, and he followed suit soon after.

In the morning he'd thought that there would be sheepish looks and
awkward goodbyes, but she'd made them breakfast in her tiny kitchen and
didn't show an ounce of shame.

``I need to change out of these clothes and get dressed for work,'' said
Eleanor. Her voice was soft and gentle. ``But if you ever need someone
to talk to, you know where I live. There's a communal telephone on this
floor, I can give you the number.''

``I'd like that,'' said Jimmy. ``I never even asked what you do. We
talked about me too much. I feel like a lout.''

Eleanor looked at him for a moment before answering. ``I work for a
private detective agency. And I really do need to get going, I'm
sorry.''

Jimmy said his goodbyes and left for \emph{The Daily Planet}. He felt
better, more at peace with what he'd seen the day before. He couldn't
imagine spending that night alone.

\begin{center}\rule{0.5\linewidth}{\linethickness}\end{center}

``Are you okay?'' asked Clark. Of course he got better right after the
biggest news story since the bombings was already on the page. It was
typical of him.

``Peachy,'' replied Lois. She'd barely slept the night before. She would
have gotten drunk, but she'd done some thinking about alcohol on the
ride out to the farmhouse. She'd become too entangled with Superman for
loose lips. So far, she'd been making up for it by smoking more, but
that didn't seem to be helping her nerves at all.

``Sorry I wasn't here,'' said Clark. ``Sorry you had to see that.''

``See what?'' she asked.

``The body,'' said Clark. ``The blood. I read your article and looked at
Jimmy's photos, the ones that didn't make it to print. It was
gruesome.''

Lois waved her hand. ``That was nothing,'' she said. ``I mean, not
nothing, but there are hundreds of millions of children in the world,
and you've got to figure that hundreds of them die every day, right?
Maybe thousands? Lots of little girls get raped. Lots of little boys get
chopped up. The only reason this is front‐page news is that they were
rich and white with a famous father, and because Superman didn't quite
get there in time.'' Clark watched her. She tried to concentrate on her
typewriter, but she couldn't even remember what she was supposed to be
typing up.

``Perry told me that Superman talked to you. What exactly did he say?''
asked Clark.

``That's between me and Superman,'' said Lois. She was being too harsh
with Clark, she could tell, but it would have taken more effort than she
was willing to spend to make her words come out nice.

``Lois, if you need someone to talk to, I'm here for you,'' said Clark.
``And I don't mean any offense, but it seems like you've got something
you need to get out.''

``Possibly,'' said Lois. She stopped for a moment to think through her
wording. Superman had come to her of all the people in the whole world
to get things off his chest, and that meant that she was important to
him. She had to assume that he was listening and watching, so talking
about Superman became a matter of framing him in the best possible
light. ``In general terms, he explained to me that being Superman can be
difficult sometimes.'' There, that didn't sound so bad as it really was.
``He said that he can't do everything.''

``And that upset you?'' asked Clark. He had a look of serious and
heartfelt concern, like she were some delicate doll that he was worried
would break under stress. She hated that. She'd had more adventures in
her life so far than Clark Kent could ever dream of, and to him it was
like she was made of glass.

``It made me think about how right he is,'' said Lois.

``Lois, look, I don't know what it was he said, but I'm sure he didn't
mean to upset you,'' said Clark.

Lois nodded. ``I agree, it wasn't his intent. But he opened my eyes up,
and if my reaction to that is to be upset with the world, then so be
it.''

Clark kept staring at her, and she kept avoiding his eyes. ``Do you know
what I think?''

Lois didn't answer, because she wasn't confident that she could speak
without snapping at him.

``My pa was in prison for a while, I told you that,'' said Clark. ``And
for a long time he never really talked about it, but I knew it was bad.
And I think that maybe talking about it would have made it less bad for
him, you know?''

``You're saying I should talk to you,'' said Lois.

``No,'' said Clark. ``I'm saying that maybe whatever Superman said to
you, he just said because he was having a bad day. Maybe he just\ldots{}
needed someone to talk to, and talking to you made whatever difficulties
he was having easier to bear.''

Lois found this far from comforting.

Superman was holding back, in nearly everything that he did. He didn't
hurt people, and certainly didn't kill people. He could fly at twenty
times the speed of sound, maybe even more, but he almost never did. He
worked quickly and efficiently towards his objectives, and most of the
time if you showed up after he'd gone there'd hardly be any evidence
that he was there at all. Everyone thought that was just who Superman
was. He was so totally and completely in control of himself that he
would never do anything truly wrong. He firmly followed the doctrine of
unambiguous goods.

It wasn't true though. People thought that Superman did everything
effortlessly, and maybe as far as the physical realm went that was true.
Inside his head though, he wasn't much more than a man. She'd heard that
Superman had walked into Calhoun's bar and let himself be hit in the
face with a gob of spit. She'd believed that Superman had been
unbothered by that, but now it was clear that Superman was human enough
to have felt something there. Superman's ideals weren't innate to him,
they took conscious effort on his part. And what would happen when
Superman had a day so bad that he decided that his ideals weren't worth
keeping?

\begin{center}\rule{0.5\linewidth}{\linethickness}\end{center}

Picture a circle. Next, picture a point outside that circle, call it
\emph{O}. Draw a line from the point such that it pierces the circle in
two places~--- a secant~--- and call those two points \emph{A} and
\emph{B}. Draw another line originating from that \emph{O} such that it
intersects the edge of the circle in only one place~--- a tangent~---
and call that point \emph{C}. The secant‐tangent theorem states that
\emph{OA} times \emph{OB} is equal to \emph{OC} squared.

If the circle is Earth and the point outside it is Superman, then that
tangent defines how far Superman can see before his vision starts to
clip the crust of the Earth. To find that distance, take the diameter of
the Earth (roughly eight thousand miles) plus Superman's distance from
the Earth (rarely seen to be more than ten miles), then multiply that by
Superman's distance from the Earth, then take the square root of that.
The result was 280 miles, the distance that Superman could see to the
geometric horizon from the height that he stayed within ninety‐nine
percent of the time.

There were 1,127 miles between Smallville, Kansas and Metropolis, New
York.

Of course, Superman had x‐ray vision, but that was stopped by lead. Lex
Luthor had consulted a book of geological science and found that the
estimated abundance of lead in the Earth's crust was one thousandth of
one percent, which meant that for every mile of earth that Superman
looked through, he was looking through sixteen millimeters of lead.
Based on Lex's calculations, it was safe to assume that it only took a
centimeter of lead to stop Superman's x‐ray vision. The upshot was that
Superman could not see what went on in Smallville unless he specifically
moved himself into a position to do so.

It allowed for a comparatively enormous amount of breathing room.

It was imperative that he get someone there as quickly as possible. What
records he could pull showed that Clark Kent at least existed on paper,
and a quick call done through layers of intermediaries confirmed that
the \emph{Smallville Ledger} had once employed him, or at least claimed
to have employed him. Lex was starting to once again doubt that Superman
was an alien, since it very much seemed that Clark Kent's backstory was
solid, but he kept digging all the same. Learning about the existence of
Clark Kent had produced numerous threads to pull on.

He needed someone in Smallville, but the constraints on hiring were
immense. He needed someone intelligent, prone to following orders,
trained in espionage, and willing to go into deep cover for an extended
period of time. He would need to instruct them to take precautions above
and beyond what any covert operation had required in the history of
spycraft, a constant cover that remained unbroken for weeks or even
months at a time. The list of people that fit the bill was very, very
short. Lex was in the middle of trying to figure out whether it would be
possible to put someone in deep cover and still keep them in the dark
about the connection between Clark Kent and Superman when the doorbell
rang.

A few minutes later, Mercy stood in the doorway of the study. ``Miss
Lane is here, requesting a moment of your time,'' she said.

``Send her in,'' said Lex.

She looked different, though Lex couldn't say exactly how. Did she know
that Superman was Clark Kent? If so, it wasn't obvious from her face.
Lex was wearing the outermost layer of his personas, the one where he
was a simple enthusiast and advocate for Superman with no knowledge of
the alien he wouldn't willingly share with the world. He mentally
prepared himself for Lois Lane to peel back the personas one at a time.
He'd been careful, but part of being careful was preparing for your
carefulness to fail you. He had stories prepared that would justify his
actions.

``Miss Lane,'' Lex said with a smile.

``Mister Luthor,'' replied Lois. He pinpointed what was different about
her; she was tense. ``I called your office and they said you were
here.''

``The businesses mostly run themselves,'' said Lex. ``I have a knack for
hiring competent people, and that's left me with the free time to pursue
my passions.''

``Superman,'' she said. She began to dig a pencil and notepad from her
purse.

``Just so,'' replied Lex.

``I've read your proposals,'' said Lois. ``What would you do, if you
were Superman?'' She began writing in the notepad.

``A common question,'' said Lex. He was about to continue on when Lois
turned her notepad around to face him. It said \emph{Can Superman be
stopped?} Lex's eyes moved to the door, to make sure it was closed. They
were encased in a hidden layer of lead. Lois had been over when the
shielding was being installed, and knew they were behind it. She was
being cautious.

``A common enough question,'' repeated Lex. ``For many it's the perfect
fantasy. People talk about setting foot on the surface of the Moon, or
going to the Olympics and dominating in every sport. They talk about
standing up to their various oppressors. My companies have been picking
up quite a few Jewish immigrants from Germany of late, and I feel that
many of them would like nothing better than to fly down and put a hole
in Hitler's face.'' He turned to look at her. ``Superman can't be
stopped. It's frightening to think what might happen if his power fell
into the hands of someone without such a strong moral compass. For
myself, I'm not sure that I would want the power. I'd use it for good as
best I could, I suppose. No flashy displays, no material wealth, just
the betterment of mankind.''

``I was wondering whether you could help me,'' said Lois, pointing to
her notepad, where the words were still written.

Lex watched her carefully. Lois Lane could easily be working for
Superman. Even if she didn't know that he was Clark Kent, she could have
been sent in to get some admission of guilt. He couldn't trust her. But
perhaps he didn't have to. ``Help you with what?'' he asked, not missing
a beat.

``I've written two books,'' said Lois Lane, ``One on the radium girls
and another on the role of women in the World War.''

``I know,'' said Lex. He pointed to his bookshelf. ``I've read them.''

Lois seemed momentarily taken aback by this, but of course he had read
them. He'd read The Daily Planet every single day for the past year, and
after he'd learned that Clark Kent was Superman he'd gone and read every
issue again. Earlier that morning, when he'd learned that Clark Kent had
once written for the \emph{Smallville Ledger}, he'd immediately started
thinking up possible methods of getting back issues of it to his home or
office without immediately allowing Superman to connect the dots.

``My new book will be about Superman,'' said Lois as she wrote in her
notebook. ``And as you and I have something of a working relationship, I
was wondering whether you would be willing to contribute.'' She flipped
the notebook towards him again. \emph{S is losing faith in us}.

``What sort of contribution?'' asked Lex.

``You're the preeminent scholar of him, and one of the greatest examples
that his efforts to be a symbol actually work,'' said Lois. She pointed
at the notepad and raised her eyebrows.

All Lex could think was that it was a trap. She would have to be a
masterful liar for that to be true, but that was certainly possible. If
he'd been willing to admit that Superman was using the disguise of Clark
Kent and lying through his teeth to everyone he interacted with on any
given day, then surely he had to admit that the same might be true of
the woman that sat next to him every day. The idea of Lois Lane turned
to his side was seductive though. And though he was well aware that the
best traps didn't look like traps until they'd been sprung, it truly
didn't look like a trap.

``I'm afraid I'm a busy man,'' said Lex. ``Though I admit that sharing
my thoughts on Superman with a wider audience appeals to me. What
precisely would be the nature of this arrangement?''

Lois wrote in the notebook. Superman could surely hear that, if he were
listening. Lex couldn't decide whether he was being too paranoid in
thinking that Superman would find it suspicious. ``I'd like you to write
two chapters,'' she said. ``They can be short. There will be a chapter
on the science that I'd like you for contribute to, and another chapter
on how he's changed the people of the city.'' She held up the notepad
again. \emph{S is more human than he lets on, might turn on us}. ``Does
that sound reasonable?''

``Let me think on it for a moment,'' replied Lex. ``In the meantime,
feel free to peruse my library, I'd be happy to give you any book that
you have an interest in. Give me five minutes, by the clock?''

Lois looked unhappy, but she nodded all the same.

Lex closed his eyes, relaxed his body, and thought.

There was too much unknown information. He could make all sorts of
educated guesses about what Lois Lane and Superman knew, but there was
so little information available that these guesses were barely worth
anything. There were dozens of configurations of truth which fit the
data as he saw it, and in some of those possible worlds it would be
correct to allow himself a partnership with Lois Lane, and in others it
would throw not just his operations but the fate of the entire planet
into jeopardy. Lex Luthor had set himself up as a follower of Superman,
highly visible and shining like a beacon. If Superman really was losing
his faith in humanity, what would happen if he learned that Lex Luthor
was responsible for the deaths of dozens, nevermind that it had been the
correct decision given the information he'd had available at the time?

He looked to Lois. If she were telling the truth, why had she chosen to
confide in him? Well, he was a billionaire with an active interest in
the betterment of humanity, the premiere scholar on everything related
to Superman, and likely one of the few people she knew who had a room
lined with lead and the sense not to immediately blurt out a strangled
``What?'' when shown a secret message. On top of that, they had an
established relationship. It made a certain sort of sense. The more he
thought about it, the more he thought it plausible that she really had
come to him in good faith.

He walked over to her and took the notepad and pencil from her hands.
She had a hopeful look.

``I've decided that I'll do my best to help,'' said Lex. He pointed to
where she'd written \emph{might turn on us}, then began to write
something of his own. ``I'm a busy man, but a partnership could benefit
us both.'' He turned the notepad towards her. \emph{Tell me everything
you think you know about Superman.} ``I have a number of things coming
up in the near future, so it would be good to get this done quickly.''

``Agreed,'' replied Lois. She grabbed the notepad from him. ``I should
warn you that I don't have a publisher lined up just yet, but it
shouldn't be a particularly hard sell.'' \emph{He can't know I'm telling
you.}

``A problem to be dealt with in due time,'' said Lex. ``If you're free
tomorrow, we could meet here? There are a few things that I'd like to
think over first. I'll try to have some initial thoughts ready.''

Lois watched him for a moment, then nodded.

\begin{center}\rule{0.5\linewidth}{\linethickness}\end{center}

The next day, Lois Lane picked up the piece of paper from Lex Luthor's
desk as he said unimportant things for the benefit of Superman.

\emph{I'm not saying that I believe you, Miss Lane. But if you think
that Superman is losing his faith in us, then that's something that
needs to be discussed, and I can only hope that if he finds out, he'll
understand that the discussion couldn't happen in front of him, as it
were. You have more exposure to the man than anyone on the planet, so
far as I know. You're the only one he's really talked to. If you have
concerns, I need to hear them, no matter how outlandish.}

``There much to the science of Superman,'' said Lex. ``His x‐ray vision,
for example, doesn't use actual x‐rays. The current best theory is that
there's an exotic type of particle which is as yet undetectable to us.
It permeates the planet, with lead atoms being the only thing that can
stop it for reasons that possibly relate to its atomic weight, electron
density, or some other property. But there's so much unknown, as with
much about Superman. I've been working on it for a year, and I still
don't have the faintest understanding of how his hearing works. I want
to make it clear that much of what I say about the science of Superman
is on the cutting edge, and not to be taken as gospel.

``I've done the liberty of typing up a very rough draft, and would be
pleased if you could take a look,'' he said. He handed her a blank sheet
of paper and a pencil. She was about to object that if they really
wanted to be secretive she'd need to leave his study with some actual
papers, but he pulled out a number of typewritten pages, already marked
up with a few corrections and notes, and set it beside her. She began to
give her account.

From time to time, she would ask Lex an inane question to keep up
appearances, and he would respond with inane answers. To Superman it
would sound like they were simply working on a book together. She wasn't
sure whether she could trust Luthor, but he was by far the most capable
man in the city, and she hoped that the worst he would do would be to
burn her notes and refuse to see her without letting Superman know what
she thought. She tried to use the strongest, most persuasive language
she could, and hoped that Superman would never learn what she really
thought of him.

Still, she left some things out. She didn't mention the possessive way
that Superman had touched her when he'd picked her up and flown her
through the air. She'd interacted with Superman on a number of
occasions, and he always seemed so familiar with her. So far as she
knew, she was his only friend, but she was also something more to him.
She could feel his eyes on her while she undressed sometimes. She could
feel him staring at her while she tried to sleep. With every
conversation she had, she imagined Superman listening in. This feeling
had grown in intensity since their last meeting. She hoped it was just
paranoia on her part. But either way, Lex Luthor didn't need to know.

\begin{center}\rule{0.5\linewidth}{\linethickness}\end{center}

The picture Lois Lane printed was a grim one.

He was now reasonably confident that she knew nothing of Superman's
alter ego. Her account of Superman was vivid and unflinching.

\emph{He can hear everything that's happening in the world, and it's
driving him to despair. I think he can shut down his hearing and tune it
all out, but that's almost worse in a way, because he still knows all of
the pain and suffering that's happening, and turning away from it
doesn't make it disappear. He sounded like a martyr to me, forcing
himself to bear witness not just to the evils but to the vast but simple
indifference of the world.}

Yet that was very different from the picture that Lex had been forming.
Superman spent time as Clark Kent, which implied a certain apathy
towards suffering. What did Superman get from maintaining the Clark Kent
persona? From what Lex's various sources could tell him, Clark Kent
didn't seem to take very many pleasures from life. He didn't drink or
smoke, and he had no romantic relationships to speak of. It seemed
unbearably dull to Lex. Even in his work life, Clark Kent was only
second best, and he didn't seem to leverage the full force of his
powers.

The first possibility was that everything Superman had said to Lois was
a ruse. Superman was an abject liar, he'd already proven as much by
spending an entire year pretending at being someone he was not. It was
possible that he was manipulating Lois Lane towards some end, though Lex
could only make the vaguest guesses as to what end. Superman should have
no need for a reporter, since he already was one. If it was
manipulation, Lex suspected that it was in pursuit of inflicting some
mental or emotional harm, but it was also possible that he had some
delusions about Lois. Lois hadn't mentioned Clark at all, and Lex hadn't
thought it prudent to bring him up.

The second possibility had taken some time to see. Lex had been under
the assumption that the persona of Clark Kent had been invented as a
cover for Superman, but it was distinctly possible that Superman was a
cover for Clark Kent. The solidity of his background information
suggested as much. Lex had told Mercy that Clark was a mockery of
humanity, but perhaps the outward appearance of Clark Kent matched his
inner feelings. Lex Luthor could almost imagine Clark Kent as a simple
man who wanted nothing from life but to be left alone, burdened by
powers that he didn't understand or desire, donning a costume and flying
through the air because the guilt of sitting at his desk simply became
too much sometimes. It was almost sad, until you remembered that he was
the most dangerous man on the planet.

If there were answers, they would be found in Smallville.

\begin{center}\rule{0.5\linewidth}{\linethickness}\end{center}

Joseph and Loretta Greene bought one of the town's two general stores.
They moved into a small house on Cherry Street, and quickly made friends
throughout the community. Joseph was always ready to ask about the
history of Smallville, a town which he seemed to have adopted as his
own, and Loretta was relentlessly social. They attended church every
Sunday at the Zion Lutheran Church. Though they didn't have any
children, they often spoke of it as an eventuality. If you could see
straight through Loretta's clothes, you would see a scar running at a
diagonal from the side of her left breast to just above her navel. If
you could see straight through Joseph's dress shirt, you would find
three puckered marks that were unmistakably bullet wounds. Joseph and
Loretta had stories ready in case anyone ever saw and asked. Those were
the only marks of their former lives.

As it turned out, Clark Kent was somewhat famous in Smallville. His name
had come up on the very first day that Loretta and Joseph had come to
town, when the previous owner of the general store had told them that
they should carry The Daily Planet, even though it would be at least two
days old by the time it arrived. Though he hadn't been especially
popular or well‐known growing up, Clark Kent had become the nearest
thing that Smallville had to a celebrity, and the people of Smallville
often talked about what Clark was up to in the big city.

Every few days, Loretta would write a letter to her family back in
Gotham City. She wrote an enormous amount, even when there wasn't much
to say, and often included some of Joseph's historical research about
the town and its residents. Joseph took to Smallville like a fish to
water, and some days could be seen two doors down talking to the small
group of men that worked at the Smallville Ledger, a once weekly
newspaper that served as the main source of news for the county.
Anything and everything of interest he learned there went into the
letters to Gotham.

From time to time, a letter would come back.

The player piano had effectively died out in 1929 with the stock market
crash, and few of the things were produced anymore, since radio had
effectively taken its place. Player pianos worked through pneumatic
action to play music, and the different songs were recorded on sheets of
perforated paper. Joseph and Loretta had brought a player piano with
them when they moved in, and a very careful observer might note that it
routinely seemed to break down just after one of these letters from
Gotham City came in. Joseph would take the perforated sheet of paper
with the music out of the machine and go to work repairing whatever was
wrong, and Loretta would lay the sheet on top of the letter. The
typewritten letter would perfectly line up with perforated sheet music,
revealing a scattering of letters that formed a message. Those brief
seconds were the only time that someone watching through the walls from
hundreds of miles away would know that they were something more than
just rural shopkeepers.

``Do you think we'll ever know?'' asked Loretta one night over dinner.

``No,'' said Joseph.

``How much longer, do you think?'' she asked.

``No idea,'' said Joseph. He leaned over and kissed her on the cheek.
``Let's not talk about these things.''

Five thousand dollars were deposited into a Kansas City bank account
every week for each of them, courtesy of a trust that had been set up
according to the will of Joseph's fictitious uncle. They had no idea who
their employer was, only that he was fanatically paranoid. Joseph and
Loretta weren't their real names, but all the proper records were in
place if anyone went looking. If asked about the money, they would
confess that they simply liked the small and quiet life of a small town
and didn't want to complicate things.

\begin{center}\rule{0.5\linewidth}{\linethickness}\end{center}

Hershel Whitman sat on the veranda of the governor's mansion. It was
early in March, and too cold for the veranda, but he didn't like to be
inside the house anymore. He'd have never thought that so soon after
winning an election he would feel like leaving his office. People had
offered their condolences and paid their respects, but it had been more
than a month now, and mostly all that was left were awkward glances and
sad looks. June was shut up in her room, and Robert was buried in the
Oakwood Cemetery.

Superman landed in the yard and started walking towards the house.
Hershel tried not to react. Early on he'd wanted to yell at Superman for
failing to save his children. He had yelled, in fact. Late at night,
after June had been brought back and Robert hadn't, when Hershel
couldn't sleep, he would walk a mile or so from the mansion and scream
at the sky. He didn't know if Superman had listened, or if Superman
cared. He felt somewhat guilty about that now. If it hadn't been for
Superman, June might not have come back at all.

``Superman,'' said Hershel. His voice caught.

``Governor Whitman,'' Superman replied. ``I never said how sorry I
was.''

``No,'' replied Hershel. ``You didn't.''

``I came here to ask a favor,'' said Superman. ``Thirteen minutes ago
Francis Pasqua spoke with his lawyer about getting immunity. He named
William Calhoun as the man who gave the orders.''

``Immunity,'' said Hershel. ``You want me to give him immunity in
exchange for testimony.''

``No,'' said Superman. ``I need to know what June heard them talk about,
and if it's enough, I need her to testify.''

``Just kill him,'' said Hershel. His voice was barely a whisper. ``Just
fly in and kill him. No one would stop you, no one could stop you. Hell,
use a gun and no one would even think of you. There are a dozen people
with cause to kill Willie Calhoun. You want my daughter to take the
stand against him, to say that his name was thrown around by those men?
Calhoun would have the right to face his accuser, and that means cross
examination. No. I won't put her through that.''

``He needs to be brought to justice,'' said Superman.

``Do you know why it didn't happen in the last trial?'' asked Hershel.
He'd had two whiskeys before Superman had shown up, and swayed slightly
as he stood. ``It's because you let him. The criminals don't care about
you. They know you won't hurt them. They know how to hide from you.
Ronald Oakes. That was the name of the man driving my children, and
everyone forgets about him. They slit his throat because they knew that
if they didn't he would call for you. You're not making them stop,
you're just making them adapt.''

``Crime has dropped ninety percent since I've come to Metropolis,'' said
Superman. ``You can ask the chief of police. I know you're angry, but if
we don't have the rule of law, we don't have anything.''

Hershel crumpled into his chair. Arguing was no use. ``If June agrees,''
said Hershel. ``If June agrees to talk, and she knows enough to convince
the district attorney, and the jury listens to her and then they say
he's not guilty, if all that happens\ldots{} you'll just let him go?''

``No,'' said Superman.

``No?'' asked Hershel.

``No,'' replied Superman.

\begin{center}\rule{0.5\linewidth}{\linethickness}\end{center}

In 1911, a baby boy was left in the hallway of a tenement in Metropolis.
He was taken to the Metropolis Foundling Hospital and from there became
part of the Orphan Train program. In Metropolis the abandonment of
children was a continual problem, while in the Midwest there was a
continual shortage of labor. The inventive solution to these twin
problems was for the children and babies to be delivered to the
heartland of America by railway. At every stop the children would be
taken out and displayed before the gathered crowd, sometimes having
their muscles felt and teeth checked. Some would be selected for
indentured servitude and possibly adoption, while others would be put
back on the train and sent to the next stop. When the orphan train
stopped in Oskaloosa, the foundling, Clark, was selected by Martha and
Jonathan Kent. They adopted him a few years later.

So far as Lex could tell, that was the official story that was believed
by the residents of Smallville. Though the orphan trains had fallen out
of favor, the Metropolis Foundling Hospital was still standing. As Lex
Luthor was funding five different orphanages in Metropolis, it wasn't
terribly hard for him to get the records from the Foundling Hospital,
and more importantly, it wouldn't look too terribly suspicious,
especially when it was known that Lex Luthor was looking to expand his
charitable giving. It had taken only a day of looking through the
records to see that they contained no mention of a boy named Clark
leaving the train at Oskaloosa, and no record of the Kents as sponsors
for a child.

This in itself was nothing too out of the ordinary. Lex had found that
few people took record keeping seriously. Ownership of the records
changed, people developed new formats, and sometimes entire years worth
of data were destroyed by insects, acids in the paper, or an excess of
humidity. Yet it still felt suspicious to Lex. If you were trying to
hide someone's parentage, you couldn't do much better than the orphan
trains. Clark Kent had the perfect excuse for not having a birth
certificate.

According to the reports he received from his two agents, Martha Kent
owned a farmhouse outside of town, which she shared with a live‐in
farmhand named Elias Clayton. His agents had spoken to her, and remarked
only that she was a nice woman who went to church every Sunday and spent
most of her time on the farm. Jonathan Kent had died a year before Clark
had come to Metropolis, and if that was a deception, someone had at
least given him a gravestone.

Clark Kent had grown up in Smallville. There were dozens of people who
could recall him as a boy. His worn and faded initials were carved into
desktops and trees. The evidence of his existence was so utterly
convincing that it couldn't be denied. There were aberrant incidents in
and around Smallville that suggested the powers characteristic of
Superman extending back to the time that Clark Kent was eleven years
old. Superman had not actually arrived in a spaceship, he had grown up
on a farm in the middle of Kansas. Even if Lex believed this, it didn't
help to clear up anything. The power had to have come from somewhere.

The solution had to be on the Kent farm.

\begin{center}\rule{0.5\linewidth}{\linethickness}\end{center}

Floyd Lawton had come into Smallville as a drifter looking for room and
board with barely a dime in his pocket. He'd walked down the dusty dirt
roads, going door to door looking for work, until finally he'd happened
upon a small house that belonged to a greying old lady. He'd gone down
the path and up the steps to the front porch, then knocked with a ready
smile on his face.

``Missus Kent?'' Floyd had asked as she came to the door.

``Yes? Do I know you?'' she'd asked. She was in her sixties, maybe even
older, with white hair tied up in a loose bun. Her dress was simple and
blue.

``No ma'am, sorry, the name was on the mailbox. Name's Floyd Lawton.''
He took off his hat and clutched it to his chest. ``Sorry to trouble you
on this fine day, but I've been on the road a long while and I'm looking
to settle down for a spell of work. If you have something that needs
doing, or if you know some neighbors that need some work, I'd do it just
for room and board, whatever's asked of me.''

Martha Kent gave him a warm smile. ``Why you know, I had a live‐in
farmhand up until just two days ago, Elias Clayton. He was a strong and
able man, helped with the few animals I still keep, the garden, and the
maintenance on the old barn. We made enough to keep ourselves afloat,
along with the money brought in by leasing out the land to the Parkers,
and I paid him a good wage. Well Elias had aspirations, you see, but he
was a black and so work didn't come too easy, especially not the kind of
work that he was keen on doing, which was acting. Then just a week ago a
director of movies came out to Smallville, right out of the blue. He
said that he was going to make the great American movie, and said that
Smallville would make the perfect location for it. Well now, Elias took
the day off to go speak with that director. I thought nothing of it of
course, until Elias came back and told me that he'd been discovered. He
said it happens all the time, if you can believe that, so I said to him
that he wasn't to leave until he'd finished putting up new chicken wire
around the coop. I was thinking it might be I'd try taking this year by
myself for a change, but if you're looking for work, then boy do I have
some.''

Floyd nodded through all this, a slightly desperate grin on his face
like he thought a real drifter would have. Martha mostly seemed happy to
have someone to talk to though, and they'd moved the conversation
inside. They'd come to an agreement over homemade lemonade that had too
much pulp in it for Floyd's liking.

Later that day, Floyd had picked up his meager belongings from the
Greene house in Smallville, where he'd rented out a room for the night.
He had a rifle slung over his shoulder, and two pistols in a wooden box
that draw Martha's attention.

``There's not much use for pistols out here,'' said Martha with a frown.
``We have a shotgun, and a few rifles for dealing with the coyotes and
wolves, or for bringing in more meat.''

``They were my father's,'' said Floyd with a smile. ``Hand‐crafted and
fine quality pieces, and I'm only thankful that I've never had to sell
them.''

``My husband Jonathan, may he rest in peace, he abhorred pistols,'' said
Martha. ``He was pacifist and an absolutist, and thought every war was a
crime against God's own will.''

``He's lucky he didn't get drafted then,'' said Floyd with a smile.

Martha's face became very serious. ``Oh, my Jonathan was drafted
alright. He'd applied to be a conscientious objector. When I say he was
a pacifist, I don't mean that he thought it was better not to kill, I
mean he believed with every fiber of his being that it was simply
something a good person doesn't do, no matter the circumstances. He went
to prison for his beliefs.''

``Ma'am, if you don't want me bringing pistols into your \\ home\==='' Floyd
began.

``No, no,'' said Martha. ``There were more than a few things that
Jonathan and I didn't see eye to eye on. You don't use those pistols
lightly though. If someone tries to steal from our farm, I'd rather just
let them take what they came for. It's not worth killing a man over a
pair of chickens.''

Floyd breathed a silent sigh of relief. He loved his pistols. He liked
to use both at once, feeling them kick in tandem. He'd once cleared out
an entire poker den with those two pistols, killing thirteen men with
twelve bullets and earning him the nickname ``Deadshot''. He was handy
with a rifle too, and had been briefly trained in sharpshooting by the
military before a dishonorable discharge that had left him perfectly
positioned to become an assassin. He was very explicit on that term, and
had maimed more than one thug who called him a mere hitman.

He'd met men who didn't want to kill before. Hell, most men didn't want
to kill. But he'd never met a man who'd prefer jail over being in the
army, except perhaps those cowards that only wanted to stay out of the
fighting because they were afraid for their own safety. In his opinion,
Jonathan Kent was probably just a slacker, but he held his tongue.

He settled into a routine at the Kent house. He would listen to Martha
Kent yap away during an early morning breakfast, go out and do whatever
work needed to be done until lunchtime, take a break during which he'd
work on composing a letter to his completely fictitious sister, and then
keep working on the farm until nearly sunset, when he'd go into town,
grab a copy of the Smallville Ledger, and on occasion mail off his
letter for the week.

``Why is there a lock on the storm cellar?'' asked Floyd.

``Oh, that old thing,'' said Martha. ``It kept blowing open, so I put a
lock on it a while back and somehow forgot the key.''

``I could cut the lock,'' said Floyd. ``I wouldn't want to get caught in
a tornado without a storm cellar.''

``It's rusted shut anyway, I think,'' said Martha. ``And there's a small
basement room we can go to if the storms ever get too bad. I wouldn't
worry about it dear.''

Floyd had gone back and looked at the doors to the storm cellar more
closely. They were made of metal, and when he looked closely at the
seams, he could see that the whole thing had been welded shut. It was
hard to make out with all the rust, but the storm cellar had been sealed
shut as tightly as possible.

So far as he could guess, whatever was down there was the entire reason
for his being on the Kent farm. He made sure to mention the storm cellar
in his letters to his fictitious sister, cloaking the information in
long paragraphs about how he was afraid of tornadoes. Hopefully his
employer was smart enough to read between the lines.

\begin{center}\rule{0.5\linewidth}{\linethickness}\end{center}

\emph{Author's Note: Orphan trains were a real thing. Whether this was
slavery by another name or an ingenious solution to the societal
problems of abandoned children and a lack of cheap labor is left as an
exercise to the reader.}

\emph{If you have an interest in reading more about the treatment of
conscientious objectors in WWI, search out ``Armed with Prayer in an
Alcatraz Dungeon'', which does a lot more justice to the topic than I
can do here. It's interesting reading even if you disagree with the
moral philosophy of it. My grandfather was a conscientious objector in
WWII. One of my strongest memories of him was when he told me about how
he was routinely spit on while building bridges and roads around the
Midwest by people who thought that sticking to his beliefs was somehow
the height of cowardice.}

\emph{As always, I appreciate the favorites / follows / reviews /
recommendations. A special thanks to my wife Alyssa for being my beta
reader.}
