\hypertarget{private-wars}{%
\chapter{Private Wars}\label{private-wars}}

\emph{Superman has a day job.}

It was just a joke, the kind of thing that the brain coughs up when it's
trying to match a pattern. Kant said that humor was expectation strained
until it suddenly dissolved into nothingness. Lex had been making maps
and doing complex math for weeks on end, and if that was a joke, it made
sense that the punchline was simply that Superman walked the streets of
Metropolis as a human. The very thought of the most powerful entity in
the world choosing to work a nine to five job in downtown Metropolis
should have caused any right thinking person to burst into laughter. But
as Lex turned the idea over in his head, the humor faded. And once the
idea had presented itself, it refused to leave.

``Son of a bitch.''

People liked to believe that brilliance was a matter of sudden insight.
Isaac Newton was sitting beneath an apple tree and just happened to be
struck on the head with an apple, which led to him developing the theory
of universal gravitation. Archimedes sat in the bathtub and realized
that an object displaces its equal volume of water. Friedrich August
Kekule realized the structure of the benzene molecule after having a
dream of a snake eating its own tail. These were the stories that people
liked to tell, because it made thinking seem like magic, and no matter
that the stories weren't true. Even where there was a grain of truth
behind the story of the insight, there were hundreds of hours of thought
and study before it, and another hundred hours of proving it afterward.
Another thing that was never mentioned was how often a startling insight
proved to be rubbish.

Some years ago, he'd spent days trying to make what he called a
battlesuit a practical reality. It was going to be a callback to the
knights in shining armor, creating a solitary soldier encased in
impenetrable armor and capable of advancing on enemy lines with
impunity, mounted machine guns firing away the whole time while a diesel
engine belched smoke. He'd drawn up schematics and eventually began
stripping parts away, replacing those things that thrilled the
imagination with those that would work practically and reliably. The
steel legs were replaced with treads. The arms were removed in favor of
a larger cockpit with buttons and levers. The center of gravity was
lowered, until the cockpit sat between or just on top of the treads. He
still remembered the feeling of looking down at his design and realizing
he'd done nothing more than make a better tank. LexCorp now owned two
factories that made them, building up a stockpile to sell to the
European powers when the next inevitable war broke out. Still, the whole
project had been borne out of a vision he'd had, of diesel powered
mechanized armor striding across the battlefields of the next war. The
fact that he'd spent so much time pursuing that vision was a source of
embarrassment. It had been a valuable lesson in critically examining
those ideas that came to him suddenly and struck him on some emotional
level.

What Lex needed was someone who would ask some pointed questions and act
as a foil to his enthusiasm, a devil's advocate. He made a quick
calculation of the risks of speaking out loud, and found them
acceptable. If he was right, Superman engaged in surveillance far less
than he had supposed, and if he was wrong, there was no harm in it.
There was only one person that he trusted enough to discuss the idea
with, and conveniently she was sitting in the same room as him, drinking
tea and reading a book.

``Mercy, your attention for a moment?'' asked Lex. He used French, a
language that they both shared, as a weak form of security.

``Of course sir,'' she said as she put down her book with a finger
resting between the pages.

``Convince me that Superman doesn't have a secret identity,'' he said.

``A secret identity?'' she asked, as though she had never heard of the
concept. On the long list of wonderful things about Mercy Graves was her
ability to effortlessly take the role of the ignorant in their dialogues
when it was required of her. Lex found being forced to define himself
quite helpful.

``Like a spy,'' said Lex. ``Or a philanderer, I suppose. Superman leads
a double life, and in the second one he doesn't wear the costume.''

Mercy took a sip from her tea. ``And what does he do in this second
life?'' she asked.

``I don't know,'' said Lex. ``I'd have to guess at motivations, and if
he has an alter ego I know less of his psychology then I had thought.''
Lex ran a hand across his hairless head. ``What does a man need? Food,
water, sleep, shelter. Superman has never displayed any need for those.
Perhaps he eats and drinks in secret, but playing at being human would
be the least efficient way to go about satisfying those needs. Sex or
family\ldots{} it's possible. He'd have no trouble convincing women to
sleep with him or bear his children as his costumed self though. So it
must be something more ethereal, something that he can't get as
Superman. True, honest friendship untainted by his brute strength and
speed, not to mention his celebrity? Or perhaps just the thrill of
deception? There's some historical precedent for it. Tsar Peter of
Russia used to dress up like a workman and go among his people.''

``Peter the Great was six foot eight in a time when the people of Russia
were starving,'' said Mercy. ``It was because he was tsar that no one
dared broach the subject, but surely they knew the man by his height
alone. It's the same with Superman. They'd recognize him.''

``Perhaps,'' said Lex. ``But when people look at Superman, what are they
really seeing? They see the emblem on his chest, the bright colors of
his costume, and brilliant smile and the curl of hair that hangs down
just so. If you saw Superman in the street wearing a suit and tie, would
you recognize him in that new context?''

``Most likely,'' replied Mercy.

``Photographs of him are surprisingly rare,'' said Lex. ``When people
think of Superman, they don't think of him as he really exists, they
think of Norman Rockwell's painting of him on the cover of \emph{The
Saturday Evening Post}. Superman has posed for a single photograph, the
one that showed him and Miss Lane, and all the rest are of the man doing
some impossible thing, lifting cars above his head or flying through the
air, and the focus is seldom on his face. He keeps his interactions with
people short. The photographs from the courthouse, at least the ones
I've seen published, are always from a distance, the better to take in
his full appearance. They emphasize the muscles and the costume, not the
face. And they're published by the newspapers in terrible quality.
Perhaps putting him in a suit and tie wouldn't be enough, but if you
added a hat, an overcoat to hide his bulk\ldots'' Lex scratched his
chin. ``A change of mannerisms, a slouch, makeup, prosthetics, wigs, a
false moustache or beard, glasses, speaking in a low or high voice, or a
false accent, well, there are a large number of ways he could disguise
himself and go unnoticed. Charlie Chaplin once lost a
look‐alike‐contest, or at least that's what he told me. Very rarely do
people distinguish faces by their component parts, they look at
demeanor, gait, gestures, that sort of thing. They think in
caricatures.''

``You're getting dangerously close to pure ex post facto rationalization
for something you want to believe is true,'' said Mercy.

``I am,'' said Lex after a moment. Mercy could cut straight to the heart
of matters like few other people. ``I find it attractive because it
would reveal something hitherto unknown about the man. I've run into
failure after failure in trying to understand Superman, and this is the
first theory that might actually lead somewhere. Even if the probability
is low, we have to pursue it. Can we at least agree that Superman might
stand to gain something from having an alter ego and that he might be
able to pull off a long running disguise?''

``I can accept that perhaps he would be able to walk into a deli and
purchase something to eat without arousing any suspicion,'' said Mercy.
``But you're suggesting a sustained deception.''

Lex nodded. ``The quiet period, when he's less active, lines up too
nicely with working hours, and not just because there are fewer crimes
around that time. His movements point to a specific location that he
keeps going to or coming from. That data is fuzzy enough that it
suggests to me an inexpert attempt to hide the pattern, or perhaps just
an attempt by someone who wasn't clear on what methods could be used to
reveal the truth. Superman doesn't strike me as a mathematician.''

``They would still know,'' said Mercy. ``If they ever saw Superman in
the flesh, they would see his alter ego for what it is.''

``Perhaps not,'' replied Lex. ``No one is looking around for Superman in
disguise, because the concept is nearly unthinkable to them. No one
believes that they would work a day job if they had his powers. They
would become filthy rich and live a life of celebrity and hedonism.
Perhaps it occurs to his coworkers that the man they work with looks
like Superman, but they wouldn't immediately make the leap to thinking
that he actually \emph{was} Superman. Maybe they would make jokes, but
he would deflect them, or play along. Maybe he even has a few people in
his confidence. Think about it. Superman doesn't wear a mask. If he wore
one, people would wonder what was behind it. Many people have thought
that Superman was hiding something, but they think it's his spaceship,
or invasion plans, when all along it's just so\ldots{} mundane.''

``You've made up your mind,'' said Mercy. It wasn't a question, and
wasn't said with any trace of disapproval. She was simply informing him
of what she had observed, and as usual, she was right.

``Thank you Mercy,'' said Lex. In French this was rendered as ``Merci
Mercy'', a minor bit of wordplay that nevertheless brought a rare smile
to Mercy's lips. ``I believe that this lead is worth the resources
required to pursue it. Even if the odds of it being true are somewhat
low. The only question is the methodology.'' He smiled. ``Perhaps an
investment in the arts.''

``If you find him, will you expose him?'' asked Mercy. She asked without
any real curiosity or concern, and Lex was certain it was only intended
him to get him thinking about the answer before he walked too far down
the path. Mercy could convey quite a bit of information with a flat
affect.

``Lord no,'' Lex replied.

Lex Luthor saw antagonizing Superman in and of itself as having no
value, or more likely negative value. If Lex Luthor and Superman were
the only actors on the stage, Lex might even have refrained from using
the bombs, and instead relied solely on those methods that revealed no
foul play at all. It would have been more difficult, but on balance
probably worth it. Unfortunately, the stage was crowded with actors, and
some of them seemed to find great sport in trying to take Superman down
a peg. In Lex Luthor's public role as Superman's champion, he'd done
everything from funding legal efforts to defend Superman to penning
articles in favor of Superman's ridiculous moral stances. In the context
of the other actors, antagonism became a more acceptable risk only
because it would blend into the background.

Superman's presumed secret identity was a vector of attack, but not one
that Lex had any intention of using against him. The people who thought
they had something to gain from disrupting Superman's emotional state
were fools.

\begin{center}\rule{0.5\linewidth}{0.5pt}\end{center}

``The judge is dropping the case,'' said Clark as he laid his phone in
its cradle. He was visibly upset, which was rare for him. He pouted in a
way that might have looked adorable on a small child but just didn't fit
a grown man.

``There wasn't enough evidence,'' said Lois. ``It shouldn't have even
made it to the judge in the first place.''

``Calhoun is guilty,'' said Clark. ``I know he is.''

``You think he is,'' corrected Lois. ``And even if he's got to be guilty
of something, there's no guarantee that he's actually guilty of
manipulating Kramer. I know this story is near and dear to your heart,
but maybe it's time to let it go.''

``It's an injustice,'' Clark insisted.

``I should introduce you to my friend Vicki Vale,'' said Lois. ``She
works for The Gotham Gazette and I'm sure she could regale you with some
stories about real injustice. Actually, you might like her, I think
she's your type.''

After she'd said it she realized that it sounded like a bit of a low
blow instead of an olive branch. Lois knew Clark still had a crush on
her, and to him it might have sounded like she was making fun of him and
saying that he had a thing for female reporters. But Vicki Vale really
would be his type, and she really could set them up the next time that
Vicki came to town. Lois was never actively cruel to Clark, she just
liked to push his buttons. She liked to see him get all uncomfortable
when she swore around him. She would watch his face while she sucked
back a cigarette or took a glass of whiskey at her desk, neither of
which Clark approved of. These were small, harmless pleasures. Clark was
like a puppy dog in a lot of ways. She didn't want to hurt him.

``I didn't mean it like that,'' said Lois.

``Mean it like what?'' asked Clark. Clark had always seemed like the
kind of guy who would blush at the drop of a hat, but Lois hadn't seen
it once. He would get visibly embarrassed, but even after all this time
she kept looking for a hint of red in his cheeks or ears.
Disappointingly, it was never there.

``Nothing,'' said Lois. ``I was just thinking that she would like you.''

Clark gave her one of his familiar grins. Lois worried she'd gone too
far in rolling back what she'd said, but turned back to her typewriter.
She wasn't in charge of Clark Kent. And if Clark got his feelings hurt
because he misinterpreted something she said, well, it wasn't the end of
the world.

\begin{center}\rule{0.5\linewidth}{0.5pt}\end{center}

William Calhoun should have felt relieved that the judge had dropped the
case, but instead he just felt angry. He'd been accused of being in
cahoots with that bomber on charges so paper thin it would almost be
laughable. Willie had spent five of his fifty‐eight years in prison
though, and he never laughed about time in the clink. He'd sat down with
his lawyer and looked through the evidence himself, and could admit that
there was an implication there, but it wasn't even firm enough that he
could say he'd been framed. Even if it was just coincidence, it pissed
Willie off to get called out on something he'd had no part in,
especially considering all the things that he was actually guilty of.

It was Superman's fault. Superman had barged into Willie's bar and
announced as much, and it must have been Superman who whispered in the
right ears to get the case moving forward. Superman was a prick of the
highest order. Worse, people listened to him.

Luckily, Willie's schemes were paying off. The barrage of lawsuits had
mostly been a nuisance to keep Superman tied up in court, but some of
them had been taken further than he'd expected. Three decisions were due
to come down from the Supreme Court, and if Preethi v. New York went the
right way, Superman would be bound by all sorts of rules. Superman had
already agreed to abide by the rulings no matter what they were, and so
far the man had never broken his word. It made him predictable, and
Willie hoped he could use that against him.

One of Willie's early tactics had been to have people accuse Superman of
everything under the sun, to try to smear the alien's name if not
actually get him in trouble with the police. Willie had paid a young
girl to claim rape, and a few other people as witnesses. No one had
believed it though, and the girl had crumbled after a confrontation with
Superman on the steps of the courthouse where he'd been kind, courteous,
and forgiving. After that it was tough to find people to make false
allegations, and though Willie had searched, he'd never found someone
with a real criminal complaint. It occurred to him that Superman was
becoming so universally loved that even if Superman did do something
truly evil most people wouldn't believe it.

Slander and libel weren't working, and Willie was being bled dry.
Business had been brought to a near halt. There had to be a way to turn
the tide against Superman, and Willie was willing to do anything to
figure out what.

\begin{center}\rule{0.5\linewidth}{0.5pt}\end{center}

Hershel Whitman had become governor of New York when Franklin Delano
Roosevelt had won the Presidency in '32, and he was in it for the long
haul. The state of New York was most famous for Metropolis, its crown
jewel, and nearly half of the people in the state hailed from that city
or the surrounding greater metropolitan area. Ever since Superman had
shown up from out of deep space, politicians had been clamoring to be
seen as associating with Superman, and Whitman was certainly no
exception.

From a politician's perspective, Superman was perfect. He didn't upset
the apple cart, he didn't hold public opinions, he'd had nothing but
positive effects on the rate of crime in Metropolis. As the incumbent,
it would be nearly impossible for Whitman to lose his next election if
the people were happy, nevermind that he hadn't had all that much to do
with Superman. Most of the hard work of governance was in building roads
and bridges, passing funding measures, and wrestling with the other
parts of state government to hammer out laws. The vast majority of
people didn't place their votes because of anything sensible like the
actual work that was done, they would see Superman flying through the
air and think ``governor Whitman must be doing something right''. The
bombings had been a black mark, but the city was recovering better than
anyone could have hoped for, and thankfully the bomber had hung himself
and spared everyone the ordeal of a lengthy trial. Whitman hated the
inevitable appeal for clemency from death row inmates.

Whitman would have taken a meeting with Lex Luthor no matter what it was
about. The man was a billionaire after all. When Luthor had asked to
discuss a public‐private partnership of the arts, Whitman couldn't help
but feel that someone up there was looking out for him. Whitman was a
strong supporter of the New Deal policies, and there could be no
downside to adding in a billionaire's funds.

``There's much discussion about you, you know,'' said Whitman with a
smile. Prohibition had been brought to an end, thankfully, which meant
that a man could enjoy a martini on his veranda without having to worry
about scandal. A hot summer had made way for a cool autumn, and
Whitman's two children played in the yard.

Luthor shared the smile. ``I'm sure that tongues will wag. What do they
say, I wonder? That I came up from nowhere?''

``Things of that nature,'' said Whitman. ``I dare say there's a risk
you'll be named Metropolis's most eligible bachelor. There's a mystery
about you people quite like. You were born in Southside, if I recall
correctly, and the charitable work you've been doing there has been
admirable. Yet prior to Superman's arrival, you were known only inside
the world of business, and then more as a name than a man.''

Luthor shrugged, and looked out at the yard at the children. They'd
invented some game that involved ever more elaborate cartwheels. ``I've
never wanted fame,'' said Luthor. ``For a time I wanted money, but I
think I have enough of it to last me for a good long while. No, now is
the time for giving back. Superman has shown me that. And that's
precisely what I'm here to talk to you about.''

``You have my full attention,'' said Whitman.

``Simply put, I would like to fund the arts. I'm not an artist myself, I
can acknowledge that, but I have certain ideas that I think would help
towards increasing the beauty of our beloved city and showing off its
character. Now I'm aware that the Public Works of Art Program has run
its course, but I was just speaking with Harry Hopkins over the phone,
and he suggested that a pilot program might be just the thing. They're
getting close to putting together a second New Deal, which they hope to
include some arts in, and I think we might be finished with what I had
in mind before the bill goes through Congress. It might help grease some
votes, as it were.'' Luthor sipped at his martini. ``I would put in a
good deal of the funding of course, but I was thinking that perhaps
working jointly with the state could be mutually beneficial. That sort
of partnership isn't unheard of.''

``Of course,'' said Whitman quickly. Lex Luthor was becoming known as
quite the philanthropist, and the photo opportunities would help in an
election year. There were vague rumors about something criminal in
Luthor's past, but the man had been born in Suicide Slums and if
anything he was stronger for the narrative of reform.

``I have three in mind,'' said Luthor. ``The first is a statue, that I
think would look nice in Fort Hob's Park, though of course that's
negotiable. Not one of Superman, but something close I think, clearly
inspired by him. The idealized man, cast in bronze and standing tall, a
reminder for each of us to be the best person that we can possibly be. I
believe this is the lesson that Superman intends for us to take. It
would capture the zeitgeist, don't you think?''

``I do like the sound of that,'' replied Whitman. There would be an
unveiling ceremony, and Whitman would be standing in front of the statue
holding a pair of oversized scissors. He rather enjoyed the mental
image.

``The second is a large mural that will grace the length of Gerald
Ordway Drive, along a length of the West River between the Queensland
Bridge and Dockside,'' said Luthor. ``I have no specific vision there
beyond it showing a progression of the city from its humble origins to
the future we're striving for, perhaps something in mosaic. Metropolis
is the City of Tomorrow after all, and I think it would be nice to pay
some tribute to our roots as well as our aspirations.''

``Very doable,'' said Whitman. ``I'll need to speak to the mayor and the
city council about it, but very doable indeed.''

``Of course,'' replied Luthor. ``I'll be sure to put in a few words as
well.''

``And the third?'' asked Whitman.

``For the third, I want a photography exhibit,'' said Luthor. ``Sharp,
candid photographs of the people of the city. As I picture it, we'd hire
some photographers and park them downtown, to get a full sampling of the
lifeblood of Metropolis and the rhythm of workers coming and going. When
we're finished, we'll gather these photographs together and display them
in a gallery --- I have just the one in mind --- packed from wall to
wall in order to show the full breadth of humanity from the immigrant
populations to the high financiers. I believe it would be a marvelous
demonstration of both our similarities and our differences. More than
that, people who aren't normally interested in the arts might stop by to
try to find their own photo, or the photos of their friends. My
provisional title is `Faces of Metropolis'. I'd like some creative
control over that one, since the artistry will be in how we compare and
contrast the people we capture rather than the photographs themselves.''

Whitman nodded, still thinking about the political opportunities. He was
up for re‐election in November, and while there was little chance that
the projects could be completed by then, he'd be able to use this deal
with Luthor in his stump speech. He could spin it to sound like his own
idea, a melding of business and government for the improvement of the
lives of the citizens of the state. The project would surely create
jobs, but more importantly it would be a highly visible way of creating
jobs.

His children ran towards the house and poured themselves tall glasses of
lemonade before dashing back off into the yard again. June was eleven
and Robert was nine, and a father couldn't ask for better.

``I enjoy children,'' said Luthor. ``I've thought about having a few
from time to time. But more and more I find myself thinking of
Metropolis as my child. I want nothing more than to help her grow, to
protect her from harm, and to make her into the best city she can
possibly be.''

Whitman nodded. He found himself quite liking Lex Luthor.

\begin{center}\rule{0.5\linewidth}{0.5pt}\end{center}

``Calhoun just got arrested again,'' said Clark with a smile.

``What are the charges?'' asked Lois. ``Something solid this time?''

``Racketeering, murder, conspiracy to commit murder, loansharking,
illegal gambling, obstruction of justice, bribery, and tax evasion,''
said Clark.

Lois let out a low whistle. ``That's a long list. Any of them that will
stick?''

``All of them,'' said Clark with confidence.

``You're too close to the story, Clark,'' said Lois. ``And it's back
page material anyway. If Superman's involved it might be one of the
first cases that hinges on the outcome of whatever the Supreme Court is
doing, but that only bumps it up to page four or five.'' She looked him
up and down. Usually Clark wasn't so happy. The bombings had begun to
fade into the background, but Lois had found that it affected people in
different ways. She'd gone drinking in one of the clubs, and the
conversation had dropped into awkward silence when someone mentioned
that they'd had a friend who died in one of the blasts. Clark seemed
certain Calhoun was behind it, and Lois didn't think he'd get his
closure until Calhoun was in jail or dead. ``Look Clark, take your mind
off this. Justice takes time. Write up the story and then just forget
about it until the verdict comes in. Perry's not going to want to devote
too much space to it.''

``Alright,'' said Clark, but Lois didn't miss the pleased look on his
face as he pecked away at his typewriter.

\begin{center}\rule{0.5\linewidth}{0.5pt}\end{center}

A dozen photographers were sent downtown, where they snapped picture
after picture of people going to or leaving from work. They had cards to
hand out, and by and large most people were game. Pictures were taken
even of the ones that didn't seem too keen on the idea. The shots ranged
from candid to posed, with some being simple headshots and others taken
from a balcony or second story to capture everyone on the streets.
Ideally, Superman would be hiding somewhere among them. Of course, it
was possible that Superman would see the photographers and simply turn
the other way to avoid them, but Lex had been trying to work out the
alien's psychology for a while now, and felt that it was unlikely. If
Superman really did have a secret identity, it was probable that he
enjoyed being a normal human, and what could appeal to the alien more
than being simply one of many, a face in a sea of faces? Besides that,
Superman wouldn't want to be seen avoiding the cameras, because that
would be just as conspicuous.

There were too many people to photograph them all. The Emperor Building
and the Daily Planet Building were each within the four block area, and
the Emperor Building alone had 10,000 workers. Still, a good number of
people could be photographed, and if Lex was right, Superman himself
would be attracted to the photographers, no matter how ill‐advised that
would be. If the plan failed to work, there were other, more risky
plans. Private investigators could be set to work, company payrolls
could be combed through, and hard data could be examined. The trick was
to find out who Superman's secret identity was --- if he had one ---
without tipping him off.

It was late November by the time Lex and Mercy sat together in his
lead‐lined cabin some distance from Metropolis and sorted through the
photographs.

``Dark haired white male, likely above six feet tall,'' Lex had said
when they'd first begun. ``Superman is six feet and four inches tall,
when he's actually got his feet on the ground. We can't rule out that
the identity we're looking for has a slouch, or an affected limp, but
there'd be no changing his physical size, not unless there's some power
we haven't seen yet. We can't rule out that he wears a wig in his daily
life either, so set aside all those photographs with tall blonde men as
well.''

``Yes sir,'' said Mercy. She worked with quiet efficiency, sorting
photographs into various piles with Lex. It was boring work, and quite
slow, especially as the faces and people all began to meld together. It
was in the second day of this that Mercy found a picture of Lois Lane.
When she slid it across for him to look at, Lex saw Superman standing
next to her.

``It's him,'' said Lex, and Mercy moved around to look over his
shoulder.

``Are you sure?'' asked Mercy. ``I would have put it in the pile for
later review, but I'm less immediately convinced than you are.''

``He's the perfect mockery of humanity,'' said Lex.

The man clearly didn't want to be there. Lois Lane was as feral and
energetic as ever, staring directly into the camera with a winsome
smile, but the man was looking slightly off to the side. He was tall and
large, and looked slightly disheveled. Your eyes were attracted to the
notepad he tucked into his jacket pocket, then to the glasses that were
so thick you could barely see his eyes through them. Almost immediately
you'd peg the man as an oaf. He was so unlike Superman that it had to be
him.

``Superman always holds his head high, with his jaw thrust out,'' said
Lex. ``This man spends most of his time looking down, with his chin
tucked in. It disrupts the lines of his face, makes him less noticeable.
But the nose, you can tell from the nose it's the same man. It's him.
It's Superman.'' Lex flipped over the photograph. The idea had been for
the photographers to capture essential information from their subjects
wherever possible, but from the sampling so far it was clear that not
all of them had been so diligent. In this particular case, Lex Luthor
got lucky, and a number of nascent schemes for manipulating Lois Lane
into giving up information were quickly put to rest.

\emph{Lois Lane and Clark Kent, reporters, outside Daily Planet bldg.}

\begin{center}\rule{0.5\linewidth}{0.5pt}\end{center}

\emph{Author's Note: This chapter once again grew too long, so again I'm
splitting it up into what I think works best for the story breakdown.
Ten thousand words seems a little bit long for a chapter, and that's
what I was approaching. Chapter 7 will be posted on Sunday.}
