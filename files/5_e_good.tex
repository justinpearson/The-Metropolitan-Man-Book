\chapter{A Stopped Clock}\label{a-stopped-clock}

Lois Lane had been walking down 15th Avenue looking for a place to eat
breakfast when she'd heard the bang from the next block over. She'd
started running towards it seconds after she'd heard it, while everyone
else on the street was looking around like they'd missed something. If
they'd read the paper, they'd know that Superman had predicted that the
Clockwork Bomber would be back. Two weeks had passed and whatever leads
Superman had been following, they hadn't led anywhere, since no arrests
had been made. Lois looked down at her wristwatch as she ran --- it was
almost exactly nine in the morning.

Lois hadn't been close enough to the other bombs to get there in time;
when she or Clark showed up, the whole thing was already over, and the
mad panic and confusion that followed a bombing had given way to shock
and grief. This one was different, a chance to be close enough that she
would be one of the first on the scene. As she turned the corner and ran
past an appliance store, she could see the debris strewn out over the
street and the broken windows. People were still picking themselves up,
and a few were bleeding, but it didn't look nearly as bad as the other
bombs had. She was fishing a pencil and paper out of her purse when the
second bomb went off. It was small and subdued, much softer than the
first, and there were shouts of surprise but few of pain.

Lois started forward, just as Superman arrived on the scene.

He moved at speed, darting into the damaged storefront so fast he was
little more than a blur, and leaving minutes later. He was carrying what
looked like a box, and trailing yellow‐brown smoke behind him. Lois
tried to follow his movements, but after only a few seconds she'd lost
him. He was back half a minute later, and landed right in front of Lois.

``Can you smell anything?'' he asked.

``Horseradish,'' said Lois. The pieces clicked into place. ``It's
mustard gas.''

Superman nodded, and was back in the fray in moments. Lois sprang into
action, calling out to people to get away from the site of the
explosion. If she could smell the mustard gas, that meant that she was
too close. She tried to remember what the medics used on people who'd
been exposed mustard gas. Her father was a general in the army, and had
fought in the Great War, but it was too far before Lois's time. The most
she could do was to get people away from the gas, so that they wouldn't
suffer from exposure. Mustard gas was an insidious poison mostly because
it took a while to take effect, and if you didn't know what the odor
meant you wouldn't think to take action until long after it was too
late. Lois concentrated on getting people to safety, and yelling out
instructions. It caused blisters, not just where it touched exposed skin
but in the nose and throat as well. It could damage the eyes so badly
that you'd go blind. If you weren't killed by the swelling of the
throat, you could still be made mute. She ripped at her blouse and
fashioned a crude mask for herself, and helped others to do the same.

After everyone was clear of the gas, or at least in an area where they
could no longer smell it, Superman landed next to her.

``Call the radio stations, tell people to stay inside and keep their
windows closed. If I'm right, the next one will come in six hours.'' She
nodded, but didn't really need to be told what to do. She could keep a
cool head under pressure. Superman was crouched down and ready to launch
into the air when a buried thought surfaced.

``Wait!'' she called, worried that he would be a mile away by the time
the words left her lips. But he stood up and looked at her, puzzlement
on his face. She took a deep breath. ``You said that the bomber was
trying to get to you,'' she said. ``Why not let him have you? We could
send a message out over the radio, and make a deal. Even if we can't
disarm the bombs it'll let us evacuate people.''

Superman hesitated. She could swear she saw his eyes blur as they moved
around to take in the crowd in a fraction of a second. ``Lois, I don't
know whether or not these things will kill me. I don't even know if the
mustard gas is going to have an effect on me. I breathed in more than
anyone else before I realized what it was. He wants to kill me, that's
the only kind of deal I think he'd listen to.''

``But you'd rush in to save people anyway,'' said Lois. ``We're just
cutting out the possibility of collateral damage. You'll be fine.''

Superman stared at her, and she was sure they were both painfully aware
that everyone around them was listening in on their conversation. Some
were outright staring.

``No,'' said Superman. ``I don't negotiate. And if the bomber wants me
at the site of these bombs, I'm not going to play into his hands.''

He hurled himself into the sky and flew away before she could figure out
how to respond to that. Had Superman just said that he was handing
Metropolis over to the bomber?

\begin{center}\rule{0.5\linewidth}{\linethickness}\end{center}

``An officer Kennedy for you, sir,'' said Mercy.

``Thank you dear, I'll take it now if you please,'' replied Lex. He
calmed himself, and got into character, becoming a man who knew nothing
about what was happening across town. ``Officer Kennedy?'' he asked with
a pleasant voice. ``What is this regarding?''

``Ah, Mr. Luthor we really appreciated your donation to the Policeman's
Ball this year,'' said the policeman, ``And the chief was saying how you
wanted updates on anything real important having to do with Superman, so
he just thought I should give you a call to keep you up to date.'' It
wasn't a bribe per se, just a mutually beneficial friendship.

``Has something happened?'' asked Lex. He allowed some genuine‐sounding
concern creep into his voice.

``Well sir, it seems like the Clockwork Bomber is back, and he's working
with some nasty stuff. The boys say that it's mustard gas, like from the
trenches of the Great War?''

``I'm familiar, yes,'' said Lex. ``Superman came in to save the day?''

``Well, here's the thing,'' said Kennedy. ``He came down and pulled
people out of the gas, and said it was the Bomber come back, but then
that lady reporter told him that he should try to make some kind of deal
with the Bomber, because she seemed to think that the Bomber was trying
to kill him and that maybe Supes could save everyone a lot of trouble by
letting him try, for all the good it would do because he's invincible
right?''

``I see,'' Lex replied. ``And his response?''

``He said he wouldn't negotiate,'' said Kennedy. ``And then he just flew
off like he didn't want to hear any more about any bombs. Like he was
done helping out with them.''

``He's abandoning us?'' asked Lex.

``I don't know sir,'' said Kennedy. ``But it sort of sounded like it.''

``Thank you for the update, officer,'' said Lex. ``Tell the captain that
the police of this city continually reaffirm my faith in them.'' Lex set
down the phone without waiting to hear a response. He steepled his
fingers for a moment before remembering that he needed to keep up with
the role he was playing. Someone might notice if Lex responded to news
of an attack on Metropolis with only a look of quiet contemplation.

``Mercy, it appears that the Clockwork Bomber is back, and using
different tactics. I believe chemical agents were mentioned,'' he said.
Radioactive and biological too, though the person he was pretending to
be didn't know that. ``We should be safe in this building, but I want
you to start on calling the managers and telling them to follow the
drills and keep tuned to the radio. If it's like last time, the next
bomb will be in six hours.''

``Yes sir,'' said Mercy, picking up the phone before he was even done
speaking. She was invaluable, and likely could have handled the entire
crisis on her own without his instruction. She didn't know the full
extent of Lex's plans, but she knew enough to implicate him in a vast
number of crimes, and that was a mark of the extreme faith he placed in
her. Lex turned on the radio in his office. It was there more as
camouflage than to provide any information. Superman's movements and
actions were the most important thing right now, and he was skeptical of
the radio's ability to provide that information. Lex's other channels
were slower but more reliable, and there were enough of them that any
unreliability in one could be compensated for in the others.

Lex had a contingency plan in place. There were two couriers waiting by
phones in separate locations within the city. A message could quickly be
relayed to them which would send them to the nearest police station. One
courier had an encoded message, while the other had the one‐time pad
needed to decode it. Both pieces were in envelopes lined with lead. Once
decoded, the message would give the locations of all of the bombs, and
the buildings could be evacuated, saving lives and likely preventing
property damage. The only question was whether this was the proper time
to deploy that plan.

Lex had killed for the first time when he'd been fifteen. Willie Calhoun
had entered him in a bare knuckle boxing fight, and Lex had landed a
lucky punch that burst an artery in his opponent's neck. He'd been
rewarded with a twenty dollar bill and a slap on the back. He'd
committed his first actual murder later that year, when a shop owner had
gotten wind of the plan for a nighttime robbery and decided that the
best course of action was to lay in wait with a revolver and the lights
turned off. That shootout was the closest that Lex had ever come to
dying. His hands had been trembling when he shot the shopkeeper in the
face. He'd been less hardened then.

Lex took no special pleasure in murder. It stirred no passion in him to
see the life leave a man's eyes. It gave him no glee to hear about the
people who died or were injured by the bombs, just a certain sense of
sadness that he imagined other people might feel more keenly. He
certainly didn't feel any guilt. Lex sat back and looked at his watch.
The next bomb would be going off soon. He tried to make a careful
consideration of the possibilities.

It was possible that Superman had a weakness to biological, chemical, or
radiological attacks, as it was one of the only vectors of attack that
hadn't yet been tried. Numerous witnesses had reported seeing Superman
breathing, and none had specifically noted that he wasn't drawing
breath. Though Superman had never been seen coughing or sneezing, and
had surely endured smoke inhalation on an absurd level while engaging in
fire rescue, it was still reasonable to assume that he had a biology of
sorts, and that this biology could be disrupted in some way. He had not
yet been observed eating, drinking, or sleeping, but that might have
been something done during what Lex thought of as the quiet periods when
Superman was less active.

Superman might be afraid of dying to the bombs. If true, this would be
incredibly valuable information, and assuming that Superman didn't
radically alter his modus operandi in response to these attempts, it
would be fairly simple (in the scheme of plans that had to work around
Superman's powers) to stage some event to attract his attention and
deliver the poison while Superman suspected nothing. The reason that Lex
hadn't done it this way in the first place was the enormous amount of
planning, expense, and exposure that would have to go into doing that
for each of the thirty candidate attacks that he had planned. Mustard
gas, phosgene, chlorine, contact poisons, pesticides, polonium; it saved
an enormous amount of time to simply allow Superman to know that someone
was trying to kill him and put him in a position where he would either
expose himself or expose an unwillingness to intervene.

The second possibility was that Superman was thinking of the future.
Superman had routinely refused to make deals with criminals who had
taken hostages, presumably because he knew that if he did, more
criminals would begin taking hostages in order to put themselves in a
position to strike a bargain. Similarly, if the bombs were only being
placed because the bomb‐maker expected Superman to show up, then
Superman's best course of action to prevent bombs from being placed was
to stop showing up. Of course, that would only lead to a change in
tactics, and not one that would likely result in better outcomes from
Superman's perspective. Lex had dozens of ideas on how to administer the
poisons if Superman refused to touch the bombs. But perhaps Superman was
under the delusion that his unseen enemy would stop trying over a little
thing like changing strategies.

Ultimately, Lex decided against using the contingency plan, at least in
the short term. The message from the police officer had been too vague,
and even if Superman had directly stated his intent to leave the rest of
the bombs untouched, it was possible that the alien was bluffing. Lex
didn't particularly like the prospect of martial law being declared, nor
the unfortunate economic impacts of a sustained series of bombs in the
largest city in America, but the dice were already cast. If Superman had
anticipated the bombings and outright stated his refusal, perhaps Lex
would never have spent the time and money going down that path, but with
everything set up the majority of the costs had already been sunk.

\begin{center}\rule{0.5\linewidth}{\linethickness}\end{center}

``Where the hell have you been?'' asked Lois. She'd been seen by a
doctor, gone home to shower, changed clothes, and then gone back to
work. It turned out that there wasn't all that much you could do about
mustard gas, and while the doctor had wanted her to wait it out to see
whether she would develop any symptoms, she was pretty confident that
she'd had a low enough dose, so she'd slipped out the door when he was
seeing to someone else. No way was Lois Lane sitting on her ass when
there was news being made. From what the doctor had said, people didn't
get worse all at once, so if it got bad she'd go in. She'd called Perry
to let him know she was alright, and then kept calling Clark because she
wanted updates.

Clark sat at his desk, typing up an article. He typed with both his
index fingers, punching the keys down one at a time. As she watched he
took a glance at the keyboard to see which key was which. Lois could
type so fast she very nearly hit the mechanical limits on her Underwood.
Speed didn't matter all that much for a reporter, but it was still
grating to watch him do such a poor job of something so basic.

``Clark?'' she asked. ``Where have you been?''

``Sorry,'' he said. He pointed to the typewriter in front of him. ``I
got a call in from the Midwest, apparently there was a Superman
sighting.'' He hadn't answered her question, but then again, Clark was
hopeless. ``You're taking point on the return of the bomber?'' He always
said ``bomber'', not ``Clockwork Bomber'', which Lois felt was a bit
petty. He was sore that he hadn't been the one to name him.

``I am,'' said Lois. ``Superman's flown the coop. He's said he's not
going to help out with the Clockwork Bomber.''

Clark turned to look at her. ``Why not?''

Lois shrugged. ``I don't know. I guess Superman wasn't sure whether it
would affect him or not.''

``Makes sense,'' said Clark.

Lois raised an eyebrow.

``I mean, let's say that you walked down a dark alley and got shot, only
to find out that the bullet didn't do much more than tear up your
blouse,'' said Clark. ``You might try shooting yourself again to see
whether you really were bulletproof, but maybe you'd be too scared that
you'd just end up with a gunshot wound. And you certainly wouldn't go
drink some poison, because maybe it would kill you.''

``I understand that,'' said Lois. ``Even if I buy that maybe Superman
doesn't know the full extent of how he's protected, he's still supposed
to be a hero. It doesn't take a whole lot of courage to walk up to a guy
with a gun when you know that his gun can't hurt you. Superman says he
wants to be a symbol and then runs away the first time he might get
hurt? That's what I don't get.''

``I guess,'' said Clark. He frowned. ``With Superman's powers, isn't it
better for him to stay alive and saving people instead of risking death?
I mean, how many people does he save in a week?''

Lois shook her head, and pulled a cigarette out. ``Clark, you're not
thinking in the long term. Superman might think that there's some risk
of dying, right? And he's got a general stance that he doesn't negotiate
with criminals, for the obvious reasons. But let's assume that this
bomber's got huge amounts of money, no morals, and an honest desire to
kill Superman, all of which I think are probably true. If Superman's
going to stay away from the chemical end of things out of a sense of
self‐preservation, then assuming Superman still intends to operate
within Metropolis that means that the bomber is just going to resort to
tricks. He's going to \ldots{} I don't know, cause a train derailment
and vent pesticides over the area. Superman shows up thinking it's a
legitimate threat, and then bam --- poison right in his face.''

``Because Superman can't figure out whether or not there's going to be a
trap,'' said Clark slowly.

Clark Kent wasn't as dumb as he looked. It had taken Lois a long time to
figure him out, but she was pretty sure that she knew what games he
played. Clark Kent wanted to be underestimated, because it would make it
easier for him to exceed expectations. People clapped with delight when
Clever Hans had done math, not because the math was impressive, but
because it was impressive for a horse. It was the same way with Clark.
You saw this four‐eyed Midwestern guy in the middle of Metropolis,
looking for all the world like he'd taken a wrong turn leaving the farm,
and then when he actually put out a competent story you couldn't help
but feel like he'd done something amazing --- like he was a horse that
could do math. But the thing was, if you were actually good at math you
wouldn't need anyone to think that you were a horse. There was more to
Clark than met the eye, but once you'd lived and worked with him for
long enough and recalibrated your expectations of him, Clark was simply
below average in every way that really mattered to Lois. He typed with
two fingers for Christ's sake.

``Superman's got a problem either way,'' said Lois. ``That problem is
that someone with means, motive, and intellect is trying to kill him. If
he doesn't deal with the bombs, it's going to be something else,
something that he won't see coming, I'm pretty sure of that. Making a
deal isn't ideal, but it would at least help for him to actually be the
symbol he talks about being.''

Clark looked to the ceiling, which was quickly becoming a universal sign
that powerful ears might be listening. ``Are we having this conversation
for his benefit?'' asked Clark.

Lois shrugged, which meant yes. She knew that Superman could hear her.
It would be better for Metropolis to not have a war between Superman and
whoever was behind the bombings --- and she had a few ideas of who that
might be. She was about to add to her argument when Perry's door slammed
open.

``There's been another bombing,'' he shouted.

``But it's too early,'' said Lois. ``Last time there was six hours
between bombs.''

``Either the Clockwork Bomber screwed the pooch, or the schedule's been
stepped up,'' said Perry.

``Let's go,'' said Lois as she turned towards Clark, but Clark was
already gone.

\begin{center}\rule{0.5\linewidth}{\linethickness}\end{center}

Sal Maroni was a Superman spotter, which really just meant that he sat
on a rooftop with a notebook and drank beers while looking out over the
skyline. He listened to the radio, usually some kind of music, and
smoked like a chimney. Spotting didn't pay all that well, but there
wasn't an easier job in the entire city. Sal had worked as a security
guard once, and this was just like that except there wasn't ever the
slightest amount of danger. In addition to the radio, the smokes, and
the beer, he had a comfortable chair he'd pulled up from his apartment
on the fifth floor and a parasol he'd bought at a flea market to block
out the worst of the sun. On an average day he'd see Superman half a
dozen times, and he would faithfully write down his best guess of
Superman's location, speed, and direction of travel. On a few occasions
Sal had been tempted to just take a nap and then make things up, but
he'd been told that his observations would be checked against what the
other spotters put down. He could see a few of them on other rooftops.

He'd heard the sirens earlier, and WGBS has switched from \emph{The
Adventures of Lolly Lemon} to reporting on the return of the Clockwork
Bomber. It was about two hours after that when Superman rose up from
near The Daily Planet Building, moving so fast that Sal might have
missed it if he hadn't been paying attention. In his notebook, Sal wrote
down the details, making some best guesses. There was a man named Lonnie
who sat at Grecco's Cafe. He took in the notes from the spotters once
night fell, and had taught them how to make the most accurate
estimations of speed, distance, and direction.

Sal enjoyed being a spotter. It was boring, most of the time, but boring
was the same as relaxing if you looked at it the right way. Another perk
of the job was getting to see the news in the making. Sal had seen
Superman go in for a slow landing on top of Daily Planet Building, and
then the next day he'd read the interview in the paper. It was nice, to
be able to see Superman flying and connect the dots later on. Sal would
read the newspaper and be able to make sense of what his notes actually
meant. More often than not, the crimes he stopped were small or private,
but sometimes something big would happen in Metropolis, and Sal would
get a glimpse of it.

When the radio started talking about bombs, Sal cracked open another
beer and settled in. Today would be a busy day for spotting.

\begin{center}\rule{0.5\linewidth}{\linethickness}\end{center}

Superman responded to the second bomb, and Lex felt a sense of relief.
There was no way to know whether it had been a bluff or simple
indecision, or maybe even poor information, but for whatever reason
Superman had decided to stick his neck out. Lex would have to arrange
another interview with Lois Lane in order to find out what Superman had
really said to her, but it would have to wait. That Superman hadn't
tried to make a deal with the bomber was not wholly surprising.

The selection of attacks to try had taken careful consideration.
Anything that caused a death throes had to be avoided, and Lex put a
preference towards those agents which would cause weakness or paralysis
in humans. There was no way of knowing whether Kryptonian biology was
similar, of course. Lex had considered the possibility that in
attempting to destroy Superman he might unintentionally cause the
disaster he wished to avert, but Lex was certainly not the only player
in this game, and their plots were far more dangerous than his. All the
more reason to take minor risks to kill Superman, when the other players
sometimes seemed to be doing nothing more than trying to piss him off.

There were forty‐eight bombs, spread out over four days, one every two
hours.

After the third bomb had gone off, he'd sent all his employees besides
Mercy home for the day and sequestered himself in his office. He had
adequate food and water, a set of fresh suits hanging in his closet, and
a private bathroom. It was more or less everything that he needed.
During this time of crisis, Lex would play things safely, and do nothing
too terribly out of character. He would offer a reward for information
leading to the bomber, he would offer to help the police in any way that
he could, and he would listen to the reports as they came in. The facts
could be collected afterwards, when the whole ordeal was over, but Lex
didn't think that the man he was pretending at being would apply harsh
scrutiny during a time of crisis.

There would be immense scrutiny. If the bombs simply stopped, the police
would go on the hunt. Lex made the call that would tie up the loose ends
and divert attention away from him. He was extremely skeptical that a
path could be drawn back to him, but Lex Luthor was cautious, and so a
false trail had been laid instead.

\begin{center}\rule{0.5\linewidth}{\linethickness}\end{center}

Officers Milheiser and Kennedy walked up the stairs, sweating in the
summer heat. They'd been working back to back shifts ever since the day
before when the bombings had started up again, as had most of the police
and firefighters in the greater Metropolis area. The mayor had briefly
talked about instituting martial law, but no one was keen on that. The
compromise was double shifts. The elevator in the building was out, and
it was just their luck that the apartment was on the tenth floor. It was
more or less how the last few days had been going for them.

``Any reason the captain wants us chasing this down?'' asked Milheiser.

``He said an anonymous tip is more trustworthy,'' said Kennedy.

Milheiser nearly stopped. ``How does that figure?''

``Well, there's a big reward out for information, right?'' asked
Kennedy. ``More tips have been flooding in than we could ever take a
look at, because there's no penalty for making stuff up and maybe if you
get lucky you get a little piece of the pie. So we got people sending us
all sorts of crap, gossip about their neighbors, reports about people
that they just don't like, paranoid fantasies and all that. Ten thousand
dollars is in the pot right now thanks to Luthor, and that's enough to
attract all kinds of crazies.''

``So the captain thinks that an anonymous tip is more trustworthy,
because no one stands to gain from it?'' asked Milheiser.

``You got it,'' said Kennedy with a strained smile. The heat was getting
to him.

``And the captain didn't stop to think maybe someone else would figure
that and send the pair of us to a building with no working elevator so
we'd have to sweat our asses off climbing to the top?'' asked Milheiser.

Kennedy had no response to that. He might have said that no one would do
that in a time of crisis, but he knew Metropolis well enough to know
that wasn't the case. He'd seen enough rioting and looting to come to
the conclusion that people were bastards.

When they got to the tenth floor, they knocked on the door, and found
that it swung in to the touch. Kennedy and Milheiser shared a glance and
drew their revolvers. It occurred to both of them that perhaps the
Clockwork Bomber had lured them there, just to make a point, but they
entered anyway.

In the center of the apartment, a young man was hanging from the rafters
by his belt. He'd been dead for hours, and the smell was utterly
offensive. Milheiser rushed to the bathroom to throw up, while Kennedy
made sure the place was cleared. It was a pretty cut and dry suicide,
with a kicked out chair beneath the young man, but Kennedy went through
the motions anyway. He stood the chair up and made sure that the hanged
man would actually have been able to stand on it, since he'd heard that
sometimes people would stage a murder to look like a suicide but forget
the details. He was vaguely disappointed when the chair was the right
height.

Kennedy had moved on to a small workshop area by the time Milheiser
walked out of the bathroom, wiping his mouth with the back of his
sleeve.

``Looks like our guy was a tinker, at least,'' said Kennedy. He leafed
through a set of schematics, pulling some out from the bits of
electrical wire and springs, trying to make heads or tails of it. There
were copious notes and detailed drawings, but it didn't snap into focus
until Milheiser unearthed a book titled ``The Manufacture of
Explosives''.

``It's really him,'' said Milheiser with a shake of his head as Kennedy
began laying out the papers. The body was in the other room, and would
have to be dealt with, but neither of them relished the thought of going
up and down the stairs again, which they'd surely have to do a few times
before the day was out.

``Let's call it in,'' said Kennedy. ``Looks like there's an address
here, might be the place where the bombs were made.''

\begin{center}\rule{0.5\linewidth}{\linethickness}\end{center}

Lex Luthor was a people person. People told him their problems, and he
found solutions. It had been that way ever since his childhood on the
streets of Suicide Slums, the worst neighborhood that Metropolis had to
offer. So far as anyone besides Mercy knew, Lex had gone legitimate. The
vast majority of his criminal enterprises were run through various
intermediaries, who knew him only by codewords over the phone. Since
Superman's arrival, Lex had let much of that go to rot. It was easy
enough to make money in perfectly legitimate ways if you had a mind as
keen as Lex's. Instead, he used his network of slush funds and discreet
contacts in order to facilitate his private war against Superman.

Harry Kramer had been a piece of serendipity. He'd been an expert in
explosives by the age of sixteen, thanks in part to a father who had
done demolition work at a mine upstate before losing his life to a
faulty detonator. Kramer liked to blow things up, and got involved in
professional fireworks before he was discharged after an incident that
lost his boss the use of two fingers. It was when Kramer got hired on to
do a bank job that he came to the attention of Lex. The job had been an
abject failure, though it was through no fault of the explosives, which
had worked perfectly. Kramer had been willing to hire himself out again,
but he was difficult to work with, and there wasn't much call for an
explosives expert in the criminal underworld of Metropolis. Harry had
been working as a grocery bagger until Lex needed his expertise. Lex
could design the bombs easily enough, but wasn't willing to put himself
in a position where he could be seen making or delivering them. He'd
given Harry a new apartment and a workshop, along with a large amount of
freedom.

A careful examination of the evidence would reveal a hidden hand behind
the Clockwork Bomber. Harry Kramer had received a large amount of money
from an uncle down in Georgia, and if that thread were tracked down the
sham would be revealed, and point back to Metropolis. This was part of
Lex's design.

\begin{center}\rule{0.5\linewidth}{\linethickness}\end{center}

There were forty‐eight bombs in total. Thirteen were found by Superman
prior to detonation, and he managed the evacuation and the removal or
controlled detonation of the bomb. Any hesitance he'd displayed in front
of Lois was completely gone, and over the course of the extended
bombing, the enactment of martial law, and everything else, he'd proven
himself to be a complete hero in every way. When he wasn't helping with
rescue efforts or stopping the bombs, he could regularly be seen
watching over the city.

``You look like shit, Clark,'' said Lois when they got back to work.
Most of the businesses had temporarily closed after the second day;
\emph{The Daily Planet} had closed on the fourth, when some people were
saying that the bombs would keep going off forever.

``I didn't get much sleep,'' he replied with a yawn. ``I kept worrying
that my apartment was going to explode out from under me and I'd die
choking.''

Lois had escaped the mustard gas with only a small blister on her left
hand and a light cough. She considered herself lucky. No one had died
from the mustard gas, but it was one of the tamest things that had come
out of the bombs. She'd spent the days off from work pacing back and
forth, sleeping heavily, and using her home phone to try to get a break
in the story, though the phones were nearly as useless as the radio.

``Who do you think did it?'' asked Lois.

``They caught him Lois,'' said Clark.

``One man, working alone, and you believe that?'' she asked.

``He came into a lot of money,'' said Clark. ``He was smart and
deranged. Everyone who knew him thought that it made sense after the
fact, and some of them had even reported him to the police. If he hadn't
switched apartments they'd have got him.''

``Sure,'' said Lois. ``And if you buy that I've got a bridge to sell
you. The police are investigating it all. They've found a few of the
guys that planted the bombs, and a couple of places that delivered the
materials used for construction. I don't know anything about making
bombs, but I can believe that a single person might be able to make as
many as he did, if given enough time. But add on all the logistics on
top of that, all the scoping out of locations and arrangements for
delivery? No, no way he was acting alone. I'm not saying that we can
solve it from our desks, but think about it Clark.'' She looked at him.
``Someone intelligent, resourceful, wealthy, with deep criminal
connections and a strong desire to see Superman dead. There's one guy
head and shoulders above everyone else on that list.''

``William Calhoun,'' said Clark.

``The last crime boss of Metropolis,'' said Lois with a nod. ``If you
could follow the trails well enough, I have no doubt that they'd lead
back to him.''

\begin{center}\rule{0.5\linewidth}{\linethickness}\end{center}

William Calhoun was fifty‐eight years old, which was ancient for a crime
boss. When Superman had come along, organized crime had to either
toughen up or flee the city, and Willie seemed to be one of the only
ones willing to toughen up. Boss Moxie had continued on like nothing was
different, and now he was sitting in Sing‐Sing. Johnny Stitches and Toby
Whale had left for Gotham City, while Angelo Baretti simply evaporated
like mist. And that left Willie as a big fish in the biggest pond in the
world, with the only problem being that the pond was being shot full of
holes by a nut with a tommy gun. Willie had been working on the metaphor
for a while, and it still wasn't quite right.

Willie was looking over the books in his lead‐lined office, and trying
to figure out a way to get people to pay their bookies when there was a
commotion downstairs in the bar. Not really having any enthusiasm for
the drudgery of what he'd been looking at, Calhoun wandered down the
stairs. His two guards followed.

Superman stood in the middle of the Elephant Club, with everyone around
him giving him a wide berth. Superman was staring at Willie from the
moment he started walking down the stairs, and maybe even before that.
He could see through walls, the prick.

``Hello William,'' said Superman. His voice was calm and gentle as a
breeze.

Willie put on his most casual demeanor. He kept telling the boys that
they had nothing to be worried about when it came to Superman. Sure,
Superman would foil crimes and get them locked up, but he never hurt
anyone, not even in the process of arresting them. Micky Fingers had
stabbed Superman in the eyes and Superman had just stood there like a
statue. But it was hard not to think about what the man could do.

``You're trespassing,'' said Willie. He tried to keep his voice light.

``This establishment is open to the public,'' said Superman.

``Well you're blacklisted then,'' said Willie. ``I'll have to put up a
sign that says `No Supermen'.'' This brought a round of nervous chuckles
from the crowd.

``I'll be leaving soon,'' said Superman. ``I just wanted to let you know
that I'm watching you. You've been careful, but not careful enough.
There's nowhere that you can hide from me. There's nothing that I won't
do to bring you to justice.''

``Oh really?'' asked Willie, striding towards the Superman with a
confidence that he almost felt. ``Anything? Then I've got a deal for
you. Tear off one of my arms, and I'd be in so much pain I'd give you a
full confession for whatever it is you think I did. Go on, do it.''

Supeman didn't move. ``I'm not a monster,'' he said evenly.

``No, you're a monster alright, you just don't want people thinking that
you are. You don't want to get your hands dirty,'' said Willie. ``I've
heard from a bunch of guys that you're nothing but a big fat pussy, and
standing here looking at you I can see it's ab‐so‐lutely true.'' Willie
could feel his blood pumping in his ears. Months of frustrations at the
hands of Superman were coming to a head. Willie had tried to stay low,
but his organization could only stay starved of cashflow for so long.
Willie'd been funding lawsuits against Superman, false accusations and
red tape, along with whatever else he could think of. Some of the guys
talked about killing Superman, but that was a fool's errand --- the
bombs had proven that. Willie just wanted him to leave, to go bother
Gotham City or Blüdhaven.

``No one likes you,'' said Willie. ``No one wants you here. Get that
through your thick alien skull. You think the government doesn't have
plans to kill you? Hell, you think that they haven't tried?'' That was
Willie's best guess as to who was behind the bombings after talking it
out with Luthor. ``You do whatever the fuck you feel like doing and
expect us to praise you. Well I got news for you, it's not going to
happen. Eventually someone is going to find a way to kill you, and I'll
be first in line to piss on your grave.'' Willie spat at Superman, and
watched as the glob of phlegm hit him in the cheek. Superman could have
dodged it, probably could have reached across the room and grabbed a mug
to catch it in, but he'd just let it hit him.

``I just wanted to let you know that I know,'' said Superman. ``In
everything that you do, be aware that I'm watching you. When you're
arrested, it will be completely by the books. When you're convicted to
life in prison, I hope that they're able to rehabilitate you.'' Superman
didn't touch the spit on his cheek. He just turned and walked out the
door. The bar exploded into conversation, and Willie went back upstairs
to think about what it was that Superman had actually known.

\begin{center}\rule{0.5\linewidth}{\linethickness}\end{center}

Forty‐eight bombs, and not so much as a cough or a sneeze from Superman.

In his lead‐lined study was a large map of Metropolis, five feet to a
side, which took up a place of prominence on one wall. Stuck into this
map were pins with small flags on them, each of them a recorded Superman
sighting. The information had been collected from various sources, from
newspaper reports to eyewitnesses. Lex had dozens of people around the
city who worked as Superman watchers, and they would sit atop tall
buildings and make notes of the lone figure flying through the sky
whenever they could.

Lex was looking for patterns. Which directions did Superman come from?
Which directions did he go? What crimes did he tend to respond to, and
which did he ignore? What were his hours of activity? Lex had long
hypothesized that Superman had a base of operations somewhere, likely
the same place that his spaceship was stashed. Finding it would be a
godsend. The arrival of the Clockwork Bomber had provided a wealth of
data. Lex sat down to do some math.

Each arrival and departure could be defined by a vector, and these were
represented on the map by small lines drawn moving away from the pins in
different colors. Lex compared the times and directions, and began by
throwing out all of those vectors with known destinations. When he was
done, he was left with one‐thousand eight‐hundred sixty‐one vectors to
manipulate. He began mapping them in different ways, to see whether
Superman favored one side of the city over the other, or whether he
consistently came into the city from one direction. He found a slight
eastward inclination to arrivals and westward inclination to departures,
though given that the entire United States was to the west of
Metropolis, that might have just been because Superman often responded
to large‐scale crises outside of the city. Following that middling
success, Lex did some complicated math to make another map that showed
where vectors converged. He eventually circled ten square blocks in the
center of Metropolis. It was there that Superman kept going towards,
though that might have simply been because Superman spent his time
waiting in the center of the city.

It was close to a futile exercise. The data was bad. It was cobbled
together from too many sources, and too many of those sources were
unreliable. There were certainly data points that were lies told by
people who wished they had more interesting lives. Lex couldn't properly
trust the data, and so couldn't properly trust the conclusions that he
drew from the data. Worse, Superman was aware that people were watching
him. Still, it was better to grasp at straws than to simply give up.

Lex began segmenting the vectors into blocks of time. Even with
unreliable data, it was well‐established that Superman was less active
during working hours, and so perhaps it might be that paring down the
data would help to reveal something more. The big problem there was that
there was that the data became thinner, and even less reliable.
Nevertheless, Lex continued on. There were other plates spinning that
wouldn't need to be touched for a while, and in the meantime Lex could
pretend that he was getting somewhere. The math was somewhere between
difficult and tedious, and not at all pleasant.

When he was done, Lex frowned at the result. He circled four city blocks
on the map, slightly away from the direct center of downtown. He turned
to look at Mercy, who sat in a padded chair drinking tea and reading a
book.

``Mercy darling, my brain is failing me,'' said Lex.

``Sorry to hear that sir,'' replied Mercy, not bothering to look up.

``I've been staring too closely at this for far too long,'' said Lex.
``Eight o'clock in the morning to five o'clock at night. I can feel
something refusing to spring to mind there, something that's not quite
clicking.''

``It's standard working hours for most of downtown,'' said Mercy.

Lex turned back to the map. He stared at it. There was something he was
missing, some piece of the puzzle. Nine to five, but not on weekends. It
was fuzzy, painfully fuzzy, but the data was clear and the correlations
were real. Lex was on the verge of a breakthrough, if only he could ---

``Son of a bitch,'' said Lex softly.

\begin{center}\rule{0.5\linewidth}{\linethickness}\end{center}

\emph{Author's Note: As always, thanks for the
reviews/favorites/follows, which are always a pleasure to see. Thanks to
those people who've pointed out typos; you're making the story better
for people who read it after you.}
