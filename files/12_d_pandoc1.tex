\chapter{Finale, Part 1}\label{finale-part-1}

``And he's just gone forever?'' asked Jimmy. His girlfriend Eleanor sat
beside him, opposite Lois and Clark. It was somewhat emphatically not a
double date. Lois was trying her best to shift her position on Clark in
a way that he would actually believe. Eventually she would pretend to
see the light, or give him a chance, and they would presumably have a
relationship built on a foundation of lies. It left a sour taste in her
mouth, but since Superman was impervious to physical damage, he needed
to be anchored to the mortal world.

``It won't be forever,'' said Lois. ``He'll probably come back, once
he's figured some things out.''

``It's been a week,'' said Clark. ``Maybe what he'll figure out is that
he just doesn't want to help people anymore.''

Lois looked at him. A full week had passed, and it was still hard not to
marvel at how completely she'd been duped by him. Her pride was only
slightly salved by how much effort Clark seemed to put into it. He'd
changed everything about himself to put on the Clark Kent persona, and
there were a thousand subtleties to the performance that she hadn't been
consciously aware of seeing. Everything about Clark was a lie, only
there to fool people.

``He still cares,'' said Lois. ``He still considers himself an American,
I think. And there's been a lot less backlash than there could have
been.'' That had been thanks in part to the media embargo that had let
Lois get a head start on influencing public opinion. Superman had
powerful friends too, not least of which was the governor himself. There
had been no attempt to put out a warrant for Superman's arrest, and so
far as Lois could tell, no one was seriously considering trying to stop
him aside from her and Lex. A fair number of people even seemed to think
that Superman had done the right thing.

``You're the expert,'' said Clark. He shot her a smile, reveling in a
joke that he thought only he could understand. When she thought back
through all their conversations, she could see that he peppered in these
winks and nods to the truth, though he never said anything that
Clark‐the‐ordinary‐reporter‐with‐no‐secrets wouldn't say. It reminded
her of playing the game of double meanings with Lex. That was different,
because at least they were both in on it, and there was a point to it
other than gloating.

It did cross her mind that she was being uncharitable to Clark. Of
course she couldn't actually tell what his smiles meant, and it was just
as likely that he felt a fondness towards her extensive reporting on
him. But since Superman thankfully couldn't read minds, and since Lois
had to keep up a front at all times, thinking mean thoughts about Clark
was a form of private rebellion, and helped her to keep her sanity.

After dinner had wrapped up, the couples went their separate ways. Jimmy
had been dating the same girl for a while now, and things were getting
serious between the two of them. It made her unexpectedly sad, since all
the future seemed to hold for her was a sham of a relationship with
Clark, and abstinence from any meaningful --- or even meaningless ---
romances for fear of how he'd react.

``Can I ask a delicate question?'' asked Clark. They walked together
down the city streets. Even after being worn down by the city and losing
some of his innocence, he was still a gentleman, and had insisted on
walking her back to her apartment. It was questionable how real the
transformation had been in the first place, since he had been Superman
all along, and how could he have had any innocence left when he could
see how big of bastards people were to each other?

``Out with it,'' said Lois.

``When I asked you out, why did you say no?'' asked Clark.

``A delicate question indeed,'' said Lois. She let silence settle on
them while she thought about it. To his credit, he made no attempt to
rush her. ``You'd been working at the paper for two weeks,'' she finally
said. "I'd been working at \emph{The Daily Planet} for eight years,
since I was a teenager, and you were far from the first of our coworkers
to ask me out. I care about my job, Clark. Dating someone from the
office --- it doesn't matter who --- would be a recipe for professional
disaster. Even outside the office I have to think about whether my
relationships are going to be kosher. If I went on a date with a
politician, people would start saying that I was sleeping with him to
get a story. I can handle the rumors that crop up just from being in the
public eye, and the way people talk when they see a woman in a position
of power or authority, but I'm not going to invite more of it on myself,
and I think I would always have a small voice in the back of my head
that said they were right if I stepped over some abstract line."

She took a breath. All that had been true, but it wasn't all that needed
to be said to Clark. She'd been preparing for this conversation in one
way or another for the last week, and she was grateful that she hadn't
been the one to start it. ``And I didn't like you, not when you first
came on. You're different now. You've changed. I'd thought that the city
would chew you up and spit you out, but you didn't end up going back to
Smallville, you stuck with it and persevered. You're a better reporter
now too, someone who doesn't just rely on ---'' \emph{being able to see
through walls and listen in on private conversations} ``--- luck.'' The
pause had been barely perceptible. ``You'd better not hold this over me,
but I respect you now.''

She'd expected Clark to grin, but he only nodded. ``You've changed
too,'' he said. ``Especially after Superman showed up. You said that you
didn't want to invite rumors, but with him you just set that rule
aside.''

Lois stared at him. ``Clark, you can't possibly be jealous that Superman
and I --- no, it's ridiculous.'' Nothing had ever happened between her
and Superman, they'd gone on a single date together, and Clark
\emph{knew} all that. It was true that her attempts at playing the role
of Superman's girlfriend had been painful from a professional
standpoint, but Clark had no reason to be upset with her. They were the
same person. Unless the problem lay somewhere else, in which case Lois
thought she knew what to say. ``Clark, do you know what I liked about
Superman?''

``Past tense?'' asked Clark. ``You do know he might be listening in,
right?''

The gall it must have taken for Clark Kent to say that almost left Lois
impressed. She knew if she tried to have a conversation about Superman's
eavesdropping she'd be liable to go incandescent with rage, so she
skipped right past it.

``What I liked about Superman is that he was kind and gentle,'' said
Lois. ``He was good. And he liked me, even though I'm not very
likable.''

``You're likable,'' said Clark quickly.

``No, I'm really not,'' said Lois. ``I'm opinionated and hot‐headed, and
I like to push people's buttons. I work more than anyone really asks me
to, I stick my nose where it doesn't belong, and I turn every tragedy or
triumph into a story for consumption by the masses without really even
thinking about it anymore. People die and I think about what the
headline is going to be, and part of me knows that's just a way of
shielding myself. I know my good qualities, but likeability just isn't
one of them.''

``You're intelligent, driven, principled ---''

``Clark, I said I know my good qualities,'' said Lois. She had to wonder
whether any of that was what had attracted him, or if he'd simply caught
a glimpse of her legs and worked backward from there. And with Clark it
was always possible that it was another lie. ``We've gotten off on a
tangent, but what I was trying to say is that he liked me, and I liked
him, and the rules I'd set down for myself seemed really arbitrary. If
Superman had picked me out of all the reporters in the world --- hell,
all the women in the world --- then maybe I was just being obstinate
about how I wanted to be seen by the people around me.'' The lies
spilled out easily, but the next part would be harder. ``It wasn't that
I broke the rule for Superman, it was that Superman made me see that it
was a rule worth breaking.''

They had reached her apartment building. Lois turned to look at Clark.
``Look, I don't know whether you still feel the same way about me, but I
do like you Clark. And if you asked me out again, maybe my answer would
be different.''

\begin{center}\rule{0.5\linewidth}{\linethickness}\end{center}

Creating more kryptonite proved to be a challenge.

Lex took a minor risk and shipped portions of the kryptonite to two
different facilities which both operated as part of the Scientific and
Technological Advanced Research Labs. Robert Meersman had wanted to
create a series of research laboratories which were disconnected from
any corporate or governmental interests. Lex had quickly seen that the
end result of such a philosophy would surely result in either collapse
at worst or organizational drift towards the very same set of problems
which it was trying to escape at best. A combination of money and mild
coercion had put S.T.A.R. Labs in his pocket, though few people knew the
source of their funding, and fewer still knew the primary beneficiary of
their research.

The kryptonite was given the name PU‐356. It had supposedly been found
in the core of a meteorite, and transferred for analysis shortly
afterward, all of which was backed up by a trail of falsified
documentation. It was the work of the labs to analyze the PU‐356 into
its component pieces, generate a full list of its properties, and then
attempt to make more of it. It was semi‐crystalline in nature, and after
only a few days of work it was suggested that more might be made by
introducing a shard of it into a super‐saturated solution which
contained the composite elements. It wasn't entirely clear how the
structure of the PU‐356 produced the properties that it did, but the
elements which made it up were eventually sorted out, and a multitude of
experiments were run to achieve synthesis.

This was part of the reason that Lex had taken the risk of shipping the
kryptonite outside of his immediate control; it would have taken him an
enormous amount of time to arrive at the proper solution to creating
more kryptonite, which involved enormous amounts of energy, a small
shard as a catalyst, and a wide variety of purified elements in very
precise quantities. The process was slow and costly, but more of the
kryptonite was produced with every passing day.

On those nights when he knew that Clark Kent would be occupied with Lois
Lane, Lex began a slow renovation of his house.

\begin{center}\rule{0.5\linewidth}{\linethickness}\end{center}

Before she'd known that he was Superman, Lois had imagined Clark Kent's
apartment as being relatively bare, with little more than a picture of
his mother and father and a large cross. After she'd seen the truth,
she'd imagined only a more extreme kind of minimalism; no toiletries, no
toothbrush, no food in the cupboards or any other sign that a real
person lived there, because no real person \emph{did} live there. The
world was Superman's playground, and he had no real needs beyond those
he decided to indulge in.

She had been wrong. Clark's apartment was slightly smaller than her own,
but just as packed with mementos, curios, and pictures. Where Lois had
accumulated souvenirs over a lifetime of travel, Clark had instead
pulled in pieces of Metropolis. It wasn't just the photographs that
lined the walls, there was a collection of bric‐a‐brac on top of one of
the short bookshelves; a model of the Emperor building,
three‐dimensional map of the city made with pressed tin, and a signed
baseball among others.

``Jimmy took most of the pictures,'' said Clark. He seemed nervous,
though there was no way of knowing whether that was his usual act or
whether he was actually tentative about letting her see how he lived.

Her anger was starting to fade, which was a problem. It had been five
weeks, and though she still felt hot sparks of rage, it was hard to stay
as angry as she'd been in the beginning, especially when she was wearing
a layer of deception over her feelings. Lying to Clark day in and day
out meant building up an image of how she would feel about him if she
didn't know, and almost by definition that meant some level of empathy.

She'd dealt with a number of battered women in her time, either as part
of covering a story or through one of the social programs she was part
of, and she had always found it puzzling that they would sit there with
a black eye and say that their husband or lover had done nothing wrong,
or that it wouldn't happen again. It was a lie, but it was one that they
were able to convince themselves of. She'd never thought that she would
be a woman like that, but now that she and Clark were courting, she
could see it happening to her. She would tell the lies so much that she
would start to believe them, because the alternative was making Superman
upset. And if she tried to get help, she would be laughed off and
alienated, and of course it would only make him angry. There would be no
escape.

She could imagine Clark hitting her. She could imagine his fist going
straight through her skull, pushing aside bone and flesh like it wasn't
even there, just like he'd done with Calhoun. She kept more than enough
secrets from Clark, and a few of them might set him off. She could have
stopped meeting with Lex, but that would mean giving up hope that
Superman could be brought down to mortal levels. She was willing to give
up her personal happiness if it meant keeping Clark pacified, but she
had to know for sure that there was no other way --- some more permanent
solution. Lex had not yet declared that it was hopeless, but when he
did, Lois would focus all of her efforts on being a good girlfriend, and
eventually a good wife.

``Do you like it?'' asked Clark.

``It's not what I expected,'' said Lois. A glint of metal in the corner
caught her eye, and she walked over to stare at it. ``You can't possibly
expect me to believe that you play the saxaphone.''

``No,'' said Clark with a bashful smile. ``I bought it thinking that I
would learn, but it turns out that I don't really have an ear for
music.''

She had to wonder whether that was another bluff. Once you knew that
Clark Kent was a disguise, it called into question everything he said.
His appearance was a lie, the thick glasses most of all. His apartment
was clean, in a way that suggested that it wasn't always so pristine,
and she wondered whether that was another piece of the elaborate
deception he'd woven for her. She felt a flash of anger coming on, and
did her best to divert it.

``My father made me take harp lessons,'' she said. ``He must have
thought that it would make me more ladylike, but I hated the harp and
never practiced. After I gave up, he kept the harp in the living room,
and it was like an albatross around my neck. And we moved a lot, you
remember, so for years my father just carried the harp with us from
place to place.''

``I'd be interested to meet him,'' said Clark with a frown.

``It probably won't happen, at least for a while,'' said Lois. ``He got
pulled out of an early retirement to work on some secret military
project. He wasn't the best father, but he trained my sister and I well,
and I think we're tougher for it. He wanted boys, and didn't get them,
and on top of that he raised us alone.''

``We have that in common,'' said Clark. ``Not being raised alone, but
unconventional childhoods.'' He tapped a photo of Martha and Jonathan,
which held a place of privilege on his wall --- the only piece of his
life in Kansas that was visible. ``They were too old to be raising a
child, by most people's estimations. Sometimes I think everyone has
their own story that's just as unique and interesting as your own, if
you could only get to know them.''

Clark made a dinner of stuffed chicken and mashed potatoes. It wasn't
really a surprise that he was a better cook than she was, since he would
almost have to be, but it still irked her just a bit. The thing was,
there wasn't really anything wrong with Clark if you could subtract out
the Superman business. If he were truly, honestly Clark, he wouldn't be
so bad, especially given the ways that he'd changed over the past year.
He had actual stories to tell now. He was kind and courteous, and he'd
left the naiveté behind him. Most of all, he treated her like an equal,
despite his infatuation. There had never been a moment in their time
working together when she felt like he was dismissive of what she was
saying, which was more than she could say for any other man that she
worked with, except perhaps for Perry. Lois could practically feel the
part of her that wanted to believe that she'd been wrong about him being
Superman. If it was all just a bad case of paranoia, and Superman was a
separate person that just looked like Clark despite all the other
evidence, it wasn't like she and Clark would live some idyllic life of
marital bliss, but at least she could see how she would find him
compelling, and possibly even attractive.

But no, Clark was an unrepentant liar. She wasn't sure whether he was an
alien that had forged a human identity for himself or a farm boy who had
developed astonishing powers, but it didn't really matter much either
way. He was cold and callous, and sat by while bad things happened in
favor of reporting on the news in the least efficient possible way. Lois
wasn't terribly religious, and much of it had to do with a conversation
she'd had with a priest when she was eleven years old about why God let
bad things happen. Most of the same arguments applied to Superman, even
if he wasn't perfectly omniscient and omnipotent. When seen through the
new lens of Clark Kent, it was possible to imagine that he'd never cared
about doing the most good at all. Being a symbol for the people
coincided with getting the highest amount of public acknowledgement, and
that seemed a little too convenient. It was easy to look at Clark as
Superman and think that it had all been about his ego all along.

After dinner they sat down on his couch together and listened to the
radio. After debating it for a few minutes, Lois yawned and then curled
up against him. It was a momentary shock to remember that he had the
same hard, defined muscles that Superman did, but she tried her best to
play the oblivious girlfriend that Clark wanted. The show that Clark had
picked out was a fanciful bit of science fiction about a man meeting
aliens on the surface of Mars, which didn't really hold her interest.
Lois slowly fell asleep against the Man of Steel.

The radio show ended and the commercials started up, which was when Lois
began to wake up. Clark leaned forward and shut it off. He turned
towards her, cupped her chin in his hand, and kissed her. His lips were
soft, and if it weren't for the thought that his hand could crush her
jawbone in a heartbeat, she might have actually enjoyed herself. He
wasn't awkward and fumbling like she had thought he would be, just calm
and tender.

When Clark backed away, he looked sad. ``How long have you known?'' he
asked.

Lois swallowed. She was still sleepy, but she knew this wasn't good.
``What are you talking about?''

``I kept waiting for you to slap me across the face,'' said Clark.
``From practically the moment I put on the suit, I was waiting for you
to figure everything out and\ldots{} I don't know what I thought that
your reaction would be. I guess I thought you'd be angry with me, but
I'd hoped that you would help to keep my secret.'' He sighed. ``Lois,
how long have you known?''

She wanted to deny it, but it was clear that wouldn't do any good. A
surge of fear was working its way through her brain, clearing up her
thoughts. ``Since just after you retired Superman,'' said Lois.

``Ah,'' replied Clark. He took off his glasses and set them on an end
table, then laid back against the couch. Some of the Clark Kent
posturing faded away. ``And that's why we're dating now.'' It wasn't a
question. Lois kept herself very still. ``I feel like I've made a mess
of everything.''

``You haven't ---''

``Stop,'' said Clark, and so she did. ``I love you Lois. One of the
things that I always loved about you, right from the start, is that you
never held back. You said the things that other people kept to
themselves. In Smallville people talk in circles and hide barbs in their
words. My mother ---'' His voice caught. ``My mother always disliked it.
You'd ask to borrow a cup of sugar, and they'd happily give it to you,
and then afterward they'd complain about the inconvenience. It was worse
for me, since I could hear all of the words said in private. But you
were never like that. You talked to artists, urchins, and politicians
all the same. There was an honesty to you, I guess. And then Superman
showed up, and you were different. It took me so long to see. Here was
someone that you were actually scared of, someone that you had to watch
your words around. You lied to him --- to me. Even your affection was a
lie, because you were scared. So please, no more lies. We need to have
it out, one way or another. If you hate me, I need to know.''

Lois watched him carefully. She took a few moments to consider. Clark
already knew that she had been lying to him, and nothing short of the
truth --- or at least \emph{a} truth --- would convince him. ``Do you
really want that?''

Clark nodded.

``You're squandering your power,'' said Lois. "You're invincible, and
people are \emph{dying}, and you're just\ldots{} sitting here. If I had
your powers, I wouldn't stop for a single instant. Lying to everyone
around you is one thing, and killing a man in cold blood was another,
but what I can't stand is that you're so indifferent to the suffering of
the world." Perhaps it was more than he wanted to hear, but he had asked
for the truth, and she hoped that he could hear it in her voice.

``You don't see the hypocrisy there?'' asked Clark. He was perfectly
calm, and it was hard to see whether that was another mask. ``People say
that all the time. They claim that if they had infinite power they would
protect the weak and heal the sick. And then they eat out at fancy
restaurants and buy expensive cigars. It's easy to say that someone else
should do something, but it's hard to do it yourself. I've been in your
apartment. I've seen how many things you could do without, if you were
really serious about doing the most good to the detriment of your own
personal satisfaction.''

``I work twelve hour days,'' said Lois. ``I work for and head up social
programs in my free time.'' She could feel her face flush. ``When I
waste an hour on something small and petty, the cost isn't measured in
terms of lives.''

Clark didn't seem the least bit hurt by this. ``The rich have a duty to
the poor. But they also have a right to do as they please with their
money, don't they? Lex Luthor engages in philanthropy, but you don't
begrudge him his mansion, or the excessive amounts of money he's spent
on lead shielding, among other things. I'm not talking about what should
be legally required of us, and I don't think you are either. I have a
moral obligation to the people of the world, as do you, but that
obligation isn't all‐encompassing. I'm not a slave.''

Lois frowned. ``I didn't say you were a slave.''

``You just think you're better than me?'' asked Clark.

``Clark, you lied to me, over and over. But even before that, you were
so powerful and so strange. You crushed rocks into dust in your hands
and you thought I would be impressed, and it seemed so hopelessly naive
to me.'' She spoke slowly, trying to find the right words. "You lifted
me up into the air like it was nothing, and flew me out a half mile
above the city like it was second nature for my life to be in your
hands. What you can do is objectively terrifying, and anyone who doesn't
see that is just engaging in wishful thinking. I'm sorry that I tried to
pretend at being the woman you wanted me to be, the one who you could
settle into a life with, but Clark, it wasn't all an act. If things had
been different --- hell, things \emph{are} different now, if we can be
open and honest with each other, and tear down the lies\ldots{} I'm not
promising anything, you understand, but I think we'd both like to start
over." There. Just the right notes of contrition, and something that was
close enough to the truth that it could pass the sniff test.

``Starting over,'' said Clark. He looked out the window at the city.
``Alright then.'' He held out his hand. ``My name is Clark Kent. I
masquerade as Superman. I can bend steel with my bare hands and move so
fast that bullets look like they're frozen in the air, among other
things.''

She shook his hand. Relief flowed through her; she'd been worried that
his outward calm was only for show. ``Lois Lane,'' she replied.
``Professional snoop. You're really from Smallville then? That wasn't
all made up when you came to Earth?''

``I was raised in Kansas,'' said Clark. ``Everything I've ever told you
about my childhood is true, but I left out all the interesting bits. My
parents found a spaceship in their field one day, and they took it as a
sign from god. I was just an unremarkable baby back then. They adopted
me without much discussion, and hid the spaceship beneath a tarp until
my father could hook a tractor up to it and stick it in the storm
cellar. I was raised like any other boy, until I started to get my
powers.'' He paused. ``How much of this do you want to know?''

``All of it,'' replied Lois. There wasn't much reason to believe it was
anything but another deception beyond her gut feeling, but he was
painting a picture for her, and either way he seemed to want to share.

``The hearing came first,'' said Clark. ``I was six years old, and I
thought I was going crazy. You can imagine my relief when I realized
that I was just hearing conversations from the next county over. It got
more powerful as the months went on, and I learned to shut it down, so
that I didn't have to listen to everything that people said or did. I
didn't tell my parents, but I thought that the hearing was what made me
special --- what God had put me on the earth for. And then I got the
vision when I was eight. I could see straight through things. I could
count the feathers on a hawk from ten miles away. That was when I looked
inside the cellar that my parents had kept shut and saw the spaceship.''

``The spaceship that didn't burn up over the Atlantic,'' said Lois.

``It was one of the lies I told you,'' said Clark. ``Sorry.''

``Wait, this doesn't make sense,'' said Lois. ``You said that you were
baby when the spaceship came down. But the story you told me was that
you learned English from our radio waves on the way over. Was everything
about Krypton a lie then? Because if you didn't know you were an alien
until you were eight years old, I don't see how you would know anything
about the planet you came from.''

``I'm getting to that,'' said Clark. ``And I know that you're skeptical,
but you're going to have to bear with me. I asked my parents about the
spaceship, and showed them what I could see and hear. They told me
everything, and we went down into the cellar. Almost as soon as I
touched the spaceship it grabbed a hold of my mind and showed me a
vision of Krypton as it had been. The ghost of my real father was there,
and he told me about the planet as it had been.''

Lois stared at him. ``A ghost,'' she said flatly.

``Not really a ghost,'' said Clark. ``A simulacrum. A shard of my
father's personality. Krypton was a sprawling place of crystalline
spires and flying cars, and my father sat me down to explain everything
to me. He told me how my powers would grow, and tried to instruct me on
how to help avoid the fate of his planet.''

``And he said all of this in English?'' asked Lois. If Clark wanted her
as she truly was, that was what he was going to get. Skepticism as
practically second nature to her.

``I only thought to ask that later, when I was a teenager,'' said Clark.
``I'd read enough history books by that point to see that Jor‐El was
wearing a modified toga. All of the buildings and plants were inspired
by Greece, mixed with a few more artistic flourishes, but it seemed too
much like what I knew of Earth. I asked him about that, and he told me
that what I was being shown was just a representation that would make
sense to me. The real Krypton was a dark planet covered in black water,
and the real Kryptonians were something like a cross between a spider
and an eel. Before the ship landed, it mapped out human civilization and
drew in samples of humanity to examine. I'm not really a Kryptonian, I'm
something that the ship built. I actually think I was born on American
soil. Jor‐El showed me an analog of their world that I could understand,
but I think they were even further beyond us than I could imagine.''

The conversation continued on, and Clark talked about the defining
moments of his childhood. Lois listened closely, and made mental notes
for later, occasionally sharing her own anecdotes that kept him in rapt
attention even though they didn't involve godly powers and alien ghosts.
The important thing was that Clark was being honest with her now, and
his secrets were spilling out into the open. She had told him off, and
he'd called her a hypocrite, but somehow that didn't mean they couldn't
still be friends. She debated telling him about her arrangement with
Lex, but decided that was one secret to keep to herself. Nothing had
ever really come of that partnership anyway. And besides that, all the
talk with Clark hadn't really changed that much about how she felt. He
was more human to her now, but still as negligent as he'd ever been, for
all his protests. Some of the fear had left her, but not enough that she
was about to let Clark know she'd actively tried to work against him.

\begin{center}\rule{0.5\linewidth}{\linethickness}\end{center}

The first attempt involved the drinking water at the Daily Planet
Building. The kryptonite was ground into a fine powder and put into both
the water cooler and the water main connecting to the building. Lex had
run tests on it before using it on people, more because he was worried
about overplaying his hand than because he was concerned about what
effect it would have on the people. A week passed with no indication
that there had been any change, though his channels of information from
within the building were rather incomplete, especially since Lois had
cut back her visits to practically nothing. She hadn't told him
Superman's secret identity, despite his best efforts to pry it out of
her. He was working on a way to have plausibly ferreted it out without
exposing himself, but that was doubly difficult now that Superman was no
longer active.

The second attempt involved aerosolizing the kryptonite powder. Lex
thought it unlikely to work, given that the concentration would be
measured in parts per million. The kryptonite seemed to lose the
signature glow when reduced to pieces smaller than a gram, and Lex
suspected that the still‐mysterious source of the radiation required
sufficient mass in close proximity in order to continue emitting its
waves or particles. At any rate, this too seemed to have no noticeable
effect on Clark Kent or anyone else in the building.

The third attempt involved exposure to the kryptonite. A small,
thumb‐sized piece was given to a man who had only the simplest of
instructions: to walk past Clark Kent. Two spotters were put into
position to watch. Their report was typed up and broadcast in code,
which eventually made its way back to Lex. He hadn't been able to give
them full instructions for fear that they would discover too much, but
they hadn't noticed any real change in Clark's behavior, not even when
the patsy came within arm's length of him.

Brief exposure likely wasn't going to do the trick, especially not at a
distance. The fullest test of the kryptonite would be to place it
directly next to Superman for as long as possible. The spaceship's
creche had a large piece of kryptonite directly next to it, and a
relatively thin layer of lead was apparently sufficient shielding, which
said quite a bit about the danger that it posed. The kryptonite would
have to be close, nearly in range of skin contact. That meant using Lois
Lane. Unfortunately, Lois could lead Superman right back to Lex, but
that was what contingency plans were for.

\begin{center}\rule{0.5\linewidth}{\linethickness}\end{center}

Superman waited, and watched.

\begin{center}\rule{0.5\linewidth}{\linethickness}\end{center}

\emph{Author's Note: Some post‐publication edits have been made --- if
some of the reviews don't make sense anymore, that's why.}
