\hypertarget{like-clockwork}{%
\chapter{Like Clockwork}\label{like-clockwork}}

Harry Kramer loved explosives. He loved the danger of working with them
and the thrill of watching them go off. A properly made bomb was an
amazing piece of engineering, a compact device of wires, springs, and
explosives all set up in a very precisely and ordered way. When the bomb
went off, all that hard work evaporated in a single transformative
moment. It was like taking a piece of fine crystal and hurling it
against the side of a brick wall, and how could someone not feel joy at
that? How could someone not see that there was something magical that
only existed in that single solitary moment when the product of labor
and a thoughtful mind became nothing more than garbage? Though there
wasn't anything sexual about it, the best word that Harry had found for
it was orgasmic.

A thick letter came in the mail for him. He ran a few simple tests to
see whether it might contain a bomb, sniffing at it and hefting it
carefully. Letter bombs were tricky to do, because you couldn't reliably
set them on a timer unless you knew for certain when they'd be opened.
The letter also had to make it through the postal service without
detonating or being discovered, which was a challenge in and of itself.
The most common way to make a letter bomb was to fill an envelope with
two chemicals that were explosive when mixed, separating them with
layers of paper. Another chemical trigger was placed along the top where
the paper was going to be ripped. The chemicals would mix when the
letter was opened, and the bomb would explode, but that was often messy
because people didn't always open their own mail. It was easier to make
a larger package that would explode, because then you didn't need to
worry about the bomb being bent or squeezed, but there was a very clear
distinction between a ``letter bomb'' and a ``mail bomb'' owing to the
restrictions on construction.

Harry had a recurring fantasy about being sent a letter bomb. In the
fantasy he would smell the metallic powders and carefully disarm the
bomb in his workshop, pulling it apart to expose its secrets. Written
inside the letter bomb would be words of congratulations for showing
caution, and a coy invitation to begin using his skills in earnest. In
the fantasy he and the other bomber would engage in a conversation
written across the city in explosive force, needing nothing more than
concussive blasts to speak to each other. There was something raw and
primal about destroying the ordered world of the city. Eventually Harry
would prove himself the superior of the two, and she would reveal
herself to him, and declare her undying love for him. It was always a
woman, of course. They would exchange hot, hungry kisses on the rooftop
of his apartment as Harry's bombs leveled the city.

The letter he'd received wasn't a bomb. Instead, there was an offer of
employment. Beneath that, the bulk of the envelope containing crisp
twenty‐dollar bills, enough to pay for his apartment for two full years.
The letter was concerning, because it meant that someone knew about him,
but it was exciting, because it meant that he was going to get to do
something that he loved. It wasn't some simple job that required only a
simple demolition or death, it was finally a chance to be unchained and
fully funded. No longer would he have to cobble something together from
bits and pieces. He was going to make something beautiful.

\begin{center}\rule{0.5\linewidth}{0.5pt}\end{center}

``What makes a person do a thing like this?'' asked Clark.

Lois rolled her eyes.

Clark was a heavy man, thick without really seeming muscular, though you
could tell from a glance that he'd never learned to buy clothes that
fit. He had terrible posture, his hair was messy, and he wore glasses so
thick you could hardly see his eyes through them. He seemed to get sick
constantly, and he was so out of shape that whenever they had to move
quickly he could be seen gasping for air afterwards. He had the desk
right next to Lois's, and so she'd had time to examine each and every
one of his faults --- that was just a small sampling of the physical
problems with Clark. Much to her consternation, he was somehow the
second best reporter at \emph{The Daily Planet}. They were often paired
together for the big stories, since it allowed Perry to run a companion
story to a front‐pager. More often than not, Lois found that being
around Clark tried her patience. It was made worse by the fact that he'd
quite obviously developed an infatuation with her from nearly the day
that he started working at the Planet. He'd asked her out during his
second week, and she'd politely but firmly told him no, but he was still
hung up on her. One of the only good things about Clark was that he was
as transparent as glass. His crush was more sad than annoying, most days
anyway.

Lois and Clark were standing outside the remains of an apartment
building. It had exploded earlier in the day at around noon, sending
bricks, wood, and personal belongings in every direction and shattering
a number of windows all around the block. Two people had died, and a lot
more had been seriously injured. The apartment was still standing, but
three of the upper floors were now just a gaping hole, and it was likely
that there was enough structural damage that the building was a total
loss. Everyone talked about how much worse it could have been. It was
front page material for sure.

``Some people are just evil,'' said Lois.

``I don't think a person is born a certain way,'' said Clark. ``People
make choices, for good or evil. Free will is part of God's design. I
just can't understand why someone would make this choice.''

Lois tried to stop herself from rolling her eyes again. ``Some design,''
she said, as she spotted a severed arm in the rubble that no one seemed
to have picked up yet.

Lois and Clark had done their interviews, talking to the victims,
police, firefighters, and neighbors. There was little question that the
explosion had been deliberate. The police were already chasing down some
promising leads, though Lois knew that half the time they only said that
to keep people reassured.

They'd been back at the Daily Planet Building working late when the
second bomb had gone off, exactly six hours after the first. This one
was at a sales office downtown. Most of the staff had gone home, but the
rescue workers had pulled a few corpses from the wreckage. She overheard
one of the onlookers say that it was a tragedy that people had died
because they'd stayed late to work. She made sure to put that in her
article.

The third bomb exploded in Superman's face. He'd found it in the freezer
of a grocery store, and got people out of the way before he'd tried
moving it, which was when it had blown up. Lots of people reported
seeing a gaping hole torn right in the center of his costume. Superman
had spoken directly with the chief of police, giving him as much
information as possible. Lois had come back into the office late at
night in order to write about it, and found that Clark was already there
in a wrinkled shirt, looking for all the world like he'd never stopped
working when she'd left at eight. Though he finished his article before
her, she came up with the better moniker --- the Clockwork Bomber. Perry
groused about them being too competitive and wasting effort writing the
same story, then decided to run Lois's article in the morning edition
with the headline ``Clockwork Bomber Strikes Midnight!''. The long hours
were worth it just for the forlorn look on Clark's face.

Lois set her alarm for five in the morning. The first bomb had been at
noon, the second at six, and the third Superman had detonated just
before midnight. The pattern was obvious to anyone with half a brain.
Ten minutes before six o'clock in the morning she heard a distant rumble
from across the city, and she was ready to trek off towards it in her
most sensible shoes. Clark was nowhere to be seen, and despite being
tired as hell, Lois felt a warm glow of satisfaction that she'd beat him
to the punch.

The mayor and the chief of police held a press conference, where they
promised that they would find the man or men responsible. No one made
any demands, and no one claimed credit. Everyone braced themselves for
another bomb at noon, but it didn't come. Four bombs had claimed the
lives of six people, and there didn't seem to be a point to it. The
casualties had been much lower than they could have been, given the time
of day that the bombs had gone off and the locations that they'd been
placed, but it was anyone's guess what that said about the bomber.

A few days passed, and eventually things began to settle down again.

Lois was surprised when she found a second letter on her desk, addressed
to Miss Lane and requesting to meet her in the same place as before. She
was ready this time, and grabbed a sheet of paper with a series of
questions from inside her desk. She stopped by Perry's desk to tell him
where she'd be going, just in case something happened. Perry looked
ecstatic, but Lois felt her nerves getting the best of her.

She prided herself on being utterly fearless. She'd stood on the spire
of the Emperor Building as the first airship came in, strapped in with
what amounted to a thick belt. She'd hunted big game with Hemingway over
a memorable summer in Kenya. She'd braved storms while sailing the North
Atlantic in a yacht, the closest she'd ever come to actually dying. She
found these adventures exhilarating instead of terrifying. Yet there was
something about Superman that tickled some animal part of her brain. She
did her best to ignore it, and made the trek up to the rooftop where the
Man of Steel was waiting.

\begin{center}\rule{0.5\linewidth}{0.5pt}\end{center}

``Hello Lois,'' he said as he turned around. His smile was gentle, but
it didn't help her nerves. Luthor had said that Superman moved faster
than muscles alone would dictate, but that didn't make the muscles look
any less impressive. It was impossible for her to look at him and not
think about the fact that he could cross the distance between them
faster than she could blink.

``Hello Superman,'' she replied. ``I've got some questions for you.''

``I know,'' he said.

Lois immediately imagined him staring through the walls, looking over
the questions she'd prepared for him and composing answers. It felt
utterly invasive --- she would never allow an interview subject to look
over the questions like that, not at this stage in her career. She
really should have gotten one of those lead‐lined drawers. Of course,
maybe he'd just meant that he knew she had questions because everyone
had questions. She found herself unwilling to give him the benefit of
the doubt.

``Go on,'' said Superman. ``But I can't answer everything.''

``Where is your ship?'' she asked.

``It burnt up over the Atlantic on my way in,'' Superman replied.

``Could you find the wreckage?'' she asked.

``There wouldn't be anything left,'' said Superman. ``Even if there
were, I wouldn't hand it over. If humanity were able to work backwards
and figure out how it was made, I fear the results would be disastrous.
It would be like giving a gun to a baby.''

Lois frowned. ``And you're the final arbiter of what's good for
humanity, what we can and can't handle?''

``I am the arbiter of myself,'' said Superman. ``I can only do what I
think is best, and hope that humanity gives the same consideration to
their own actions.''

``Okay,'' said Lois. ``But are you really doing the most good? I mean,
I've seen proposals for what other people would be doing with your
powers, digging canals or generating power, searching out veins of ore,
the amount of money---''

``I don't need money,'' said Superman. He interrupted her so delicately
that she momentarily lost her train of thought.

``You don't,'' she replied slowly. ``But the rest of us do. These are
lucrative jobs that could bring in millions, and with that you could
fund orphanages, women's shelters, homeless shelters, or whatever
charitable organizations you wanted. We could set up a trust. It
wouldn't matter that you were using your powers for a profit, because
that profit would be directly translated into good works that would
overwhelm positive effects of the crime fighting and general heroism you
do now.'' Lex Luthor's words were coming out of her mouth. ``And if you
embraced the celebrity that you already have you could charge enormous
amounts of money for the use of your image. People are already making
lunchboxes and trading cards with your emblem, and I've heard that
they're making two different movies about you. These things are going to
happen whether you're involved with them or not, and you could at least
make some money that you could use for good causes.''

``Saving people from violent crime is an unambiguous good,'' said
Superman. ``Bringing money into it isn't, and I don't know that I should
be supplying humanity with a brawn that it doesn't and shouldn't have
yet. I've tried my best to confine myself to acting only when there is a
clear good to be done. I'm trying not to bend the course of human
history, or force my morality on anyone else. I do that by operating
within the laws of the country and avoiding controversy as much as
possible. I have as few points of interference with a citizen's daily
life as possible.''

``You think that an avoidance of controversy is part of the greater
good?'' asked Lois. ``Do you think that the laws of this country are
anywhere close to just?'' She pointed across the city to the docks, and
the channel where ships were streaming in and out of the harbor. ``A
hundred years ago there were slaves being sold here. If you'd shown up
then would you have stopped slavemasters from beating their slaves? Do
the laws of men mean that much to you that you'd actually let such an
injustice stand?''

``You're losing your cool,'' said Superman.

Lois looked down at her notebook. She hadn't asked him a question from
it for quite some time. ``You're right,'' she said. ``I'm sorry. It's
just that sometimes I think about what I would do if I had your powers,
and in comparison you seem so\ldots{}''

``Reluctant?'' asked Superman.

``Yes,'' Lois replied.

``During Prohibition, as part of an effort to stop people from drinking
industrial alcohol, it was denatured and methyl alcohol was added,
making it toxic. They thought that people would change their behavior.
The end result was that the United States government killed ten thousand
of its own citizens.''

``I wrote an article about that,'' said Lois. ``It never made it to
print.''

``I know,'' said Superman. He looked out towards the city in quiet
contemplation. ``I believe that the people who poured their poisons into
the vats truly believed that they were doing good. They just couldn't
see what the end result would be. Even with the work I've been doing,
there have been unwanted side effects.'' He pursed his lips. ``I get the
distinct impression that people are less cautious with their lives now
that they have me around. People shout for me to save them instead of
taking action. There was a fire in an apartment building three days ago,
and half the occupants ran up to the roof and screamed for me to come
pick them up. If I'd been dealing with some other more serious disaster
at the time, those people would have died. These are the things that
happen on even a small scale when humanity is saved from their own
mistakes and steered away from forging their own path. I'm sure you
could think of half a dozen other examples of the unintended
consequences.''

She could. The budgets for the police and fire department in Metropolis
were up for review, and both looked like they were going to be cut by a
large percent, because the city saw no point in paying the same amount
for services when Superman had taken up much of their duties. Those
elements of the underworld with sufficient mobility were moving to
Gotham City, causing a crime wave there the likes of which hadn't been
seen in a decade. The ones that stayed in Metropolis were more organized
than before, with a higher propensity towards subterfuge, trickery, and
crimes which didn't make a sound. Superman didn't speak anything but
English, and so there had been an explosion in language learning. That
was above and beyond the general insanity that came from having a man
that flew through the air, and the world's first extraterrestrial.

There were many things that Lois wanted to say, but she was worried
she'd get too wrapped up in argument again. A good reporter pressed
their subject, but didn't get heated. If she were speaking to him
outside of her role as a professional, she might have called a policy of
non‐intervention the definition of moral laziness. She might have told
him that he had the most inconsistent moral system she'd ever had the
displeasure of encountering. The truth was, she didn't like Superman.
They'd both read the various proposals and the pleas for aid. There were
so many things that he could do, and he simply refused to do them. It
might have been one thing if he'd engaged in reasoned debate, but
Superman had acted unilaterally, thinking that he knew what was best for
humanity. Her thoughts returned again to when he'd scooped her up like a
child. Superman was a man --- an alien --- of presumptions.

But Lois Lane was a good reporter, and so resisted the urge to berate
him.

``How long were you on the planet before you began intervening?'' asked
Lois.

``Two weeks,'' said Superman. ``I learned English on the way over from
your radio signals and spent a good deal of time watching from above and
getting a more in‐depth understanding of your culture and the ways of
your people, as well as the relevant laws.''

``And did you anticipate what followed?'' asked Lois.

``For the most part,'' said Superman. ``Celebrity, shock, awe, analysis
--- that was predictable. What I hadn't counted on was the cruelty or
organization of the attempts to kill me.''

Lois furrowed her brow. ``You're talking about the people trying to
shoot you?''

``No,'' replied Superman. ``That I expected. The criminal element was
bound to try. I let them sometimes, just to prove how useless it is to
stand against me, but most of them attack me like it's going to do some
good. I stopped a mugging three weeks ago, and the man kept stabbing my
eyes. It didn't do anything more than dull his knife, and eventually he
ran out of steam. Sometimes they shoot me and look at their guns like
they're shocked that it didn't work. Maybe some people don't really
believe the stories until they see it for themselves. No, all that I
expected. I'm talking about the bombs. That's why I came to speak with
you today.''

``The Clockwork Bomber,'' said Lois.

``Yes,'' said Superman. ``All the bombs were meant for me. They were
encased in lead and had mechanisms inside to prevent me from doing
anything with them. I think someone was making an effort to kill me.''

``It seems obvious that wouldn't work,'' said Lois. ``Even on the face
of it.''

``The bombs were special,'' said Superman. ``They used focused blasts
and a variety of different materials. I think one was an attempt to
blind me. They're probing for a weakness.''

``But it didn't work,'' said Lois.

``No,'' said Superman. ``I've been looking over the city and trying to
connect the dots. Whoever set the bombs up is going to try again. I need
you to warn the people of Metropolis. If I'm right, next time it's going
to be worse.''

\begin{center}\rule{0.5\linewidth}{0.5pt}\end{center}

Ninety‐nine percent of the time, ripping a handful of wires out of a
bomb will safely defuse it, either by removing the fuse from the
detonator or the detonator from the explosive material. Most people who
made bombs were unsophisticated, and most bombs were designed not to be
found until after they had detonated. There wasn't much point in making
them particularly hard to defuse or move, and there weren't many people
with the technical skill to do it.

The bombs that Harry designed were complex, above and beyond the
complexity designed into them by his benefactor. They had to be, because
their target was Superman.

Many things could be made fail‐safe. The railways used air brakes, in
which a piston was held up by compressed air. To apply the brake, some
air was let out of the system, causing the piston to lower and the brake
to be applied. If any of the components of the system failed, the brake
would be engaged by the loss of pressure, stopping the railcar and
preventing it from going out of control. Fail‐safe design was becoming
more and more important as a method of stopping machines from
self‐destruction.

The bombs Harry made were fail‐deadly. The detonator was connected to a
timer, but the timer didn't cause the bomb to explode --- it prevented
the explosion from happening. Removal of the timer would collapse a
circuit and cause the bomb to explode. Removal of the detonator would
cause a circuit to collapse and trigger a secondary hidden detonator.
Several small glass tubes were filled with beads of mercury which were
part of the circuit, and if the bomb was tilted too far in any direction
a circuit would complete and cause the bomb to explode. No one would
ever be able to see this hard work, not even Superman, because the whole
thing was encased in lead shielding. Wires were affixed to the interior
of the casing, and if the lead shielding was removed the bomb would
detonate.

Most bomb makers didn't make their bombs this complex. It was more work,
and with the work came a higher risk of accidental detonation. With the
amount of explosives that Harry was using, it wasn't really a concern
for him. What he feared was a small explosion that left him limbless and
bleeding out, but given the number of pounds of cyclonite he was working
with, an accident would leave him vaporized. It didn't seem like such a
bad way to go. In truth, Harry liked the heightened sense of reality
that came from being one mistake away from utter destruction. The
benefactor had designed the bombs to be dangerous things, and Harry had
modified them to be nearly reckless.

``Be careful with that,'' said Harry as the workmen took the first bomb
out of the workshop that had been rented for him. ``It's fragile.''

They hadn't smiled at his joke, but then they didn't know what was in
the crate they were carrying out. The circuit with the mercury switches
was on a separate timer to ensure that the bomb wouldn't blow up in
transit, but there was still more risk than most people would want to
take. Harry had no idea where the workmen had come from. Like many
things, the benefactor took care of it.

He also had no idea where the bomb was headed, but he couldn't help
smiling as his bomb ventured off into the world. He'd headed back into
his shop to make some variations on the theme.

\begin{center}\rule{0.5\linewidth}{0.5pt}\end{center}

Lex had tried doing things cleanly. The Conference on Extraterrestrial
Science had put out a plea to Superman, asking him to attend a meeting
of minds so that they might make a cultural bridge between human and
Kryptonian science. Superman could have come forward and simply spoken
to them about what the true limits of his powers were, but he hadn't
even responded to them. The invitation carried nearly every important
name among the scientific elite, and the lack of response couldn't be
seen as anything but an insult. Lex had put forward a mountain of plans
and proposals that would allow him to get close to Superman, and almost
all of them would allow for an advancement in what most people would
consider to be the common good. Superman hadn't responded to any of it.

The bombing campaign served multiple goals, as any good plan did.

Superman was an extinction level event waiting to happen, and where
those were concerned there were no second chances. If Superman ever
decided to kill everyone, there would be no stopping him, and so it
stood to reason that humanity should take every possible precaution to
prevent that from happening. The most direct path would be through
killing Superman. Lex had written multiple letters to the editor under
various pseudonyms, but none had ever been published, and his point of
view seemed entirely unpopular. It was always one that he voiced from a
position of anonymity, because in public he was playing the role of
Superman's champion.

People were bad at estimating the risk that an extinction posed, because
no one had ever lived through one. People were also quite bad at
imagining a catastrophe so large. A woman might weep when you mentioned
the possibility of her child dying from consumption, but the total
obliteration of Earth‐originating life would produce only a shrug. It
was too vast for people to think about rationally. Worse, they assumed
that ``Superman is the greatest threat to humanity'' was a shorthand for
some decision on Superman's part, when in truth that was only a part of
it.

Many people accepted Superman's story at face value; the last son of a
dying planet, the only one of his kind to exhibit such incredible
powers, with little aid from technology save for the ship that had
provided him with a trip through the stars. There were many parts of the
story that Lex was skeptical of, but he found it most terrifying to
think that the story was true, namely because of what it suggested about
Kryptonian science.

Huntington's disease was a hereditary degenerative disease with
cognitive and psychiatric symptoms, one of which was psychosis.
Huntington's was seen in perhaps one in eight thousand people, and
psychosis was seen in perhaps one in ten of those. If a randomly
selected human of Superman's apparent age were to obtain Superman's
powers, there would be a one in eighty thousand chance that they would
both have Huntington's disease and symptoms of psychosis, the result of
which would probably be casualties that would dwarf the Great War by a
large margin. If Superman was telling the truth about the culture that
he came from, his society wasn't much further advanced than humanity,
and so likely hadn't grown past degenerative diseases and hereditary
defects. Even if Superman were perfectly good in some abstract sense,
the onset of a mental disease might be just around the corner.

Worse, if Superman's powers weren't the result of engineering and
carefully controlled science (a hard pill to swallow) then no one had
made sure that they were safe, and perhaps some day something internal
to him would simply unravel, unleashing enough energy to destroy an
entire hemisphere. If Superman was to be believed, his powers had come
from seemingly nowhere, and yet everyone simply trusted them as though
it were the most natural thing in the world.

Estimates were difficult to make, given Superman's silence. His second
interview with Lois Lane had provided little illumination. Nevertheless,
numbers could be pulled from thin air in order to get a sense of things.
There was the possibility that something would happen that was
completely outside of Superman's control which would result in Superman
destroying the Earth. There was the possibility that Superman could
simply have a bad day and decide to kill a large number of people, which
many people seemed to think was absurd. There were also failure modes
which didn't involve the destruction of humanity but would nevertheless
result in an effective end to humanity as Lex Luthor knew it, the most
probable of which seemed to be that Superman would turn into a tyrant.
When these probabilities were multiplied together, the final very rough
estimate was that Superman had a one in ten chance of bringing about a
global scale human catastrophe of some kind in the next thirty years.
Even if the odds had been one in a hundred, Lex would have taken a
similarly extreme course of action.

The collateral damage caused by the bombs was negligible in comparison
to the threat of Superman.

But of course the bombs were unlikely to kill Superman. The first four
had been for calibration, built with a small device which gave a series
of loud chirps prior to detonation to allow Superman time to get to it
before it exploded. The next series of bombs would introduce more exotic
methods of harm which hadn't yet been conclusively ruled out, but the
prospects looked grim.

The secondary goal was to probe for a weakness. Lex had it on good
authority that Superman had taken the equivalent of a direct hit from
navy artillery to his chest when the third bomb exploded. He'd simply
looked surprised that he'd set it off. The magnesium and phosphorus
compounds had done nothing to blind him, and he'd been talking with the
police soon afterwards with no ill effects. Lex had suspected as much,
but perhaps something would be found that could harm him but not kill
him, or otherwise give Lex an advantage. Lead had been a boon, and
allowed Lex a level of freedom that was gratifying until he remembered
how free he'd been before Superman's arrival.

The third objective was testing Superman's limits. Lex kept a detailed
log of Superman's movements in his study, as well as a large map of
Metropolis which was covered in small color‐coded labels that
corresponded to Superman sightings or activities. Superman's patterns
had been mapped against the general patterns of crime and emergency in
Metropolis, and Lex had not been all that surprised to find that the
patterns didn't quite match one another, even taking into account
Superman's preferences for certain crimes and emergencies over others.
There were two lulls, one during the daytime that seemed to start around
eight in the morning and end around five in the afternoon, and one in
the dead of night from three in the morning to five in the morning. Lex
had no idea what to make of it, but kept the information safely locked
away behind lead walls. Perhaps Superman needed to sleep, or needed to
recharge in some other way, but sustained and consistent bombings would
allow for information to be gathered.

The fourth objective was to identify the place that Superman retired to
when he wasn't flying around the city, since Superman demonstrably
didn't spend all of his time on heroics. Lex strongly suspected that the
ship hadn't broken up over the Atlantic, and was in fact located
somewhere in or near Metropolis. Depending on the size it would be
difficult to hide, but Superman could surely lift the craft up and move
it at will, which meant that it could be nearly anywhere. All that was
under the assumption that Superman was an alien --- there was still an
outside possibility that there was some other explanation. If the
spaceship existed, finding it was of utmost importance. Lex had already
hired a team of private investigators to see if they could find some
trace of a ruined ship in the Atlantic, though without eyewitness
accounts of where the spaceship had burned up it would be impossible.
With them it would merely be very, very difficult. Still, it was worth
trying.

The next wave of bombs would be planted in two weeks time. Perhaps Lex
would get lucky and Superman would prove to have a weakness.

\begin{center}\rule{0.5\linewidth}{0.5pt}\end{center}

\emph{Author's Note: This chapter was getting too long, so I split it in
half. The next half will come at the regularly scheduled time on Sunday
night.}

\emph{Any numbers that Lex or anyone else gives is their own best guess
based on what might have been knowable in the 1930s before the age of
the internet. I don't guarantee that these are at all period accurate,
and obviously we're dealing with an alternate universe where a city
named Metropolis exists.}

\emph{A note on geography: I'm writing on the assumption that Metropolis
replaces New York City and Gotham City replaces Chicago, though with
different city layouts and some changes to small scale geography of the
region.}

\emph{Two historical notes: First, the American eugenics movement was
still alive and well at this time, so if you see references to it pop up
here and there, just remember that this was an opinion you could voice
without anyone really raising an eyebrow. Second, the United States
really did denature alcohol, which wound up killing more people than
Superman probably saves in a decade. The more you know!}

\emph{As always, I appreciate any corrections, comments, or general
feedback, and thanks for reading.}
