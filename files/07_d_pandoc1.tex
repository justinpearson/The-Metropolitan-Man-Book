\hypertarget{choices}{%
\chapter{Choices}\label{choices}}

From \emph{Preethi v New York} 293 U.S. 367 (1934):

The State of New York has provided such significant encouragement, both
overt and covert, that the actions of Superman must be judged to be that
of the State. {[}\ldots{]} It is this Court's considered opinion that
there would not be much use to Constitutional protections if the State
could do an end run around those protections through the use of private
parties. By engaging in the same type of work as the Metropolis police
department, and with their cooperation and approval, Superman may fairly
be described as a state actor.

\begin{center}\rule{0.5\linewidth}{0.5pt}\end{center}

From \emph{Shoe v New York} 293 U.S. 377 (1934):

Obtaining by enhanced senses any information regarding the interior of
the home that could not otherwise have been obtained without physical
intrusion into a constitutionally protected area constitutes a search.
{[}\ldots{]} In permitting the use of this evidence upon trial, we
believe prejudicial error was committed.

\begin{center}\rule{0.5\linewidth}{0.5pt}\end{center}

From \emph{The Daily Planet}, anonymous letter to the editor, December
19th, 1934:

Taken together, there can be no question that these rulings severely
curtail Superman's ability to effectively conduct law enforcement within
the United States. In the coming months, dozens if not hundreds of
appeals will be filed on the premise that Superman has engaged in
procedural error, in which the Metropolis police department and others
were complicit. The Fourth Amendment of the Constitution has been
incorporated against the states, which many see as a worrying expansion
of federal power. Yet while people argue over what the right legal
structure for dealing with Superman is, what they seem to miss is that
Superman only obeys the laws because he chooses to. He has already
graciously said that he will abide by these rulings, yet one has to
wonder what the Man of Steel actually thinks of them. All too often, we
forget the enormity of his powers and treat him like a constant, but
what man can exist without change?

\begin{center}\rule{0.5\linewidth}{0.5pt}\end{center}

Lois and Clark stood outside the Metropolis Courthouse with the other
reporters, waiting for the verdict. Calhoun's trial had been sped
through, and there was little doubt that Superman had used pressure of
some sort to make that happen. The early portion of the trial had been
marked by an enormous amount of evidence being thrown out, with the
judge citing the new Supreme Court rulings. A number of the charges had
been dropped after that, though it was still enough to put Calhoun away
for the rest of his life. Bail had eventually been set at one hundred
thousand dollars, which Calhoun had happily paid as though it were chump
change to him. Clark no longer smiled when the topic of the case came
up. He'd submitted an article to Perry about corruption in the case. It
alleged witness intimidation, jury tampering, and juror misconduct, but
his sources were shaky and couldn't be verified to Perry's satisfaction.

``Not guilty!'' came a shout from within the courthouse. The reporters
began to crowd around, to get a picture of Calhoun or shout a question
out to him as he walked out. Lois went with the pack, but Clark stayed
behind. He had a defeated look on his face, like he'd known that it was
coming but hoped he was wrong. Lois got her comment, and Clark wrote up
an article about how Superman was nearly useless in the face of
organized crime with the laws the way they were.

A week later, someone began setting fire to the homes of known or
suspected abortionists. Superman stopped them, which caused a
significant controversy. So far as Lois could tell, that was the whole
point.

``Why do you think Superman doesn't stop abortions from happening?''
asked Lois. It was a question that many of her fellow Catholics had
asked for a long time. She'd been practically mobbed by the other
churchgoers when she'd gone to Christmas Mass, since people seemed to
think that she and Superman were as close as two peas in a pod. In their
defense, \emph{The Daily Planet} hadn't been quick to correct that view.

``He used the term unambiguous good, didn't he?'' asked Clark. Lois had
predicted that Metropolis would eventually break him, but she hadn't
thought it would be such a long, slow decline.

``Well that's the whole idea,'' said Lois. ``If Superman isn't stopping
the abortions, then that means he doesn't seem to think stopping them is
an unambiguous good.''

``He wants to avoid the controversy,'' said Clark. It was clear that his
heart wasn't in the conversation.

``Avoiding controversy outweighs unambiguous goods?'' asked Lois.

``I don't know,'' said Clark. ``The world is complicated. I'd really
rather not talk about this.''

Early on, Clark had been eager to engage her. He'd liked having her
attention of course, but he'd also been more sure about himself then,
more convinced that he could get her to come around to his way of
thinking. It wasn't just that she'd worn him down though, everything
about him had started to become so\ldots{} mechanical. It hadn't
affected his work, and if anything he had been increasing his output.
But the spark that was Clark Kent was dimming, and Lois wondered if
there was anything she could do about that. She and Clark were more
colleagues than friends, but she spent more of her time with him than
anyone else, at least when they weren't out in the city chasing down
stories.

``Do you want to go see the mural after work?'' asked Lois.

``Sullivan already covered that,'' said Clark without looking up from
his typewriter.

``I said after work. I meant more as something to look at,'' said Lois.
``For entertainment. Which I think is the point of it.'' Clark looked at
her. ``Not a date or anything like that, just friends. And maybe
afterwards we'll get a bite to eat somewhere?''

A slow, cautious smile crept onto Clark's face. ``Sure, I'd like that.''

When the mural was finished it would stretch for three city blocks, but
so far only two blocks of it had been completed. It was a mosaic made up
of small tiles, each about the size of a fingertip, visible as a
coherent image only from a few steps back. They started walking it from
the end that was supposed to represent the past, when the island that
Metropolis was built on was home to the Lenape Indians.

``It's white‐washed,'' said Clark. ``But I don't suppose anyone expected
anything else. None of the subjugation or slavery that marks the actual
history of the city. There should be men in collars somewhere
around\ldots{} there.''

``Clark, I know you're still a bit raw about Calhoun getting off,'' said
Lois. ``But you've got to snap out of it eventually.''

``It's not just him,'' said Clark. ``It's all the rest that are just
like him. Do you know how many guilty men go free?''

``Better for ten guilty men to go free than one innocent man rot in
jail,'' said Lois.

``Why that number?'' asked Clark. ``Why ten and not five?''

``It's not meant to be literal,'' said Lois.

``I'm just curious,'' said Clark. ``It's in the Bible, did you know
that? Genesis 18:23, `And Abraham drew near, and said, Wilt thou also
destroy the righteous with the wicked?' The numbers were different
though. God said that if he could find ten innocent men in the whole of
Sodom and Gomorrah he would refrain from raining down brimstone and
fire.''

``That's kind of gruesome,'' said Lois. They walked past a colonial
scene of men planting crops and raising cattle. It was unimaginable that
land in Metropolis had once been cheap enough that you could farm it.

``In the end, God destroyed Sodom and Gomorrah,'' said Clark. ``Because
it was a place of evil. But he saved the only innocents in it first,
because God is perfect, and that was within his power.''

``Unfortunately,'' said Lois. ``The justice system is run by men.
There's a distinct lack of perfection. Are you just figuring this out
now?''

``No,'' said Clark. ``Believe me, I know how imperfect people can be.''
He bit at his lip. ``I don't know, maybe I just never studied history as
closely as I should have. It's easy to forget that slavery ever
happened, you know? And there are crimes against humanity that are just
swept under the rug, forgotten by everyone, though you could still find
the mass graves if you looked hard enough.''

``Jesus Clark,'' said Lois. ``You really know how to show a girl a good
time.''

Clark was silent after that, but she could tell he was still thinking
along the same lines as before and just not saying anything out loud.
She wished that the final part of the mural had been finished, so that
they could talk about something more pleasant. She'd heard that it was
going to be like something out of science fiction, with spaceships going
to the moon and robots serving people dinner. Lex Luthor was the man
behind the project, and he'd proven himself an optimist. It was somewhat
comforting that the future history of the world was going to be written
by men like him.

``Do you think that Superman should have just killed Calhoun?'' asked
Clark.

``No,'' said Lois. ``Can you imagine the panic that would have caused?''

``No one would have to know,'' said Clark. ``Superman could just abduct
him and drop him in the middle of the Pacific to drown.''

``Superman wouldn't be that cruel,'' said Lois. ``Even the state tries
to keep their executions as clean and painless as possible. And that's
all a moot point. Superman doesn't kill, everyone knows that. Your
average criminal would rather be arrested by Superman than the cops,
because Superman is gentle.''

``You're right,'' sighed Clark. ``They take him for granted. The whole
trial with Calhoun proved that. No one feared what Superman would do
when the verdict came down. They didn't think it was suicidal to
challenge Superman's will. And they were right.''

It was time for a change of tactics. ``Clark, can you talk to me about
life in Smallville?'' asked Lois.

``You hate Smallville,'' said Clark.

``I was a military brat, and I grew up all over the country,'' said
Lois. ``I lived in a couple places like Smallville, and I was always
bored. But I think maybe I've been projecting my own experiences onto
what I've been imagining. So come on, I promise not to make fun. You
never talk about it anymore.''

Lois had been right. Smallville seemed to be just the trick. There was
no more talk about mass graves or killing the innocent along with the
guilty. Maybe it was just because she'd spent so long around Clark, but
where she'd rolled her eyes at his stories about small town Kansas
before, now she was almost interested. As he talked, he grew more
animated, until his mood had visibly improved. From there it wasn't that
difficult to keep him upbeat, and after a long talk about the
possibilities of the future in front of the unfinished section of the
mural, they'd gone out for dinner and then drinks, though Clark only had
soda water. Lois wasn't sure whether he wasn't so bad as she'd thought
he was, or whether she'd just been worn down by his constant presence.
Either way, Operation Cheer Up Clark had been a rousing success, and
when he came into work the next day he was nearly back to his old self.

Everything started to fall apart two weeks later when the governor's
children were kidnapped.

\begin{center}\rule{0.5\linewidth}{0.5pt}\end{center}

Lex Luthor was slow and careful.

He never said the name ``Clark Kent'' out loud. There were hundreds of
Clarks in Metropolis, and hundreds of Kents, but so far as Lex could
find, there was only a single Clark Kent. It wasn't inconceivable that
every time he heard his full name his super hearing kicked. Everyone
chattered about Superman all day, but surely very few people talked
about Clark Kent. He was a reporter, and his name appeared in nearly
every issue of \emph{The Daily Planet}, so perhaps there was some cover
there, but Lex wasn't about to risk it. He had Superman's secret, and it
was the most precious thing in the world.

Getting records was difficult. Lex had set himself up as one of
Superman's champions, a man inspired by a zeal for the alien that few
others had. He was the chair of the Conference on Extraterrestrial
Science and two other organizations, and somewhat noted as a collector
of information. Now this was working against him, because any connection
he formed with Clark Kent would be immediately suspect. If Lex had
simply remained an anonymous businessman, there would be nothing too
surprising about him purchasing \emph{The Daily Planet} and looking
through its files. But for Lex Luthor the Superman scholar to do it ---
well, there was no way that Superman wouldn't suspect something.

Lex was moving slowly, and the other players in the game were getting
creative. He was certain that Willie Calhoun was one of them, but didn't
know what intent would explain the actions. There were smear campaigns
and contrived moral quandaries --- attempts to put Superman in a
position where his values would be challenged. Thankfully, none of it
seemed to affect the alien. Lex would have killed Calhoun if he could
have seen a way to do it. It would have been worth it just to stop the
plots. There were so many contacts and lines of communication that had
been burned in the last few months though, and so few ways of getting
dirty work done. Worse, a failure might alert Calhoun. Lex could only
hope that he would figure something out about Clark Kent before Calhoun
or someone else made Superman angry.

\begin{center}\rule{0.5\linewidth}{0.5pt}\end{center}

Willie Calhoun was losing.

He'd won in court, but everything else was in a shambles. Crime was
dropping in Metropolis all over the place, and loyalty seemed to be a
thing of the past as more and more people moved away. The ones that were
left were animals, idiots without the proper restraints. Willie had once
had money, and a nice house, but he was in debt to the banks now with no
way he could see of getting out. He had no real skills he could use in
the real world, and no real nose for legitimate business like Luthor. He
was getting old, and this was the end of the road.

``Fuck Superman,'' said Willie to his empty office. He hoped the alien
would hear. There was hardly a day that went by without some new fantasy
of what he'd do to Superman if the alien weren't invulnerable. It was
comforting, thinking of ripping into that impenetrable flesh.

Superman had cast a spell over the city, one that grew with every
passing day. The last time that people had really doubted him was during
the bombings, when they'd wondered why it was that he wasn't doing more.
What Willie needed to do was to replicate that feeling. If the people
stopped believing in Superman, maybe he'd finally fuck off and fly away.
All the worst psychopaths of Metropolis had been left in Willie's
employ, and it was time to use them.

\begin{center}\rule{0.5\linewidth}{0.5pt}\end{center}

The governor's two children were abducted on their way home from private
school. The abductors had used chloroform on both the driver and the
children. The operation must have been carried out in nearly complete
silence to prevent Superman from hearing, but this was par for the
course in Metropolis. The driver was found laid down in the front seat
with his throat slit. By the time Superman had arrived at the governor's
mansion, an hour had passed and the kidnappers were long gone. No ransom
note ever came. The radio and newspapers latched on to the story, and
someone from somewhere had dug up a picture of June and Robert Whitman
waving at Superman as he flew through the air, which only added fuel to
the fire.

It was five days later that Lois found another letter perched on her
desk, again requesting that she come up to the rooftop. She grabbed her
pencil and notepad, then made the trek up.

``Hello Lois,'' said Superman. He stood with his back to her, looking
out over the city. His cape flowed out behind him. Even after all this
time, Lois couldn't help but see him as anything but a god.

``Superman,'' she replied. ``What brings you to my neck of the woods?''

``I found the governor's children,'' said Superman. He didn't turn
around to face her.

``And are they alright?'' she asked.

``No,'' replied Superman.

Lois was quiet for a moment. She'd been covering the story double time,
since Clark was out with the flu. She'd been hoping it wasn't the
Lindbergh baby all over again. ``Were they---''

``Off the record?'' asked Superman.

Lois hesitated for a moment, then tucked her pencil behind her ear.
``Sure.''

``I found them in a farmhouse forty miles outside of Metropolis. They
had June in the kitchen on a table,'' said Superman. ``Laid out on her
back. Only eleven years old and they were---'' Superman stopped. ``I
barely recognized her. They were taking turns with her.''

Lois felt her stomach churn. She didn't want to be hearing this.

``Robert had been put into the refrigerator,'' Superman continued.
``Nine years old, and they'd used a hatchet to get him into small enough
pieces that he'd fit on the shelves.''

Superman kept clenching and unclenching his fists, and Lois could only
think about how much power he was exerting when his knuckles went white.
Enough to turn coal into diamonds, probably.

``There were three men there,'' said Superman. ``Three men, and they
were --- animals. Monsters. June had a gag in her mouth, and she was
screaming around it.'' He took a breath. ``I flew in as fast as I could.
I pulled her out of there and flew her to the nearest hospital. She beat
against my chest the whole time, crying and shouting. Either she didn't
realize who I was or --- or she realized, and she hated me for being too
late.'' He swallowed hard. ``And then I went back for the men.''

Lois wanted to say something, but the words were stuck in her throat.

``Do you know what I did to them?'' asked Superman.

Lois took an involuntary step back. She couldn't help herself. She could
see the anger radiating off of him now, barely kept in check. It had
been there the whole time, as plain as day, she just hadn't thought to
look for it. The muscles on his neck were strained and his teeth were
clenched. ``What did you do?'' she asked in a soft, small voice.

``I arrested them,'' said Superman.

``You\ldots{} what?'' asked Lois.

``It would have been so easy to kill them,'' said Superman. ``No one's
seen the upper limits of my strength. I could have just snapped my
fingers and ---'' He did just that, and there was a thunderclap. It left
Lois's ears ringing. ``--- like that. Dead. I could have pushed my
fingers straight into their brains, faster than a speeding bullet. It
would have been better than they deserved. They deserved to be chained
up in the deepest, darkest cell I could make for them and slowly starved
to death.''

``Superman,'' said Lois, but there wasn't any set of words that could
come after that to make everything okay.

``I can't keep doing this,'' he said. He finally turned around, and she
could see tears in his eyes. ``I can't keep pretending that I'm someone
that I'm not --- some paragon of truth and justice. I'm just ---'' he
seemed to start to say something but changed his mind. ``Just an alien
from the planet Krypton. I'm not perfect.''

``No one is asking you to be,'' said Lois, but she knew that wasn't
true. Millions of people were clamoring for Superman to be a million
different things. They assumed he was perfect, they just thought he was
perfect in the wrong way. ``They just want you to try your best.''

``My best? I can hear everything going on in the world right now,'' said
Superman. ``No one thinks about what that means.'' He pointed to the
north. ``Just there, six miles away, a house is on fire. The family has
evacuated, but their possessions are burning. A little girl is crying
because she left her doll behind, and I can see it melting. She's
calling out for me to do something. Over there, two miles down the road,
a man just punched his wife in the mouth, and shouldn't I be going to
stop him from doing it again?'' He pointed east. ``There was a flash
flood in China a handful of minutes ago. I can hear three women choking
to death. If I left now, I might be able to save them.'' He pointed to
the south. ``There was a car accident near Atlanta, eight seconds ago.
When the windshield shattered it sliced a man across his neck. If I left
now, I might be able to get him to the hospital before he bleeds out.''
He shook his head. ``But I'm not doing anything to help anyone. I'm
standing here on this rooftop, talking to you.''

Superman stared out over the city, unmoving. Lois watched him.

``It's not selfish to take time for yourself,'' said Lois. She tried to
keep her hands from shaking. She was scared of him, and she wondered
whether he could tell. ``If that's what keeps you sane, there's no shame
stopping to take a breath.''

``Of course there is,'' said Superman. ``Do you know why I wanted to
kill those men? It wasn't just because of what they'd done. It's because
I didn't do enough. I was busy taking time for myself. Those men were
monsters, but I'm a monster for not doing more. I'm a fraud.''

He was silent for a long moment, staring out into space while he
listened to people die. ``I really should be going.'' Lois tried to
think of something to say, but Superman stepped backwards off the roof
and plummeted downwards. The last thing she saw was his cape fluttering
behind him.

Her heart was hammering away in her chest. Her palms were sweaty. There
was no force in the world that could stop Superman. He was being pushed
harder than he could handle, and she was the only one that knew. He'd
revealed himself to her in confidence, but what she now knew was bigger
than any promise. Superman was unstable. She had no idea what to do
about that.

\begin{center}\rule{0.5\linewidth}{0.5pt}\end{center}

Lex Luthor had done some quick, sloppy math.

Superman spent a minimum of four hours a day as Clark Kent. He didn't
spend the entire day in the office, and was often out in the field
reporting on something or another, which gave him some time to be
Superman. Lex Luthor had read every article written by Clark Kent over
the past year, and there were some trends that suggested to him that
much of the information was gathered through the use of x‐ray vision and
super‐hearing. Clark Kent rarely used direct quotes, and rarely claimed
that he'd asked someone a question. He also had a tendency towards
unnamed sources. So call it four out of every eight hours of every
workday as Clark Kent. Forget for a moment that Superman went about his
do‐goodery in an incredibly inefficient way and just crunch the numbers
with best guesses about the variables and probabilities.

The existence of Clark Kent cost four people their lives in the average
day. A human life was worth less to Superman than the ability to sit at
a desk for an hour. And that was just actual death. If you included
rape, assault, property damage, and theft, it became even more
atrocious. Lex immediately revised his estimate of the existential risk
posed by Superman upwards by a substantial amount.

Lex had investigated the Clark Kent issue as much as he could from as
remote a distance as possible. There were a number of troubling aspects
to it, aside from what it implied about Superman's psychology and the
value that Superman placed on human life.

Clark Kent's first byline for The Daily Planet had preceded Superman's
arrival by three months. Superman had claimed to study the world for two
weeks before intervening in human affairs, but that was clearly a lie.
And where had Clark Kent come from? You couldn't just get hired without
paperwork and references. It was admittedly possible that a number of
people were in on the deception, but Lex thought it unlikely. He'd
spoken to Lois Lane in person on a number of occasions, and she hadn't
let even the smallest false note slip. Even if she were a masterful
liar, now that Lex knew the truth he should have been able to spot
something in retrospect. He would speak to her again to make sure, but
if Superman's interviewer weren't in on the secret, Lex couldn't imagine
anyone else would be either.

No, the signs pointed to Clark Kent existing in some respect prior to
his arrival in Metropolis, and this buried past was where Lex needed to
be looking. He hired out a private investigator to strike up a
conversation with a photographer at The Daily Planet named Jimmy Olsen,
and when the topic of a recent article came up, Jimmy was all too ready
to spill the beans on Clark Kent. He'd been obliging enough to provide a
location: Smallville, Kansas.
