\hypertarget{the-allseeing-eye}{%
\chapter{The All‐Seeing Eye}\label{the-allseeing-eye}}

Lex Luthor wasn't the only one gathering information. As the days
passed, people began to make their observations, and a few things began
to become known.

Superman would show up at misdemeanors in downtown Metropolis, felonies
in the greater metropolitan area, and large disasters in the continental
United States. Those who had done the math would point out that Superman
could reach any point on the planet within an hour, but he only rarely
seemed to use this ability; he went to a mine collapse in Peru, a
landslide in Bangladesh, and an earthquake in China, but he seemed
inconsistent in his ranging.

He prioritized crimes against people above crimes against property.
Murder and forcible rape were almost sure to bring a response, while
burglaries often went unstopped. He avoided controversy and grey areas,
and tended to stay away from incidents where both parties were at fault.
He tended to avoid crimes committed by people in the immigrant
neighborhoods, and there was some question about whether this was the
result of a language barrier or because Superman harbored some ideas
about class or racial purity. There were some members of the Eugenics
Society of Metropolis that pointed out that Superman was white.

Superman didn't participate in any foreign wars, despite repeated
requests. There was a civil war in China, and a war between Bolivia and
Paraguay in South America. Thousands died, and Superman did nothing,
presumably because of his claimed neutrality. It was unknown whether
Superman would side with the United States if they once again went to
war. In Germany, the National Socialists had risen to power and
repudiated the Treaty of Versailles, which was generally agreed to be a
worrying development. When the Nazis killed eighty‐three people in a
political purge, there was much discussion about whether Superman's
absence from Germany had been a calculated effort to avoid becoming
embroiled in global politics, a tacit endorsement of their politics, or
whether he simply hadn't known about it until it was too late.

Certainly Superman wasn't active all the time, and he'd proved to be far
from omniscient. Even with him going on patrol and being visible high
above the city, murders still happened with some frequency. The United
States was slowly creeping its way out of the Great Depression, with
Metropolis as the vanguard. Where there had been three murders per day
before his arrival, there was now an average of one. Some people
grumbled that he should do more.

Superman was in the news on a regular basis. He pulled Pretty Boy Floyd
out of a rathole hotel in Gotham City, and requested that the reward be
donated to charity. When the SS Morro Castle caught fire and burned on
the way up from Havana, Superman swooped in and saved the lives of
hundreds. He stopped a tornado in Kansas, and a hurricane moving towards
Florida. He was undeniably a hero.

Through it all, the lawsuits began to pile up. A good number of
criminals came forward with complaints of brutality, and some had the
injuries to prove that they'd at least taken a hit to make their story
plausible. There were accusations of rape that no one believed. Not
every legal issue was so spurious. Superman was sued for theft after
taking steel girders off the back of a truck to shore up a collapsing
factory. He was subpoenaed as a witness to all manner of man‐made
disasters. The case of Shoe v. New York was working its way towards the
Supreme Court. At issue was whether Superman's x‐ray vision could be
used to obtain a warrant for arrest or whether that unreasonably
infringed upon the right to liberty guaranteed by the Fourteenth
Amendment. Most of the court watchers predicted that a half dozen cases
would end up going to the Supreme Court in the coming year. It was a
wonderful time for those with an interest in jurisprudence.

Lex Luthor existed in the background. In public, he was a champion for
Superman, arguing in favor of the stances he believed Superman to favor
and heading the first Conference on Extraterrestrial Science which of
course had Superman as its sole focus. In private, he was the world's
most cautious puppet‐master.

\begin{center}\rule{0.5\linewidth}{0.5pt}\end{center}

``You sure we should be doing this?'' asked Ted. ``It's not exactly
acting.''

``It's acting,'' said Claire defensively. ``We're pretending at being
different people for an audience.''

That Ted had landed a bit part in a doomed production of The Stationary
Man wouldn't have been worthy of note if not for the fact that this made
him more successful than Claire. It was a constant source of tension
between them, and the subtext of nearly all of their conversations.

``Easy for you to say,'' Ted replied. ``You're not the one who's going
to go to jail.''

``Oh hush,'' said Claire. ``The pay is good enough.''

``We probably shouldn't be talking about this where he can hear,'' said
Ted. He fidgeted with the gun tucked into the waistband of his pants. It
wasn't loaded, and he was thankful for that. Guns made him nervous.

``There's nowhere Superman can't hear, the papers said so,'' said
Claire. ``Now come on, I'm ready to go.''

``You'll drop the charges?'' he asked.

``Who on earth do you think I am?'' asked Claire. ``Of course I'll drop
the charges. This whole thing is going to last a single night, tops.
Maybe he won't even show up and we can get paid to do this again.''

``Fine,'' said Ted. He pulled the ski mask down over his head and
whipped out the gun. ``Gimme your goddamned money and you don't get
hurt.''

Claire glanced nervously from side to side. ``Please, I need that money
to feed my baby sister.''

``Hand over the dough,'' said Ted. ``Just hand over the goddamned dough
or I swear to God I will shoot you right in your pretty little mouth and
steal the money off your warm corpse.''

``Superman!'' screamed Claire at the top of her lungs. ``Superman, save
me!''

``Shut your mouth, bitch,'' said Ted with what he hoped was a convincing
sneer. But then he saw Claire's face when he said the b‐word, and
instantly regretted it. He was about to break character and tell her he
was sorry when Superman appeared between them. Neither had seen him
arrive. He was simply there with a rush of air.

``What seems to be the problem?'' asked Superman with half a grin on his
face. He plucked the gun from Ted's hand.

``This bastard was trying to mug me,'' said Claire.

``I wasn't,'' said Ted. He didn't have to feign the fear in his voice.
He'd never realized how tall Superman was before. Odd that it would have
such an effect, when that was the least impressive thing about him.

``Ted and I will be going to the police station,'' said Superman. Ted
felt his stomach tie into a nervous knot at Superman saying his name
before realizing that Superman had probably just read it off of one of
the cards in his wallet. ``If you'd make a statement it would help to
put this man behind bars.''

As Claire looked at him, Ted felt another jolt of honest fear run
through him. She looked like she was going to agree to it. But at the
last second, her face softened, and she shook her head.

``I need to get home to my baby sister,'' said Claire. ``I'll file
something with the police in the morning.''

``Very well,'' said Superman. ``Have a good day.'' Then he flew up into
the air, carrying Ted with him.

A homeless man watched from a distance, and wrote something in his
notebook in an extremely neat script. The next day, a curious personal
ad appeared in The Daily Planet. Lex Luthor made a point of reading
through both of Metropolis's daily newspapers each morning, and so even
if Superman had been watching, there would be nothing suspicious about
the way that Luthor's eyes flickered over the page. There was no copy of
the key to be found anywhere on Lex's person~--- it had been committed
entirely to memory. The actors had been hired by an intermediary who had
no knowledge of Lex Luthor, and the man who'd watched them received
payment from a slush fund that Luthor had cut his connection to years
ago.

\begin{center}\rule{0.5\linewidth}{0.5pt}\end{center}

Leroy Barnes pulled his mask down over his face and hefted his tommy
gun, then charged straight in through the revolving doors of the
Commerce Bank of Metropolis. He used the butt of his gun to smack the
security guard hard in the nose as Sean ``Moustache'' Murphy and Big
Paul Castellano followed closely behind him. Leroy fired off five rounds
into the ceiling, bringing plaster down on the customers. They scrambled
to the floor without having to be told, men in fine suits and women in
glitzy dresses pressing themselves up against the immaculate marble of
the Big Apricot's most prestigious bank.

``God dammit Leroy,'' said Murphy, ``We were supposed to do this
clean.'' Murphy picked up the guard's gun and stuffed it into the burlap
sack they'd be using to carry the money.

``This is a robbery!'' yelled Big Paul, a short man who had once worked
as a jockey down at the Apricot City Racetrack before he'd broken his
leg. He limped, but it didn't slow him down much. ``Get down on the
floor! We don't wanna bump off nobody, so no funny business and we'll be
through this caper in a flash!''

The three of them walked towards the cash registers, guns held out in
front of them, trying their best to cover the whole room. The idea was
to get in and out before the cops had a chance to show up. There was the
question of the Big Blue rearing his ugly head, but that was what
contingencies were for.

``Empty the cash register sweetheart,'' Big Paul said to one of the
cashiers. He was careful not to point his gun straight at her, just in
her general direction. He'd found that people panicked with a gun to
their head. It was better to hold the gun like you didn't want to use
it, instead of like you were seconds away from killing them. ``Throw it
all in this sack and we won't have any trouble.''

``There's no need for that,'' said a voice from the front of the room.
Everyone turned to look at Superman. He'd entered the bank silently, and
stood with his cape hanging down behind him. The revolving door spun
around behind him. Superman looked the same as he ever did, a god
striding among men.

``Stop right there,'' said Murphy. ``We planned for this, ya see?
There's hostages, planted all around the city, and you can stop us or
save them, but not both.''

``I can do both,'' said Superman. ``And I don't negotiate.''

Superman glanced rapidly between the three robbers and closed the
distance to Murphy in the space of a heartbeat. He bent the barrel of
the tommy gun with one hand, and reached into Murphy's jacket with the
other. Murphy dropped the gun and tried to beat against Superman, but it
was like slamming his fists into granite. Superman pulled out a thin
metal case from Murphy's pocket and stared at it with a frown. It was
locked shut, but Superman pried it open with ease and pulled out a slip
of paper. He let the note flutter to the ground after reading it, then
moved forward and tied up both of Murphy's hands with the sleeves of the
man's own jacket.

Leroy and Big Paul had started running away as soon as Superman had
grabbed Murphy, the promise of money forgotten. Big Paul, with his limp,
was falling behind. Superman came at them from behind as they ran,
ripping the guns from their hands and setting both men on the ground.

``You gonna kill us?'' Leroy spat at him. ``Or are you some kind of
pussy?'' Superman turned his implacable gaze towards the criminal, and
Leroy lost his bravado at once, like a balloon being popped.

``No,'' said Superman. He seemed about to say more, but tied them up and
dashed through the revolving door of the bank, leaving it spinning
behind him. On the floor of the bank, huddled among the other customers,
Lex Luthor smiled.

\begin{center}\rule{0.5\linewidth}{0.5pt}\end{center}

Watching the robbery had been a risk, but Lex Luthor had wanted to see
the Man of Steel at least once in person, just in case it would stir
something loose within his mind. Lex stopped by the Commerce Bank three
times a week at the same time of day, and so there was little unusual
about him being there when the robbers arrived. There was nothing that
Superman could use to trace the robbery back to Lex, unless Superman had
been watching as Lex planned it. Even then it was unlikely given the
precautions that Lex had taken.

In his home, Lex Luthor had built a keyboard which connected to the
phone lines. Many nights he could be seen pressing the keys while
staring at his coded notebook, with no apparent output. When he hammered
down the keys to, they didn't produce the normal solid clack of metal
levers pressing up against a ribbon of ink. Though it looked much like a
typewriter, the keys were attached to an electrical mechanism which
translated each press of a key into a tone, which was in turn sent down
the phone lines.

Someone watching Lex Luthor's hands from above might try to observe what
he was typing, but that would be a useless exercise since Lex Luthor was
typing in a crude code on keys that were completely unmarked. Someone
with absurdly superior hearing might find the terminus to the phone
connection at an office building in downtown Metropolis, where the tones
were magnetically recorded on a steel wire and later translated into a
still‐encrypted paper copy by a somewhat bewildered secretary. The paper
copies were filed away, and from time to time Lex Luthor could be seen
stopping by to leaf through them, seemingly able to decode them without
need for a cipher.

The line was split of course, and the terminus in LexCorp offices was a
decoy. The coded messages that filled the cabinets were nonsense, the
letters randomized past the point of recovery, not that Superman had
shown himself to be much of a code‐breaker. The real coded message was
received by a small office out in Star City, California, where it was
decoded into a set of instructions, with a header in English and the
rest in some other language. The people who worked at the office knew
little about who they worked for or what purpose their work served. The
English portion of the message was for them, and told them who to send
mail to, or occasionally who to call, while the second part was for
their recipient, and invariably in a language that the people at this
small office didn't speak~--- an additional protection against Superman,
though it was really more of a minor inconvenience than good security.
The people at the office assumed that their secret master was the United
States government.

This circuitous route was a bit paranoid, even given Superman's
demonstrated surveillance capabilities. Superman had repeatedly been
shown to need to focus on stimulus, and it was Lex's working theory that
Superman's brain filtered out the vast majority of the input that it
received from his ears and eyes. Superman could prime himself to listen
for a gunshot, or the sounds of shouting, but he didn't have total
information processing. For this very reason, most murders in Metropolis
were now surprise attacks using melee weapons that would eliminate the
victim's ability to produce sound. A gunshot was distinctive, while the
sound of a knife slicing flesh was not. In a way, Superman's arrival had
made the underworld a more brutal place.

It was likely that Lex could have skated by on lesser security
precautions than he took, but he'd woken up to nightmares of having his
skull crushed between Superman's hands too many times. In the dream he
was just one in a long line of people that stretched out on either side
of him, an endless number of people waiting to be killed by Superman.
The alien did the work calmly and cleanly, and Lex was the only one who
was trying to fight back. Precautions were the order of the day.

``Mustache'' Murphy hadn't known why he'd been asked to rob the bank.
The jeweler on 4th St and 16th Ave hadn't known why he'd been asked to
make a small case lined with lead. Leroy Barnes hadn't known why he'd
been asked to fire off his gun towards the ceiling. All these men knew
was that they were being paid. Strings had been pulled and messages had
been sent.

The end result had been that Lex Luthor discovered that Superman
couldn't see through lead. More than anything, he was upset that
something so stupid had worked.

It was what you would try if you knew a little bit about x‐rays. Lead
was used to block x‐ray radiation, even people who didn't have a clue
what x‐rays were knew that, so it made sense that Superman's vision
could be blocked by it. Yet Superman's x‐ray vision fairly conclusively
did not use x‐rays. That was obvious just from thinking about it, and of
course Lex had tested it by having patsies carry around sealed strips of
x‐ray film and subject themselves to his gaze. Furthermore, Superman was
able to distinguish colors using his x‐ray vision, and in all respects
treated it simply as ``the ability to see through objects'' instead of
something that made any sense. Yet lead blocked it all the same. It was
a victory to learn that, but utterly infuriating. Lead was used to block
x‐rays because it was dense, and yet it was apparent that any amount of
lead stopped Superman's vision, even a few centimeters. If lead blocked
Superman's vision in the same way that it blocked x‐rays, Superman
shouldn't have been able to see through a solid foot of steel or three
feet of concrete either. Thinking of new physical laws which would
explain this behavior made Lex Luthor frustrated, though this wasn't
terribly unusual where Superman's powers were concerned.

Lex had to wonder whether Superman realized he'd given something away by
reaching to grab the case instead of getting the information through
some other means. Lex had other plans in place~--- when Superman
eventually followed the trail of clues that started with the piece of
paper in the case, he would be confronted with a number of challenges to
his x‐ray vision, and he would be forced to give up a bit of information
at each one. The clues would lead to three locations; a diving bell
beneath a hundred feet of water, a large Faraday cage, and a steel vault
in a closed down bank. There were no hostages to speak of, and either
Superman would use his so‐called x‐ray vision to confirm this or be seen
by spotters engaging in a rescue for someone that wasn't there. Either
would give information.

As the day passed, the reports came back from the spotters. Superman
wasn't seen at any of the locations he should have been led to. If lead
was the only thing that would stop Superman's vision, then it would have
to be lead that Lex would use.

\begin{center}\rule{0.5\linewidth}{0.5pt}\end{center}

Simply lining a room with lead would be of limited use, since it would
give Superman the incentive to pry in precisely the places that his
attention was least wanted. It would be like erecting a sign that said
``Don't look here''. The only way around that was to make lead shielding
so common that Superman wouldn't be able to keep track of them all, and
for that Lex developed a plan.

A scientific paper was mailed out to a number of universities and
businessmen with the cryptic title ``Non‐Röntgenian Vision; An
Exploration from Inference''. The paper used complicated words where
simple ones would do, and meandered over twenty pages when its findings
could properly be summed up in two. There were numerous digressions and
spelling errors, and the author identified himself as a former professor
of physics living in a cabin in the Adirondacks who had been exiled from
Harvard some decades earlier due to indiscretions which the author
implied were fabricated by his jealous colleagues. It was for the most
part scientifically sound, but so mired in authorial problems that it
had no hopes of being properly published in any journal of note. You
would have to read it three times before understanding that it was
talking about Superman.

The disgraced professor had died some years earlier in a Prohibition
speakeasy that had been owned by Lex Luthor. The professor's body had
been dumped in the river and never identified, and his death was known
by very few. The paper's true author was Lex Luthor, who had crafted it
carefully using information made available to members of the public
through the police and the newspapers. The original incidents which
demonstrated Superman's inability to see through lead had been
engineered by Lex himself, both the first one at the bank and a host of
others used to confirm the finding. Taken on its own, the conclusions
were tenuous, but it was enough to get the ball rolling.

The paper was mailed to the office of Thomas Nivas, a Dutch businessman
with no obvious connection to Lex Luthor, and he made a show of reading
it carefully. Where others would dismiss the professor as a crank, Nivas
would take a gamble and begin immediately buying up shares in the
handful of companies that mined or traded in lead. Within two weeks,
Nivas would announce to the world that lead conclusively stopped
Superman's vision, and publicly challenged the Man of Steel to
demonstrate otherwise. Superman never showed up, and though that proved
little, Nivas began to see a trickle of customers. One of the first of
these was Lex Luthor.

It was a happy bit of serendipity when Lois Lane scheduled a second
interview.

\begin{center}\rule{0.5\linewidth}{0.5pt}\end{center}

``Well, of course I trust Superman,'' said Lex. Lois Lane sat across
from him in one of his leather chairs. Across the hallway, the sounds of
construction could be heard, as his study was ripped apart in
anticipation of lead lining on all the walls, the floor, and the
ceiling. When the sheets of lead were in place, the fine woodwork would
be replaced and the room would look exactly like it was before.

Lois Lane had apparently asked Nivas for the name of one of his clients,
and Nivas had mentioned Lex Luthor. It was a minor betrayal of
confidence, but Lex guessed that Nivas had given up Lex's name because
of the conversation they'd had wherein Lex had put forth what he
believed was the most cogent possible argument in favor of a perfectly
innocent man obtaining protection from the eyes of a watchful and
seemingly benevolent god. Nivas didn't know that Lex was the one behind
the funding, nor the author of the paper he'd been mailed.

``I trust Superman,'' said Lex, ``But do you believe that Superman is
perfectly good?''

``Perfectly?'' asked Lois. ``That's a high standard. But he's as damned
close as we're going to get. He's been here four months now, and he's
saved hundreds if not thousands of lives. He doesn't act as a law unto
himself, he just flies through the air and helps people like it was the
most natural thing in the world. He hasn't killed anyone, and despite
what people might allege, I don't believe that he's ever seriously
injured anyone either.''

``All true,'' said Lex with a smile. ``But given that he isn't perfect,
do you think that it's unreasonable to take precautions against the
possibility that he one day acts in some unconscionable way?''

``Is it really worth however many hundreds or thousands of dollars this
renovation is costing you?'' she asked.

``There are a number of factors that go into determining that,'' said
Lex. ``I have enough money that the expense is somewhat trivial to me,
and I have enough intellectual property that having it stolen would be
quite damaging to me~--- patents, ideas, formulas, processes, and half a
hundred other things. Beyond that, there is a value to me in not being
watched, even when I'm not doing anything of note. It brings me peace of
mind, which is worth something even when the actual risk is low. I
suspect much of the sales of this shielding will go to husbands who want
to know that their wives aren't being spied on in the bath.''

``\,`Humans have an intrinsic right to privacy','' said Lois. ``Navis
told me that, and I suspect that he heard it from you.''

``I believe I said something like that, yes,'' said Lex. ``It's one of
the great flaws of our Constitution that a right to privacy is not among
those enumerated. It's funny, isn't it? No one would begrudge you from
having frosted windows in the bathroom or drawing your curtains when
company is over, but as soon as Superman enters the picture many people
think that such measures are somehow indicative of criminality, or
morally wrong in and of themselves.''

``I didn't bring up crime,'' said Lois.

``But you will, in the article?'' asked Lex.

``Of course,'' Lois nodded.

``Then I have a further argument for you,'' said Lex. ``Perhaps you
perfectly trust Superman not to look at you while you change, or perhaps
you have no secrets you'd rather he not be privy to, but do you believe
that Superman will always be the only one with his abilities? We can
infer that there are other aliens out there, and here on Earth there are
plenty of scientists~--- myself included~--- who are working to
reverse‐engineer the things they see him do. If tomorrow my rivals in
business can see through my walls, they'll find my defenses already in
place, which is only prudent.''

``I suppose,'' said Lois. She looked down at her notebook. ``I think I
have everything I need. More than I need, actually. The article isn't
going to be particularly long.''

``You can admit that you enjoy talking to me,'' said Lex.

``It's stimulating, I'll give it that,'' said Lois. ``But I also came
here to thank you. The ERA passed the Senate and moved onto the House,
and even if it fails there I'll consider you to have held up your end of
the bargain.''

``I'm a man of my word Miss Lane,'' said Lex. ``Though I have to warn
you that prospects are bleak. The Eighteenth Amendment has made people
shy of modifying our founding document.''

``All the same,'' said Lois.

There was a moment where perhaps Lex could have asked her to dinner, but
he let it pass by. Lois was tenacious and decisive, intelligent and
principled, and in another time he might have tried to see whether she
could sustain his interest in the long‐term. Now was a time of action,
and the threat of Superman was too great to permit for such idle
distractions. Later perhaps, when Superman lay dead in the street, Lex
would go on the pursuit.

After they'd said their goodbyes, Lex sat in his smoking room and
thought about explosives. The actual designs would have to wait until
his study had been coated in lead, but until that time he could refine
his plans within his head. He would need to find someone to carry out
his will, someone without a strong moral compass, but he thought that he
had just the right person in mind.
