\chapter{Dust to Dust}\label{dust-to-dust}

Lex needed to know what was in the storm cellar. It was a matter of
practical necessity, but there was an emotional component as well. He'd
spent nearly a year of his life in pursuit of what lay there, slowly
working his way backwards from Superman to Clark Kent and Clark Kent to
Smallville. He had three agents in Smallville, one of whom was living on
the farm itself, so close that it ached. In Lex Luthor's fantasy, he
stood in a clean, pressed suit and watched over a workman using an
oxy‐fuel cutting torch. When the doors were opened, he would stride down
into the cellar and find whatever was hidden there.

It couldn't be a secret laboratory. If it were, there was no way that
Lex would have been able to insert his agent onto the farm so easily. It
was possible that the storm cellar was a decoy of some kind, but Lex
found that doubtful. The game wouldn't be at this stage if Superman were
such a supremely paranoid person. More likely, the storm cellar was
booby‐trapped, or simply impassible by human means. Those metal doors
could hide explosive devices or three solid feet of steel. And contained
within the cellar could be anything. All of that made planning a mission
difficult.

Lex Luthor had involved himself in a number of thefts, especially in his
youth. Stealing an unknown object from behind unknown defenses with a
guard that had a nearly unlimited surveillance ability would be
challenging but not, strictly speaking, impossible. Removing Martha Kent
from the farm for the day would be easiest part. Superman could be
distracted by a disaster of some sort, or more likely a series of them.
Getting the proper equipment into place would be trivial, and the
thieves themselves already had their cover identities. It might even be
possible to break into the storm cellar, retrieve whatever was in there,
and then weld it back shut without Superman even knowing a theft had
taken place until he flew over Kansas and used his x‐ray vision to
check. Whether Superman ever checked at all was an open question, but
Lex found it unlikely. Neither Clark Kent nor Superman had been seen in
or around Smallville in the time that his agents had been there.
Superman could watch from high up in outer space, but from what Lex knew
of his psychology, this too was unlikely --- though not so unlikely that
the theft could be done without precautions.

Judging how Superman might react to the theft was more difficult.
Superman would find out that someone knew his secret identity, and he
would know that someone had whatever was in the storm cellar. Obviously
that was far from ideal, but it might be worth it if the cellar
contained the means to defeat, depower, or contain him. Lex Luthor laid
his plans.

\begin{center}\rule{0.5\linewidth}{\linethickness}\end{center}

The letter arrived on April 7th. The date was written at the top of it
was ``4/3/35'' rather than ``April 3rd, 1935'', which was a prearranged
indication that it contained a coded message. The code was fairly
simple, as these things went, and it was solvable without the use of
pencil and paper so long as you knew that the variety of salutation
defined which of the six codes was being used. In this case, ``Dearest
Floyd,'' meant to take the last letter of every word and put it into a
four by four grid which was then read from bottom to top and right to
left. Floyd deciphered it quickly. \emph{Go to Greene shop and get
tickets. April 14th leave from church and take Martha with you. Keep her
away until after the show. Top priority.}

Floyd went down to the grocery store owned by Joseph and Loretta Greene.
He had no idea what their level of involvement in this scheme was, just
as he had no idea what the goal of the scheme itself was. So far as he
could tell, they were either patsies with no real knowledge of what they
were doing, or very skilled deep cover agents. Sometimes Floyd thought
he could see something hard and dangerous behind Joseph Greene's smiles,
but he might have just been imagining things. If they were something
more than store owners, their employer wanted to keep them
compartmentalized, since he'd never been told much about them. When he'd
stayed with them, they'd acted as nothing more than shopkeepers looking
to help out a traveler.

``Floyd!'' called Loretta. ``Good to see you. Would you like to buy a
raffle ticket?''

Floyd smiled at her. She had pretty, blue‐grey eyes. He could imagine
her as a killer, if he tried, but it wasn't clear on her face. ``Well
that depends now, what's the raffle for?''

``Two tickets to see \emph{Anything Goes} in Wichita,'' said Loretta.
She smiled with her eyes. ``A nickel to enter, though a few folks around
here have bought a few entries to increase their chances.''

``Well that sounds lovely,'' said Floyd. ``I think I have a nickel on
me, as a matter of fact.''

It was no surprise to hear that he'd won a week later.

``All the way in Wichita?'' asked Martha when he'd asked her to come
with him.

``We can leave from the church, have lunch in the city, and then see the
show,'' said Floyd. ``I have some money saved up, and I wouldn't mind
spending some of it to show you a nice time. It'll be good to see the
big city.''

``I suppose you're right,'' said Martha. ``I haven't been outside of
Smallville since Jonathan passed.''

In truth, the two of them didn't get along that well. Martha was clearly
lonely. Her husband was dead and her only son was two days away in
Metropolis. In the first week she'd shown him everything that was
required of him, but even after that, she would sit down on a tree stump
with a glass of lemonade and talk at length while he worked on mucking
out the chicken coop or tended to the small garden. Floyd tried to smile
and encourage her. Listening to her stories was half of the reason he
had been hired, but she had a way of rambling on that irked him. She
quite proudly held opinions that might have made her outspoken among the
people of rural Kansas, but were practically pedestrian by the standards
of the people Floyd had met throughout his life. Martha talked about the
exodusters coming to town when she was a little girl, her involvement in
the radical temperance movement, and working the farm with Jonathan
through tornadoes, blizzards, hail, floods, grasshoppers, and droughts.
Floyd tried his best to pretend to be interested, and most of what she
said went into the letters.

Her distaste for pistols aside, the first time they'd really locked
horns was when he brought home a small jar of moonshine. While the rest
of the nation might have recognized Prohibition for the folly that it
was, Kansas had laws against alcohol long before the amendment was
passed, and had kept them in place after it was repealed. The ban was
mostly thanks to little old ladies like Martha Kent. She'd shamed him
for bringing moonshine into the house, told him it was against the law
as if he didn't already know that, and then made him dump it out on the
ground just beside the front steps. The only reason he hadn't gotten his
pistols and shot her three times in the head was the enormous and
ever‐growing amount of money waiting for him once his work was done.

Floyd Lawton was a professional, but the job was getting to him. In the
normal course of his work he would get a job and then spend some time
doing the homework and employing a wide range of skills in things like
lockpicking, disguise, forgery, and so on. The actual murder itself took
a day at the most, and then he'd make his getaway and spend his newfound
wealth on women and booze. His life consisted of long periods of
debauchery punctuated by razor‐sharp focus on a task that had been set
before him. This particular job upset that natural rhythm. For the last
few weeks, he'd been working as a farmhand with no clear end in sight.

When he'd initially arranged for the job, he'd been told that the term
of employment was indefinite, but he hadn't really thought that it would
be so long, especially with the amount that was being put into his
account on a daily basis. Sitting in the woods with a rifle trained on a
cabin for three days was easy for Floyd; this required a different kind
of patience that he wasn't sure he had.

\begin{center}\rule{0.5\linewidth}{\linethickness}\end{center}

The SS \emph{Excelsior} caught fire at nine in the morning. It was a
cruise ship which had taken a recent turn as a ritzy floating restaurant
in order to drum up business for its next voyage. Three times a day it
would pull into the harbor and exchange passengers, giving a large
number of people the chance to experience what a life of luxury on the
seas was like. Sunday was Lois's day off, but her work as a journalist
was never far from her mind, and she didn't have a real affection for
personal time. There was a story somewhere on the \emph{Excelsior},
something that went beyond just the glitz and glamour of it. If there
wasn't a story, then a day of eating fine foods on a fancy ship was a
small price to pay. Luthor was a part owner of the ship, and had paid
her way.

Lois was first alerted to the fire when a crewman hurried across the
dining room. She'd set her fork down and rushed after him, and it was
when she heard the panic in their voices that she began to smile. It
wasn't many days that she got to be so close to a story as it developed.
She was an excellent swimmer, and in the worst case scenario could tread
water for long enough to get rescued, if not outright swim to shore. The
water wasn't too cold, and hypothermia wouldn't be an issue. All in all,
it was a pleasant enough time and place to be on a sinking boat.

It took a full fifteen minutes for Superman to show up, by which point
the electrical cables and hydraulic lines had both been burned through,
leaving the ship adrift and without radio. The \emph{Excelsior} had been
at its furthest distance from Metropolis when the fire started, and just
making the turn back towards the city. If it weren't a Sunday, the ports
would have been busier, but as it was the effort to provide them a
rescue was looking pitiful.

Superman moved low to the water as he came in, splashing up waves behind
him, and entered straight through the side of the burning ship. The fire
was out within half a minute, though smoke and steam still rose around
the ship. The ship was listing to one side, and Lois held firm to the
railing. A number of the lifeboats had been lowered into the water, and
the women and children were being put onto them. Someone had tried to
grab Lois's arm and lead her away, but of course she was having none of
it. She felt a lurch from the keel of the ship, followed by a loud
snapping sound.

``The ship is too damaged for me to move,'' said Superman from just
beyond the side of the ship. He'd moved there so quickly she couldn't be
sure he hadn't been there all along. He stood in mid‐air with his feet
pointed down, and talked clearly and loudly with a rich baritone. Lois
doubted that there was a person on the ship who couldn't hear him.
``Everyone stay calm, the fire has been put out and you're in no
danger.''

The evacuation was neat and orderly, and done with a minimum of fuss. A
small boy laughed and jumped into the water, and Superman pulled him out
and put him on a lifeboat with a stern admonition not to engage in
foolishness. With Superman there, no one really feared for their lives.
Lois heard a man say that it was impossible to die when Superman was
standing next to you.

He landed on the deck next to Lois. ``Do you need assistance Miss
Lane?'' he asked with a half grin, as though nothing had ever passed
between the two of them. Worse, he said it like there weren't hundreds
of people dying all over the world with every passing minute. His mask
was so complete that she almost believed it.

``I can make it to the lifeboat by myself,'' said Lois. She'd been
thinking what about she and Luthor had been talking about of late, and
forced the next words out. ``But if you're heading back into Metropolis
anyway, I wouldn't mind a direct flight.'' She smiled, and could feel
herself showing too much teeth, but Superman smiled back and returned to
helping people into their life boats. When everyone had been evacuated
from the ship and the Coast Guard were on their way, Superman once again
landed beside her and held out an arm towards her. Trying not to think
about it too much, she stepped towards him and allowed herself to be
swept up in his arms.

She'd been sitting in Lex's study two weeks prior ago he'd brought up
the idea.

\emph{You're one of the anchors holding Superman in place}, wrote Lex.
\emph{You need to bind yourself tighter to him, so that he'll listen to
you. He's attracted to you. Use that.}

She made the hand signal for \emph{No}. She and Lex had some two dozen
signals that they used for messages that were too short for paper, a
sign of how long they'd kept up their charade. The book was nearing
completion, with Lex as a full co‐author, and there was nothing close to
a solution for the Superman problem. He'd suggested that they begin work
together on a new book after the one on Superman was done, but Lois
wasn't sure that there was a point in continuing.

\emph{Why?} he signed back.

Lois sighed and started writing a message. \emph{He would know that I
was lying. His senses are too sharp for me to fool him. And I could only
keep it up for so long before he would figure it out.} She paused with
her pencil poised over the page. \emph{It would increase the scrutiny on
me. And I don't like him. He's too powerful.}

\emph{You've criticized me for not doing enough}, Lex wrote back.
\emph{This is a good plan. Scrutiny we can deal with. I understand that
you don't like him, but if you're truly worried about him going rogue,
this is one of the best ways to stop it from happening.}

She and Luthor had gotten to know each other well over the course of
their two person conspiracy, but she still wasn't entirely sure that he
took what she'd told him seriously. He'd expanded his charitable efforts
and began contributing to various legal efforts on Superman's behalf,
but it never felt quite as concrete as she might have hoped. Luthor
wanted to deal with Superman on an ideological or psychological level,
and when she'd told him that wouldn't be enough, he'd quirked an eyebrow
and asked what more they could possibly do. Superman was invincible,
everyone knew that.

She couldn't argue with the logic of providing an anchor for Superman,
but the thought of courting him made her skin crawl. He was strong,
handsome, popular, and powerful, but she hadn't been able to shake the
sense of danger she felt on their first meeting, and after his breakdown
she'd stopped trying to see him in a more favorable light. He was an
alien pretending at being a moral exemplar when really he was much
closer to an ordinary man. Who knew what personality lay in wait behind
the mask he wore? Lois had never had anything resembling a lasting
relationship, but she'd gone on dozens if not hundreds of dates. Some of
the men were creeps right off the bat. With others it didn't become
clear until the third drink, when she'd already begun thinking about the
next date. And just once, the guy she'd been dating was arrested for
beating a woman to death. She'd been dating him for two weeks at that
point, and wouldn't have believed he was actually guilty except for the
fact that she had contacts within the police department who'd shared the
evidence with her. It had taken a long time for her to actually want to
spend time in the company of a man after that.

\emph{It's fine if you don't want to do it}, wrote Lex. \emph{But it's
important to make the distinction between you having a personal distaste
for your involvement and the plan actually being a poor one.}

Lois thought about her objections. Superman could use his incredible
senses to watch a person's breathing and listen to their pulse, but so
far he hadn't shown any real ability to translate that into an ability
to see whether someone was lying. She didn't want to be his girlfriend
or anything else, but it was difficult to argue that humanity as a whole
would be in a better position if Superman had someone that he actually
listened to. Superman already cared about her in some way, and she
already had to assume that he was watching her. There was a risk that
Superman would discover that she was trying to play him, but that came
down to whether Lois was good enough to keep it up. She would just have
to become a better liar.

She didn't give Lex an answer, but had started preparing for the next
time her path crossed with Superman's all the same. And that was how she
ended up in his arms, flying over the Lower Metropolis Bay.

It wasn't so bad as before. He kept the speed gentle and stayed close to
the water, so that if he dropped her it would only be unpleasant and not
instantly fatal. Lois had her arms wrapped around his neck, and pressed
her face against his chest to keep it out of the wind. So far as she
could tell, it was exactly what he wanted. Her fear was still present,
but if any of it showed perhaps he would mistake it for something else.

He set her down gently, near the stretch of river where Luthor's long
mural stood.

``Thank you,'' said Lois. She placed a hand against his chest, and stood
close to him. ``For everything.'' She tried to ignore the people
watching them.

Superman seemed about to say something, then cocked his head to the
side. ``There's a chemical spill down in Dockside,'' he said. ``If
you're ever in need, just call my name.''

And with that he was off, flying through the air towards some new
disaster. Lois's hands were trembling slightly, but it had gone better
than she'd thought it would.

\begin{center}\rule{0.5\linewidth}{\linethickness}\end{center}

``Why doesn't Superman do something about this drought?'' asked Bill
Parker.

Martha Kent always made it a point to go to church early, and Floyd sat
with her. Attending the Zion Lutheran Church was more about community
than religious fulfillment, and Martha never missed a chance to chime
in, no matter the topic of conversation.

``And how would he do that Bill?'' asked Martha. ``He can fly, not
control the weather.''

``Well,'' said Bill. ``Well he could spin around a bunch and pull some
water to us.'' He spun his finger around in front of him to demonstrate.

``He'd be liable to flood our farms if he tried that, and where on Earth
would he get the water to do it?'' asked Martha.

``Lake Superior,'' replied Bill. ``Fresh water, more than we'd ever
need, and he could just funnel it up like that. Five hundred miles or so
ain't nothing to him. And there are waterspouts, ain't there? Same
thing.''

``It wouldn't work,'' said Martha with her arms crossed in front of her.

``Then a canal, say,'' replied Bill. ``We can't take much more of these
dust storms.''

``Sit back and enjoy the clear day,'' said Martha.

``Superman doesn't do hard labor,'' said Pete Ross, who ran the auto
repair place.

``Well we could use a canal,'' said Bill. ``I don't care how we get
it.''

Floyd tried to resist rolling his eyes, and settled in for another
sermon. The pastor was young, and his lessons were obvious by the time
he was three sentences in. Floyd was far from being a religious man, but
he'd always thought that the true meaning of what was said shouldn't be
revealed until near the end, when it all came together and made itself
clear.

After the sermon was over, Floyd waited next to the truck for Martha.
The musical was showing at two in the afternoon, which left them just
enough time to have some lunch in the city. The skies were clear and
blue. Martha liked to talk to the other church goers for a good long
while, and if Floyd owned a watch he would have been looking at it every
few seconds. There was no real hurry though. The whole point of the
operation was for him to keep Martha away from the farm for as long as
possible, and it didn't matter whether she was talking to friends or on
the road.

``The barometer's dropping fast,'' said Martha as she walked towards the
truck. ``There's going to be a dust storm.''

``Skies look clear and blue to me,'' said Floyd with a ready smile.
There wasn't a cloud in sight. ``I'm sure if there's a storm we can take
cover in Wichita better than on the farm.''

Martha shook her head. ``No, the radio says it's going to be bad, and we
can't be out on the open road. Besides that, we need to prepare the farm
to weather it as best we can.''

Floyd thought on that. The Greenes hadn't been in church, which probably
meant that they were already on the farm --- probably cracking open the
storm cellar. He couldn't very well go back to the farmhouse with Martha
and come across them in some incriminating position.

``Please Missus Kent, I'm sure we'll be fine. Worse comes to worst we
pull over and take shelter in someone's cellar. Folk in Kansas are nice,
I can't imagine that anyone would turn us away.'' Floyd smiled, and
hoped he didn't seem to desperate. ``I've never seen a musical before,
and if we miss this one I think I might never.''

``I know you had your heart set on it,'' said Martha. ``But we have to
go home. If we don't seal those windows the house will be full of dirt,
to say nothing of what's going to happen to the chickens.''

``Alright,'' said Floyd. ``Maybe someday I'll save up enough to go see a
musical all on my own.''

``If it's as bad as I think it is, the theater would be closed anyway,''
said Martha. ``Now let's get going.''

The next step was sabotage. Floyd could choke the engine and then
disconnect some vital part when he popped the hood of the truck to see
what was wrong. He was just about to do this when Martha spoke.

``I've heard some unpleasant rumors, Floyd,'' said Martha. Floyd spared
a glance at her and saw a frown on her face.

``Rumors?'' he asked, though he could guess right away what they were.

``You and that Betty Graber,'' said Martha. ``There's some talk that the
two of you are an item, and I can't say that I could tolerate you living
with me if that's true.''

Betty Graber had made eyes at him from nearly the moment he'd set foot
into town. She was nearly sixteen, and naive enough to think that there
was something romantic about a drifter. More likely than not she thought
she could change him, but better women had tried and failed at that. She
would chat with him whenever their paths crossed, and pretend to be
going in the same direction as he was so they could walk the two blocks
that made up Smallville's downtown together. If he'd been smart, he
would have avoided her, but Smallville had little to offer in the way of
entertainment and booze was prohibited. He'd taken her virginity in a
grassy field, and she'd cried the whole time. Afterwards, she followed
him around like a puppy that was particularly desperate for affection.
It was the very definition of trouble, and it was only after he'd been
with her that he could see that with any clarity.

``There's no truth to it,'' said Floyd. ``She's keen on me, I can tell,
but I would never take advantage.''

Martha said nothing, and Floyd risked a glance over at her. She looked
upset. He couldn't be sure how much she had heard from the gossipmongers
at church, but there was a serious risk that his room, board, and dollar
a week were about to disappear, and that meant in turn that his enormous
salary was going to disappear too. Depending on what Betty had let slip,
there might be some way to salvage things. He was mulling this over when
he realized that he could see the Kent farm ahead of him. He choked the
engine, and popped out of the car.

``I'll see what's wrong,'' he said quickly.

``I'll just walk the rest of the way,'' said Martha. She pointed back
behind her. ``I can see the dust storm on the horizon already.'' And in
fact she was right, to the north the horizon was muddled and blackish
brown. The storm was moving fast.

Floyd could see a large truck on the farm from where they were, and it
was just a matter of time until Martha noticed it too. The mission was
blown, and now it was a question of what his employer would want. The
problem was, he just hadn't been given enough information, because he
wasn't supposed to be anywhere near the operation. The storm cellar had
to be the target. It was a question of whether it would be better to let
the operation be discovered by Martha or better for Floyd to lose his
cover. The fact that Martha was threatening to kick him out made the
decision easy.

Floyd Lawton pulled out his gun. He caught up with Martha in a few short
strides, and smacked her in the head with the butt of the pistol as she
turned to look at him. From there it was just a matter of half a minute
to pick up her light and frail body and set it in the back of the truck.
With some quick work with ropes and a handkerchief he had her bounded
and gagged. He restarted the truck and drove towards the house.
Hopefully his employer would understand.

``What the hell are you doing here?'' asked Joseph Greene as Floyd
pulled up.

``Change of plans,'' said Floyd. ``I'm here to help you two out. There's
a dust storm coming and we need to get out ahead of it if we can.'' That
was when he spotted Loretta beside the large delivery truck they had,
aiming a rifle at his chest. Either he was getting rusty or she was
well‐trained.

Joseph stared at him. ``Where's Missus Kent?''

``Knocked out,'' said Floyd. ``In the back of the truck.''

Joseph swore. ``Alright, we need to move then, quickly. Come this way.''

The moving truck was backed up towards the storm cellar, which had its
doors cast wide open. Beside it was an array of cutting tools. Joseph
stepped down into cellar, and Floyd walked behind him. A small lantern
cast light on the object.

``What in the hell is that?'' asked Floyd.

It looked like a kite had swallowed an enormous egg. There was hardly a
straight angle on it, save for the tips of the wings which were set two
thirds of the way back, and it was easily six feet wide. The metal was
gleaming a dull gold where the dust had been wiped away. There were no
openings or protrusions of any kind, just pleasantly sweeping curves.
When Floyd looked closely at the area that had been cleaned of dust, he
could see that it was tiled in an intricate pattern.

``No idea,'' said Joseph. ``Now come on, no talking, we need to get
moving. Now.''

The three of them heaved at it, and eventually the two men got their
shoulders beneath the two stubby wings and managed to lift it up enough
to start moving it up the wooden steps. They stopped to rest once it was
outside, then with another burst of effort got it up into the back of
the truck. Floyd collapsed against the side of the truck. He'd initially
thought that the object would be unmovable without wrapping ropes around
it and using the truck to pull it, but instead it was just obscenely
heavy.

``Storm's coming,'' said Loretta. A black cloud stretched from one end
of the horizon to the other, hanging low. She walked over to Floyd's
truck and turned to look at him with a frown. ``Take that truck and
drive as far away from here as you can. Keep Martha with you.'' The
Greenes moved swiftly, and were already on the move by the time that
Floyd had gotten the truck started up again. In his rear view mirror he
could see Martha Kent, folded up like a doll. After a moment of looking
at her, he realized that she didn't seem to be breathing. Floyd swore
and hopped out of the truck, but as he reached down to check for a pulse
he could see that he was far too late. Blood had trickled out from her
nostrils and dried in place, and her eyes had gone milky. He swore
again, and got back in the truck.

The storm was a godsend, so far as Floyd was concerned. It would cover
up both the death of Martha Kent and his disappearance from Smallville.
Dust storms didn't usually kill unless they caught you by surprise and
choked you out, but if this one was bad, maybe that's what people would
assume happened. He and Martha would both be missing, along with the
truck, and surely the police would draw their own conclusions. The empty
storm cellar with its doors blown open would only contribute to that.
Floyd's money was held in a bank in Kansas City, and he'd make a
withdrawal before anyone knew what had happened.

Floyd was a dozen miles away from Kansas City when the storm front
caught up with him. Visibility dropped down to nothing, and he kept
going more through the feel of the road than because he could see what
was in front of him. A strong gust of wind hit the truck, nearly sending
it sliding sideways. When Floyd looked back, Martha's body was gone.

The driver's side door flew away in a tumble of twisted metal and broken
glass, and Floyd was wrenched from his seat and flung into the dirt. He
closed his eyes tight and spat out a mouthful of blackened soil. Half a
second later the wind whipped him hard, pulling him up into the air. He
fell, twisting in the wind, for what seemed like a long time. He was
stopped when his shirt snagged on something, suspending him off the
ground. He wiped at his eyes, trying to clear the dirt away. It was only
slowly that he realized he was being held by a man. The dirt wasn't
blowing anymore, because the clouds were now below them, sweeping over
the Midwest like a horde of black demons. He had been thrown up into the
sky and caught by a god.

Superman --- for it could only be Superman --- was covered in the same
fine soil that Floyd was. His hair was a mess and his face was caked
with dirt, save for just below his eyes where there were twin streaks of
pink flesh. He was crying. Floyd didn't move, and didn't say anything.
His employer had been taking precautions against the arrival of
Superman, and now Superman was here. The only thing to strive for was
getting out of this alive, and the only way to do that was to convince
Superman to bring him back down to the ground. Superman didn't kill
people, but he wasn't supposed to cry either. Floyd was being held up by
the cheap, dirty fabric of his shirt, which was pressing uncomfortably
against his armpits.

``You killed her,'' said Superman in a voice filled with cold fury.

``It was an accident,'' said Floyd. His voice was hoarse. He must have
swallowed quite a bit of dirt on his way up. ``I meant to knock her out,
not kill her. I just hit her too hard.''

``An accident,'' spat Superman. ``I spend my every waking second
treading lightly, trying not to go too fast, trying not to break your
fragile little bodies. Do you understand how careful I was in bringing
you up out of the storm? How easily I could have broken your bones, or
liquified your muscles? Do you think I have one single \emph{ounce} of
sympathy for you?'' Superman let out a raw and primal scream that left
Floyd momentarily deaf. It was so loud his very bones had vibrated. And
even then, he could tell that Superman had been holding back.

``I'm sorry,'' said Floyd, barely able to hear his own words.

``She was my mother,'' replied Superman.

Floyd had a sudden moment of clarity. He'd had a dozen conversations
with Martha Kent about her son, and all of them had been given their
context. His employer's paranoia now seemed reasonable. There were
pictures of Clark throughout the house, and as he stared at the dirty
and distraught face in front of him, he realized the truth.

``Listen Clark,'' said Floyd quickly.

One of Superman's hands flickered forward and wrapped around Floyd's
throat, stopping the attempt at persuasion before it could even begin.
The pressure was firm but gentle. If not for the other hand still
twisted around and grabbing Floyd's shirt, he'd be choking to death.

``Don't call me that,'' said Superman. He stared at Floyd with hatred in
his eyes for a long moment. Floyd wondered whether this was the end.
Surely Superman wouldn't let him live with the knowledge of his second
identity.

``Is there a point to your life?'' asked Superman. ``Did God have any
purpose behind your creation other than to test me?''

Floyd tried his best to nod. Slowly, Superman released his throat.

``My employer,'' said Floyd. ``I can help you get to him. He never
showed his face, but we have ways of communicating, and there's a bank
account he puts money into.''

Superman nodded. ``Talk.''

\begin{center}\rule{0.5\linewidth}{\linethickness}\end{center}

\emph{We were able to remove the foreign contaminant from the lab's
water supply. The source of it was a large, singular deposit beneath the
surface, which has now been safely separated out. The origin of the
contaminant is unknown, but initial tests have shown it to be somewhat
exotic. In other news, our biological research is going well, but
unfortunately our prized test subject has been injured, perhaps
mortally. We suspect mishandling by one of the other workers in the lab.
While that experiment was originally going to be a double‐blind, we now
believe that some bias may have crept in. With that said, we're proud to
report that our total cycle time is down to just an hour and a half.}

Lex stared at the after‐action report. The storm cellar had contained a
spaceship --- or something similar enough to it --- and was now housed
three hundred feet below the ground in a lead mine near Pleasanton,
Kansas. In the next part of the plan it would be encased in a quantity
of refined lead, and from there shipped out to an atomic research
laboratory in Hub City which had been set up far in advance. The man and
woman posing as Joseph and Loretta Greene were long gone, and their
usefulness was at an end, given that Superman might have seen their
faces.

There had been no word from Floyd Lawton.

That Martha Kent was injured and probably dead was troubling. Superman
had tethers to the world, and she was one of them. From what Lex had
been able to find out, Clark Kent had few friends, and none that
extended beyond his employment at \emph{The Daily Planet}. In all
likelihood, Superman now knew that his secret identity was compromised,
which was another point of worry. Events were not yet spiraling out of
control, but if the plan had followed the happy path, managing Superman
would have been much easier.

With the spaceship in Lex's possession, hopefully a solution could be
found before Superman broke free of his moral constraints.

\begin{center}\rule{0.5\linewidth}{\linethickness}\end{center}

\emph{Author's Note: ``Black Sunday'' was the worst dust storm of the
era, and shortly afterwards the term ``Dust Bowl'' was coined.}
